\newcommand{\customDir}{../}
\RequirePackage{ifthen,xifthen}

% Input inkl. Umlaute, Silbentrennung
\RequirePackage[T1]{fontenc}
\RequirePackage[utf8]{inputenc}

% Arbeitsordner (in Abhängigkeit vom Master) Standard: .LateX_master Ordner liegt im Eltern-Ordner
\providecommand{\customDir}{../}
\newcommand{\setCustomDir}[1]{\renewcommand{\customDir}{#1}}
%%% alle Optionen:
% Doppelseitig (mit Rand an der Innenseite)
\newboolean{twosided}
\setboolean{twosided}{false}
% Eigene Dokument-Klasse (alle KOMA möglich; cheatsheet für Spicker [3 Spalten pro Seite, alles kleiner])
\newcommand{\customDocumentClass}{scrreprt}
\newcommand{\setCustomDocumentClass}[1]{\renewcommand{\customDocumentClass}{#1}}
% Unterscheidung verschiedener Designs: htw, fjs
\newcommand{\customDesign}{htw}
\newcommand{\setCustomDesign}[1]{\renewcommand{\customDesign}{#1}}
% Dokumenten Metadaten
\newcommand{\customTitle}{}
\newcommand{\setCustomTitle}[1]{\renewcommand{\customTitle}{#1}}
\newcommand{\customSubtitle}{}
\newcommand{\setCustomSubtitle}[1]{\renewcommand{\customSubtitle}{#1}}
\newcommand{\customAuthor}{}
\newcommand{\setCustomAuthor}[1]{\renewcommand{\customAuthor}{#1}}
%	Notiz auf der Titelseite (A: vor Autor, B: nach Autor)
\newcommand{\customNoteA}{}
\newcommand{\setCustomNoteA}[1]{\renewcommand{\customNoteA}{#1}}
\newcommand{\customNoteB}{}
\newcommand{\setCustomNoteB}[1]{\renewcommand{\customNoteB}{#1}}
% Format der Signatur in Fußzeile:
\newcommand{\customSignature}{\ifthenelse{\equal{\customAuthor}{}} {} {\footnotesize{\textcolor{darkgray}{Mitschrift von\\ \customAuthor}}}}
\newcommand{\setCustomSignature}[1]{\renewcommand{\customSignature}{#1}}
% Format des Autors auf dem Titelblatt:
\newcommand{\customTitleAuthor}[1]{\textcolor{darkgray}{Mitschrift von #1}}
\newcommand{\setCustomTitleAuthor}[1]{\renewcommand{\customTitleAuthor}{#1}}
% Standard Sprache
\newcommand{\customDefaultLanguage}[1]{}
\newcommand{\setCustomDefaultLanguage}[1]{\renewcommand{\customDefaultLanguage}{#1}}
% Folien-Pfad (inkl. Dateiname ohne Endung und ggf. ohne Nummerierung)
\newcommand{\customSlidePath}{}
\newcommand{\setCustomSlidePath}[1]{\renewcommand{\customSlidePath}{#1}}
% Folien Eigenschaften
\newcommand{\customSlideScale}{0.5}
\newcommand{\setCustomSlideScale}[1]{\renewcommand{\customSlideScale}{#1}}
\newcommand{\customSlideHeight}{9.63cm}
\newcommand{\setCustomSlideHeight}[1]{\renewcommand{\customSlideHeight}{#1}}
\newcommand{\customSlideWidth}{12.8cm}
\newcommand{\setCustomSlideWidth}[1]{\renewcommand{\customSlideWidth}{#1}}

%\setboolean{twosided}{true}
%\setCustomDocumentClass{scrartcl}
%\setCustomDesign{htw}
%\setCustomSlidePath{Folien}

\setCustomTitle{Stochastik}
\setCustomSubtitle{Vorlesungsskript}
\setCustomAuthor{Falk-Jonatan Strube}
%\setCustomNoteA{TitlepageNoteBeforeAuthor}
\setCustomNoteB{Vorlesung von Prof. Dr. Schwarzenberger}

%\setcustomSignature{\footnotesize{\textcolor{darkgray}{Mitschrift von\\ \customAuthor}}	% Formatierung der Signatur in der Fußzeile
%\setcustomTitleAuthor{\textcolor{darkgray}{Mitschrift von #1}}	% Formatierung des Autors auf dem Titelblatt

%-- Prüfen, ob Beamer
\ifthenelse{\equal{\customDocumentClass}{beamer}}{
%%% TODO: andere Layouts für Beamer außer HTW
	\documentclass[ignorenonframetext, 11pt, table]{beamer}
	
	\usenavigationsymbolstemplate{}
	\setbeamercolor{author in head/foot}{fg=black}
	\setbeamercolor{title}{fg=black}
	\setbeamercolor{bibliography entry author}{fg=htworange!70}
	%\setbeamercolor{bibliography entry title}{fg=blue} 
	\setbeamercolor{bibliography entry location}{fg=htworange!60} 
	\setbeamercolor{bibliography entry note}{fg=htworange!60}  
	
	\setbeamertemplate{itemize item}{\color{black}$\bullet$}
	\setbeamertemplate{itemize subitem}{\color{black}--}
	\setbeamertemplate{itemize subsubitem}{\color{black}$\bullet$}
	\makeatother
	\setbeamertemplate{footline}
	{
	\leavevmode
	\def\arraystretch{1.2}
	\arrayrulecolor{gray}
	\begin{tabular}{ p{0.167\textwidth} | p{0.491\textwidth} | p{0.089\textwidth} | p{0.103\textwidth}}
	\hline
	\strut\insertshortauthor & \insertshorttitle & Slide \insertframenumber{}% / \inserttotalframenumber{}
	 & May 4, 2016\\
	\end{tabular}
	}
	\setbeamertemplate{headline}
	{
	\leavevmode
	\setlength{\arrayrulewidth}{1pt}
	\hspace*{2em}	
	\begin{tabular}{p{0.63\textwidth}}
	\rule{0pt}{3em}\normalsize{\textbf{\insertsection\strut}}\\
	\arrayrulecolor{htworange}
	\hline
	\end{tabular}
	\begin{tabular}{l}
	\rule{0pt}{4em}\includegraphics[width=3.25cm]{\customDir .LaTeX_master/HTW_GESAMTLOGO_CMYK.eps}\\
	\end{tabular}
	}
	\makeatletter	
}{	
	%-- Für Spicker einiges anders:
	\ifthenelse{\equal{\customDocumentClass}{cheatsheet}}{
		\documentclass[a4paper,10pt,landscape]{scrartcl}
		\usepackage{geometry}
		\geometry{top=2mm, bottom=2mm, headsep=0mm, footskip=0mm, left=2mm, right=2mm}
		
		% Für Spicker \spsection für Section, zur Strukturierung \HRule oder \HDRule Linie einsetzen
		\usepackage{multicol}
		\newcommand{\spsection}[1]{\textbf{#1}}	% Platzsparende "section" für Spicker
	}{	%-- Ende Spicker-Unterscheidung-if
		%-- Unterscheidung Doppelseitig
		\ifthenelse{\boolean{twosided}}{
			\documentclass[a4paper,11pt, footheight=26pt,twoside]{\customDocumentClass}
			\usepackage[head=23pt]{geometry}	% head=23pt umgeht Fehlerwarnung, dafür größeres "top" in geometry
			\geometry{top=30mm, bottom=22mm, headsep=10mm, footskip=12mm, inner=27mm, outer=13mm}
		}{
			\documentclass[a4paper,11pt, footheight=26pt]{\customDocumentClass}
			\usepackage[head=23pt]{geometry}	% head=23pt umgeht Fehlerwarnung, dafür größeres "top" in geometry
			\geometry{top=30mm, bottom=22mm, headsep=10mm, footskip=12mm, left=20mm, right=20mm}
		}
		%-- Nummerierung bis Subsubsection für Report
		\ifthenelse{\equal{\customDocumentClass}{report} \OR \equal{\customDocumentClass}{scrreprt}}{
			\setcounter{secnumdepth}{3}	% zählt auch subsubsection
			\setcounter{tocdepth}{3}	% Inhaltsverzeichnis bis in subsubsection
		}{}
	}%-- Ende Spicker-Unterscheidung-else
	
	\usepackage{scrlayer-scrpage}	% Kopf-/Fußzeile
	\renewcommand*{\thefootnote}{\fnsymbol{footnote}}	% Fußnoten-Symbole anstatt Zahlen
	\renewcommand*{\titlepagestyle}{empty} % Keine Seitennummer auf Titelseite
	\usepackage[perpage]{footmisc}	% Fußnotenzählung Seitenweit, nicht Dokumentenweit
}

% Input inkl. Umlaute, Silbentrennung
\RequirePackage[T1]{fontenc}
\RequirePackage[utf8]{inputenc}
\usepackage[english,ngerman]{babel}
\usepackage{csquotes}	% Anführungszeichen
\RequirePackage{marvosym}
\usepackage{eurosym}

% Style-Aufhübschung
\usepackage{soul, color}	% Kapitälchen, Unterstrichen, Durchgestrichen usw. im Text
%\usepackage{titleref}
\usepackage[breakwords, fit]{truncate}	% Abschneiden von Sätzen
\renewcommand{\TruncateMarker}{\,…}

% Mathe usw.
\usepackage{amssymb}
\usepackage{amsthm}
\ifthenelse{\equal{\customDocumentClass}{beamer}}{}{
\usepackage[fleqn,intlimits]{amsmath}	% fleqn: align-Umgebung rechtsbündig; intlimits: Integralgrenzen immer ober-/unterhalb
}
%\usepackage{mathtools} % u.a. schönere underbraces
\usepackage{xcolor}
\usepackage{esint}	% Schönere Integrale, \oiint vorhanden
\everymath=\expandafter{\the\everymath\displaystyle}	% Mathe Inhalte werden weniger verkleinert
\usepackage{wasysym}	% mehr Symbole, bspw \lightning
%\renewcommand{\int}{\int\limits}
%\usepackage{xfrac}	% mehr fracs: sfrac{}{}
\let\oldemptyset\emptyset	% schöneres emptyset
\let\emptyset\varnothing
%\RequirePackage{mathabx}	% mehr Symbole
\mathchardef\mhyphen="2D	% Hyphen in Math

% tikz usw.
\usepackage{tikz}
\usepackage{pgfplots}
\pgfplotsset{compat=1.11}	% Umgeht Fehlermeldung
\usetikzlibrary{graphs}
%\usetikzlibrary{through}	% ???
\usetikzlibrary{arrows}
\usetikzlibrary{arrows.meta}	% Pfeile verändern / vergrößern: \draw[-{>[scale=1.5]}] (-3,5) -> (-3,3);
\usetikzlibrary{automata,positioning} % Zeilenumbruch im Node node[align=center] {Text\\nächste Zeile} automata für Graphen
\usetikzlibrary{matrix}
\usetikzlibrary{patterns}	% Schraffierte Füllung
\usetikzlibrary{shapes.geometric}	% Polygon usw.
\tikzstyle{reverseclip}=[insert path={	% Inverser Clip \clip
	(current page.north east) --
	(current page.south east) --
	(current page.south west) --
	(current page.north west) --
	(current page.north east)}
% Nutzen: 
%\begin{tikzpicture}[remember picture]
%\begin{scope}
%\begin{pgfinterruptboundingbox}
%\draw [clip] DIE FLÄCHE, IN DER OBJEKT NICHT ERSCHEINEN SOLL [reverseclip];
%\end{pgfinterruptboundingbox}
%\draw DAS OBJEKT;
%\end{scope}
%\end{tikzpicture}
]	% Achtung: dafür muss doppelt kompliert werden!
\usepackage{graphpap}	% Grid für Graphen
\tikzset{every state/.style={inner sep=2pt, minimum size=2em}}
\usetikzlibrary{mindmap, backgrounds}
%\usepackage{tikz-uml}	% braucht Dateien: http://perso.ensta-paristech.fr/~kielbasi/tikzuml/

% Tabular
\usepackage{longtable}	% Große Tabellen über mehrere Seiten
\usepackage{multirow}	% Multirow/-column: \multirow{2[Anzahl der Zeilen]}{*[Format]}{Test[Inhalt]} oder \multicolumn{7[Anzahl der Reihen]}{|c|[Format]}{Test2[Inhalt]}
\renewcommand{\arraystretch}{1.3} % Tabellenlinien nicht zu dicht
\usepackage{colortbl}
\arrayrulecolor{gray}	% heller Tabellenlinien
\usepackage{array}	% für folgende 3 Zeilen (für Spalten fester breite mit entsprechender Ausrichtung):
\newcolumntype{L}[1]{>{\raggedright\let\newline\\\arraybackslash\hspace{0pt}}m{\dimexpr#1\columnwidth-2\tabcolsep-1.5\arrayrulewidth}}
\newcolumntype{C}[1]{>{\centering\let\newline\\\arraybackslash\hspace{0pt}}m{\dimexpr#1\columnwidth-2\tabcolsep-1.5\arrayrulewidth}}
\newcolumntype{R}[1]{>{\raggedleft\let\newline\\\arraybackslash\hspace{0pt}}m{\dimexpr#1\columnwidth-2\tabcolsep-1.5\arrayrulewidth}}
\usepackage{caption}	% Um auch unbeschriftete Captions mit \caption* zu machen

% Nützliches
\usepackage{verbatim}	% u.a. zum auskommentieren via \begin{comment} \end{comment}
\usepackage{tabto}	% Tabs: /tab zum nächsten Tab oder /tabto{.5 \CurrentLineWidth} zur Stelle in der Linie
\NumTabs{6}	% Anzahl von Tabs pro Zeile zum springen
\usepackage{listings} % Source-Code mit Tabs
\usepackage{lstautogobble} 
\ifthenelse{\equal{\customDocumentClass}{beamer}}{}{
\usepackage{enumitem}	% Anpassung der enumerates
%\setlist[enumerate,1]{label=(\arabic*)}	% global andere Enum-Items
\renewcommand{\labelitemiii}{$\scriptscriptstyle ^\blacklozenge$} % global andere 3. Item-Aufzählungszeichen
}
\usepackage{letltxmacro} % neue Definiton von Grundbefehlen
% Nutzen:
%\LetLtxMacro{\oldemph}{\emph}
%\renewcommand{\emph}[1]{\oldemph{#1}}
\RequirePackage{xpatch}	% ua. Konkatenieren von Strings/Variablen (etoolbox)
\usepackage{xstring}	% String Operationen
\usepackage{minibox}	% Minibox anstatt \fbox{} für Boxen mit Zeilenumbruch


% Einrichtung von lst
\lstset{
basicstyle=\ttfamily, 
%mathescape=true, 
%escapeinside=^^, 
autogobble, 
tabsize=2,
basicstyle=\footnotesize\sffamily\color{black},
frame=single,
rulecolor=\color{lightgray},
numbers=left,
numbersep=5pt,
numberstyle=\tiny\color{gray},
commentstyle=\color{gray},
keywordstyle=\color{green},
stringstyle=\color{orange},
morecomment=[l][\color{magenta}]{\#}
showspaces=false,
showstringspaces=false,
breaklines=true,
literate=%
    {Ö}{{\"O}}1
    {Ä}{{\"A}}1
    {Ü}{{\"U}}1
    {ß}{{\ss}}1
    {ü}{{\"u}}1
    {ä}{{\"a}}1
    {ö}{{\"o}}1
    {~}{{\textasciitilde}}1
}
\usepackage{scrhack} % Fehler umgehen
\def\ContinueLineNumber{\lstset{firstnumber=last}} % vor lstlisting. Zum wechsel zum nicht-kontinuierlichen muss wieder \StartLineAt1 eingegeben werden
\def\StartLineAt#1{\lstset{firstnumber=#1}} % vor lstlisting \StartLineAt30 eingeben, um bei Zeile 30 zu starten
\let\numberLineAt\StartLineAt

% BibTeX
\usepackage[bibencoding=ascii,
%backend=bibtex8,
%style=authortitle, citestyle=authortitle-ibid,
%doi=false,
%isbn=false,
%url=false
]{biblatex}	% BibTeX
\usepackage{makeidx}
%\makeglossary
%\makeindex

% Grafiken
\usepackage{graphicx}
\usepackage{epstopdf}	% eps-Vektorgrafiken einfügen
\usepackage{transparent}	% transparent nutzen: {\transparent{0.4} ...}
%\epstopdfsetup{outdir=\customDir}
% Prüft, ob Grafik existiert (mit \ifvalidimage{}{}) [Quelle: https://tex.stackexchange.com/a/99176]:
\makeatletter
\newif\ifgraphicexist
\catcode`\*=11
\newcommand\ifvalidimage[1]{%
    \begingroup
    \global\graphicexisttrue
    \let\input@path\Ginput@path
    \filename@parse{#1}%
    \ifx\filename@ext\relax
    \@for\Gin@temp:=\Gin@extensions\do{%
        \ifx\Gin@ext\relax
        \Gin@getbase\Gin@temp
        \fi}%
    \else
    \Gin@getbase{\Gin@sepdefault\filename@ext}%
    \ifx\Gin@ext\relax
    \global\graphicexistfalse
    \def\Gin@base{\filename@area\filename@base}%
    \edef\Gin@ext{\Gin@sepdefault\filename@ext}%
    \fi
    \fi
    \ifx\Gin@ext\relax
    \global\graphicexistfalse
    \else 
    \@ifundefined{Gin@rule@\Gin@ext}%
    {\global\graphicexistfalse}%
    {}%
    \fi  
    \ifx\Gin@ext\relax 
    \gdef\imageextension{unknown}%
    \else
    \xdef\imageextension{\Gin@ext}%
    \fi 
    \endgroup 
    \ifgraphicexist
    \expandafter \@firstoftwo
    \else
    \expandafter \@secondoftwo
    \fi 
} 
\catcode`\*=12
\makeatother
\usepackage{letltxmacro}	% Latex-Befehle unter anderem Namen neu definieren
\LetLtxMacro{\forceincludegraphics}{\includegraphics}	% neuer Befehl für includegraphics
\renewcommand{\includegraphics}[2][]{	% altes includegraphics neu definieren, damit es auch nicht vorhandene einfügt
\ifvalidimage{#2}{
\forceincludegraphics[#1]{#2}
}{
\message{Achtung: Grafik wurde nicht gefunden: '#2'}
\minibox[frame]{
\textbf{\StrSubstitute{#2}{_}{\_}}  \ifthenelse{\isempty{#1}}{}{\\\textit{#1}}}
}}

% pdf-Setup
\usepackage{pdfpages}
\ifthenelse{\equal{\customDocumentClass}{beamer}}{}{
\usepackage[bookmarks,%
bookmarksopen=false,% Klappt die Bookmarks in Acrobat aus
colorlinks=true,%
linkcolor=black,%
citecolor=red,%
urlcolor=green,%
]{hyperref}
}

%-- Unterscheidung des Stils
\newcommand{\customLogo}{}
\newcommand{\customPreamble}{}
\ifthenelse{\equal{\customDesign}{htw}}{
	% HTW Corporate Design: Arial (Helvetica)
	\usepackage{helvet}
	\renewcommand{\familydefault}{\sfdefault}
	\renewcommand{\customLogo}{HTW-Logo}
	\renewcommand{\customPreamble}{HTW Dresden}
}{
% \renewcommand{\customLogo}{HTW-Logo.eps}
}

% Nach Dokumentenbeginn ausführen:
\AtBeginDocument{
	% Autor und Titel für pdf-Eigenschaften festlegen, falls noch nicht geschehen
	\providecommand{\pdfAuthor}{John Doe}
	\ifdefempty{\customAuthor} {} {\renewcommand{\pdfAuthor}{\customAuthor}}
	\providecommand{\pdfTitle}{}
	\providecommand{\pdfTitleA}{}
	\providecommand{\pdfTitleB}{}
	\providecommand{\pdfTitleC}{}	
	\ifdefempty{\pdfTitle}{
		\ifdefempty{\customPreamble} {} {\renewcommand{\pdfTitleA}{\customPreamble{} | }}
		\ifdefempty{\customTitle} {\renewcommand{\pdfTitleB}{No Title}} {\renewcommand{\pdfTitleB}{\customTitle}}
		\ifdefempty{\customSubtitle} {} {\renewcommand{\pdfTitleC}{ - \customSubtitle}}
	}{}
	
	\newcommand{\customLogoLocation}{\customDir .LaTeX_master/\customLogo}
	\hypersetup{
		pdfauthor={\pdfAuthor},
		pdftitle={\pdfTitleA\pdfTitleB\pdfTitleC},
	}
	\ifthenelse{\equal{\customDocumentClass}{beamer}}{
		\title{\customTitle}
		\author{\customAuthor}
	}{
		\automark[section]{section}
		\automark*[subsection]{subsection}
		\pagestyle{scrheadings}
		\ifthenelse{\equal{\customDocumentClass}{report} \OR \equal{\customDocumentClass}{scrreprt}}{
		\renewcommand*{\chapterpagestyle}{scrheadings}
		}{}
		%\renewcommand*{\titlepagestyle}{scrheadings}
		\ihead{\includegraphics[height=1.7em]{\customLogoLocation}}
		%\ohead{\truncate{4cm}{\customTitle}}
		\chead{\truncate{.5\textwidth}{\headmark}}
		\ohead{\customTitle}
		\cfoot{\pagemark}
		\ofoot{\customSignature}
		% Titelseite
		\title{
		\includegraphics[width=0.35\textwidth]{\customDir .LaTeX_master/\customLogo}\\\vspace{0.5em}
		\Huge\textbf{\customTitle}
		\ifdefempty{\customSubtitle} {} {\\\vspace*{0.7em}\Large \customSubtitle}
		\\\vspace*{5em}}
		\author{
		\ifdefempty{\customNoteA} {} {\customNoteA \vspace*{1em}}\\ 
		\ifdefempty{\customAuthor} {} {\customTitleAuthor}
		\ifdefempty{\customNoteB}{}{\vspace*{1em}\\\customNoteB}
		}
		
		\ifthenelse{\equal{\customDocumentClass}{cheatsheet}}{
			\pagestyle{empty}
			\setlist{nolistsep}
	%		\usepackage{parskip}	% Aufzählung Abstand
	%		\setlength{\parskip}{0em}
			\lstset{
	    belowcaptionskip=0pt,
	    belowskip=0pt,
	    aboveskip=0pt,
			tabsize=2,
			frame=none,
			numbers=none,
			showspaces=false,
			showstringspaces=false,
			breaklines=true,
			}
		}{}
	}
}

% Unterabschnitte
%\newtheorem{example}{Beispiel}%[section]
%\newtheorem{definition}{Definition}[section]
%\newtheorem{discussion}{Diskussion}[section]
%\newtheorem{remark}{Bemerkung}[section]
%\newtheorem{proof}{Beweis}[section]
%\newtheorem{notation}{Schreibweise}[section]
% LaTeX master Datei(en) zusammengestellt von Falk-Jonatan Strube zur Nutzung an der Hochschule für Technik und Wirtschaft Dresden: https://github.com/genericFJS/htw
\RequirePackage{xcolor}
\RequirePackage{amsmath}
\RequirePackage{letltxmacro}

% Horizontale Linie:
\newcommand{\HRule}[1][\medskipamount]{\par
  \vspace*{\dimexpr-\parskip-\baselineskip+#1}
  \noindent\rule[0.2ex]{\linewidth}{0.2mm}\par
  \vspace*{\dimexpr-\parskip-.5\baselineskip+#1}}
% Gestrichelte horizontale Linie:
\RequirePackage{dashrule}
\newcommand{\HDRule}[1][\medskipamount]{\par
  \vspace*{\dimexpr-\parskip-\baselineskip+#1}
  \noindent\hdashrule[0.2ex]{\linewidth}{0.2mm}{1mm} \par
  \vspace*{\dimexpr-\parskip-.5\baselineskip+#1}}
% Mathe in Anführungszeichen:
\newsavebox{\mathbox}\newsavebox{\mathquote}
\makeatletter
\newcommand{\mq}[1]{% \mathquotes{<stuff>}
  \savebox{\mathquote}{\text{"}}% Save quotes
  \savebox{\mathbox}{$\displaystyle #1$}% Save <stuff>
  \raisebox{\dimexpr\ht\mathbox-\ht\mathquote\relax}{"}#1\raisebox{\dimexpr\ht\mathbox-\ht\mathquote\relax}{''}
}
\makeatother

% Paragraph mit Zähler (Section-Weise)
\newcounter{cparagraphC}
\newcommand{\cparagraph}[1]{
\stepcounter{cparagraphC}
\paragraph{\thesection{}-\thecparagraphC{} #1}
%\addcontentsline{toc}{subsubsection}{\thesection{}-\thecparagraphC{} #1}
\label{\thesection-\thecparagraphC}
}
\makeatletter
\@addtoreset{cparagraphC}{section}
\makeatother


% (Vorlesungs-)Folien einbinden:
% Folien von einer Datei skaliert
\newcommand{\slide}[2][\customSlideScale]{\slides[#1]{}{#2}}
\newcommand{\slideTrim}[6][\customSlideScale]{\slides[#1 , clip,  trim = #5cm #4cm #6cm #3cm]{}{#2}}
% Folien von mehreren nummerierten Dateien skaliert
\newcommand{\slides}[3][\customSlideScale]{\begin{center}
\includegraphics[page=#3, scale=#1]{\customSlidePath #2.pdf}
\end{center}}

% \emph{} anders definieren
\makeatletter
\DeclareRobustCommand{\em}{%
  \@nomath\em \if b\expandafter\@car\f@series\@nil
  \normalfont \else \scshape \fi}
\makeatother

% unwichtiges
\newcommand{\unimptnt}[1]{{\transparent{0.5}#1}}

% alph. enumerate
\newenvironment{anumerate}{\begin{enumerate}[label=(\alph*)]}{\end{enumerate}} % Alphabetische Aufzählung

% Hanging parameters
\newcommand{\hangpara}[1]{\par\noindent\hangindent+2em\hangafter=1 #1\par\noindent}

%% EINFACHE BEFEHLE

% Abkürzungen Mathe
\newcommand{\EE}{\mathbb{E}}
\newcommand{\QQ}{\mathbb{Q}}
\newcommand{\RR}{\mathbb{R}}
\newcommand{\CC}{\mathbb{C}}
\newcommand{\NN}{\mathbb{N}}
\newcommand{\ZZ}{\mathbb{Z}}
\newcommand{\PP}{\mathbb{P}}
\renewcommand{\SS}{\mathbb{S}}
\newcommand{\cA}{\mathcal{A}}
\newcommand{\cB}{\mathcal{B}}
\newcommand{\cC}{\mathcal{C}}
\newcommand{\cD}{\mathcal{D}}
\newcommand{\cE}{\mathcal{E}}
\newcommand{\cF}{\mathcal{F}}
\newcommand{\cG}{\mathcal{G}}
\newcommand{\cH}{\mathcal{H}}
\newcommand{\cI}{\mathcal{I}}
\newcommand{\cJ}{\mathcal{J}}
\newcommand{\cM}{\mathcal{M}}
\newcommand{\cN}{\mathcal{N}}
\newcommand{\cP}{\mathcal{P}}
\newcommand{\cR}{\mathcal{R}}
\newcommand{\cS}{\mathcal{S}}
\newcommand{\cZ}{\mathcal{Z}}
\newcommand{\cL}{\mathcal{L}}
\newcommand{\cT}{\mathcal{T}}
\newcommand{\cU}{\mathcal{U}}
\newcommand{\cX}{\mathcal{X}}
\newcommand{\cV}{\mathcal{V}}
\renewcommand{\phi}{\varphi}
\renewcommand{\epsilon}{\varepsilon}
\renewcommand{\theta}{\vartheta}

% Verschiedene als Mathe-Operatoren
\DeclareMathOperator{\arccot}{arccot}
\DeclareMathOperator{\arccosh}{arccosh}
\DeclareMathOperator{\arcsinh}{arcsinh}
\DeclareMathOperator{\arctanh}{arctanh}
\DeclareMathOperator{\arccoth}{arccoth} 
\DeclareMathOperator{\var}{Var} % Varianz 
\DeclareMathOperator{\cov}{Cov} % Co-Varianz 

% Farbdefinitionen
\definecolor{red}{RGB}{180,0,0}
\definecolor{green}{RGB}{75,160,0}
\definecolor{blue}{RGB}{0,75,200}
\definecolor{orange}{RGB}{255,128,0}
\definecolor{yellow}{RGB}{255,245,0}
\definecolor{purple}{RGB}{75,0,160}
\definecolor{cyan}{RGB}{0,160,160}
\definecolor{brown}{RGB}{120,60,10}

\definecolor{itteny}{RGB}{244,229,0}
\definecolor{ittenyo}{RGB}{253,198,11}
\definecolor{itteno}{RGB}{241,142,28}
\definecolor{ittenor}{RGB}{234,98,31}
\definecolor{ittenr}{RGB}{227,35,34}
\definecolor{ittenrp}{RGB}{196,3,125}
\definecolor{ittenp}{RGB}{109,57,139}
\definecolor{ittenpb}{RGB}{68,78,153}
\definecolor{ittenb}{RGB}{42,113,176}
\definecolor{ittenbg}{RGB}{6,150,187}
\definecolor{itteng}{RGB}{0,142,91}
\definecolor{ittengy}{RGB}{140,187,38}

\definecolor{htworange}{RGB}{249,155,28}

% Textfarbe ändern
\newcommand{\tred}[1]{\textcolor{red}{#1}}
\newcommand{\tgreen}[1]{\textcolor{green}{#1}}
\newcommand{\tblue}[1]{\textcolor{blue}{#1}}
\newcommand{\torange}[1]{\textcolor{orange}{#1}}
\newcommand{\tyellow}[1]{\textcolor{yellow}{#1}}
\newcommand{\tpurple}[1]{\textcolor{purple}{#1}}
\newcommand{\tcyan}[1]{\textcolor{cyan}{#1}}
\newcommand{\tbrown}[1]{\textcolor{brown}{#1}}

% Umstellen der Tabellen Definition
\newcommand{\mpb}[1][.3]{\begin{minipage}{#1\textwidth}\vspace*{3pt}}
\newcommand{\mpe}{\vspace*{3pt}\end{minipage}}

\newcommand{\resultul}[1]{\underline{\underline{#1}}}
\newcommand{\parskp}{$ $\\}	% new line after paragraph
\newcommand{\corr}{\;\widehat{=}\;}
\newcommand{\mdeg}{^{\circ}}

\newcommand{\nok}[2]{\binom{#1}{#2}}	% n über k BESSER: \binom{n}{k}
\newcommand{\mtr}[1]{\begin{pmatrix}#1\end{pmatrix}}	% Matrix
\newcommand{\dtr}[1]{\begin{vmatrix}#1\end{vmatrix}}	% Determinante (Betragsmatrix)
\LetLtxMacro{\originalVec}{\vec}
\renewcommand{\vec}[1]{\underline{#1}}	% Vektorschreibweise
\newcommand{\imptnt}[1]{\colorbox{red!30}{#1}}	% Wichtiges
\newcommand{\intd}[1]{\,\mathrm{d}#1}
\newcommand{\diffd}[1]{\mathrm{d}#1}
% für Module-Rechnung: \pmod{}
\newcommand{\unit}[1]{\,\mathrm{#1}}
\LetLtxMacro{\ntilde}{\tilde}
\renewcommand{\tilde}{\widetilde}
\newcommand{\gdw}{genau dann wenn}
\newcommand{\lecdate}[1]{\begin{flushright}\textcolor{gray}{Vorlesung am #1}\end{flushright}}

\setlist[enumerate,1]{label=(\arabic*)}
\renewenvironment{anumerate}{\begin{enumerate}[label=(\alph*)]}{\end{enumerate}} % Alphabetische Aufzählung


%\bibliography{\customDir .Literatur/HTW_Literatur.bib}
\begin{document}

%\selectlanguage{english}
\maketitle
\newpage
\tableofcontents
\newpage

\chapter*{Vorbemerkung}
Lernraum: Dienstag 17:00 S327, S329

\chapter*{Stochastik}

\section*{Was ist Stochastik}
Stochastik…
\begin{itemize}
\item … kommt etymologisch aus dem Griechischem; Bedeutung: „Kunst des Vermutens“
\item … beschäftigt sich mit der Beschreibung und dem Untersuchen von zufälligen Ereignissen (z.B. Lotto, Wurf eines Würfels, Lebensdauer einer Glühbirne, …)
\item … beinhaltet die Teilgebiete
\begin{itemize}
\item Wahrscheinlichkeitsrechnung:\\
Zu Grunde liegende Gesetzmäßigkeit des Zufalls bekannt. Frage nach Wahrscheinlichkeiten „interessanter“ Ereignisse

Bsp. Würfel: Jede Seite fällt mit Wahrscheinlichkeit $\frac{1}{6}$. \\
Wie groß ist die Wahrscheinlichkeit, dass unter $10$ Würfen mindestens $4$ mal 6 kommt? 
\item Statistik:\\
Zur Grunde liegende Gesetzmäßigkeit des Zufalls ist unbekannt. Idee: Nutze Stichproben/Daten um diese Gesetzmäßigkeiten zu erkennen.

Bsp.: Gesamtproduktion $100\,000$ Teile, Stichprobe von $100$ Teilen enthält $2$ defekte. \\
Kann davon ausgegangen werden, dass die Fehlerquote von $1\%$ nicht eingehalten wird?
\end{itemize}
\end{itemize}

\chapter{Wahrscheinlichkeitsrechnung}
\section{Zufallsexperimente, Ereignisse und Wahrscheinlichkeiten}
\subsection{Zufallsexperimente und Ereignisse}
Erster wichtiger Begriff:
\cparagraph{Definition}
Ein \emph{Zufallsexperiment} ist ein Vorgang
\begin{itemize}
\item der beliebig oft unter gleichartigen Bedingungen wiederholt werden kann und
\item dessen Ergebnis nicht mit Sicherheit vorhergesagt werden kann
\end{itemize}
$\Omega:=$ Ergebnismenge (oder Ergebnisraum) ist die Menge aller möglichen Ergebnisse

\cparagraph{Bemerkung} Drei wichtige Fälle
\begin{itemize}
\item $\Omega$ endlich, d.h. $\Omega=\{\omega_1,\omega_2,\dots,\omega_n\}$
\item $\Omega$ abzählbar unendlich, d.h. $\Omega=\{\omega_1,\omega_2,\dots\}$ (Ereignisse lassen sich mit den natürlichen Zahlen aufzählen)\footnote{zu natürlichen Zahlen (in dieser VL): $\NN=\{1,2,3,\dots\}, \; \NN_0=\{0,1,2,3,\dots\}$}
\item $\Omega$ überabzählbar unendlich, d.h. $\Omega=\RR$ oder $\Omega [0,1)$
\end{itemize}

\cparagraph{Beispiel} 
\begin{itemize}
\item Würfel: $\Omega=\{1,2,3,4,5,6\}$
\item Anzahl der defekten Glühbirnen in einer Stichprobe von 100 Stück: $\Omega=\{0,1,2,\dots,100\}$
\item Anzahl der Anrufe im Call-Center zwischen 8:00 und 9:00
\begin{enumerate}
\item Möglichkeit 1: $\Omega = \{0,1,2,\dots\}=\NN_0$
\item Möglichkeit 2: $\Omega = \{\omega_1,\omega_2,\dots, \omega_{100}\}$ mit $w_i =\begin{cases}
i \text{ Anrufe, falls }i\leq 99\\
100 \text{ oder mehr Anrufe, falls }i=100
\end{cases}$
\end{enumerate}
\item Downloadzeit einer Datei: $\Omega = (0,\infty)$
\end{itemize}
Wir interessieren uns oft nicht allein für das Eintreten von einem $w\in \Omega$, sondern dafür ob ein $w$ aus einer gewissen Teilmenge aus $\Omega$ eingetreten ist (z.B. sind weniger als $3$ Glühbirnen defekt). Daher:
\cparagraph{Definition} Ein \emph{zufälliges Ereignis} $A$ ist eine Teilmenge des Ergebnisraums $\Omega$. 

Spezielle Ereignisse:
\begin{itemize}
\item $A = \emptyset$ \tab … das unmögliche Ereignis ($\omega\in \emptyset$ tritt nie ein)
\item $A=\Omega$ \tab … das sichere Ereignis ($\omega \in \Omega$ tritt immer ein)
\item $A=\{\omega\}$ \tab … Elementarereignis (für ein $\omega \in \Omega$)
\item $\bar A = \Omega \setminus A$ \tab … Gegenereignis zu $A$
\end{itemize}
Sprechweise: „Das Ereignis $A$ tritt ein“, falls ein $\omega \in A$ beobachtet wird.

\cparagraph{Beispiel} (Würfel)\\
$A=\{\text{„gerade Zahl fällt“}\}$\\
$\Rightarrow A=\{2,4,6\} \subseteq \Omega = \{1,\dots,6\}$\\
Gegenereignis: $\bar A = \{1,3,5\}$

\cparagraph{Bemerkung} Da Ereignisse Teilmengen von $\Omega$ sind, lassen sich alle Rechenoperationen für Mengen anwenden. Seien $A,B \subseteq \Omega$.
\begin{itemize}
\item $A \subseteq B$ … $A$ ist Teilereignis von $B$
\item $A=B$, gleiche Ereignisse
\item Durchschnitt: $A \cap B$, „$A$ und $B$“ (beide Ereignisse treten gleichzeitig ein)
\item Vereinigung: $A \cup B$, „$A$ oder $B$“ (entweder $A$ oder $B$ treten ein)
\item Differenz: $A \setminus B$, „$A$ ohne $B$“ ($A$ tritt ein, $B$ aber nicht)
\item Negation/Gegenereignis: $\bar A = \Omega\setminus A$ ($A$ tritt nicht ein)
\item gilt $A \cap B = \emptyset$, so heißen $A$ und $B$ \emph{unvereinbar/disjunkt}.
\end{itemize}

\cparagraph{Beispiel} (Würfel)\\
$\Omega = \{1,\dots,6\}$,\\
$A=\{2,4,6\},\; B=\{2,3,5\},\; C =\{1,3\}$\\
Bestimme: $A \cup B$, $A \cap B$, $A \cap C$, $C \cup \bar C$\\
$A \cup B = \{2,3,4,5,6\}$\\
$A \cap B = \{2\}$\\
$A \cap C = \emptyset$\\
$C \cup \bar C = \Omega$

\cparagraph{Satz} (Rechenregeln) Es seien $A$, $B$ und $C$ Ereignisse. Dann gilt:
\begin{itemize}
\item $A \cap B = B \cap A \quad A \cup B = B \cup A$ \tab(Kommutativgesetz)
\item $A \cap (B \cap C)=(A \cap B) \cap C$\\
$A \cup (B \cup C) = (A \cup B ) \cup C$ \tab\tab (Assoziativgesetze)
\item $A \cap (B \cup C) = (A \cap B) \cup (A \cap C)$\\
$A \cup(B \cap C) = (A \cup B) \cap (A \cup C)$ \tab (Distributivgesetze)
\item $\overline{A\cap B} = \bar A \cup \bar B$\\
$\overline{A \cup B} = \bar A \cap \bar B$ \tab\tab (De Morgansche Regeln)
\item aus $A \subseteq B$ folgt $\bar B \subseteq \bar A$ und $A\setminus B=A \cap \bar B$
\end{itemize}

\cparagraph{Definition} Sei $\Omega$ eine Menge. Ein Mengensystem $\cA\subseteq \cP(\Omega)$ heißt $\sigma$-Algebra, falls gilt
\begin{itemize}
\item $\Omega \in \cA$
\item $A \in \cA \Rightarrow \bar A \in \cA$
\item $A_1, A_2, A_3, \dots \in \cA \Rightarrow \bigcap_{i=1}^\infty A_i \in \cA$
\end{itemize}
(Sprich: die Menge, alle Komplemente und die Schnitte und Vereinigungen aller Teilmengen müssen in $\cA$ liegen [Mächtigkeit der $\sigma$-Algebra ist bei einer endlichen Grundmenge immer eine 2er-Potenz!])

\cparagraph{Bemerkung} Sei $\cA$ eine $\sigma$-Algebra auf $\Omega$. Dann gilt:
\begin{itemize}
\item $\emptyset \in \cA$
\item $A, B \in \cA \Rightarrow A \setminus B \in \cA$
\item $A_1, A_2,A_3,\dots \in \cA \Rightarrow \bigcup_{i=1}^\infty A_i \in \cA$
\end{itemize}

\cparagraph{Beispiel} (Würfel)
\begin{itemize}
\item $\cA=\{\{1\},\{2\},\emptyset, \{1,2\}, \{3,4,5,6\}, \{2,3,4,5,6\}, \{1,3,4,5,6\}, \underset{=\Omega}{\{1,2,3,4,5,6\} }\}$ ist eine $\sigma$-Algebra über $\Omega=\{1,\dots,6\}$
\item $\cA=\{A \;|\; A \subseteq \Omega\} = \cP(\Omega)$ ist auch ein $\sigma$-Algebra
\end{itemize}

\cparagraph{Bemerkung} Besteht $\Omega$ aus $n$ Elementen, so enthält $\cP(\Omega)$ genau $2^n$ Elemente.

\subsection{Definition der Wahrscheinlichkeit}

Ziel: Ordne zufälligem Ereignis $A$ eine Wahrscheinlichkeit $\PP(A)$ zu, die die Chance beurteilt, dass $A$ eintritt.

\cparagraph{Definition} (Kolmogorov 1933)\\
Gegeben sei eine Ereignismenge $\Omega$ und eine $\sigma$-Algebra $\cA$. Eine Funktion $\PP : \cA \to [0,1]$ heißt \emph{Wahrscheinlichkeitsmaß auf $(\Omega,\cA)$}, falls
\begin{enumerate}
\item $\PP (\Omega) =1$
\item für paarweise disjunkte $A_i \in \cA, \; i=1,2,\dots$ (d.h. $A_i \cap A_J = \emptyset$ falls $i \not = j$) gilt $\PP(A_1 \cup A_2 \cup \dots ) = \PP(A_1)+\PP(A_2)+\dots$.
\end{enumerate}
Weitere Bezeichnungen:
\begin{itemize}
\item $\PP(A)$ … Wahrscheinlichkeit des Ereignisses $A$
\item $(\Omega,\cA, \PP)$ … Wahrscheinlichkeitsraum / Wahrscheinlichkeitsmodell
\end{itemize}

\cparagraph{Bemerkung} (Allgemeines Vorgehen, Vereinfacht Darstellung)
\begin{enumerate}
\item Theoretische Untersuchungen (Kombinatorik, physikalische Gesetze); Beobachtung der relativen Häufigkeit (deskriptive Statistik)
\item Schritt (1) liefert für gewisse Grundereignisse die Wahrscheinlichkeiten (exakt oder zumindest näherungsweise)
\item Bestimmen der Wahrscheinlichkeiten für alle interessierenden Ereignisse (mittels Rechenregeln, siehe später)
\end{enumerate}

\cparagraph{Satz} Seien $A,B,C$ sowie $A_1, A_2, \dots ,A_n$ zufällige Ereignisse. Dann gilt:
\begin{anumerate}
\item $\PP(\emptyset)=0$
\item $\PP(\bar A) = 1 - \PP (A)$
\item $A \subseteq B \Rightarrow \PP(A) \subseteq \PP (B)$
\item $\PP(A \cup B)=\PP(A)+\PP(B)-\PP(A \cap B)$\\
$\PP ( A \cup B \cup C) = \PP(A) + \PP(B) + \PP(C)-\PP(A \cap B) - \PP(A\cap C) - \PP(B\cap C ) + \PP(A \cap B \cap C)$\\
(Gut zu veranschaulichen durch Venn-Diagramme)
\end{anumerate}
Es gilt sogar der Additionssatz:
\cparagraph{Satz} Seien $A_1, \dots, A_n$ zufällige Ereignisse. Dann gilt: \\
$\PP(A_1 \cup \dots \cup A_n)=\sum_{i=1}^n \PP(A_i)-\sum_{i<j} \PP(A_i \cap A_j) + \sum_{i<j<k}\PP(A_i \cap A_j \cap A_k) - \dots + (-1)^{n+1} \PP(A_1\cap A_1 \cap \dots \cap A_n)$

\subsection{Laplacesches Modell}
Nun wollen wir ein spezielles, einfaches, aber oft sehr nützliches WK-Maß einführen.
\cparagraph{Definition} Ein WK-Modell $(\Omega, \cA, \PP)$ heißt \emph{Laplacesches Modell}, falls $\Omega = \{\omega_1, \omega_2, \dots , \omega_n\}$ endlich ist, $\cA = \cP ( \Omega)$ und $\PP(\{\omega_1\})=P(\{\omega_2\})=\dots = \PP(\{\omega_n\})=\frac{1}{n}$ gilt.

\cparagraph{Bemerkung} für beliebiges $A \in \cA$ gilt im Laplaceschen Modell:\\
$\PP(A) = \frac{|A|}{|\Omega|}=\frac{m}{n}$, wobei $m=|A|$ die Anzahl der Elemente in $A$ ist (und $|\Omega|=n$).\\
Also $\PP(A)=\frac{\text{Anzahl der günstigen Elementarereignisse}}{\text{Anzahl der möglichen Elementarereignisse}}$.\\
Man sagt auch: $\PP$ ist dann die diskrete Gleichverteilung auf $\Omega$.

\cparagraph{Beispiel}
\begin{anumerate}
\item (fairer Würfel) Wie groß ist die WK eine Zahl größer $4$ zu würfeln?\\
$\Omega = \{1,2,3,4,5,6\}$, $A=\{5,6\}$ und es gilt $\PP(\{1\})=\PP(\{2\})=\dots=\PP(\{6\})=\frac{1}{6}$\\
Daher: $\PP(A) = \frac{|A|}{|\Omega|}=\frac{2}{6}=\frac{1}{3}$
\item (2 faire Würfel) Wie groß ist die WK mit 2 Würfeln mindestens eine 11 zu würfeln?
\begin{align*}
\Omega=\{ & (1,1), (1,2), \dots , (1,6)\\
&\vdots\\
&(6,1), (6,2), \dots , (6,6)\}
\end{align*}
$A=\{(6,5), (5,6), (6,6)\}$ und es gilt $\PP(\{(i,j)\})=\frac{1}{36}$ für beliebiges $i,j \in \{1,\dots,6\}$. Also liegt Laplace Modell vor.\\
Daher gilt: $\PP(A) = \frac{3}{36}= \frac{1}{12}$.
\end{anumerate}
Um in Laplace-Modellen die Größe (Mächtigkeit) von Ereignissen zu bestimmen, sind oft spezielle „Abzähltricks“ sinnvoll. Diese liefert die Kombinatorik.

\subsection{Kombinatorik}
Fragestellung: Wie viele Möglichkeiten gibt es aus einer $n$-elementigen Menge $k$ Elemente auszuwählen? Dabei sind die Spielregeln zu klären:
\begin{itemize}
\item Spielt die Reihenfolge eine Rolle?
\item Dürfen Elemente mehrfach ausgewählt werden (mit Zurücklegen oder ohne)?
\end{itemize}

\cparagraph{Satz} In einer Urne befinden sich $n$ (voneinander unterscheidbare) Elemente. Wir ziehen $k$ davon…
\begin{anumerate}
\item … mit Zurücklegen, unter Berücksichtigung der Reihenfolge, dann gibt es
\[\bar v_n^k = n^k\]
viele Möglichkeiten (Variation von $n$ Elementen zur $k$-ten Klasse mit Wiederholungen).
\item … ohne Zurücklegen, unter Berücksichtigung der Reihenfolge, dann gibt es
\[v_n^k=n\cdot (n-1) \cdot (n-2) \cdot \dots \cdot (n-(k-1))=\frac{n!}{(n-k)!}\]
viele Möglichkeiten (Variation von $n$ Elementen zur $k$-ten Klasse ohne Wiederholungen).
\item … mit Zurücklegen, ohne Berücksichtigung der Reihenfolge, dann gibt es
\[\bar c_n^k=\binom{n+k-1}{k}=\frac{(n+k-1)!}{k!(n-1)!}\]
viele Möglichkeiten (Kombination von $n$ Elementen zur $k$-ten Klasse mit Wiederholungen).
\item … ohne Zurücklegen, ohne Berücksichtigung der Reihenfolge, dann gibt es
\[c_n^k=\binom{n}{k}=\frac{n!}{k!(n-k)!}\]
viele Möglichkeiten (Kombination von $n$ Elementen zur $k$-ten Klasse ohne Wiederholungen).
\end{anumerate}

\cparagraph{Bemerkungen}
\begin{itemize}
\item $n!=n\cdot (n-1) \cdot \dots \cdot 2\cdot 1$ mit $0!=1$
\item Spezialfall in (b): $n=k$, dann $v_n^k=n!$. Dies beschreibt die Anzahl der möglichen Anordnungen von $n$ Elementen (Permutationen).
\item Spezialfälle in (d): 
\begin{itemize}
\item $n=k$, dann $c_n^k=1=\binom{n}{n}$
\item $k=0$, dann $c_n^0=\binom{n}{0}=1$
\item $k=1$, dann $c_n^1=\binom{n}{1}=n$
\end{itemize}
\end{itemize}

\cparagraph{Beispiel}
\begin{anumerate}
\item Wie viele mögliche Zieleinläufe gibt es beim 100m-Lauf mit 8 Teilnehmern?\\
$8!=40320$
\item Wie viele Möglichkeiten gibt es beim Lotto (6 aus 49)\\
$\binom{49}{6}=13\;983\;816$
\item Wie viele Möglichkeiten gibt es ein Nummernschild der Art „DD-Buchstabe Buchstabe Ziffer Ziffer Ziffer“ zu konstruieren?\\
$26^2\cdot 10^3=676 \; 000$
\item Wie viele Möglichkeiten gibt es 5 (nicht unterscheidbare) Äpfel auf 3 Kinder aufzuteilen?\\
$\binom{3+5-1}{5}=\binom{7}{1}=21$
\end{anumerate}

\subsection{Bedingte Wahrscheinlichkeit}
Frage: Wie verändert sich die Wahrscheinlichkeit eines Ereignisses, falls ich Zusatzwissen mit einfließen lasse? 

\cparagraph{Beispiel} HIV Prävalenz liegt weltweit bei $0,8\%$, also:\\
$\PP_1 (\{\text{zufällig ausgewählte Person ist HIV-positiv\}})=0,008$\\
Modell 1: $\Omega =\{0,1\},\; \PP_1 ( \{1\}) = 0,008, \; \PP_1(\{0\})=0,992$\\
Zusatzwissen: ausgewählte Person ist Europäer und Prävalenz in Europa: $=0,2\%$, also:\\
$\PP_2 (\{\text{zufällig ausgewählte Person ist HIV-positiv\}})=0,002$\\
Modell 2: $\Omega = \{0,1\}, \; \PP_2(\{1\})=0,002=1-\PP(\{0\})$\\
Problem/Frage:
\begin{itemize}
\item Wie kombiniert man beide Modelle?
\item Wir wollen nicht mit 2 verschiedenen $\PP$s rechnen.
\item WK für HIV positiv unter Nicht-Europäern?
\end{itemize}

\cparagraph{Beispiel}
\begin{itemize}
\item Von insgesamt $800$ Schülern besitzen $440$ ein Smartphone.
\item Unter den Smartphone-Besitzern sind $60\%$ männlich.
\item Unter den Nicht-Smartphone-Besitzern sind $35\%$ männlich.
\item Unter allen $800$ Schülern wird ein Smartphone verlost.
\end{itemize}
Fragen:
\begin{anumerate}
\item Wie groß ist die Wahrscheinlichkeit, dass der Gewinner bereits ein Smartphone besitzt?
\item Wie groß ist die WK, dass der Gewinner bereits ein Smartphone besitzt, wenn man schon weiß, dass ein Mädchen gewonnen hat?
\end{anumerate}

\cparagraph{Definition} Sei $(\Omega, \cA, \PP)$ ein Wk-Raum und seien $A, B \subset \Omega$ Ereignisse mit $\PP(B)>0$. Dann definieren wir $$\PP(A|B):=\frac{\PP(A \cup B)}{\PP(B)}$$ und nennen $\PP (A|B)$ die Wahrscheinlichkeit von $A$ bedingt auf $B$.\\
Interpretation: „Wie groß ist die Wk,von $A$, wenn ich schon weiß, dass $B$ eingetreten ist?“

\cparagraph{Beispiel} (Smartphone, s.o.)\\
$\Omega = \{ (S,M), (\bar S, M), (S, W), (\bar S, W)\}$\\
$S$ … Gewinnende Person besitzt Smartphone\\
$\bar S$ … Gewinnende Person besitzt kein Smartphone\\
$M$ … Gewinnende Person ist männlich\\
$W$ … Gewinnende Person ist weiblich\\
\begin{tikzpicture}[scale=2]
\node (v1) at (5,1) {$M/W$ / $S/\bar S$};
\node (v2) at (3.5,0) {$S$};
\node (v3) at (6.5,0) {$\bar S$};
\node (v4) at (3,-1) {$M$};
\node (v5) at (4,-1) {$W$};
\node (v6) at (6,-1) {$M$};
\node (v7) at (7,-1) {$W$};
\node at (v4) [below = 1em, align=center]{$0,33$\\$(=0,55\cdot0,6)$};
\node at (v5) [below = 1em]{$0,22$};
\node at (v6) [below = 1em]{$0,1575$};
\node at (v7) [below = 1em]{$0,2925$};
\draw (v1) -- node[left, pos =.5]{$0,55\left(=\frac{440}{800}\right)$} (v2);
\draw (v1) -- node[right, pos =.5]{$0,45$}(v3);
\draw (v2) -- node[left, pos =.5]{$0,6$}(v4);
\draw (v2) -- node[right, pos =.5]{$0,4$}(v5);
\draw (v3) -- node[left, pos =.5]{$0,35$}(v6);
\draw (v3) -- node[right, pos =.5]{$0,65$}(v7);
\end{tikzpicture}% ABB S1
\\
gegeben: \\
$\PP(\{(S,M)\})=0,33$\\
$\PP(\{(S,W)\})=0,22$\\
$\PP(\{ (\bar S, M )\})=0,1575$\\
$\PP(\{ (\bar S, W )\})=0,2925$\\
Antwort auf Fragen:
\begin{anumerate}
\item $0,55$ (klar)
\item Intuition: Wir wissen, dass nur noch die Stränge mit „W“ interessieren. Die Stränge ohne „W“ sollten wir „streichen“. Wie groß ist die WK der Kombination (S,W) im Vergleich zu allen, wo W vorkommt? Also:$$\frac{\PP(\{(S,W)\})}{\PP(\{(S,W),(\bar S,W)\})}=\frac{0,22}{0,22+0,2925}=0,4293$$
Was hat das mit der bedingten WK aus Def. 1.1.25 zu tun? \\
$A:=\{\text{Person besitzt Smartphone}\}=\{(S,M),(S,W)\}$\\
$B:=\{\text{Person ist weiblich}\} = \{(S,W), (\bar S ,W)\}$ \\
$\PP(A|B)=\frac{\PP(\{(S,W)\})}{\PP(\{(S,W), (\bar S, W)\})}=\dots =0,4293$
\end{anumerate}

\cparagraph{Satz} (Rechnen mit bedingten WK)\\
Sei $(\Omega, \cA, \PP)$ ein WK-Raum und $A, A_1, A_2, B \in \cA$ Ereignisse mit $\PP(B)>0$. Dann gilt:
\begin{itemize}
\item $\PP(B|B)=1$, $\PP(\emptyset|B)=0$
\item Falls $A$ und $B$ disjunkt, gilt $\PP(A|B)=0$
\item $\PP(\bar A | B) = 1-\PP(A|B)$
\item $\PP(A_1 \cup A_2 | B ) =\PP(A_1 | B ) + \PP (A_2|B) - \PP (A_1 \cap A_2 | B)$
\item Falls $B\subseteq A$, so gilt $\PP(A|B)=1$
\item Falls $A \subseteq B$, so gilt $\PP(A|B)=\PP(A)$
\end{itemize}

\cparagraph{Beispiel} Auf einer E-Mail Adresse kommen im Schnitt $80\%$ Spam-Mails und $20\%$ gute Mails.\\
Eine „gute“ Mail enthalte mit $2\%$ WK das Wort „Viagra“. In einer Spam-Mail liegt dieser Anteil bei $60\%$. Berechnen Sie die WK, dass eine Spam-Mail vorliegt, falls man weiß, dass das Wort „Viagra“ enthalten ist.\\
Lösung:\\
$A=\{\text{Mail enthält „Viagra“}\}$\\
$\bar A=\{\text{Mail enthält kein „Viagra“}\}$\\
$B=\{\text{Mail ist Spam}\}$\\
$\bar B=\{\text{Mail ist kein Spam}\}$\\
4-Felder-Tafel:\\
\begin{tabular}{l | c | c | l}
& $B$: Spam & $\bar B$ kein Spam & \\
\hline
$A$, mit Viagra & $0,8\cdot 0,6=0,48$ & $0,2 \cdot 0,002=0,004$ & $0,484$\\
$\bar A$, ohne Viagra & $0,32$ & $0,196$ & $0,516$\\
\hline 
& $0,8$ & $0,2$ & $1$
\end{tabular}\\
Gesucht ist $\PP(B|A)=\frac{\PP(B\cup A)}{\PP(A)}=\frac{0,48}{0,484}=0,9917$
Auch interessant ist die WK, dass die Mail kein Spam ist, wenn man schon weiß, dass „Viagra“ nicht enthalten ist. $\PP(\bar B | \bar A)=\frac{0,196}{0,516}=0,3798$

\cparagraph{Satz} (Multiplikationssatz)\\
Seien $A$ und $B$ Ereignisse mit $\PP(A)>0,\; \PP(B)>0$. Dann gilt:
$$\PP(A\cup B ) = \PP(A ) \cdot \PP(B|A) = \PP(B) \cdot \PP(A|B)$$
Sind $A_1,\dots,A_n$ Ereignisse mit $\PP\left( \bigcap_{i=1}^{n-1} A_i\right) >0$, dann gilt sogar:
$$\PP(A_1\cap A_2 \cap \dots \cap A_n)=\PP(A_1)\cdot \PP(A_2|A_1) \cdot \PP (A_3|A_1 \cap A_2)\cdot \dots \cdot \PP(A_n|A_1 \cap \dots \cap A_{n-1})$$

\cparagraph{Beispiel} In einer Los-Trommel befinden sich $20$ Lose. Jemand zieht $3$ nacheinander. Es gibt insgesamt $5$ Gewinnlose. Wie groß ist die WK, dass alle $3$ gezogenen Lose Gewinnlose sind?\\
$A_k=\{\text{Gewinn beim $k$-ten Los}\}, \; k=1,2,3$\\
Gesucht: $\PP(A_1\cap A_2 \cap A_3)$\\
Satz 1.1.29 liefert: \\
$\PP(A_1 \cap A_2 \cap A_3)=\PP(A_1) \cdot \PP(A_2 | A_1) \cdot \PP(A_3 | A_1 \cap A_2)$\\
$\PP(A_1) = \frac{5}{20}= \frac{1}{4}$ (5 Günstige in 20 Losen)\\
$\PP(A_2 | A_1) = \frac{4}{19}$\\
$\PP(A_3 | A_1 \cap A_2) = \frac{3}{18}$\\
$\Rightarrow \PP(A_1 \cap A_2 \cap A_3)=\frac{1}{4}\cdot \frac{4}{19}\cdot \frac{3}{18}=\frac{1}{114}=0,0087$

\cparagraph{Satz} (Formel der totalen WK)\\
Sei $(\Omega, \cA, \PP)$ ein WK-Raum und seien $B_1,\dots,B_n \in \cA$ mit 
\begin{itemize}
\item $\bigcup_{i=1}^n B_i = \Omega$
\item $B_i \cap B_j = \emptyset $ für $i \not= j$
\item $\PP (B_i) >0$ für alle $i=1,\dots,n$
\end{itemize}
Dann gilt:
$$\PP(A) = \sum_{i=1}^n \PP(A|B_i) \cdot \PP (B_i)$$\\
ABB S2

\cparagraph{Beispiel} (Prävalenz von HIV)
\begin{itemize}
\item HIV-Prävalenz weltweit: $0,8\%$
\item HIV-Prävalenz in Europa: $0,2\%$
\item es gibt $7$ Mrd. Menschen auf der Erde
\item es gibt $740$ Mio Menschen in Europa
\end{itemize}
Gesucht:
\begin{itemize}
\item WK, dass zufällig ausgewählter Europäer HIV-positiv ist.
\item WK, dass zufällig ausgewählter Nicht-Europäer HIV-positiv ist.
\end{itemize}
Lösung:\\
$E:=\{\text{ausgewählte Person ist Europäer}\}$\\
$P:=\{\text{ausgewählte Person ist HIV positiv}\}$\\
Wir wissen: \\
$\PP(P)=0,008$, $\PP(E)=\frac{74}{700}\approx 0,1057$\\
$\PP(P|E)=0,002$.\\
Wir wollen wissen:
\begin{itemize}
\item $\PP(\bar P | E)=1-\PP(P|E)=1-002=0,998$
\item $\PP(P | \bar E)=\PP(B|\bar E) \cdot \PP(E)+\PP(P|E) \cdot \PP(E)$ (mit $B_1=\bar E$ und $B_2=E$)\\
Umstellen liefert:\\
$\PP(P|\bar E) = \frac{\PP(B)-\PP(P|E)\cdot\PP(E)}{\PP(\bar E)}=\frac{0,008-0,002\cdot 0,1057}{1-0,1057}=0,008709$
\end{itemize}

\cparagraph{Satz} (Formel von Bayes)\\
Sei $(\Omega, \cA, \PP)$ WK-Räume und seien $B_1,\dots,B_n \in \cA$ mit 
\begin{itemize}
\item $\bigcup_{i=1}^n B_i = \Omega$
\item $B_i \cap B_j = \emptyset $ für $i \not= j$
\item $\PP (B_i) >0$ für alle $i=1,\dots,n$
\end{itemize}
Dann gilt für beliebige $A \in \cA$ mit $\PP(A) >0$ und beliebiges $j\in \{1,\dots,n\}$:
$$\PP(B_j|A)=\frac{\PP(A|B_j)\cdot \PP(B_j)}{\PP(A)}=\frac{\PP(A|B_j)\cdot \PP(B_j)}{\sum_{i=1}^n \PP(A|B_j) \cdot \PP(B_j)}$$
Formel von Bayes dreht also die Bedingung um.

\cparagraph{Beispiel} (Ziegenproblem)\\
In einer Spielshow steht der Kandidat vor $3$ verschlossenen Türen. Eine Türe verbirgt den Hauptgewinn, ein Auto. Hinter den beiden anderen Türen sind Ziegen. Der Kandidat zeigt auf eine der Türen, der Spielleiter (der weiß, wo das Auto steht) öffnet gemäß der Spielregeln eine der beiden anderen Türen um eine Ziege zu präsentieren. \\
Der Kandidat darf nun seine Wahl ändern. Sollte er das tun?\\
ABB S3\\
Lösung: \\
Wir legen uns fest, dass der Kandidat Tor 1 gewählt hat und Moderator Tor 3 öffnet(ohne Beschränkung der Allgemeinheit(oBdA): sonst Umnummerieren).\\
Ergebnismenge: $\Omega = \{(i,j)\;|\; i,j=1,2,3\}$ mit $(i,j)$ … Gewinn ist hinter Tor $i$, Moderator öffnet Tor $j$.\\
Definiere die Ereignisse \\
$G_i:= \{ \text{Gewinn hinter Tor }i\}=\{(i,1), (i,2), (i,3)\}$ und\\
$M_j:=\{\text{Moderator öffnet Tor }j\}=\{(1,j), (2,q), (3,q)\}$\\
Wir wissen:\\
$\PP(G_i)=\frac{1}{3}$ für alle $i=1,2,3$\\
$\PP(M_3|G_1)=\tfrac{1}{2}$\\
$\PP(M_3|G_2)=1$\\
$\PP(M_3|G_3)=0$\\
Gesucht: $\PP(G_2|M_3)$
\begin{align*}
\PP(G_2|M_3)&=\frac{\PP(M_3|G_2)\cdot \PP(G_2)}{\PP(M_3|G_1)\cdot \PP(G_1)+\PP(M_3|G_2)\cdot \PP(G_2)+\PP(M_3|G_3)\cdot \PP(G_3)}\\
&=\frac{1\cdot \frac{1}{3}}{\frac{1}{2}\cdot \frac{1}{3}+1\cdot \frac{1}{3}+0\cdot \frac{1}{3}}\\
&=\frac{2}{3}
\end{align*}
Dieses scheinbare Paradoxon ist gut zu veranschaulichen, wenn man sich nicht 3 sondern 100 Tore vorstellt. Wenn man eines der 100 auswählt und der Moderator von den restlichen 99 Toren 98 öffnet, ist offensichtlich, dass die Wahrscheinlichkeit zu gewinne höher ist, wenn man das Tor wechselt. Die gesamte Wahrscheinlichkeiten der geöffneten Tore „sammeln“ sich hinter dem nicht geöffneten, nicht ausgewählten Tor.

\cparagraph{Beispiel} (Zuverlässigkeit diagnostischer Tests)\\
Betrachten eines Test zum diagnostizieren einer Krankheit. Dieser kann entweder „positiv“ oder „negativ“ sein.\\
Gegebene Ereignisse:\\
$P:=\{\text{Test positiv}\}$ … Test tippt darauf, dass Krankheit vorliegt.\\
$\bar P:=\{\text{Test negativ}\}$ … Test tippt darauf, dass Krankheit nicht vorliegt.\\
$K:= \{\text{Person ist krank}\}$\\
$\bar K:= \{\text{Person ist nicht krank}\}$
\begin{itemize}
\item $\text{Sensitivität}:=\PP(P|K)$ (WK, dass Test „positiv“ anzeigt, wenn man tatsächlich auch krank ist. D.h. richtig-positiver Test)
\item $\text{Spezifität}:=\PP(\bar P|\bar K)$ (WK, dass Test „negativ“ anzeigt, wenn man tatsächlich gesund ist. D.h. richtig-negativer Test)
\end{itemize}
\begin{tabular}{r | c c}
& krank & gesund\\
\hline
Test positiv & richtig-positiv & falsch-positiv\\
Test negativ & falsch-negativ & richtig-negativ
\end{tabular}

Problem: Typischerweise sind Sensitivität und Spezifität gegeben, aber eigentlich interessieren uns $\PP(K|P)$ oder $\PP(K|\bar P)$.

\subsection{Unabhängigkeit}
Wir untersuchen die Frage, ob sich Ereignisse gegenseitig beeinflussen.

\cparagraph{Definition} Zwei Ereignisse $A,B \in \cA$ heißen (stochastisch) unabhängig, wenn 
$$\PP(A\cup B)=\PP(A) \cdot \PP(B)\text{.}$$
Die Ereignisse $A_1,\dots, A_n$ heißen paarweise (stochastisch) unabhängig, wenn 
$$\PP(A_i\cup A_j) = \PP(A_i)\cdot \PP(A_j)$$
für alle $i\not=j$.\\
Die Ereignisse $A_1,\dots,A_n$ heißen (stochastisch) unabhängig (in ihrer Gesamtheit), wenn 
$$\PP(A_{i_1}\cap A_{i_2}\cap \dots \cap A_{i_k}=\PP(A_{i_1})\cdot \dots \cdot \PP(A_{i_k})$$
für jede beliebige Auswahl von $k \;(2\leq k \leq n)$ der $n$ Ereignisse.

%\newpage
%\printbibliography

\end{document}