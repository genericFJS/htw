\section{DIN ISO Überblick}
ISO (International Organization for Standardization) wurde 1946 auf einer internationalen Konferenz nationaler Normungsorganisation ins Leben gerufen. ISO hat über 150 Mitgliedsnationen und hat über 16.000 Standards veröffentlicht.

EN steht für Europäische Norm und CEN für Comité Européen de Normalisation. CEN wurde 1961 gegründet und zählt mehr als 30 Mitgliedsnationen aus ganz Europa.

DIN (Deutsches Institut für Normung e.V.) wurde 1920 gegründet, seit 1951 ist DIN Mitglied der ISO. Die wohl bekannteste DIN Norm stammt aus dem Jahre 1922 und normiert die Papierformate.

\begin{itemize}
\item DIN ISO 12119: Informationstechnik. Software-Erzeugnisse – Qualitätsanforderungen und Prüfbestimmungen

1995 veröffentlicht; mittlerweile veraltet.

Die Gebrauchstauglichkeit bzw. Nutzungsqualität ist nur ein Teil der Qualitätsanforderungen.
\item DIN EN ISO 9241: Ergonomische Anforderungen für Bürotätigkeiten mit Bildschirmgeräten 

1996 veröffentlicht. Sie wird laufend aktualisiert und ergänzt.

Es ist die bekannteste Norm.
\item DIN EN ISO 13407: Benutzerorientierte Gestaltung interaktiver Systeme
1999-2000 veröffentlicht Kurz gehalten. Ihr Ziel ist es, Projektmanager bei der Umsetzung Benutzer orientierter Gestaltungsprozesse zu unterstützen.
 
Es handelt sich also nicht um ein Produktkatalog sondern und ein Prozesskatalog.
\item DIN EN ISO 14915: Software-Ergonomie für Multimedia-Benutzungsschnittstellen

2002-2004 veröffentlicht 

Diese Norm findet ihren Einsatz bei multimedialen Anwendungen.
\end{itemize}

\section{DIN ISO 9241}
\slides{07_SE-Gestaltungsrichtlinien}{6}
Die Internationale Norm DIN EN ISO 9241 hieß zunächst „Ergonomische Anforderungen für Bürotätigkeiten mit Bildschirmgeräten“ und bestand aus 17 Teilen, die nach und nach ergänzt worden sind. Auch der Titel hat sich mittlerweile geändert, es heißt nun „Ergonomie der Nutzer-System Schnittstelle“ (Ergonomics of Human-System Interaction).

Spätere  Kapitel der Norm widmen sich konkreteren Anwendungsbereichen:
\begin{itemize}
\item Barrierefreies Design\\
20 und 171 („Leitlinien für die Zugänglichkeit der Geräte und Dienste in der Informations- und Kommunikationstechnologie“ bzw. „Leitlinien für die Zugänglichkeit von Software“)
\item Internet-Design\\
Kap. 151 (“Leitlinien zur Gestaltung von Benutzungsschnittstellen für das World Wide Web“)
\end{itemize}

\subsection{Leitsätze}
Teil 11 der ISO-Norm 9241 behandelt die Leitsätze der Software-Ergonomischen Gestaltung:

\begin{itemize}
\item Effektivität (Effectiveness): \\
das bedeutet, dass die Anwendung die Vollendung der angestrebten Aufgaben ermöglicht (objektives Merkmal oder Variable).
\item Effizienz (Efficiency): \\
das bedeutet, dass die Anwendung die Vollendung der angestrebten Aufgaben mit einem für den Benutzer vertretbaren Aufwand ermöglicht (objektives/subjektives Merkmal oder Variable).
\item Zufriedenheit (Satisfaction): \\
das bedeutet, dass die Anwendung den Nutzer zufrieden stellt (subjektives Merkmal oder Variable).
\end{itemize}

\subsection{Dialoggestaltung}
Teil 10 der Norm erläutert die Richtlinien zur Dialoggestaltung mit folgenden Worten:

\begin{itemize}
\item Aufgabenangemessenheit\\
„Ein Dialog ist aufgabenangemessen, wenn er den Benutzer unterstützt, seine Arbeitsaufgabe effektiv und effizient zu erledigen“
\item Selbstbeschreibungsfähigkeit\\
„Ein Dialog ist selbstbeschreibungsfähig, wenn jeder einzelne Dialogschritt durch Rückmeldung des Dialogsystems unmittelbar verständlich ist oder dem Benutzer auf Anfrage erklärt wird“
\item Erwartungskonformität\\
„Ein Dialog ist erwartungskonform, wenn er konsistent ist und den Merkmalen des Benutzers entspricht, z.B. seinen Kenntnissen aus dem Arbeitsgebiet, seiner Ausbildung und seiner Erfahrung sowie den allgemein anerkannten Konventionen“
\item Steuerbarkeit\\
„Ein Dialog ist steuerbar, wenn der Benutzer in der Lage ist, den Dialogablauf zu starten sowie seine Richtung und Geschwindigkeit zu beeinflussen, bis das Ziel erreicht ist“
\item Fehlertoleranz\\
„Ein Dialog ist fehlertolerant, wenn das beabsichtige Arbeitsergebnis trotz erkennbare fehlerhafte Eingaben entweder mit keinen oder mit minimalen Korrekturaufwand seitens des Benutzers erreicht werden kann“
\item Inidivdualisierbarkeit, Adaptivität\\
„Ein Dialog ist individualisierbar, wenn das Dialogsystem Anpassungen an die Erfordernisse der Arbeitsaufgabe sowie an die individuellen Fähigkeiten und Vorlieben des Benutzers zulässt“
\item Lernförderlichkeit, Erlernbarkeit\\
„Ein Dialog ist lernförderlich, wenn er dem Benutzer beim Erlernen des Dialogsystems unterstützt und anleitet“
\end{itemize}

\subsubsection*{Beispiel}
\slides{07_SE-Gestaltungsrichtlinien}{9}
Der Dialog „Suchen und Ersetzen“ aus MS-Word ist aufgabenangemessen und erwartungskonform, da er mit allgemein bekannten Steuerelementen arbeitet: Buttons, Pulldown-Menüs, Karteikarten und gezielt die Erfüllung der Aufgabe erlaubt.  Er ist selbstbeschreibungsfähig, weil die Buttons verständlich beschriften sind und ausgegraut werden, wenn sie nicht aktiv sind. 
\slides{07_SE-Gestaltungsrichtlinien}{10}
Der Dialog ist steuerbar, da jede Funktion (Suchen, Ersetzen, Gehe zu) in einer neuen Kartei bzw. Bildschirm angeboten wird und mit der notwendigen Funktionalität ausgestattet, so dass der Nutzer den Dialog gezielt steuern kann.
\slides{07_SE-Gestaltungsrichtlinien}{11}
Die Darstellung der Funktion „Gehe zu“ ist an sich auch steuerbar: der Nutzer kann bequem aussuchen, zu welchem Element er hin gehen möchte.
\slides{07_SE-Gestaltungsrichtlinien}{12}
Der Dialog ist individualisierbar, da er Optionen für Anfänger und Fortgeschrittene anbietet. Außerdem lässt er sich mit der Maus und mit der Tastatur bedienen.
\slides{07_SE-Gestaltungsrichtlinien}{13}
Die Software bietet eine Hilfe an, so ist der Dialog auch lernförderlich und erlernbar.

\subsection{Informationsdarstellung}
Teil 12 der ISO-Norm 9241 erläutert die Richtlinien zur Informationsdarstellung mit folgenden Worten.

\begin{itemize}
\item Klarheit\\
„Der Informationsinhalt wird schnell und genau vermittelt“
\item Unterscheidbarkeit\\
„Die angezeigte Information kann genau unterschieden werden (z.B. als Überschrift)“
\item Kompaktheit\\
„Den Benutzer wird nur jene Information gegeben, die für das Erledigen der Aufgabe notwendig ist“
\item Konsistenz\\
„Gleiche Information wird innerhalb der Anwendung entsprechend den Erwartungen des Benutzers stets auf gleiche Art dargestellt“
\item Erkennbarkeit\\
„Die Aufmerksamkeit des Benutzers wird zur benötigten Information gelenkt“
\item Lesbarkeit\\
„Die Information ist leicht zu lesen“ 
\item Verständlichkeit\\
„Die Bedeutung ist leicht verständlich, eindeutig, interpretierbar und erkennbar“
\end{itemize}

\subsubsection*{Beispiel}
\slides{07_SE-Gestaltungsrichtlinien}{15}
Unsere Homepage der Informatik mit der Darstellung der Studiengängen ist …
\begin{itemize}
\item klar: die Information zu den Studiengängen wird genau vermittelt.
\item  nicht immer unterscheidbar: Überschriften UND Links sind schlecht zu unterscheiden, da beide blau sind. Manche Überschriften sind aber schwarz und fett, machen Hervorhebungen im laufenden Text sind ebenfalls schwarz und fett und daher schwer von den Überschriften zu unterscheiden. Die visuelle Gestaltung von Überschriften, Links und Hervorhebungen ist nicht durchdacht. 
\item  mehr oder weniger Kompakt: die Ausbildungsziele könnten evtl. kompakter beschrieben werden.
\item  nicht immer konsistent, denn die blaue Wörter sind nicht immer verlinkt. Sonst ist die Konsistenz aber weitgehend umgesetzt.
\item  mehr oder weniger gut erkennbar. Die drei Spalten bzw. Bereiche (Navigation, Content und Zusätzliche Links bzw. Informationen ) könnten noch besser erkennbar sein. Vor allem die dritte Spalte (Suche und zusätzliche Links) ist gelegentlich etwas konfus.
\item  gut lesbar aber die Lesbarkeit könnte etwas verbessert werden.
\item  oft (wenn nicht immer) gut verständlich.
\end{itemize}
\slides{07_SE-Gestaltungsrichtlinien}{16}

\section{DIN ISO 14915}
Die ISO-Norm 14915 (2002-04 veröffentlicht) beschäftigt sich nicht mit einzelnen Medienformen, sondern mit der Integration statischer und dynamischer Medien in einer Anwendung. Die Norm ist für Anwendungen mit fachlichem Inhalt aufgestellt, d.h. für solche die für die Arbeitserfüllung oder zum Lernen eingesetzt werden. Zum Beispiel nicht für Spiele. 
 
Es werden Prinzipien vorgeschlagen sowohl als Richtlinie für Designer und Programmierer als auch als Grundlage der Qualitätssicherung mittels formativer Evaluation . D.h. die Norm ist sowohl ein Produkt- wie auch ein Prozess-Katalog. 

Die Norm gliedert sich in drei Teilen:
\begin{itemize}
\item Gestaltungsgrundsätze und Rahmenbedingungen
\item Multimediale Navigation und Steuerung
\item Auswahl und Kombination von Medien
\end{itemize}
\subsection{Grundsätze}
Die Norm übernimmt die 7 Grundsätze der Dialoggestaltung nach DIN ISO 9241 / Teil 10. Hinzu kommen vier spezielle Grundsätze für Multimedia-Anwendungen:

\begin{itemize}
\item Eignung für das Kommunikationsziel\\
Kommunikationsziele können sein: erklären, motivieren, aktivieren, unterhalten. Die Medien und die Medienzusammenstellung sollen sich dafür eignen. 
\item Eignung für Wahrnehmung und Verständnis\\
Durch die Kombination unterschiedlicher Medien werden Wahrnehmung und Verständnis stärker beansprucht. Daher müssen die Grundsätze der Informationsdarstellung nach DIN ISO 9241 / Teil 12 besonders berücksichtigt werden. Wahrnehmungsüberlastungen sind zu vermeiden. 
\item Eignung für Exploration\\
Eine Multimedia-Anwendung soll durch freies Erkunden und Probieren genutzt werden können. Eine klare Orientierung und Navigation sind an dieser Stelle besonders nützlich. Es ist zu vermeiden, dass eine Benutzerdokumentation benutzt werden soll.  
\item Eignung für Benutzermotivation\\
Durch eine ästhetische und attraktive Gestaltung mit hoher Darstellungsqualität soll der Benutzer motiviert werden. Durch Interaktionen soll seine Aufmerksamkeit aufrechterhalten werden.
\end{itemize}

\subsubsection*{Beispiel}
\slides{07_SE-Gestaltungsrichtlinien}{20}
Die virtuelle Bibliothek entstand an der HTWD 2008/09 im Rahmen der eCampus-Initiative und wurde 2009 in einer Studienarbeit im Fach Medienpsychologie mit dem Titel „Evaluation der Site Virtuelle Bibliothek der HTW Dresden mit dem Schwerpunkt Joy of Use“ von Ellen Krüger und Pauline Ehlert evaluiert. Das sind die Ergebnisse angelehnt an die Gestaltungsgrundsätze der ISO-Norm 14915 .\bigskip

Kommunikationsziel: Bibliothek vorstellen und zur Besuch der echten Bibliothek Motivieren $\to$ erreicht

Zitat aus dem Bericht: 

„Die Anwendung vermittelte den Nutzern einen sehr angenehmen Eindruck von unserer Bibliothek. Sie erschien als hell und freundlich. Somit kann die Anwendung zusätzlich als attraktives Aushängeschild der Bibliothek bezeichnet werden, das im Nutzer Vorfreude auf einen Besuch weckt“\bigskip

Kommunikationsziel: Informationsvermittlung $\to$ bedingt erreicht

Zitat aus dem Bericht:

„Allerdings ist zu befürchten, dass der mehrmalige Gebrauch dem
Nutzer als langweilig und schwerfällig erscheinen könnte. 
Wenn die Anwendung nur als Informationsquelle genutzt werden wollte, erschienen die Filmsequenzen mit den Kamerafahrten als zu langsam, wodurch das Erlangen eines Zieles ebenfalls nur sehr langsam möglich ist und zusätzlich mit mehreren Navigationssschritten verbunden. Deshalb wurde der Lageplan als das sinnvollste Navigationsmittel zur Informationsbeschaffung ernannt.“
\slides{07_SE-Gestaltungsrichtlinien}{21}
Wahrnehmung und Verständnis $\to$ bedingt erreicht

wir haben die Lesbarkeit und überhaupt die Darstellung und Steuerung nicht sorgfältig genug geplant.

Zitat aus dem Bericht: 

„Nicht immer kann man davon sprechen, dass die Nutzer die Kontrolle über das, was sie taten, hatten. Oft kamen sie an Punkte, an denen sie nicht wussten wo sie sich befanden. Außerdem funktionierte auch nicht alles so, wie es zu erwarten gewesen wäre.“
\slides{07_SE-Gestaltungsrichtlinien}{22}
Exploration: ermöglich durch Aufzug, Etagenplan und Suchfunktion $\to$ erreicht

Zitate aus dem Bericht:

„Nach einer sehr kurzen Eingewöhnung wurden mit Hilfe der Übersichtspläne in den einzelnen Etagen alle gewünschten Informationen schnell gefunden. Die Benutzung der Navigation ist sehr intuitiv, wodurch der Spass an der Anwendung uneingeschränkt genossen werden kann.“

„Die Navigation kann vielseitig genutzt werden und macht die Erkundung zu einem Erlebnis.“

\slides{07_SE-Gestaltungsrichtlinien}{23}
Benutzermotivation $\to$ erreicht

Zitat aus dem Bericht:

„Die virtuelle Bibliothek überrascht den Nutzer immer wieder während seiner Erkundungstour mit neuen Bildern, interessanten Filmsequenzen und einer Vielzahl von Informationen gestreut mit Links zu nützlichen Dokumenten. 

Man kann zusammenfassen, dass alle Tester mit großem Interesse und Spannung agierten und bei keinem Langeweile aufkam“.

\subsection{Navigation}
\slides{07_SE-Gestaltungsrichtlinien}{24}

\subsection{Medien}
\slides{07_SE-Gestaltungsrichtlinien}{25}
\subsubsection*{Allgemeine Leitlinien}
\slides{07_SE-Gestaltungsrichtlinien}{26}

\section{Apple Human Interface Guidelines}
Was sind die Macintosh Human Interface Guidelines?
Es handelt sich um eine Industrie-Richtlinie publiziert von der Firma Apple Anfang der 90er Jahren. Sie beinhalten Grundsätze und Empfehlungen für die Gestaltung von Macintosh-Oberflächen. Diese Grundsätze sind aber vom allgemeinen Interesse, weil sie teilweise nicht nur Macintosh-bezogen sind und weil viele dieser Prinzipien von Microsoft Windows übernommen wurden.
\slides{07_SE-Gestaltungsrichtlinien}{28}

\slides{07_SE-Gestaltungsrichtlinien}{29}
\subsection{Metpahern}
Durch die Nutzung von Metaphern (d.h. Bilder aus der „realen“ Welt)  können die Benutzer von den Kenntnissen aus der Welt profitieren, auf diese Weise wird das Gedächtnis entlastet  (man spricht vom „positiven Transfer“).
\slides{07_SE-Gestaltungsrichtlinien}{30}

Bekannte und bewährte Beispiele sind die Desktop-Metapher, der Papierkorb (für Löschen), Ordner (für Verzeichnis).

Desktop = Obere Oberfläche des Schreibtisches. 

Ich stelle meinen Papierkorb aber nicht auf meinem Schreibtisch auf…

\slides{07_SE-Gestaltungsrichtlinien}{31}
Man muss ein Gleichgewicht finden zwischen dem, was die Metapher andeutet und dem, was der Computer wirklich macht. Manchmal ist dieses Gleichgewicht schwer zu finden, vor allem dann wenn Prozesse dargestellt werden sollen. Ein viel diskutiertes Beispiel ist die Animation, die das Download einer Datei darstellen sollte: ein Papierchen fliegt von einem Ordner zum anderen bzw. von der Weltkugel zum Ordner. Ist das eine gute Metapher? Ich persönlich finde sie nicht so schlecht, obwohl sie sehr viel kritisiert wurde.


\subsection{Direkte Manipulation}
Direkte Manipulation bedeutet, dass der Benutzer die Objekte direkt auf dem Bildschirm handhabt. Bekannt ist die Aktion zum Löschen von Dateien: man zieht die Datei in den Papierkorb, ein gelungenes Beispiel für direkte Manipulation.
\slides{07_SE-Gestaltungsrichtlinien}{32} 
Auch Google-Maps arbeitet mit direkter Manipulation: die Karte lässt sich direkt mit der Maus bewegen und auch die Webcams sind direkt am Platz und lassen sich dort starten. 


\subsection{Konsistentz}
Die Konsistenz aus den Mac HI Guidleines entspricht in etwa der Erwartungskonformität nach DIN 9241-110

Ein Software-System sollte konsistent sein ..
\begin{itemize}
\item mit sich selbst
\item mit anderen Software-Anwendungen
\item mit der Welt
\end{itemize}

So kann der Benutzer seine Kenntnisse von einer Anwendung auf anderen erweitern (auch hier spricht man vom „positiven Transfer“).
\slides{07_SE-Gestaltungsrichtlinien}{33}
Früher bedeutet die kleine Hand in Google-Maps „zurück zum letzten Ergebnis“; das ist eigentlich überraschend und mit anderen Anwendungen nicht konsistent. Mittlerweile ist dieses Problem behoben.
\subsection{WYSIWYG}
WYSIWYG = What you see is what you get

Was gebraucht wird, wird auf dem Bildschirm gezeigt.
\slides{07_SE-Gestaltungsrichtlinien}{34}
Beispiel: Alle Funktionen, die in OPAL benötigt werden, erscheinen kompakt auf dem oberen rechten Teil des Bildschirms


\subsection{Stabilität}
Die Nutzungsoberfläche soll stabil sein, sie sollte sich nicht verändern. Eine gleich bleibende Oberfläche ist leichter zu erinnern. 
\slides{07_SE-Gestaltungsrichtlinien}{35}
Beispiel: Wenn bestimmte Aktionen nicht verfügbar sind, werden sie nicht gelöscht sondern einfach ausgegraut. Die kleine Piktogramme unten rechts auf diesem Bildschirm sind orange, wenn sie aktiv und hellgrau, wenn sie nicht aktiv sind.
\slides{07_SE-Gestaltungsrichtlinien}{36}

\subsection{Steuerbarkeit}
Der Benutzer sollte immer die Initiative nehmen können. Diese Richtlinie ist auch in DIN 9241-110 zu finden.
\slides{07_SE-Gestaltungsrichtlinien}{37}

Beispiel: In diesem Tutorial kann der Lernende seinen Lernstrategie frei wählen und das Programm steuern. Er kann sich beliebig bewegen und Lernmodi (Lektion, Übung, Test) aussuchen.

\subsection{Feedback und Dialog}
Dem Nutzer wird unmittelbar informiert und zwar direkt und so einfach wie möglich. Diese Richtlinien entspricht der Idee von Norman in seinem Buch „Design of Everyday Things“.
\slides{07_SE-Gestaltungsrichtlinien}{38}
Ein einfaches Beispiel: Fortschritt-Anzeigen bei Ladeprozessen
\slides{07_SE-Gestaltungsrichtlinien}{39}

\subsection{Fehlertoleranz}
Aktionen können rückgängig gemacht werden. Ein Fehler des Benutzers führt nicht zu irreparablen Verlusten.
\slides{07_SE-Gestaltungsrichtlinien}{40}
Beispiel aus den Windows Commander: Bevor die Datei gelöscht wird, bekommt der Nutzer eine Sicherheitsanfrage.

\subsection{Ästhetische Integrität}
Software muss nicht nur funktionell sein sonder auch gut aussehen. 

\slides{07_SE-Gestaltungsrichtlinien}{41}
Als wir in AnOpeL-Projekt die Hilfe für OPAL neu gestaltet haben, war uns nicht nur die Usability sondern auch das Erscheinungsbild wichtig. Allerdings haben wir dann etwas übertrieben. Das Bild mit dem Mädchen und dem Fragezeichen sollte motivierend wirken aber es war überflüssig. Wir haben es später entfernt. Das Hilfecenter sah ohne zusätzliche Bilder besser und sachlicher aus.
\slides{07_SE-Gestaltungsrichtlinien}{42}
Die favorisierte Version:
\slides{07_SE-Gestaltungsrichtlinien}{43}
\subsection{Guidelines 2017}
Apple publiziert laufend Empfehlungen und Richtlinien für Hersteller von Mac-Oberflächen. Diese sind mittlerweile online verfügbar.
Einige Empfehlungen sind auf die Design-Philosophie von Mac ausgerichtet, aber die grundlegenden Prinzipien der Guidelines sind geblieben.
\subsubsection{Grundlegende Ziele}
\slides{07_SE-Gestaltungsrichtlinien}{45}
\begin{itemize}
\item Die Klarheit bezieht sich hier nicht nur auf dem Inhalt sondern auf alle Elemente der Gestaltung: Layout, Negativer Raum, Farbe, Bilder, Piktogramme, Texte. 
\item Inhalt und Nutzer verdienen Respekt. Inhalt geht vor Designer-Vorlieben vor. Z.B. der Inhalt wird auf der ganzen Fläche dargestellt.
\item Es soll ein Gefühl der Tiefe erzeugt werden, um mehr Funktionen auf dem Bildschirm vermitteln zu können.
\end{itemize}

\subsubsection{Prinzipien}
\slides{07_SE-Gestaltungsrichtlinien}{46}
Vergleich 1992:
\slides{07_SE-Gestaltungsrichtlinien}{47}

\subsubsection{Nutzer Interaktion}
\slides{07_SE-Gestaltungsrichtlinien}{48}

\subsubsection{Visuelles Design}
\slides{07_SE-Gestaltungsrichtlinien}{49}

\subsubsection{Richtlinien}
Die Richtlinien aus dem Jahre 1992 sind geblieben und nach wie vor aktuell. Es sind drei neue Prinzipien dazu gekommen.\bigskip

Neue Prinzipien:
\begin{itemize}
\item An zweiter Stelle hinzugefügt: Reflect the User's Mental Model 
\item An dritter Stelle: Explicite and Implied Actions
\item An letzter Stelle: Managing Complexity in Your Software
\end{itemize}

\subsubsection*{Mentales Modell des Nutzers}
Dieses Prinzip erinnert uns an den Vorschlag von Norman in seinem Buch “The Design of Every Day Things”: Nutzer machen sich immer ein mentales Modell darüber, wie die Sachen funktionieren.

Das gilt auch für Software: dieses Model entsteht aus der allgemeinen Erfahrung und aus der Erfahrung mit anderen Programmen und Computern.

Bevor ein Software-Design entsteht, sollte das Designerteam sich ein Bild darüber machen, mit welchem mentalen Model die Nutzer die Aufgaben lösen, die die Software übernehmen soll. Ein gutes Design ist in diesem Sinne, vertraut, einfach, zugänglich, und lädt zum selbst entdecken ein.

\subsubsection*{Explizite und Implizite Handlungen}
Explizite Handlungen bzw. Handlungsmöglichkeiten sind direkt auf dem Bildschirm sichtbar. Z.B. die Einträge in einem Menü sind direkt dort sichtbar.

Implizite Handlungen bzw. Handlungsmöglichkeiten sind auf dem Bildschirm nicht direkt sichtbar, der Nutzer muss selbst aus den Objekten, die sichtbar sind, entnehmen, dass die nicht sichtbare Handlung möglich ist. Beispiel: eine Datei und ein Papierkorb sind sichtbar, der Nutzer kann daraus entnehmen, dass er die Datei zum Papierkorb ziehen kann, um sie zu löschen. Das Ziehen selbst ist aber nicht sichtbar.

Die Empfehlung von Apple in diesem Zusammenhang ist: Beim Designprozess sollte bei jeder Handlung, die später mit der Software wird durchgeführt werden können, das Designerteam gut überlegen, ob diese explizit auf dem Bildschirm darzustellen ist oder als implizite Handlung umgesetzt werden kann. Viele implizite Handlungen entlasten die Gestaltung, müssen aber natürlich intuitiv erkennbar sein.

\subsubsection*{Managing Complexity}
Die Komplexität der Software Meistern
Hier gilt (Zitat): „ein einfaches Design ist ein gutes Design“. Je komplexer die Aufgaben sind, die die Software lösen soll, umso wichtiger ist es, die Schnittstelle zum Nutzer so einfach wie möglich zu gestalten.

Eine erprobte Maßnahme ist komplexere Funktionen in einem weiterführenden Formular unterzubringen, das nur auf Wunsch erscheint (Progressive Disclosure). In diesem Beispiel sind weitere Funktionen unter den Link „Details“ versteckt.

\slides{07_SE-Gestaltungsrichtlinien}{62}
\slides{07_SE-Gestaltungsrichtlinien}{63}
Auch sog. „Inspektor-Fenster“ haben sich bewährt: Komplexität wird versteckt, so lange sie nicht benötigt wird. Im Inspektor-Fenster sind zusätzliche Informationen und Einstellungsmöglichkeiten, die nur bei Bedarf erscheinen.
\slides{07_SE-Gestaltungsrichtlinien}{64}

\section{Anti-Mac Nielsen}
Ein Beispiel von Software-Ergonomische Designer-Empfehlungen: Die Anti-Mac-Oberfläche von Getner und Nielsen (1997)

In diesem berühmten Artikel diskutieren die Autoren die ergonomische Prinzipien der Macintosh Human Interface Guidelines (die gerade präsentiert wurden) und schlagen Alternativen vor, die diese Prinzipien weiter führen sollen. Der Artikel steht Ihnen in OPAL-Kurs zur Verfügung; die Lektüre lohnt sich sehr. Viele Ideen aus diesem Artikel sind in unserer aktuellen Software-Welt bereits umgesetzt und erscheinen uns heute als selbstverständlich. 
\subsection{Gegen Metaphern}
\subsubsection*{Falsche Voraussetzung}
Die Richtlinie „Metaphern“ geht von einer falschen Voraussetzung aus nämlich, dass die Nutzer ihr Wissen über die Welt gebrauchen. In Wirklichkeit sind die Nutzer mit dem Computer aufgewachsen. Sie sind nicht mit Ordner groß geworden, sondern mit Dateien und Verzeichnissen.
\subsubsection*{Problem}
Metapher überlasten und begrenzen das UI.
\subsubsection*{Alternativ}
Die Autoren schlagen als Alternative vor: Software sollte nicht so sein wie die Realität sondern besser als die Realität:
\begin{itemize}
\item Individualisiert. Beispiel: Word
\item Asynchron. Beispiel: Foren
\item Nicht linear. Beispiel: Hypertext
\item Anonym. Beispiel: Internet 
\end{itemize}

\slides{07_SE-Gestaltungsrichtlinien}{51}
Das Bild zeigt ein Comic, das in den 90er Jahren sehr bekannt wurde, und das die Anonymität im Internet verdeutlicht. Auch Gentner und Nielsen habe an dieser Anonymität geglaubt und sie als Vorteil angesehen. Heute wissen wir, dass es ganz anders gekommen ist. 

\subsection{Gegen Direkte Manipulation}
\subsubsection*{Probleme}
Direkte Manipulation kann ein UI belasten und zu Problemen führen …
\begin{itemize}
\item wenn die Anzahl der Objekte wächst.
\item wenn die Komplexität der Aufgaben wächst.
\item wenn Präzision notwendig ist.
\end{itemize}

\subsubsection*{Alternativ}
Beide Autoren schlagen als Alternative vor: Die Macht der Sprache nutzen z.B. mit Dialogen zwischen Mensch und Maschine. An einem UI, das auf natürliche Sprache beruht, wird seit Jahren gearbeitet. Das Ziel ist sicherlich schwer zu erreichen.
 
\slides{07_SE-Gestaltungsrichtlinien}{52}
Das Bild zeigt Ausschnitte aus dem Film von Stanley Kubrick „2001 Odyssee in Weltraum“. Der Computer im Raumschiff (namens Hall) versteht die natürliche Sprache und kann sogar  mit seinen Augen (rote Kamera unten links) die Lippen der Astronauten lesen (Bild oben). 

Der Film von Kubrick wirft die Frage auf, ob eine Maschine, die der Sprache mächtig ist, auch über Selbstbewusstsein verfügt. Jedenfalls glaubt Kubrick, dass es so sein müsste. In dem Film macht Hall Fehler und muss ausgeschaltet werden (Bild unten rechts) – sie währt sich dagegen, weil sie nicht „sterben“ will.


\subsection{Gegen Konsistenz}
\subsubsection*{Probleme}
Konsistenz kann zu Probleme führen:
\begin{itemize}
\item Sie ist schwierig zu halten in komplexen Systemen.
\item Sie kann ein UI monoton und unverständlich machen. Sie ist in der Welt gar nicht vorhanden, z.B. Bücher sehen unterschiedlich aus, wären sie alle gleich, könnten wir sie in Regal nur sehr schwer finden.
\end{itemize}

\subsubsection*{Alternativ}
Eine ausdrucksvolle Oberfläche. 
\slides{07_SE-Gestaltungsrichtlinien}{53}
Beispiel: Piktogramme für Dokumente und Programme sehen je nach Inhalt unterschiedlich aus. So kann man sie besser finden, wie ein Buch in unserem Regal.


\subsection{Gegen WYSIWYG}
\subsubsection*{Problem}
WYSIAT: What You see is all there is

\subsubsection*{Alternativ}
Eine reiche interne Darstellung von Objekten. Die Objekte besitzen mehr Informationen als die, die zunächst sichtbar sind.
\slides{07_SE-Gestaltungsrichtlinien}{54}
Zum Beispiel ist dieser Text-Datei mit Metadaten versehen.

\subsection{Gegen Stabilität}
\subsubsection*{Alternativ}
Intelligente Anwendungen, die sich an die Bedürfnisse des Nutzers anpassen.

\slides{07_SE-Gestaltungsrichtlinien}{55}
Beispiel: Man kann die Startseite von OPAL anpassen und auch die zuletzt besuchten Kursen werden gezeigt.

\subsection{Gegen Steuerbarkeit}
\subsubsection*{Alternativ}
Der Benutzer teilt die Kontrolle mit Computer-Agenten, die für ihn Aufgaben erledigen.

\slides{07_SE-Gestaltungsrichtlinien}{55}
Beispiel: Buchempfehlungen in Amazon. Der Leser muss nicht mehr allein nach neuen interessanten Lektüren suchen.

\subsection{Gegen Feedback und Dialog}
\subsubsection*{Alternativ}
Wenn der Nutzer nicht alles kontrolliert, muss er auch nicht alles wissen.

\slides{07_SE-Gestaltungsrichtlinien}{56}
Beispiel: der Adobe-Acrobat installiert sich selbständig in Menü von MS-Word.


\subsection{Gegen Fehlertoleranz}
\subsubsection*{Vorschlag}
Die Software „verzeiht“ nicht nur unsere Fehler, sie macht sich darüber hinaus ein Bild unserer Absichten. Damit werden Fehler im Vorfeld vermieden.

\slides{07_SE-Gestaltungsrichtlinien}{57}
Beispiel:  Menü mit Punkt „zuletzt Besucht“. Der Computer setzt voraus, dass der Nutzer evtl. dahin zurück möchte, wo er schon war, was oft der Fall ist.

\subsection{Gegen Ästhetische Integrität}
\subsubsection*{Vorschlag}
Die Nutzer suchen nicht nur ästhetische befriedigende Produkte sondern wollen auch neue und innovative Darstellungen.

\slides{07_SE-Gestaltungsrichtlinien}{58}
Beispiel: wegen der neuen und innovativen Darstellung und Interaktion war der iPhone von Apple ein Welterfolg: Design matters !!!






















