\slides{02_4u1-einfache-Regel}{2}
In diesem Thema werden 4 grundlegende Regel der Gestaltung nach Robin Williams vorgestellt. Die Beispiele stammen aus dem Buch von Robin Williams „The Non-Designer's Design Book“ und aus Belegarbeiten von Studierenden der HTWD. Zusätzlich wird eine ergänzende Regel nach Claudia Runk vorgestellt.
Darauf folgt eine Praktikumsaufgabe.

\section{Gestalten lernen}
\slides{02_4u1-einfache-Regel}{4}
Dieser Satz unterstreicht folgende Idee:\bigskip

Gestalten ist nicht nur eine Frage der Inspiration und schon gar nicht des persönlichen Geschmacks. Es gibt vielmehr Prinzipien bzw. Regeln der Gestaltung, die objektiv und rational gelten.

Gestalten lernen bedeutet, die Prinzipien und Regeln der Gestaltung zu lernen und in die Praxis umsetzen.

\slides{02_4u1-einfache-Regel}{5}
Johannes Itten, Künstler und Pädagoge, Autor mehrerer bekannten Farbmonografien, lädt uns dazu ein, gestalten zu lernen.

\slides{02_4u1-einfache-Regel}{6}
Auch der Satz von Tschihold, ein bekannter Deutsch-Schweizer Typograf (geboren in Leipzig 1902, gestorben in Locarno 1974), unterstreicht die Rationalität und Objektivität von Gestaltungsaufgaben.

\subsubsection*{Gestalten lernen?}
\slides{02_4u1-einfache-Regel}{7}
Allerdings muss man beachten: Gestaltungsregeln können und sollen kreativ verändert werden.\bigskip

Das ist der Sinn dieser Behauptung von Leonardo da Vinci, der selbst großer Anhänger und Verfasser von Regeln in der Malerei und in der Kunst war.

\subsubsection*{Gestalten lernen!}
\slides{02_4u1-einfache-Regel}{8}
Wichtig ist: Regel dürfen nur gezielt, bewusst und intelligent verändert werden; dafür muss man sie aber genau kennen und beherrschen. So ist dieses Zitat zu verstehen.

\section{Vier Regeln der Gestaltung nach Robin Williams}
Robin Willians erläutert in ihrem Buch „The Non-Designer's Design Book“ [Williams 1994] vier grundlegende Regel bzw. Techniken der Gestaltung.
\begin{itemize}
\item Nähe (für Gruppen nach Claudia Runk)
\item Ausrichtung (für Gruppen nach Claudia Runk)
\item Wiederholung (vergleichbar der Wiederkennung nach Claudia Runk)
\item Kontrast (kann für Blickfang nach Claudia Runk eingesetzt werden)
\end{itemize}

\subsection{Psychologische Grundlagen}
\slides{02_4u1-einfache-Regel}{11}
Die Gestaltungsregel nach Williams haben eine psychologische Grundlage: Nähe, Geschlossenheit, Ähnlichkeit sind Gesetze der visuellen Wahrnehmung. Was nah, ähnlich oder geschlossen gestaltet ist, wird als Einheit wahrgenommen: das Auge ergänzt.
 
Diese Gesetzte wurden in der Gestaltpsychologie der 20er Jahre entdeckt. Das neue dabei war, dass die Wahrnehmung als aktiver, konstruktiver Prozess erkannt wurde.
\subsubsection{Beispiel}
\slides{02_4u1-einfache-Regel}{12}
Die Linien werden eindeutig in 3 Gruppen wahrgenommen, sie werden automatisch nach Nähe gruppiert.

Die Quadrate und die Dreiecke werden nach Ähnlichkeit (und entsprechend Kontrast) gruppiert; wir sehen eindeutig vier Gruppen und nicht Einzelobjekte.

Die Klammer unter werden nach Geschlossenheit bzw. Ausrichtung gruppiert.

\section{Nähe}
\subsubsection*{Bedeutung}
Elemente, die inhaltlich zusammengehören, sollen (auf dem Bildschirm) nah dargestellt werden und somit eine visuelle Einheit bilden (Gruppen nach Claudia Runk).

\subsubsection*{Wofür?}
Das Hauptziel der Nähe ist die \emph{Organisation}. Der Bildschirm wird schneller erfasst, die Informationen sind einfacher zu finden und zu erinnern (\emph{Kommunikation}).
Außerdem kann die so gewonnene \emph{leere Fläche} („negativer“ oder „\emph{weißer Raum}“) als ästhetisches Element genutzt werden (\emph{Ästhetik})

\subsubsection*{Was tun?}
Betrachten Sie den Inhalt, lesen Sie den Text. Was gehört zusammen?
\emph{Gruppieren} Sie, was zusammen gehört, und lassen Sie deutlich \emph{leeren Raum zwischen Gruppen}.

Wenn Sie mehr als 3-5 Gruppierungen haben, fassen Sie weiter zusammen.

\subsubsection*{Was vermeiden?}
Viele getrennte, \emph{unorganisierte Elemente} auf dem Bildschirm, z.B. in Ecken oder in der Mitte.

Nähe zwischen Elemente, die \emph{inhaltlich nicht zusammen gehören}.

\emph{Gleiche Abstände} zwischen verschiedene „Bedeutungsgruppen“. Abstände vermitteln Bedeutung.

\subsection{Beispiele}
\slides{02_4u1-einfache-Regel}{15}
Bei der Visitenkarte, die rechts zu sehen ist, sind die richtige Gruppen gebildet worden: Einerseits die Person bzw. die Firma und anderseits die Kontaktdaten.

Diese Gestaltung ist einfach aber gelungen und ist den Versionen an der linken Seite vorzuziehen, die überhaupt nicht organisiert sind.

\slides{02_4u1-einfache-Regel}{17}
Diese Gestaltung beinhaltet einige Fehler, was Nähe und Gruppierung angeht:

Die Frage „Unter welchen Beschwerden…“  gehört nicht wirklich auf dem Blatt. Sie ist an der falschen Stelle bzw. in der falschen Gruppe platziert worden.
 
Die Überschriften (Diagnose, Rezept) und die zugehörige Texte bzw. Eingabefelder sollten näher aneinander rücken. Die Abstände zum Wort „Rezept“ nach oben und nach unten sind gleich und vermitteln nicht die richtige Bedeutung; sie sind falsch eingesetzt. Der Abstand nach unten zum Eingabefeld für das Rezept hätte kleiner sein sollen als der Abstand nach oben, da Eingabefeld und Überschrift zusammengehören.

\section{Ausrichtung}
\subsubsection*{Bedeutung}
Jeder Bestandteil einer Gestaltung soll eine visuelle Verbindung zu anderen Bestandteile der Gestaltung haben. Nichts auf einem Bildschirm sollte zufällig platziert werden. 

\subsubsection*{Wofür?}
Elemente die ausgerichtet sind, werden in Verbindung gesetzt: \emph{Organisation} (\emph{Kommunikation})

Mit Ausrichtung kann man \emph{Gleichgewicht} in die Gestaltung bringen (\emph{Ästhetik})

Mit strengen und ungewöhnlichen Ausrichtungen können \emph{Besondere Effekte} erzeugt werden, die Emotionen vermitteln, z.B. Dynamik oder Seriosität (\emph{Emotion})

\subsubsection*{Was tun?}
Setzen Sie Elemente in Verbindung. Finden und \emph{betonen Sie Beziehungen} durch Ausrichtung, auch wenn Elemente auf der Bildschirmfläche entfernt voneinander liegen.

\emph{Zentrierte} und \emph{symmetrische} Ausrichtungen vermitteln \emph{Ruhe, Beständigkeit, Sicherheit}; sie wirken \emph{klassisch}.

\emph{Asymmetrische} Ausrichtung vermitteln \emph{Dynamik, Instabilität, Veränderung}; sie wirken \emph{modern}.

\subsubsection*{Was vermeiden?}
Vermeiden Sie \emph{zentrierte Ausrichtungen}, denn sie sind typisch für Anfänger. Es sei denn, sie wählen die zentrierte Ausrichtung ganz bewusst. Dann sollte sie aber deutlich betont werden.

Werfen Sie nicht verschiedene \emph{Ausrichtungen durcheinander}, die nicht genügend kontrastieren.

\subsection{Beispiele}
\slides{02_4u1-einfache-Regel}{20}
\slides{02_4u1-einfache-Regel}{21}
\slides{02_4u1-einfache-Regel}{23}
Diese Gestaltung beinhaltet einige Fehler, was Ausrichtung angeht.\bigskip

Die rechtsbündige Überschrift („Der Friedenkuss“) bildet einen ungenügenden Kontrast zur zentriertem Ausrichtung des Textes. 

Außerdem hätte man die Zentrierung des Textes durch Blocksatz stärker betonnen sollen, die Zentrierung sieht eher zufällig und ungewollt aus.
\slides{02_4u1-einfache-Regel}{25}
In diesem Beispiel stehen die rechtsbündige Überschrift („Elberadweg“) und die zentrierte Elemente innerhalb des Rahmens („Willkommen etc.“)  in einem gezielten Kontrast zueinander, vor allem weil die zentrierte Ausrichtung extrem betont worden ist. Die Rechts-Ausrichtung der Überschrift soll dynamisch wirken, was durch die kursive Schrift (Rad) unterstrichen wird. Beide Ausrichtungen sind bewusst eingesetzt, deutlich betont und gezielt kombiniert.

\section{Wiederholung}
\subsubsection*{Bedeutung}
Bestimmte Elemente werden auf dem Bildschirm wiederholt.

\subsubsection*{Wofür?}
\emph{Einheitlichkeit}: alles wirkt wie aus einem Guss (\emph{Kommunikation, Ästhetik})

\emph{Attraktivität}, die Interesse weckt (\emph{Ästhetik, Emotion})

\subsubsection*{Was tun?}
Finden Sie bereits gegebene \emph{Wiederholungen} und \emph{betonen} Sie sie.

Versuchen Sie weitere Ideen für Wiederholungen zu finden.

\subsubsection*{Was vermeiden?}
Wiederholen Sie die Elemente nicht so oft, dass die Wiederholung \emph{langweilig}, \emph{aufgesetzt} oder gar ärgerlich wirkt. 

\subsection{Beispiele}
\slides{02_4u1-einfache-Regel}{28}
\slides{02_4u1-einfache-Regel}{29}
In diesem Plakat sind die Dreiecke aus den Tassen in die Gestaltung aufgenommen worden.
\slides{02_4u1-einfache-Regel}{31}
\slides{02_4u1-einfache-Regel}{32}

\section{Kontrast}
\subsubsection*{Bedeutung}
Unsere Augen lieben Kontrast, sie brauchen Unterschiede.

Wenn zwei Elemente nicht genau gleich sind, sollten sie sich deutlich unterscheiden.

\subsubsection*{Wofür?}
Der Bildschirm wirkt interessanter, \emph{attraktiver} (\emph{Ästhetik, Emotion})

Auch die \emph{Organisation} des Bildschirms gewinnt durch Kontrast. Kontrast darf nicht verwirren, sondern soll zum Verständnis des Inhalts beitragen (\emph{Kommunikation})

\subsubsection*{Was tun?}
Kontrast hinzufügen

Das allerwichtigste: Machen Sie die Unterschiede \emph{deutlich}!

\subsubsection*{Was vermeiden?}
Seien Sie keineswegs vorsichtig oder gar \emph{ängstlich} und damit \emph{unklar}: stellen Sie nicht eine mittle schwere mit einer schwereren Linie oder zwei mehr oder weniger verschiedene, mehr oder weniger ähnliche Schriften zusammen.
 
Noch einmal: die Unterschiede sollen DEUTLICH sein! ÜBERTREIBEN Sie ruhig, meistens wird das Design dadurch interessanter.

\subsection{Methoden zum einfügen von Kontrast}
Folgende Eigenschaften können für einen größeren Kontrast voneinander abgehoben werden:
\slides{02_4u1-einfache-Regel}{36}

\subsubsection{7 Farbkontraste nach Itten}
\slides{02_4u1-einfache-Regel}{37}
\slides{02_4u1-einfache-Regel}{38}

\subsubsection{Schriftklassen nach DIN 16518}
\slides{02_4u1-einfache-Regel}{39}
\slides{02_4u1-einfache-Regel}{40}

\subsection{Beispiele}
\slides{02_4u1-einfache-Regel}{35}
Rechts ist mehr Kontrast, dadurch sind Überschriften besser zu erkennen und die Gestaltung wirkt interessanter. Sie prägt sich besser ein.
\slides{02_4u1-einfache-Regel}{41}
Der Designer arbeitet mit Farbkontrast: Kalt (Blau) / Warm (Gelb). Die blaue Farbe wird heller und dunkler dargestellt (Qualitätskontrast). Bei der Schrift kontrastieren die Bezeichnungen in Kapitälchen (in der Navigation) mit dem laufenden Text.
\slides{02_4u1-einfache-Regel}{42}
Ein sehr schönes Beispiel für den gelungenen Einsatz von Kontrast:

\begin{itemize}
\item Bunt/Unbunt-Kontrast: der grauen Kopfbereich und der grauen Hintergrund kontrastieren mit den bunten Flächen der Navigation
\item Farb-an-sich-Kontrast: die bunte Flächen zeigen alle Grundfarben: Grün, Rot, Gelb, Blau
\item Auch in diesem Beispiel werden Kapitälchen eingesetzt und die Laufweite (Buchstabenabstand) variiert, um Kontrast in der Schrift zu bringen
\item Last but not least: die vertikale Ausrichtung der Texte auf den bunten Flächen kontrastiert gekonnt mit den restlichen Texten
\end{itemize}

\section{Blickfang}
\slides{02_4u1-einfache-Regel}{44}
Eine durch Nähe, Ausrichtung und Wiederholung gut strukturierte, verständliche und sogar gefällige Gestaltung ist nicht immer ausreichend, denn sie kann langweilig und nichtssagend wirken. Sie wird gut verstanden aber danach schnell vergessen – oder noch schlimmer, der Betrachter beschäftigt sich erst gar nicht mit ihr.

Mit einem richtig gewählten und entschlossen gestalteten Blickfang wirkt die Gestaltung attraktiv und zieht die Aufmerksamkeit auf sich. Der Blickfang bietet einen Punkt an, an dem das Auge mit der Betrachtung beginnen kann. Das Auge wird „gefangen“ und dann zu den anderen Elementen in Design geführt. Die Gestaltung wirkt nicht nur interessanter, auch der Inhalt kann besser erfasst werden.

\subsection{Fehlender Blickfang}
\slides{02_4u1-einfache-Regel}{45}
Das Poster ist nett und liebevoll gestaltet, aber ein Blickfang fehlt. Dadurch weißt der Betrachter nicht (mindesten nicht auf dem ersten Blick), worum es sich hier handelt, das Auge wird nicht geführt und bleibt unentschlossen.
\subsection{Blickfang richtig wählen}
Ein Blickfang sollte zunächst einmal inhaltlich richtig gewählt werden. Die Frage ist: was ist hier das WICHTIGSTE und zwar FÜR DEN BETRACHTER?
\slides{02_4u1-einfache-Regel}{46}
Oben ist der Blickfang nicht richtig gewählt. Wichtig für den Betrachter ist nicht die Inhaberin des Geschäfts sondern die Dienstleistung: das mobile Reisebüro. Unten wurde der Blickfang richtig gewählt. Der Betrachter weißt sofort, worum es sich hier handelt.
\subsection{Blickfang deutlich visualisieren}
Ein Blickfang wird nur als solcher erkannt, wenn er visuell entschlossen gestaltet wird, und zwar durch ausreichenden Kontrast. Kontrast ist die wichtigste Technik, um einen Blickfang visuell deutlich zu gestalten (vgl. Robin Williams).
\slides{02_4u1-einfache-Regel}{47}
Das Poster soll auf die Musik-Feuerwerke aufmerksam machen.

Auf der linken Version ist der Blickfang richtig gewählt und deutlich dargestellt. Der Text („Musikfeuerwerke“) kontrastiert stark durch Größe, Schriftform und Farbe und fällt ins Auge. Der Text wird durch das Feuerwerk-Bild ergänzt und kommt richtig zur Geltung. Der Betrachter weißt sofort, worum es geht.

Auf der rechten Version ist der Text „Musik-Feuerwerke“ kaum größer als die andere Texte dargestellt und als Blickfang nicht erkennbar. Durch den blauen Banner und die Ergänzung „Weltgrösstes Festival der …“ wird er regelrecht erdrückt. 

\subsection{Beispiel}
\slides{02_4u1-einfache-Regel}{48}
Der Blickfang ist auf diesem Poster durch starken Kontrast in der Farbe und in der Größe deutlich dargestellt.  Bild und Spruch ergänzen sich inhaltlich und visuell. 