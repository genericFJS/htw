\section{Planung des Projektes}
Das Ziel des Projektes ist es die Webseite von \url{fj-strube.de} zu evaluieren und damit auf Nutzerfreundlichkeit zu testen.
Es werden spekulative/empirische Personae genutzt, um mögliche Zielgruppen der Webseite zu definieren, entsprechende Testpersonen zu finden und geeignete Testfälle zu definieren.

\section{Personae}
Im folgenden sind die drei Personae beschrieben:
\begin{itemize}
\item Der Student,
\item die Firmenchefin und
\item die Wichtel organisierende.
\end{itemize}
%\section{Datenerhebung mittels Leitfaden Interviews}
%\section{Datenaufbereitung mittels Storytelling}
%\section{Datenstrukturierung mittels Cluster-Analyse}
%\section{Modellierung der Persona}

\clearpage
\subsection{Persona Musterstudent}
%\addcontentsline{toc}{subsection}{Persona Musterstudent}
% ===============================
\begin{minipage}[t][][b]{.25\textwidth}
\includegraphics[width=\textwidth]{persona_pics/persona1.jpg}\end{minipage}\hfill
%
\begin{minipage}[t][][b]{.72\textwidth}
\noindent%
\begin{tabular}{L{.25} L{.7}}
\textbf{Name:} & Max Musterstudent \\
\textbf{Alter:} & 22 Jahre\\
\textbf{Beruf:} & Student (Fakultät Informatik/Mathematik)\\
\textbf{Merkmal:} & Englischsprachig \\
\textbf{Zitat:} & \\
\end{tabular}\\
\begin{quote}
„Am liebsten lasse ich alle Vorlesungen sausen und hole mir alle Materialien von meinen Kommilitonen.“
\end{quote}
\end{minipage}\vspace*{1em}

\begin{minipage}[t][][b]{.48\textwidth}
\subsubsection*{Alltagsroutinen}% und Handlungen}
\begin{itemize}[leftmargin=*]
%\item Vorlesungen besuchen
\item Nachbereiten der Vorlesungen
\item Arbeiten an Belegaufgaben
\item Spielen von Computerspielen
\item An eigenen Projekten arbeiten
\item Deutsche Sprache lernen
\end{itemize}
\end{minipage}\hfill
%
\begin{minipage}[t]{.48\textwidth}
\subsubsection*{Probleme und Bedürfnisse}% in Zusammenhang mit dem Produkt}
\begin{itemize}[leftmargin=*]
\item Hat bisher nur deutsche bzw. schlecht übersetzte Websites, die Studienmaterialien anbieten, gefunden
\item Probleme beim lösen von Belegaufgaben
%\item Probleme beim anfertigen von Vorlesungsmitschriften
\item Möchte Abwechslung in Computerspielen mittels Mods
\end{itemize}
\end{minipage}\bigskip

\begin{minipage}[t][][b]{.45\textwidth}
\subsubsection*{Wünsche, Ziele, Motivation}% in Zusammenhang mit dem Produkt}
\begin{itemize}[leftmargin=*]
\item Gut ins Englische übersetzte Websites
\item Online verfügbare, gut strukturierte Vorlesungsunterlagen
\item Beispielprogramme zur Orientierung beim erstellen von Belegen
\end{itemize}
\end{minipage}\hfill
%
\begin{minipage}[t][][b]{.45\textwidth}
\subsubsection*{Ähnliche Anwendungen}
HTWdd.org, HTW-Professorenseiten
\begin{itemize}[leftmargin=*,label={$-$}]
\item Meist schlecht strukturiert
%\item Fehlende Konsistenz bei Vorlesungen und Belegen
\item Nicht in Englisch Verfügbar
%\item Unregelmäßige Aktualisierungen
\item Teilweise exklusiv
\end{itemize}
\begin{itemize}[leftmargin=*,label={$+$}]
\item Große Auswahl
%\item Unabhängigere Beiträge
\item Teilweise aus erster Hand
\item Teilweise direkter Ansprechpartner
\end{itemize}
\textbf{Nutzung:}
\begin{itemize}[leftmargin=*]
\item Zur Informationsbeschaffung
\item Als Lernhilfe
\end{itemize}
\end{minipage}\bigskip

\subsubsection*{Kommentar}
Die Person findet die Idee der Webseite gut und hilfreich. Er freut sich besonders auf eine gute Übersetzung. Eine regelmäßige Aktualisierung der Vorlesungsunterlagen in einem strukturierten Umfeld würde er begrüßen.

\clearpage
\subsection{Persona Musterchef}
%\addcontentsline{toc}{subsection}{Persona Musterchef}
% ===============================
\begin{minipage}[t][][b]{.25\textwidth}
\includegraphics[width=\textwidth]{persona_pics/persona2-w.jpg}\end{minipage}\hfill
%
\begin{minipage}[t][][b]{.72\textwidth}
\noindent%
\begin{tabular}{L{.25} L{.7}}
\textbf{Name:} & Marta Musterchef\\
\textbf{Alter:} & 37 Jahre\\
\textbf{Beruf:} & (Personal-)Chef des Unternehmens \emph{Special Designs Inc}\\
\textbf{Merkmal:} & Auf das Wohl des Unternehmens bedacht\\
\textbf{Zitat:} & \\
\end{tabular}\\
\begin{quote}
„Für unser Unternehmen brauchen wir nur die besten Köpfe. Ein guter Kopf ist der, der sich auch gut im Internet zu präsentieren weiß.“
\end{quote}
\end{minipage}\vspace*{1em}

\begin{minipage}[t][][b]{.48\textwidth}
\subsubsection*{Alltagsroutinen}% und Handlungen}
\begin{itemize}[leftmargin=*]
\item Lesen von Bewerbungen
\item Gewöhnliches Privatleben
\item Einstellungsgespräche führen
\item Besuchen von Jobmessen
\item Ermittlung von Hintergrundinformationen über die Bewerber
\end{itemize}
\end{minipage}\hfill
%
\begin{minipage}[t]{.48\textwidth}
\subsubsection*{Probleme und Bedürfnisse}% in Zusammenhang mit dem Produkt}
\begin{itemize}[leftmargin=*]
\item Verspielte und unübersichtliche Bewerber-Websites
\item Weit verstreute Informationen
\item Wenige Informationen über Leben, Projekte und Fähigkeiten
\end{itemize}
\end{minipage}\bigskip

\begin{minipage}[t][][b]{.45\textwidth}
\subsubsection*{Wünsche, Ziele, Motivation}% in Zusammenhang mit dem Produkt}
\begin{itemize}[leftmargin=*]
\item Zusammengetragene/zentralisierte Informationen
\item Gut strukturierte Bewerber-Websites
\item Fähigkeitsnachweise in Form von z.B. Projekten oder Programmen
\item Kontaktmöglichkeiten
\end{itemize}
\end{minipage}\hfill
%
\begin{minipage}[t][][b]{.45\textwidth}
\subsubsection*{Ähnliche Anwendungen}
XING, LinkedIn
\begin{itemize}[leftmargin=*,label={$-$}]
\item Nicht alle Personen sind dort angemeldet
\item Bezahldienst für bestimmte Funktionen
\item Zu viele verschiedene Anbieter
\item Accountbindung bei jedem Anbieter
\item Wahrheitswert
\item Informationen sind nur so gut wie sie gepflegt werden
\end{itemize}
\begin{itemize}[leftmargin=*,label={$+$}]
\item Standardmäßig strukturiert
\item Suche nach bestimmten Fähigkeiten
\item Alles auf einen Blick
\end{itemize}
\end{minipage}\bigskip

\subsubsection*{Kommentar}
Die Person begrüßt gute persönliche Websites der Bewerber die alle nötigen Informationen enthalten. Eine eigene Webseite weist zum Beispiel bereits gewisse Fähigkeiten auf. Jedoch ist es auch nicht notwendig, da es genug Alternativen in Form von Jobbörsen gibt.

\clearpage
\subsection{Persona Mustermensch}
%\addcontentsline{toc}{subsection}{Persona Mustermensch}
% ===============================
\begin{minipage}[t][][b]{.25\textwidth}
\includegraphics[width=\textwidth]{persona_pics/persona3.jpg}\end{minipage}\hfill
%
\begin{minipage}[t][][b]{.72\textwidth}
\noindent%
\begin{tabular}{L{.25} L{.7}}
\textbf{Name:} & Maria Mustergeist\\
\textbf{Alter:} & 42 Jahre\\
\textbf{Beruf:} & Fachangestellte im Bereich Spektrologie\\
\textbf{Merkmal:} & Möchte ein Wichteln organisieren\\
\textbf{Zitat:} & \\
\end{tabular}\\
\begin{quote}
„Abseits von meinem Beruf veranstalte ich gerne kleine Parties, bei denen sich die Gäste gegenseitig beschenken dürfen: Für den Betrieb, meine Kinder und auch Freunde.“
\end{quote}
\end{minipage}\vspace*{1em}

\begin{minipage}[t][][b]{.48\textwidth}
\subsubsection*{Alltagsroutinen}% und Handlungen}
\begin{itemize}[leftmargin=*]
\item Gewöhnliches Arbeits-/Schul- und Privatleben
\item Organisieren von kleinen Schul-/ Betriebsfeiern
\end{itemize}
\end{minipage}\hfill
%
\begin{minipage}[t]{.48\textwidth}
\subsubsection*{Probleme und Bedürfnisse}% in Zusammenhang mit dem Produkt}
\begin{itemize}[leftmargin=*]
\item Manuelle Eingabe der E-Mail Adressen
\item E-Mail Adressen werden benötigt
\item Anmelden des Wichtel-Organisators
\item System gebunden (Windows)
\end{itemize}
\end{minipage}\bigskip

\begin{minipage}[t][][b]{.45\textwidth}
\subsubsection*{Wünsche, Ziele, Motivation}% in Zusammenhang mit dem Produkt}
\begin{itemize}[leftmargin=*]
\item Import von E-Mail Adressen per z.B. Datei
\item Alternativen zu E-Mail Adressen z.B. Nicknames
\item Webanwendung (System unabhängig)
\item Exportfunktion für Wichtel-Zuweisungen
\end{itemize}
\end{minipage}\hfill
%
\begin{minipage}[t][][b]{.45\textwidth}
\subsubsection*{Ähnliche Anwendungen}
drawnames.de wichtel-o-mat.de
\begin{itemize}[leftmargin=*,label={$-$}]
\item Anmelden erforderlich
\item Eigenständiges Programm (benötigt Installationsrechte)
\item Keine Exportfunktion für Zuweisungen der Wichtel
\item Kann in kleinem Kreis unnötig großer Aufwand sein
\item Benötigen E-Mail Adressen
\end{itemize}
\begin{itemize}[leftmargin=*,label={$+$}]
\item Kontakte können Importiert werden (Gmail/Outlook)
\item Größere Wichtel-Aktionen mit vielen Teilnehmern sind möglich
\end{itemize}
\end{minipage}\bigskip

\subsubsection*{Kommentar}
Wichtel-Anwendungen können einem vor allem in größerem Maßstab sehr viel Arbeit abnehmen. Jedoch ist es häufig nicht möglich mit Namen zu wichteln, was besonders bei Schulklassen ein Problem darstellt.
