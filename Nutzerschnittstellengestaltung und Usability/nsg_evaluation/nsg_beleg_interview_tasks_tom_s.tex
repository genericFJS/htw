\subsection*{Begrüßung}
[...]

Wenn Du dem zustimmst, unterschreibe bitte hier:

\vspace*{1em}
Tom Schmidtgen [14.01.2018]\vspace*{-.9em}\\
\noindent\rule{8cm}{0.4pt}

\subsection*{Einführung}
[...]


\subsection*{Test-Aufgaben}
\subsubsection*{Alle Seiten besuchen}
\begin{enumerate}
\item Wie lautet die Email-Adresse des Admins der Website? Wie ist der Name des Seiteninhabers?
\interviewText{Okay, als erstes würde ich vielleicht nach einem Impressum suchen. [Sucht über die Suche nach 'Impressum' $\to$ Keine Ergebnisse.] Das war nichts. Ich scrolle nun nach unten sehe da ein Über. [Klickt drauf] Der Inhaber scheint dieser Falk zu sein und seine Mail ist die admin@fj-strube.de .}
\item Finde eine Seite, die den Text „schwarz-lila“ enthält. Wie ist der Titel dieser Seite?
\interviewText{Als erstes suche ich einfach mal so über die Buttons. [Klickt auf die Buttons 'Kreativ' und 'Code'] Hier steht nichts... Ich nutze die Suche und gebe das mal ein. [Nutzt die Suche] Ah hier kommt ein Ergebnis und zwar Half-life 2 Mod. Was ist das? [Wir gehen weiter zu nächsten Frage.]}
\item Für welche Spiele werden auf der Seite Karten/Maps/Level angeboten?
\interviewText{Bevor ich mich hier wieder lange durchklicke suche ich das einfach. [Nutzt die Suche $\to$ Tippt ein 'Maps'] Hier kommt nichts. Ich probiere mal Karten. [Nutzt die Suche erneut $\to$ Tippt ein 'Karten'] Hier stehen jetzt zwei Seiten und zwar 'Counter Strike' und wieder dieses 'Half-life 2 Mod'.}
\item Finde den Sammelband mit drei Geschichten. Was assoziierst du mit dem Titelbild?
\interviewText{[Nutzt die Suche $\to$ Tippt ein 'Sammelband'] Hier kommt nichts. Ich nutze mal die Buttons. Ich gehe über die Kategorie 'Kreativ'. Hier steht was von Geschichten aus dem Jahr 2009. Ich denke das ist es. Okay, also in dem Bild sehe ich einen weiblichen Körper und/oder Wellen. }
\item Was für Software wird auf der Webseite vorgestellt?
\interviewText{[Nutzt die Suche $\to$ Tippt ein 'Software'] Ich nutze wieder die Suche. Ich lange hier auf einer Seite die 'Code' heißt. Ich denke es ist C++.}
\item Unter welchem Betriebssystem ist die Ausleihbibliothek in C++ ausführbar?
\interviewText{Davon habe ich doch keine Ahnung. [Moderator bittet den Probanden sich die Seite durchzulesen.] Ich finde dazu leider nichts. }
\item Welche Medien sind in der Ausleihbibliothek in C ausleihbar?
\interviewText{Da kann man Bücher, CDs und DVDs ausleihen.}
\item Wo sind Studiendokumente zu finden?
\interviewText{Ich bin jetzt auf der 'Home' Seite und finde nichts dazu. Ich nutze die Suche. [Nutzt die Suche $\to$ Tippt ein 'Studienmitschriften'] Habe es gefunden, war unter 'Code' versteckt. Das passt doch gar nicht. Könnte man doch einen eigene Kategorie machen...}
\item Stelle die Sprache auf Englisch um. Findest Du eine Seite, die nicht auf Englisch übersetzt ist? Wenn ja, wie ist der Titel dieser Seite?
\interviewText{[Der Proband klickt alle Seiten durch außer die Seite 'Automatisiertes Wichtel-System'] Ich habe keine Seite gefunden.}
\end{enumerate}


\subsubsection*{Automatisiertes Wichtel System}
\begin{enumerate}
\item Benutze das Automatisierte Wichtel System:
[...]
\interviewText{Bei dem auswählen, wer sich nicht beschenken darf, also das war direkt nach dem eintragen der Namen kommt, finde ich es schwer die Namen zu lesen, da sie so von oben nach unten geschrieben sind. Das viel mir schwer. \\ Die eine Seite, welche eine Zusammenfassung ist, war mir einfach zu viel. Ich finde es gut, dass man da seine eigenen Anpassungen oder Änderungen machen kann. Aber es war irgendwie gleich alles auf einmal. Könnte man vielleicht etwas anpassen.}
\end{enumerate}

\subsubsection*{Responsive Design}
\begin{enumerate}
\item Rufe die Website auf deinem Handy auf:
[...]
\interviewText{Also auf meinem iPhone wird der Button 'Neue Wichtel erstellen und verteilen' nicht richtig dargestellt. [Der Moderator bittet den Probanden einen Screenshot davon zu erstellen.] Zudem finde ich die Anordnung komisch. Am Anfang kann ich drei Namen eintragen, aber wenn ich mehr eintragen will, dann scrolle ich doch Ewigkeiten. Es sieh auch alles nicht so mittig aus. Ich konnte auch ohnen einen Wichtel eingetragen zu haben einfach auf 'Weiter' klicken. [Der Moderator bittet den Probanden einen Screenshot davon zu erstellen.] }
\end{enumerate}