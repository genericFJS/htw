\begin{frame}
\begin{itemize}
\item<1-> Auswertung
\item<2-> Fehleridentifizierung
\item<3-> Fehlerbewertung
\end{itemize}
\note{
Die Analyse besteht aus diesen Schritten.
}
\end{frame}

\subsection{Allgemeines Feedback}

\begin{frame}{Positives}
\begin{center}
\uncover<1->{Orientierung:\\
\interviewScalaResult{6.2}{schwer}{leicht}\\\bigskip}
\uncover<2->{Bedienbarkeit (Desktop und Mobil):\\
\interviewScalaResult{6.5}{schlecht}{gut}}
\end{center}
\note{
Im Interview wurden die Probanden um eine Einschätzung verschiedener Qualitäten der Webseite gebeten. Auf einer Skala von 1 bis 7 sollten diese bewertet werden.\bigskip

Trotz einiger Probleme, die später identifiziert wurden, wurden bspw. die Orientierung und Bedienbarkeit positiv wahrgenommen.\\
}
\end{frame}

\begin{frame}{Negatives}
\begin{center}
\uncover<1->{Gesamteindruck:\\
\interviewScalaResult{3.5}{schlecht}{gut}\\\bigskip}
\uncover<2->{Farbwahl:\\
\interviewScalaResult{2.8}{schlecht}{gut}}
\end{center}
\note{
Der Gesamteindruck und die Farbwahl wurden größtenteils negativ wahrgenommen. 

Im Kontrast dazu wurde die Präsentation im Bezug auf den Inhalt als angemessen bewertet:
\begin{center}
Präsentation:\\
\interviewScalaResult{6.2}{unangemessen}{angemessen}
\end{center}
Das bedeutet, dass trotz einer angemessenen Präsentation eine andere Farbkomposition zu wählen ist.\\
}
\end{frame}

\subsection{Fehleridentifizierung}
\begin{frame}
Auswertung:
\begin{itemize}
\item<2-> Interviews
\item<3-> Testprotokolle
\end{itemize}
\note{
Zur Identifizierung der Fehler mussten die Interviews und Testprotokolle ausgewertet werden.

Die identifizierten Fehler konnten dann bewertet werden.\\
}
\end{frame}

\subsection{Fehlerbewertung}
\begin{frame}{Aspekte zur Priorisierung}
\begin{itemize}
\item[$H$]<2-> Häufigkeit
\item[$A$]<3-> Hartnäckigkeit (Auffälligkeit)
\item[$F$]<4-> Folgen
\end{itemize}
\note{
Um die identifizierten Fehler bzw. Probleme priorisieren zu können, wurden sie anhand folgender Kenngrößen bewertet:
\begin{itemize}
\item[$H$] Häufigkeit:\\
Wie vielen Probanden sind diesem Problem begegnet (Anzahl)?
\item[$A$] Hartnäckigkeit (Auffälligkeit):\\
Wie oft ist diesen Probanden diesem Problem begegnet (Abschätzung/Wertung anhand von Protokollen)?
\item[$F$] Folgen:\\
Wie schwer sind die Folgen, wenn das Problem aufgetreten ist (Wertung anhand der Protokolle und persönlicher Einschätzung)?
\end{itemize}
}
\end{frame}

\begin{frame}{Priorisierung nach Nielsen}
\begin{center}
$P=\lceil H\cdot 30\%+A\cdot 20\%+F \cdot 50\%\rceil$
\end{center}
\begin{center}
\uncover<2->{
\begin{minipage}{.5\textwidth}
\begin{enumerate}
\item[\nielA] Geringes Problem
\item[\nielB] Relatives Problem
\item[\nielC] Ernsthaftes Problem
\item[\nielD] Usability Katastrophe
\end{enumerate}
}
\end{minipage}
\end{center}
\note{
Für die Priorisierung sind vor allem die Folgen ausschlaggebend. Häufigkeit und Hartnäckigkeit spielen eine untergeordnete Rolle.\\
}
\end{frame}

\begin{frame}{Allgemein}
\footnotesize
\begin{center}
\begin{tabular}{ L{.56} | C{.11} C{.11} C{.11} | C{.11}}
Beschreibung& $H$ & $A$ & $F$ & $P$\\\hline
\uncover<2->{Farben trist (geringe Sättigung)} & \uncover<2->{\bewE} & \uncover<2->{\bewDe} & \uncover<2->{\bewBc} & \uncover<2->{\nielC}\\
\uncover<3->{Farben eintönig (verschwimmen)} & \uncover<3->{\bewC} & \uncover<3->{\bewD} & \uncover<3->{\bewCd} & \uncover<3->{\nielC}\\
\uncover<4->{Menüleiste zu groß} & \uncover<4->{\bewA} & \uncover<4->{\bewDe} & \uncover<4->{\bewDe} & \uncover<4->{\nielC}\\
\end{tabular}
\note{
Aufgeführt sind ernsthafte Probleme, auf die im Redesign eingegangen wurde.

Auffällig ist, dass die Farben fast jedem Probanden auf die eine oder andere Weise negativ aufgefallen ist.

Obwohl das kantige und schlichte Design nicht schlecht angekommen ist, wurde vor allem die Menüleiste kritisiert, die besonders in der mobilen Ansicht zu viel Platz auf dem Bildschirm einnimmt.\\
}
\end{center}
\end{frame}

\begin{frame}{Automatisierte Wichtel System}
\footnotesize
\begin{center}
\begin{tabular}{ L{.56} | C{.11} C{.11} C{.11} | C{.11}}
Beschreibung& $H$ & $A$ & $F$ & $P$\\\hline
\uncover<2->{Zuweisung: Tabelle nicht intuitiv benutzbar} & \uncover<2->{\bewD} & \uncover<2->{\bewDe} & \uncover<2->{\bewEf} & \uncover<2->{\nielD}\\
\uncover<3->{Wichtel-Eingabe: Anordnung der Eingabeflächen nicht eindeutig (mobil)} & \uncover<3->{\bewB} & \uncover<3->{\bewCd} & \uncover<3->{\bewE} & \uncover<3->{\nielC}\\
\uncover<4->{Mail-Anpassung: Textfelder zu klein (mobil)} & \uncover<4->{\bewC} & \uncover<4->{\bewD} & \uncover<4->{\bewBc} & \uncover<4->{\nielC}\\
\end{tabular}
\end{center}
\note{
Das AWS hatte am meisten spezifische Probleme, die teils sogar als Usability Katastrophe gewertet wurden.

Wenige oder unpräzise Erklärungen konnten dazu führen, dass der Nutzer das Programm gegebenenfalls falsch bedient hat.

Weiterhin kann man den AWS-Seiten ansehen, dass sie nicht für mobile Geräte konzipiert wurde.\\
}
\end{frame}

\subsection{Fazit}
\begin{frame}
\begin{center}
\uncover<2->{Nicht nur Kleinigkeiten zu korrigieren\\}
\uncover<3->{$\Rightarrow$ Umfangreiches Redesign nötig\\\bigskip}
\uncover<4->{Fehlerauswahl zum Redesign:\\nur Design-Fehler}
\end{center}
\note{
Da die wichtigsten identifizierten Fehler und Probleme nicht nur Kleinigkeiten sind, ist ein umfangreiches Redesign nötig. 

Das bedeutet, dass für die ausgewählten Fehler Vorschläge für die Veränderung erstellt wurden, die aufgrund des zeitlich eingeschränkten Rahmens teils nicht zur Perfektion ausgearbeitet werden konnten.

Dies steht im Kontrast zu einer Evaluierung, die eine bereits erprobte und von Fachkräften designte Webseite (wie beispielsweise Opal oder die HTW-Seite) bewertet: Dabei müssen potentiell nur minimale Änderungen vorgenommen werden -- wenn bspw. die Anordnung leicht ungünstig muss nur ein Element verschoben werden. 

Bei diesem Redesign sind deutlich komplexere strukturelle Änderungen vorzunehmen.

Fehler technischer Natur wurden nicht ausgewählt.\\
}
\end{frame}
