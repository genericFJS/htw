\subsection*{Begrüßung}
[...]

Wenn Du dem zustimmst, unterschreibe bitte hier:

\vspace*{1em}
Teresa Schönherr [14.01.2018]\vspace*{-.9em}\\
\noindent\rule{8cm}{0.4pt}

\subsection*{Einführung}
[...]


\subsection*{Test-Aufgaben}
\subsubsection*{Alle Seiten besuchen}
\begin{enumerate}
\item Wie lautet die Email-Adresse des Admins der Website? Wie ist der Name des Seiteninhabers?
\interviewText{Probandin schaut sich die Seite an. Entdeckt den Link „Über“. Glaubt dort müsste etwas über den Inhaber stehen.}
\item Finde eine Seite, die den Text „schwarz-lila“ enthält. Wie ist der Titel dieser Seite?
\interviewText{Da Probandin nicht weiß wo sie als erstes suchen sollte, verwendet sie die „Suche“. sie gibt „schwarz-lila“ ein und kommt dann schnell zu dem richtigen Ergebnis. }
\item Für welche Spiele werden auf der Seite Karten/Maps/Level angeboten?
\interviewText{Testperson klick auf das „FJS-Logo“ um zur Startseite zu gelangen. Als nächstes klickt sie auf „Kreativ“ weil sie vermutet dass dieser Punkt am ehesten mit Karten/Maps/Level zu tun haben könnte. }
\item Finde den Sammelband mit drei Geschichten. Was assoziierst du mit dem Titelbild?
\interviewText{Probandin befindet sich noch auf der Seite „Kreativ“. Da sie den Titel „Ohne Namen“ und den darunterliegenden Text beim Suchen der Maps schon gelesen hat, weiß sie wo sie hin muss.}
\item Was für Software wird auf der Webseite vorgestellt?
\interviewText{Die Testperson weiß nicht wo sie anfangen soll zu Suchen. Deshalb verwendet sie die „Suche“ und gibt „Software“ ein. Die Seite gibt den Link zu „Code“ zurück. Die Probandin hat Schwierigkeiten „Ausleihbibliothek“ mit einer Software zu Assoziierenn.}
\item Unter welchem Betriebssystem ist die Ausleihbibliothek in C++ ausführbar?
\interviewText{Die Testperson liest den Artikel aufmerksam und findet die Lösung. }
\item Welche Medien sind in der Ausleihbibliothek in C ausleihbar?
\interviewText{Da sie den Text über die Ausleihbibliothek vorher gelesen hatte, weiß die Probandin dass sie zurück muss. Dafür verwendet sie den „Zurück“-Button des Browser nicht der Seite.}
\item Wo sind Studiendokumente zu finden?
\interviewText{Testperson weiß nicht wo sie nach „Studiendokumenten“ suchen soll. sie geht einfach mal zurück. Dort steht was von „Studienmitschriften“. „Das müsste es wohl sein.“}
\item Stelle die Sprache auf Englisch um. Findest Du eine Seite, die nicht auf Englisch übersetzt ist? Wenn ja, wie ist der Titel dieser Seite?
\interviewText{Da die Probandin schon mehrere Male die Suche verwendet hatte, findet sie den Link „Deutsch/English“ schnell. Um eine unübersetzte Seite zu finden, klickt sie Wahllos durch die Seiten, bis sie etwas gefunden hat. Dabei fällt auf, dass auf der Seite „Ohne Namen“ der Titel nicht in Englisch übersetzt wurde.}
\end{enumerate}


\subsubsection*{Automatisiertes Wichtel System}
\begin{enumerate}
\item Benutze das Automatisierte Wichtel System:
[...]
\interviewText{Die Probandin gibt die vorgegeben Namen ein. Für sie ist unklar wie Personen sich gegenseitig zugewiesen werden. sie klickt einfach mal auf die Kästchen. Es funktioniert.}
\end{enumerate}

\subsubsection*{Responsive Design}
\begin{enumerate}
\item Rufe die Website auf deinem Handy auf:
[...]
\interviewText{Hier gibt es grundsätzlich keine Probleme. Die Bedienung am Smartphone fällt der Testperson leicht, da sie es vorher schon am PC durchgeführt hat.}
\end{enumerate}