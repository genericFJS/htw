\section{Begrüßung}
Hallo.

Danke, dass Du hier bist, um mit uns die Website \url{fj-strube.de} zu evaluieren.

Zur besseren Auswertung Deines Beitrags zur Evaluation wird ein Video- und Tonmitschnitt durchgeführt. Der Bildausschnitt ist so gewählt, dass dein Gesicht weitestgehend nicht erfasst wird. Gegebenenfalls werden einzelne anonymisierte Standbilder in der Dokumentation dieser Evaluation veröffentlicht. Ansonsten wird der Video- und Tonmitschnitt ausschließlich zur Auswertung benutzt, nicht veröffentlicht und nicht an dritte weiter gegeben.

Wenn Du dem zustimmst, unterschreibe bitte hier:

\vspace*{3em}
\noindent\rule{8cm}{0.4pt}

\section{Einführung}
Die Webseite soll auf einem PC und auch kurz auf dem Handy getestet werden. Achte darauf, dass Du das Handy so bedienst, dass es von der Kamera erfasst wird.

Die Aufgaben dauern ca. 30 Minuten und sollen ohne weitere Kommentare von unserer Seite erledigt werden. Versuche beim Abarbeiten der Aufgaben laut zu denken: Für uns ist nicht nur wichtig zu erfahren was Du tust, sondern auch warum Du es tust. Lass Dir beim Erledigen der Aufgaben ruhig Zeit, Du darfst auch, wenn dich etwas interessiert, kurz abseits der Aufgaben weiter schmökern.

Falls Du eine Aufgabe nicht verstehst, kannst Du uns zur Not auch kurz fragen.

\clearpage
\section{Test-Aufgaben}
\subsection{Alle Seiten besuchen}
\begin{enumerate}
\item Wie lautet die Email-Adresse des Admins der Website? Wie ist der Name des Seiteninhabers?
\interviewText{\interviewTextS}
\item Finde eine Seite, die den Text „schwarz-lila“ enthält. Wie ist der Titel dieser Seite?
\interviewText{\interviewTextS}
\item Für welche Spiele werden auf der Seite Karten/Maps/Level angeboten?
\interviewText{\interviewTextS}
\item Finde den Sammelband mit drei Geschichten. Was assoziierst du mit dem Titelbild?
\interviewText{\interviewTextS}
\item Was für Software wird auf der Webseite vorgestellt?
\interviewText{\interviewTextS}
\item Unter welchem Betriebssystem ist die Ausleihbibliothek in C++ ausführbar?
\interviewText{\interviewTextS}
\item Welche Medien sind in der Ausleihbibliothek in C ausleihbar?
\interviewText{\interviewTextS}
\item Wo sind Studiendokumente zu finden?
\interviewText{\interviewTextS}
\item Stelle die Sprache auf Englisch um. Findest Du eine Seite, die nicht auf Englisch übersetzt ist? Wenn ja, wie ist der Titel dieser Seite?
\interviewText{\interviewTextS}
\end{enumerate}

\clearpage
\subsection{Automatisiertes Wichtel System}
\begin{enumerate}
\item Benutze das Automatisierte Wichtel System:

Stell Dir vor, Du möchtest für Deine nächste Feier Wichtel zuweisen, damit jeder jemandem ein Geschenk mit bringt. Du lädst 6 Freunde ein: Brunhilde, Hans, Günter, Mechthild, Jacqueline und Kevin. Jacqueline und Kevin haben sich kürzlich unglücklich getrennt. Die beiden sollen sich auf keinen Fall gegenseitig beschenken!

Wenn Email-Adressen benötigt sind, verwende immer diese: nsg-aws@web.de. In einem separatem Fenster/Tab ist für Dich die Webmailoberfläche dieser Mailadresse geöffnet, falls Du sie benötigst.
%[Passwort: AutWicSys

Lass Dir Zeit und lies dir die Anweisungen auf der Webseite aufmerksam durch.

Achte darauf, ob Dir bei der Bedienung etwas einfach oder schwer fällt oder Dir etwas nicht klar ist. Du wirst später noch mal gefragt, was Dir (nicht) gefallen hat, hier kannst du Dir gegebenenfalls Notizen machen:
\interviewText{\interviewTextL}
\end{enumerate}

\subsection{Responsive Design}
\begin{enumerate}
\item Rufe die Website auf deinem Handy auf:
\begin{itemize}
\item Besuche eine Ausleihbibliothek-Seite. 
\item Führe die ersten Schritte des Automatisierten Wichtel Systems durch: Gib die Namen, E-Mail-Adressen ein und stelle ein, wer wen bewichteln darf.
\end{itemize}
Fällt Dir etwas im Vergleich zur Desktop-Version auf? Ist etwas besser oder schlechter? Du wirst später noch mal gefragt, was Dir (nicht) gefallen hat, hier kannst du Dir gegebenenfalls Notizen machen:
\interviewText{\interviewTextL}
\end{enumerate}