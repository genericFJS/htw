\subsection*{Interview-Fragen}
\subsubsection*{Zur Person}
\begin{enumerate}
\item Was ist dein Name?
\interviewText{Tobias R}
\item Wie alt bist du?
\interviewText{24}
\item Was arbeitest/studierst du?
\interviewText{Informatik an der HTW Dresden}
\item Wie oft bzw. wie lange browst du Webseiten im Internet?
\interviewText{4 Stunden am Tag}
\item Auf welchen Seiten bist du regelmäßig unterwegs?
\interviewText{
\begin{itemize}
\item Pornoseiten
\item Social Media
\item pr0gramm
\end{itemize}
}
\item Wie schätzt Du Deine Kenntnisse im Umgang mit Webseiten ein?
\interviewText{Hoch.}
\item Wofür nutzen Du das Internet (außer zum Browsen)?
\interviewText{Computerspiele, Arbeiten.}
\item War Dir diese Webseite im vornherein bekannt? Wenn ja: Woher?
\interviewText{Nein.}
\end{enumerate}

\subsubsection*{Allgemeines}
\begin{enumerate}
\item Wie leicht fiel es Dir dich auf der Seite zu orientieren?
%               -        ~        +  von bis
\interviewScala{ }{ }{ }{ }{ }{ }{7}{schwer}{leicht}
Was viel dir besonders schwer/leicht:
\interviewText{
\begin{itemize}
\item Suche, weil das Suchfeld nicht als allgemeine Suche identifiziert wurde.
\item Statt Über wurde Impressum erwartet.
\end{itemize}}
\item Gab es Probleme beim Bedienen der Website?
%               -        ~        +  von bis
\interviewScala{ }{ }{ }{ }{ }{6}{ }{viele}{wenige}
Anmerkungen:
\interviewText{
\begin{itemize}
\item Beim Zurück beim AWS hat es nicht richtig funktioniert.
\item Als beim AWS keiner erlaubt wurde, wurden viele Fehlermeldungen angmeldet.
\end{itemize}
}
\item Konntest Du die Website auch gut auf dem Handy bedienen?
%               -        ~        +  von bis
\interviewScala{ }{ }{ }{ }{ }{ }{7}{schwer}{leicht}
Anmerkungen:
\interviewText{Button bei AWS}
\item Würdest Du etwas an der Webseite verändern?
%               -        ~        +  von bis
\interviewScala{ }{ }{ }{ }{5}{ }{ }{viel}{wenig}
Wenn ja, was?
\interviewText{Responsive Design: Formatierung der Texte und Felder (Beispielmail, Eintragung der Wichtel, …)\\
Desktopansicht: Bei Codeeingabe und Enter drücken beim AWS ist die nächste Seite, ohne weiteres prüfen alle Mails abgeschlossen. Vielleicht trennen von Codeeingabe und Überprüfen der Verteilung/Mail.\\
Die Breadcrumbs ist das Home unerwartet. Dadurch ist die Tiefe maximal zwei. Damit überflüssig, nimmt nur Platz weg. Das wäre zu ändern.
}
\item Wie würdest du die Seite beschreiben? Was ist der Zweck der Seite?
\interviewText{Sammlung eines Studenten. Angebot der Maps, der Programme und der Mitschriften.}
\item Kennst Du Webseiten, die mit dieser vergleichbar sind? Wenn ja: Wie lauten diese? In wie fern sind sie besser/schlechter/ähnlich gestaltet?
\interviewText{technik-rumpelkammer.de -- Ähnlich: Startseite mit Kategorien: Codebeispiele usw. Neue Version sieht etwas besser aus.}
\end{enumerate}

\subsubsection*{Zum Design}
\begin{enumerate}
\item Wie bewertest du das Design der Seite?
\begin{enumerate}
\item Der Gesamteindruck war…
%               -        ~        +  von bis
\interviewScala{ }{ }{ }{ }{5}{ }{ }{schlecht}{gut}
Anmerkungen:
\interviewText{Schlicht. Nicht auffallend. Sehr homogen.}
\item Die Farbwahl und -gestaltung war…
%               -        ~        +  von bis
\interviewScala{ }{ }{ }{ }{ }{ }{7}{schlecht}{gut}
Anmerkungen:
\interviewText{Erwartet Funktionalität, da kein Produkt beworben wurde. Dem entsprechend passt Design und Farbwahl: Pastellfarben/Ausgeblichen.}
\item Die Anordnung der Inhalte war…
%               -        ~        +  von bis
\interviewScala{ }{ }{ }{4M}{ }{6D}{ }{schlecht}{gut}
Anmerkungen:
\interviewText{Desktop besser als Mobil.\\
Allerdings sowohl in Mobil wie Desktop zu große Navbar.\\
Im Desktop wäre was aufklappbares erwartet. Vor allem die Suche ist zu groß.\\
Die aktuelle Seite ist im Menüband nicht eindeutig zu erkennen eingefärbt.
}
\item Die Präsentation der Webseite war im Bezug auf den Inhalt…
%               -        ~        +  von bis
\interviewScala{ }{ }{ }{ }{ }{ }{7}{unangemessen}{angemessen}
Anmerkungen:
\interviewText{Farben und Anordnung passend.}
\item Die Lesbarkeit der Webseite war…
%               -        ~        +  von bis
\interviewScala{ }{ }{ }{ }{ }{6}{ }{schlecht}{gut}
Anmerkungen:
\interviewText{Menüband zu groß (bei Mobilansicht).}
\end{enumerate}
\end{enumerate}

\subsubsection*{Automatisches Wichtel System}
\begin{enumerate}
\item Sind Dir bei der Bedienung des Automatisierten Wichtel Systems Besonderheiten aufgefallen, die Du noch nicht erwähnt hast?
\interviewText{--}
\end{enumerate}

\subsubsection*{Abschließende Fragen}
\begin{enumerate}
\item Würdest du die Website weiter empfehlen? Warum? Warum nicht?
\interviewText{Keiner bekannt, den Inhalt interessiert. Ja, vielleicht wenn AWS relevant (November), bis dahin aber wieder vergessen.}
\item Würdest du die Funktionen der Seite erneut nutzen?
\interviewText{Maps mal runterladen. Das Wichtelsystem\\

Wichtelsystem zu dieser Jahreszeit uninteressant, vielleicht im November wieder. Das Ausschließen ist zwar ganz nett, aber nicht in seinen Kreisen anwendbar. Als Offlineversion vielleicht interessanter.

Wichtelsystem wird als Teil der Programmsammlung und nicht als besonders und funktionierend erkannt.}
\end{enumerate}
