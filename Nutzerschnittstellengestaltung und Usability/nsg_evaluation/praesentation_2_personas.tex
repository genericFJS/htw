% * Unsere Personas kurz vorstellen?
% * Warum haben wir die so gewählt?
% * Max. 3 Min.

\subsection{Überblick}
\begin{frame}
	\begin{itemize}
%		\item Ziel Evaluation von \url{fj-strube.de}
		\item spekulative/empirische Personae zur Definition Zielgruppen \pause
		\item[$\rightarrow$] 3 Personae für 3 verschiedene Zielgruppen \pause
		\begin{enumerate}
			\item klassischer Informatikstudent \pause
			\item Personalchef eines Unternehmens \pause
			\item Nutzer des Automatische-Wichtel-System
		\end{enumerate}
	\end{itemize}
\end{frame}

\subsection{Persona Musterstudent}
\begin{frame}
\centering
	\begin{minipage}[t][][b]{.25\textwidth}
		\includegraphics[width=\textwidth]{persona_pics/persona1.jpg}
	\end{minipage}
	%
	\begin{minipage}[t][][b]{.7\textwidth}
		\noindent%
		\begin{tabular}{L{.3} L{.65}}
			\textbf{Name:} & Max Musterstudent\\
			\textbf{Alter:} & 22 Jahre\\
			\textbf{Beruf:} & Student (Fakultät Informatik/Mathematik)\\
		\end{tabular}
	\end{minipage}\vspace*{1em}
	\note{\setstretch{1}
		\textbf{Merkmal:} Englischsprachig\\
		\textbf{Wünsche, Ziele, Motivation}\\
		\begin{itemize}
%			\item Gut ins Englische übersetzte Webseiten
			\item onlineverfügbare, gut strukturierte Vorlesungsunterlagen und Belegaufgaben
		\end{itemize}
		\textbf{Probleme und Bedürfnisse}\\
		\begin{itemize}
			\item schlecht übersetzte Webseiten die Studienmaterialien anbieten
			\item Probleme beim lösen von Belegaufgaben 
		\end{itemize}
		\textbf{Nutzung}\\
		\begin{itemize}
			\item Informationsbeschaffung/Lernhilfe
		\end{itemize}
		\textbf{Kommentar}\\
		\begin{itemize}
			\item[+] findet Idee der Webseite hilfreich
			\item[+] gute Übersetzung ist wünschenswert
			\item[+] strukturierte aktuelle Vorlesungsunterlagen sind wünschenswert
		\end{itemize}
	}
\end{frame}


\subsection{Persona Musterchef}
\begin{frame}
	\begin{minipage}[t][][b]{.25\textwidth}
		\includegraphics[width=\textwidth]{persona_pics/persona2-w.jpg}
	\end{minipage}
	%
	\begin{minipage}[t][][b]{.7\textwidth}
		\noindent%
		\begin{tabular}{L{.3} L{.65}}
			\textbf{Name:} & Marta Musterchef\\
			\textbf{Alter:} & 37 Jahre\\
			\textbf{Beruf:} & Personalchef\\
		\end{tabular}
	\end{minipage}\vspace*{1em}
	\note{\setstretch{1}
		\textbf{Merkmal:}  Auf das Wohl des Unternehmens bedacht\\	
		\textbf{Wünsche, Ziele, Motivation}\\
		\begin{itemize}
			\item Zentralisierte Informationen mit Kontaktmöglichkeiten
			\item Gut Strukturierte Bewerber-Webseiten
			\item Fähigkeitsnachweise in Form von z.B. Projekten oder Programmen
		\end{itemize}
		\textbf{Probleme und Bedürfnisse}\\
		\begin{itemize}
			\item Verspielte unübersichtliche Bewerberseiten
			\item Weit verstreute und zu wenige Informationen über Bewerber
		\end{itemize}
		\textbf{Nutzung}\\
		\begin{itemize}
			\item Informationen über den Bewerber/Person erhalten
		\end{itemize}
		\textbf{Kommentar}\\
		\begin{itemize}
			\item[+] gute persönliche Webseiten mit allen nötigen Informationen
			\item[+] Webseiten die Können zeigen
			\item[-] es gibt genug Alternativen in Form von Jobbörsen(z.B. XING)
		\end{itemize}
	}
\end{frame}

\subsection{Persona Mustermensch}
\begin{frame}
	\begin{minipage}[t][][b]{.25\textwidth}
		\includegraphics[width=\textwidth]{persona_pics/persona3.jpg}
	\end{minipage}\hfill
	%
	\begin{minipage}[t][][b]{.7\textwidth}
		\noindent%
		\begin{tabular}{L{.3} L{.65}}
			\textbf{Name:} & Maria Mustergeist\\
			\textbf{Alter:} & 42 Jahre\\
			\textbf{Beruf:} & Fachangestellte im Bereich Spektrologie\\
		\end{tabular}
	\end{minipage}\vspace*{1em}
	\note{\setstretch{1}
		\textbf{Merkmal:} Möchte ein Wichteln organisieren\\
		\textbf{Wünsche, Ziele, Motivation}\\
		\begin{itemize}
%			\item gutes Wichtel-System
			\item import von E-Mail Adressen, z.B. Datein
			\item Alternativen zu E-Mail Adressen
			\item Webanwendungen
			\item Exportfunktion für Wichtel-Zuweisungen
		\end{itemize}
		\textbf{Probleme und Bedürfnisse}\\
		\begin{itemize}
			\item Manuelle Eingabe und Benötigen von E-Mail Adressen
			\item Anmelden des Wichtel-Organisators oder System(Windows) gebunden
		\end{itemize}		
		\textbf{Nutzung}\\
		\begin{itemize}
			\item Organisation einer Wichtel-Runde
		\end{itemize}
		\textbf{Kommentar}\\
		\begin{itemize}
			\item[+] mögliche Arbeitsersparnis
			\item[-] häufig keine Möglichkeit Namen zu wichteln
		\end{itemize}
	}
\end{frame}