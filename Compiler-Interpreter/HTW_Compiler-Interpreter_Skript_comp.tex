\lecdate{13.12.2017}

\section{Anzulegende Speicherbereiche}
\begin{itemize}
\item Morphem
\item ProcMain (Prozedurbeschreibung der Main Funktion)
\item Speicherbereich für Konstanten
\item Speicherbereich für Zwischencode
\item Ausgabedatei
\end{itemize}

\section{Abarbeitung}
\begin{itemize}
\item Lexer erkennt Morphem
\item Parser startet mit Startbogen
\item Parser ruft Blockbogen auf (rekursiver Aufruf von parse)
\item Vor dem erstellen einer Konstanten/Variablen: Prüfen, ob es schon einen mit gleichen Namen gibt.
\item Einfügen einer Konstante: erst prüfen, ob Wert bereits vorhanden. Konstanten mit gleichem Wert zeigen auf gleichen Speicherbereich
\item Zuweisung einer Variablen: pusht Konstante. D.h. Konstantenspeicherbereich wird ggf. erweitert, wenn Wert noch nicht drin (als nicht benannter Eintrag)
\end{itemize}





