\section{Grundlagen}
\lecdate{17.01.2018}
\slides{V11_Yacc}{1}
\subsection{Aufbau einer yacc-Datei}
\slides{V11_Yacc}{2}
\subsection{Zusammenspiel lex -- yacc}
\slides{V11_Yacc}{3}

\section{yacc-Teile}
\subsection{Definitionsteil}
\slides{V11_Yacc}{4}
\subsection{Regelteil}
\slides{V11_Yacc}{5}

\subsection{Beispiel 1}
\begin{lstlisting}
%{
	#include <stdio.h>
	int yylex(void);
	void yyerror(char *);
%}
%token INTEGER
%%
program:	program expr '\n'		{ printf("%d\n", $2); }
					|
					;
expr:			INTEGER
					| expr '+' expr			{ $$ = $1 + $3; }
					| expr '-' expr			{ $$ = $1 - $3; }
					;
%%
void yyerror(char *s) {
	fprintf(stderr, "%s\n", s);
}
int main(void) {
	yyparse();
	return 0;
}
\end{lstlisting}
Mit \lstinline`yacc -d xyz.y` generiert sich \lstinline`y.tab.h` (gekürzt):
\slides{V11_Yacc}{7}
Mit der dazugehörigen \lstinline`lex`-Datei:
\slides{V11_Yacc}{8}

\subsection{Beispiel 2: expr, term, factor}
\begin{lstlisting}
%{
	#include <stdio.h> /* Printf */
	#include <stdlib.h>/* exit */
	#define YYSTYPE int
	int yyparse(void);
	int yylex(void);
	void yyerror(char *mes);
%}
%token number
%token QUIT 254
%%
 // Kommentar eins eingerückt
program :	program expr '\n'		{printf("= %d\n", $2);}
					|
					;
expr : 		expr '+' term				{puts("expr 1");$$ = $1 + $3;}
					| expr '-' term			{puts("expr 2");$$ = $1 - $3;}
					| term							{puts("expr 3");$$ = $1;}
					| QUIT							{exit(0);}
					;
term : 		term '*' fact				{puts("term 1");$$ = $1 * $3;}
					| term '/' fact			{puts("term 2");$$ = $1 / $3;}
					| fact							{puts("term 3");$$ = $1;}
					;
fact : 		number							{puts("fact 1");$$ = $1;}
					| '-'	number				{puts("fact 1");$$ = -$2;}
					| '(' expr ')'			{puts("fact 2");$$ = $2;}
					;
%%
int main() {
	printf("Enter expression with + - * / ( ) \n");
	yyparse(); return 0;
}
void yyerror(char *mes) {fprintf(stderr,"%s\n", mes);}
\end{lstlisting}
Mit dem Lexer:
\begin{lstlisting}
%{
	#include "t2.h" /* eigentlich y.tab.h !! */
	#include <stdlib.h> /* yacc -d -o t2.c erzeugt*/
	void yyerror(char *); /* auch t2.h */
%}

%%
[09]+				{ yylval = atoi(yytext);
							return number;
						}
q(uit)? 		return QUIT;
[-+()=/*\n]	{ return *yytext; }
[ \t] 			; /* skip whitespace */
.						yyerror("Unknown character");
%%
int yywrap(void) {
	return 1;
}
\end{lstlisting}

\section{Präzendenzregeln}
\slides{V11_Yacc}{12}
\subsection{Ausnahme}
\slides{V11_Yacc}{13}
\subsection{Beispiel}
\slides{V11_Yacc}{14}
\slides{V11_Yacc}{15}
\slides{V11_Yacc}{16}
Damit ergibt $12+3*4=24$, aber $12+3*-4=0$, da $12+(3*-4)=0$.

\section{pl/0 mit yacc und lex}
\slides{V11_Yacc}{17}
\subsection{pl0.y}
\subsubsection{Vereinbarungsteil}
\slides{V11_Yacc}{18}
\subsubsection{Funktionenteil}
\slides{V11_Yacc}{19}
\subsubsection{Regelteil}
\slides{V11_Yacc}{20}
\slides{V11_Yacc}{21}
\slides{V11_Yacc}{22}
\slides{V11_Yacc}{23}
\subsection{Lexer}
\slides{V11_Yacc}{26}
\subsubsection{Funktionenteil}
\slides{V11_Yacc}{27}
\subsection{Regelteil}
\slides{V11_Yacc}{28}
\slides{V11_Yacc}{29}
\slides{V11_Yacc}{30}
\slides{V11_Yacc}{31}
\subsection{Die Funktionen}
\slides{V11_Yacc}{32}
\slides{V11_Yacc}{33}



