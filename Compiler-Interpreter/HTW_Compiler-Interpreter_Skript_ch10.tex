\section{Grundlagen}
\lecdate{10.01.2018}
\slides{V10_lex}{1}
\subsection{Verwendung}
\slides{V10_lex}{2}
\subsection{Reguläre Ausdrücke}
\slides{V10_lex}{3}
\slides{V10_lex}{4}
\subsection{Besondere Zeichen}
\slides{V10_lex}{5}
\subsection{Wiederholungen}
\slides{V10_lex}{6}
\subsection{Reihenfolge der Pattern}
\slides{V10_lex}{7}

\section{Aufbau einer flex-Quelldatei}
\slides{V10_lex}{8}
\subsection{Rules division}
\slides{V10_lex}{9}
\subsection{Vordefinierte Symbole}
\slides{V10_lex}{10}

\section{Beispiele}
\subsection{Beispiel Leerzeichen}
\slides{V10_lex}{11}
\subsection{Zeichen/Zeilen zählen}
\slides{V10_lex}{13}
\subsection{Wörter zählen}
\slides{V10_lex}{14}
\subsection{Beispiel Regeln}
\slides{V10_lex}{20}

\section{Lexer für PL/0 Compiler}
\begin{lstlisting}[language=C]
%{
/* Deklarationsteil */
/*
lexikalische Analyse mit lex für
graphengesteuerten PL/0 Einpasscompiler
*/
#include "lex.h"
#include "list.h"
#include "debug.h"
extern tMorph Morph; /* globale Morpemvariable */
FILE * pIF; /* Eingabedatei */
void MorSo(int Code); /* Function zum Bau eines SymbolTokens */
%}
%%

/* Leer- und Trennzeichen */
[ \t]+
/* Zeilenwechsel */
[\n] {Morph.PosLine++;}
/* Schluesselwoerter (Wortsymbole) */
/* werden wie Sonderzeichen behandlt */
"begin" {MorSo(zBGN);return;}
call {MorSo(zCLL);return;}
const {MorSo(zCST);return;}
do {MorSo(zDO); return;}
else {MorSo(zELS);return;}
end {MorSo(zEND);return;}
if {MorSo(zIF); return;}
odd {MorSo(zODD);return;}
procedure {MorSo(zPRC);return;}
then {MorSo(zTHN);return;}
var {MorSo(zVAR);return;}
while {MorSo(zWHL);return;}

/**********/
/* Zahlen */
/**********/
[0-9]+ {
Morph.MC=mcNumb;
Morph.Val.Numb=atol(yytext);
Morph.MLen=strlen(yytext);
return;
}
/***************/
/* Bezeichner, */
/***************/
/* muessen hinter Schluesselwoertern aufgefuehrt werden */
[A-Za-z]([A-Za-z0-9])* {
Morph.MC=mcIdent;
Morph.Val.pStr=yytext;
Morph.MLen=strlen(yytext);
return;
}

/* Sonderzeichen */
("?") {MorSo('?' );return;}
("!") {MorSo('!' );return;}
("+") {MorSo('+' );return;}
("-") {MorSo('-' );return;}
("*") {MorSo('*' );return;}
("/") {MorSo('/' );return;}
("=") {MorSo('=' );return;}
(">") {MorSo('>' );return;}
("<") {MorSo('<' );return;}
(":=") {MorSo(zErg);return;}
("<=") {MorSo(zle );return;}
(">=") {MorSo(zge );return;}
(";") {MorSo(';' );return;}
(".") {MorSo('.' );return;}
(",") {MorSo(',' );return;}
("(") {MorSo('(' );return;}
(")") {MorSo(')' );return;}
/* String */
(\".*\") {Morph.MC=mcStrng;
Morph.Val.pStr=yytext;
Morph.MLen=strlen(yytext);
return;}

%%
tMorph* Lex()
{
yylex();
return &Morph;
}
void MorSo(int Code)
{
Morph.MC=mcSymb;
Morph.Val.Symb=Code;
Morph.MLen=strlen(yytext);
return;
}
int initLex(char* fname)
{
char vName[128+1];
strcpy(vName,fname);
if (strstr(vName,".pl0")==NULL) strcat(vName,".pl0");
pIF=fopen(vName,"rt");
if (pIF!=NULL) {yyin=pIF; return OK;}
return FAIL;
}
\end{lstlisting}
















