\section{Motivation}
\slides{V1_Grundlagen}{5}
\subsection{Der Compiler}
\slides{V1_Grundlagen}{6}
\subsubsection{Aufbau eines Compilers}
\slides{V1_Grundlagen}{7}
\subsubsection{Beispiel}
\slides{V1_Grundlagen}{8}
\subsubsection*{Abstrakter Syntaxbaum}
\slides{V1_Grundlagen}{9}
\slides{V1_Grundlagen}{10}
\subsection{Dreiadressmaschine vs Stackmaschine}
\slides{V1_Grundlagen}{11}
\slides{V1_Grundlagen}{12}
\subsection{Syntaxgesteuerter Einpasscompiler}
\slides{V1_Grundlagen}{13}

\section{Grundlagen}
\slides{V2_Ableitung}{1}
\subsection{Sprache}
\slides{V1_Grundlagen}{14}
\subsection{Alphabet}
\slides{V1_Grundlagen}{15}
\subsubsection*{Beispiel}
\slides{V1_Grundlagen}{16}
\subsection{Zeichenkette}
\slides{V1_Grundlagen}{17}
\subsection{Freie Halbgruppe (A*)}
\slides{V1_Grundlagen}{18}
\subsection{Syntaxregeln}
\slides{V1_Grundlagen}{18}
\subsection{Grammatik und Parser}
\slides{V2_Ableitung}{6}
\subsection{BNF}
\slides{V2_Ableitung}{7}
\slides{V2_Ableitung}{8}
\subsubsection*{Beispiel}
\slides{V2_Ableitung}{9}
\subsection{EBNF}
\slides{V2_Ableitung}{10}
\slides{V2_Ableitung}{11}
\subsection{Syntaxgraph}
\slides{V2_Ableitung}{12}
\slides{V2_Ableitung}{13}
\slides{V2_Ableitung}{14}
\slides{V2_Ableitung}{15}
\subsection{Grammatik von PL/0}
\slides{V2_Ableitung}{16}
\slides{V2_Ableitung}{17}
\subsubsection*{Beispiele}
\slides{V2_Ableitung}{18}

\section{Grammatik nach Chomsky}
\slides{V1_Grundlagen}{20}
\subsubsection*{Beispiel}
\slides{V1_Grundlagen}{21}
\subsection{Ableitung}
\slides{V1_Grundlagen}{22}
\subsubsection*{Direkte Ableitung}
\slides{V1_Grundlagen}{23}
\slides{V1_Grundlagen}{24}
\subsection{Chomsky Typ 0}
\slides{V1_Grundlagen}{25}
\subsection{Chomsky Typ 1}
\slides{V1_Grundlagen}{26}
\subsection{Chomsky Typ 2}
\slides{V1_Grundlagen}{27}
\subsection{Chomsky Typ 3}
\slides{V1_Grundlagen}{28}
\subsection{Chomsky Typ 4}
\slides{V1_Grundlagen}{29}
\subsection{Beispiel}
\slides{V1_Grundlagen}{30}
\subsubsection*{Analysen}
\slides{V1_Grundlagen}{31}
\slides{V1_Grundlagen}{32}
\subsection{Analysestrategien}
\slides{V1_Grundlagen}{33}

\section{Morpheme, Token, Atome}
\slides{V1_Grundlagen}{34}
\subsection{Morpheme einfacher Ausdrücke}
\slides{V1_Grundlagen}{35}
\slides{V1_Grundlagen}{36}
\subsubsection*{Beispiel 3+5*12}
\slides{V1_Grundlagen}{37}
\subsection{Ein einfacher Lexer}
\begin{lstlisting}[language=C]
typedef struct morph{
	int mc;
	double dval;
	char cval;
}MORPHEM;

enum mcodes{
	mempty,
	mop,
	mdbl
};

static MORPHEM m;

void lex(char * pX){
	static char * px;
	/* Initialisierung*/
	if (pX!=NULL){
		m.mc=mempty;
		px=pX;
	}
	/* lexiaklische Analyse */
	if (*px=='\0'){
		m.mc=mempty;
	} else {
		for (;*px==' '||*px=='\t';px++);
		if (isdigit(*px) || *px=='.'){
			m.dval=strtod(px,&px);
			m.mc =mdbl;
		} else
			switch (*px){
				case '+':
				case '':
				case '*':
				case '/':
				case '(':
				case ')':
					m.cval=*px++;
					m.mc=mop;
					break;
				default :
					printf("wrong ...: %c\n-- canceling --\n",*px);
					exit (1);
			}
	}
}
\end{lstlisting}

\section{Grammatik 1: Einfache Ausdrücke}
\slides{V1_Grundlagen}{39}
\subsection{Rekursiver Abstieg}
\slides{V1_Grundlagen}{40}
\subsubsection*{Alternative Regeln}
\begin{lstlisting}[language=C]
if (lookahead == First(A1)){
	A1();
} else if (lookahead == First(A2)) {
	A2();
}
/* ... */
else if (lookahead == First(An)){
	An();
} else {
	error();
}
\end{lstlisting}
First ist das jeweils erste terminale Symbol einer alternativen
Regel
\subsection{Anforderungen}
\slides{V1_Grundlagen}{42}

\section{LL(1) Grammatiken}
\slides{V1_Grundlagen}{43}
\slides{V1_Grundlagen}{44}

\section{Grammatik 2: Einfache Ausdrücke ohne Linksrekursion}
\slides{V1_Grundlagen}{45}
\subsection{Linksfaktorisierung}
\slides{V1_Grundlagen}{46}
\subsubsection*{An Grammatik}
\slides{V1_Grundlagen}{47}
\subsection{First/Follow}
\slides{V1_Grundlagen}{48}
\subsection{Expression}
\begin{lstlisting}[language=C]
double expr(void){
	double tmp=term();
	if (m.mc==mop && m.cval=='+'){
		lex(NULL); 
		tmp+=expr();
	}
	return tmp;
}
\end{lstlisting}
\subsection{Term}
\begin{lstlisting}[language=C]
double term(void){
	double tmp=fac();
	if (m.mc==mop && m.cval=='*'){
		lex(NULL);
		tmp*=term();
	}
	return tmp;
}
\end{lstlisting}
\subsection{Faktor}
\begin{lstlisting}[language=C]
double fac(){
	double tmp;
	if (m.mc==mop)	{
		if (m.cval=='(')	{
			lex(NULL);
			tmp=expr();
			if (m.mc != mop || m.cval != ')')
				exit (1);
			lex(NULL);
		} else 
			exit (1);
	}	else
		if (m.mc==mdbl){
			tmp=m.dval;
			lex(NULL);
		}	else 
			exit (1);
	return tmp;
}
\end{lstlisting}
\subsection{Main}
\begin{lstlisting}[language=C]
int main(int argc, char*argv[]){
	char *pBuf=argv[1];
	printf("%s\n",pBuf);
	lex(pBuf);
	printf("%10.4f\n",expr());
	free(pBuf);
	return 0;
}
\end{lstlisting}
\subsection{Ausführen}
\begin{lstlisting}
beck@linux:~/COMPILER> ./tPM 5+2*3
5+2*3
11.0000
aber:
beck@linux:~/COMPILER>./tPM 12/2*3
\end{lstlisting}
$\to$ Grammatik rechnet nicht korrekt!

\section{Umformung der Grammatik}
\lecdate{25.10.2017}
\slides{V1_Grundlagen}{54}

\subsection{Grammatik 3: Beseitigung der Linksrekursion durch Iteration}
\slides{V1_Grundlagen}{55}

\subsection{EBNF}
\slides{V1_Grundlagen}{56}

\subsection{Expression als C-Code}
\slides{V1_Grundlagen}{57}

\subsection{Beseitigung der Linksrekursion bei klassischer BNF}
\slides{V1_Grundlagen}{58}

\section{Grammatik für einfache Ausdrücke}
\slides{V1_Grundlagen}{59}

\section{First/Follow}
\slides{V1_Grundlagen}{60}
\subsection{Implementation}
\slides{V1_Grundlagen}{61}
\subsubsection*{Syntaxgraph}
\slides{V1_Grundlagen}{62}

\section{Wiederholung}
\slides{V1_Grundlagen}{63}






