\section{Grundlagen}
\lecdate{29.11.2017}
\slides{V7_VirtuelleMasch}{2}

\subsection{Dreiadressmaschine}
\slides{V7_VirtuelleMasch}{3}
\subsection{Zweiadressmaschine}
\slides{V7_VirtuelleMasch}{4}
\subsection{Einadressmaschine}
\slides{V7_VirtuelleMasch}{5}
\subsection{Nulladressmaschine - Stackmaschine}
\slides{V7_VirtuelleMasch}{6}

\section{Befehlssatz}
\slides{V7_VirtuelleMasch}{7}
\begin{footnotesize}
\begin{lstlisting}[language=C]
typedef enum TCODE{
/*--- Kellerbefehle ---*/
puValVrLocl,/*00 (short Displ)	[Kellern Wert lokale Variable]								*/
puValVrMain,/*01 (short Displ)	[Kellern Wert Main Variable]									*/
puValVrGlob,/*02 (short Displ,short Proc) [Kellern Wert globale Variable]			*/
puAdrVrLocl,/*03 (short Displ)	[Kellern Adresse lokale Variable]							*/
puAdrVrMain,/*04 (short Displ)	[Kellern Adresse Main Variable] 							*/
puAdrVrGlob,/*05 (short Displ,short Proc) [Kellern Adresse globale Variable]	*/
puConst,		/*06 (short Index)	[Kellern einer Konstanten]										*/
storeVal,		/*07 ()							[Speichern Wert ->Adresse, beides aus Keller]	*/
putVal,			/*08 ()							[Ausgabe eines Wertes aus Keller nach stdout]	*/
getVal,			/*09 () 		[Eingabe eines Wertes von stdin -> Addr. im Keller ]	*/
/*--- arithmetische Befehle ---*/
vzMinus,		/*0A ()						 	[Vorzeichen ]																	*/
odd,				/*0B ()						 	[ungerade -> 0/1]																*/
/*--- binäre Operatoren kellern 2 Operanden aus und das Ergebnis ein ---*/
OpAdd,			/*0C ()						 	[Addition] 																		*/
OpSub,			/*0D ()						 	[Subtraktion ] 																*/
OpMult,			/*0E ()						 	[Multiplikation ] 														*/
OpDiv,			/*0F ()						 	[Division ] 																	*/
cmpEQ,			/*10 ()						 	[Vergleich =  -> 0/1] 												*/
cmpNE,			/*11 ()						 	[Vergleich #  -> 0/1] 												*/
cmpLT,			/*12 ()						 	[Vergleich <  -> 0/1] 												*/
cmpGT,			/*13 ()						 	[Vergleich >  -> 0/1] 												*/
cmpLE,			/*14 ()						 	[Vergleich <= -> 0/1]													*/
cmpGE,			/*15 ()						 	[Vergleich >= -> 0/1] 												*/
/*--- Sprungbefehle ---*/
call,				/*16 (short ProzNr) [Prozeduraufruf]															*/
retProc,		/*17 ()							[Rücksprung]																	*/
jmp,				/*18 (short RelAdr) [SPZZ innerhalb der Funktion]									*/
jnot,				/*19 (short RelAdr) [SPZZ innerhalb der Funkt.,Beding.aus Keller]	*/
entryProc,	/*1A (short lenCode,short ProcIdx,short lenVar)										*/
/*--- Zusätzliche, (für uns) nicht benötigte Befehle ---*/
putStrg,		/*1B (char[]) 																										*/
pop,				/*1C																															*/
swap,				/*1D								Austausch Adresse gegen Wert									*/
EndOfCode 	/*1E 																															*/
} tCode;
\end{lstlisting}
\end{footnotesize}
\subsection{Arbeitsweise der Stackmaschine}
\slides{V7_VirtuelleMasch}{8}

\section{Datenstrukturen der VM}
\slides{V7_VirtuelleMasch}{9}

\section{Register der VM}
\slides{V7_VirtuelleMasch}{10}

\section{Prozedurtabelle}
\slides{V7_VirtuelleMasch}{11}

\section{Initialisierung}
\slides{V7_VirtuelleMasch}{12}
\slides{V7_VirtuelleMasch}{13}

\section{Steuerschleife}
\begin{lstlisting}[language=C]
typedef int (*fx)(void);
fx vx[]={
	FcpuValVrLocl, FcpuValVrMain,
	FcpuValVrGlob, FcpuAdrVrLocl,
	FcpuAdrVrMain, FcpuAdrVrGlob,
	FcpuConst, FcstoreVal,
	FcputVal, FcgetVal,
	FcvzMinus, Fcodd,
	FcOpAdd, FcOpSub,
	FcOpMult, FcOpDiv,
	FccmpEQ, FccmpNE,
	FccmpLT, FccmpGT,
	FccmpLE, FccmpGE,
	Fccall, FcRetProc,
	Fcjmp, Fcjnot,
	FcEntryProc, FcputStrg,
	Fpop, Fswap,
	FcEndOfCode, Fcput,
	Fcget
};

while (!Ende){
	vx[*pC++]();
}

while (!Ende){
	switch (*pC++){
		case puValVrLocl:FcpuValVrLocl(); break;
		case puValVrMain:FcpuValVrMain(); break;
		case puValVrGlob:FcpuValVrGlob(); break;
		case puAdrVrLocl:FcpuAdrVrLocl(); break;
		case puAdrVrMain:FcpuAdrVrMain(); break;
		case puAdrVrGlob:FcpuAdrVrGlob(); break;
		case puConst :FcpuConst(); break;
		case storeVal :FcstoreVal(); break;
		case putVal :FcputVal(); break;
		case getVal :FcgetVal(); break;
		case vzMinus :FcvzMinus(); break;
		case odd :Fcodd(); break;
		case OpAdd :FcOpAdd(); break;
		case OpSub :FcOpSub(); break;
		case OpMult :FcOpMult(); break;
		case OpDiv :FcOpDiv(); break;
		/* ... */
		case pop :Fpop(); break;
		case swap :Fswap(); break;
		case EndOfCode :FcEndOfCode(); break;
		case put :Fcput(); break;
		case get :Fcget(); break;
	}
}
\end{lstlisting}

\section{Prozeduraufruf}
\slides{V7_VirtuelleMasch}{16}
\begin{lstlisting}[language=C]
int Fccall(void){
	short ProcNr;
	ProcNr=gtSrtPar(pC);pC+=2;
	push4((int4_t)(pInfProc[iCProc].pVar));
	push4((int4_t)pC);
	pC=pInfProc[ProcNr].pCode;//Sprung zu entryProc
	push4(iCProc);
	return OK;
}
int FcRetProc(void){
	pS=pInfProc[iCProc].pVar;
	iCProc =pop4();
	pC =(char*)pop4();
	pInfProc[iCProc].pVar=(char*)pop4();
	return OK;
}
\end{lstlisting}
\slides{V7_VirtuelleMasch}{18}
\begin{lstlisting}[language=C]
/* Entry Procedure */
int FcEntryProc(void){
	short lVar;
	/* Codelaenge ueberlesen */
	PC+=2;
	/* Prozedurnummer lesen und einstellen */
	iCProc=gtSrtPar(pC);pC+=2;
	/* Variablenlaenge lesen */
	lVar =gtSrtPar(pC);pC+=2;
	/* Zeiger auf Variablenbereich */
	pInfProc[iCProc].pVar=pS;
	/* Neuer Stackpointer */
	pS+=lVar;
	return OK;
}
\end{lstlisting}

\section{Rekursion}
\slides{V7_VirtuelleMasch}{20}

\section{Prozedur mit Parametern}
Parameter wird vor Aufruf auf den Stack geschrieben. Dieser hat die relative Adresse -16 (die erste lokale Variable hat die relative Adresse 0, Die Adresse der Funktion sind 3 Byte, daher haben dann Parameter die Adresse ab -16).
\slides{V7_VirtuelleMasch}{21}

