\subsection{Aussagen, Wahrheitswert}

\paragraph{Aussage:} (im weiteren Sinne) Sprachlich sinnvoller, konsatierender Satz. In diesem Abschnitt werden nur zweiwertige Aussagen betrachtet, d.h. Aussagen, die entwoder wahr oder falsch sind.

\subparagraph{Bsp. 1:} 
\begin{itemize}
\item[(1)] Es gibt unendlich viele Primzahlen (wahr)
\item[(2)] Es gibt unendlich viele Primzahlzwillinge, z.B. (3,5), (5,7), (11,13), (17,19) usw. (Wahrheitswert nicth bekannt!)
\item[(3)] $5+7=13$ (falsch)
\item[(4)] Wie spät ist es? (keine Aussage)
\item[(5)] Diese Aussage ist falsch! (keine Aussage, paradox)
\item[(6)] Am 30.06.2016 wird es in Dresden regnen.
\end{itemize}

(1)--(3) sind zweiwertige Aussagen, (4) und (5) sind keine Aussagen, (6) ist keine zweiwertige Aussage (Wahrscheinlichkeit, d.h. Zahl zwischen 0 und 1 angebbar).

\subparagraph{Bezeichnungen:} \parskp
\begin{tabular}{l l}
p, q, r, &… Aussagen,\\
0 &… falsche Aussage, \\
1 &… wahre Aussage\\
\end{tabular}

\paragraph{Wahrheitswert:} \parskp
$w(p)=\begin{cases}
1 & \text{(falls p wahr)} \\
0 & \text{(fallls p falsch)}
\end{cases}$
    
$p \equiv q$ (p \emph{identisch} q) … p und q haben denselben Wahrheitswert
    
\subsection{Aussagesverschiebung}

\begin{enumerate}
\item \emph{Negation} $\overline{p}$ („nicht p“) [oft auch $p!$ bzw. $\neg p$]\\
\begin{tabular}{c | c}
$p$ & $\overline{p}$\\
\hline
0 & 1 \\
1 & 0 \\
\end{tabular}

\item \emph{Konjunktion} $p \wedge q$ („p und q“)
\item \emph{Disjunktion} $ p \vee q $ („p oder q“) [Alternative -- nicht ausschließendes Oder!]\\
\begin{tabular}{c | c |c |c}
$p$ & $q$ & $p \wedge q$ & $p \vee q$\\
\hline
1 & 1 &1&1\\
1 & 0 &0&1\\
0 & 1 &0&1\\
0 & 0 &0&0\\
\end{tabular}

\item \emph{Implikation} $(p \Rightarrow q ):\equiv \overline{p} \vee q$ („aus p folgt q“, „wenn p, dann q“)\\
\begin{tabular}{c | c |c|c}
$p$ & $q$ & $\overline{p}$ & $p\Rightarrow q$\\
\hline
1&1&0&1 \\
1&0&0&0\\
1&1&1&1\\
1&0&1&1\\
\end{tabular}\\
Begriffe: $p \Rightarrow q$ (p: \emph{Prämisse}, q: \emph{Konklusion})\\
Eine Implikation ist genau dann falsch, wenn die Prämisse richtig und die Konklusion falsch ist!
\paragraph{Bsp. 2:}
\begin{itemize}
\item $-1=1$ (falsch) $\Rightarrow$ $1=1$ (wahr) [durch Quadrieren]
\item $-1=1$ (falsch) $\Rightarrow$ $0=2$ (falsch) [Addition von 1]
\end{itemize}
Aus einer falschen Aussage lassen sich durch richtiges Schließen sowohl falsche als auch richtige Aussagen gewinnen.

Andere Sprechweisen: „p ist \emph{hinreichend} für q“, „q ist \emph{notwendig} für p“
\item \emph{Äquivalenz} $(p\Leftrightarrow q):\equiv (p\Rightarrow q) \wedge (q \Rightarrow p)$ („p äquivalent q“, „p ist notwendig und hinreichend für q“, „p genau dann wenn q“)\\
(ist genau dann wahr, wenn p und q den selben Wahrheitswert besitzen)
\end{enumerate}

\subsection{Logische Gesetze (Tautologien)}
Eine Tautologie $t$ ist eine Aussagenverbindung, die unabhängig vom Wahrheitswert der einzelnen Aussagen stets wahr ist (d.h. $t\equiv 1$).
\paragraph{Bsp. 3:}\parskp
Einige wichtige Tautologien
\begin{enumerate}
\item $p\Leftrightarrow \overline{\overline{p}}$ \tab \tab(Negation der Negation)
\item $p \vee \overline{p}$\tab \tab (Satz vom ausgeschlossenem Dritten)
\item a) $\overline{p\wedge q} \equiv \overline{p} \vee \overline{q}$\\
b) $\overline{p\vee q} \equiv \overline{p} \wedge \overline{q}$ \tab(de Morgansche Regeln)
\item $(p\Rightarrow q) \equiv (\overline{q} \Rightarrow \overline{p})$ \tab(Kontrapositionsgesetz)
\item $p\wedge (p\Rightarrow q)) \Rightarrow q$ \tab(direkter Beweis)
\item $p\wedge (\overline{q} \Rightarrow \overline{p})) \Rightarrow q$ \tab(indirekter Beweis)
\end{enumerate}
Beweise mittels Wahrheitstafeln (vgl. Übung 1).\\
Bemerkung zu 1., 3., 4.: Eine Äquivalenz ist genau dann eine Tautologie, wenn beide Seiten identisch sind, z.B. $p\equiv \overline{\overline{p}}$.

\paragraph{Beweistechniken:} \parskp
Zu beweisen ist $q$.
\begin{enumerate}
\item Direkter Beweis:
\begin{itemize}
\item Nachweis von $p$ (Voraussetzung)
\item Richtiger Schluss $p\Rightarrow q$\\
Dann $q$ wahr (Behauptung)
\end{itemize}
\item Indirekter Beweis: Annahme von $\overline{q}$ auf Wiederspruch führen (auf unterschiedliche Weise möglich, vgl. folgendes Bsp).
\end{enumerate}

\subparagraph{Bsp. 4:}\parskp
$q = $„$\sqrt{2}$ ist irrational“ (keine rationale Zahl)

Beweis indirekt: \\
Es gelte $\overline{q}$, d.h. $\sqrt{2}$ ist rational, dann gelten folgende Schlüsse: $\sqrt{2} = \frac{m}{n}$ mit teilerfremden natürlichen Zahlen $m$ und $n$.\\
$\Rightarrow 2=\frac{m^2}{n^2} \Rightarrow 2 \cdot n^2 = m^2 \Rightarrow 2|m^2$\\
$\Rightarrow \boxed{2|m}$ (2 ist Teiler von m)\\
$\Rightarrow 4|m^2$ (mit $m^2=2n^2$)\\
$\Rightarrow 4|2n^2 \Rightarrow 2|n^2 \Rightarrow \boxed{2|n}$\\
Widerspruch: Da $m$ und $n$ teilerfremd sind.	\#

\paragraph{Weitere Gesetze}

\begin{itemize}
\item $p\wedge q \equiv q \wedge p$\\$p\vee q \equiv q \vee p$ \tab \tab \tab(Kommutativgesetze)
\item $(p\wedge q)\wedge r \equiv p\wedge (q\wedge r)$\\$(p\vee q)\vee r \equiv p\vee (q \vee r)$ \tab \tab(Assoziativgesetze)
\item $(p\wedge q)\vee r \equiv (p \vee r) \wedge (q \vee r)$\\$(p\vee q)\wedge r \equiv (p\wedge r) \vee (q \wedge r)$ \tab(Distributivgesetze)
\item $p\wedge 1 \equiv p$, $p\vee 1 \equiv 1$, $p\wedge p \equiv p$\\$p\wedge 0 \equiv 0$, $p\vee 0 \equiv p$, $p\wedge p \equiv p$
\item $p \vee (p\wedge q ) \equiv p$ \tab \tab(Absorptionsgesetz)
\end{itemize}

\subsection{Aussagefunktionen, Quantoren, Prädikatenlogik} \label{subsec:Aussagefunktionen}
$X$ sei eine Menge (Gesamtheit von Objekten $x$ mit einem gemeinsamen Merkmal, vgl.  Abschnitt \ref{sec:Mengen})\\
$x\in X$ … $x$ ist Element von $X$. Die Objekte haben Eigenschaften (\emph{Prädikate})

\paragraph{Aussagefunktion} (auch Aussageform) $p(x)$:
Jedem $x\in X$ ist eine Aussage $p(x)$ zugeordnet. Dabei steht $x$ für ein Objekt, $p$ für ein Prädikat.
\subparagraph{Bsp. 5:}\parskp
$X$ … Menge der positiven natürlichen Zahlen (1, 2, 3, …)\\
$p(x):=$„$x$ ist eine Primzahl“\\
$p(5)$ … wahr, $p(10)$ … falsch

\paragraph{Quantoren:} \parskp
Betrachtet werden folgende Aussagen:
\begin{enumerate}
\item „Für alle $x$ (aus $X$) gilt $p(x)$“ $\equiv$ $\boxed{\forall x\; p(x)}$ (\emph{universeller Quantor} / Allquantor)
\item „Es existiert (wenigstens) ein $x$, für welches $p(x)$ gilt“ $\equiv$ $\boxed{\exists x \; p(x)}$ (\emph{existenzieller Quantor})
\end{enumerate}
Zur Schreibweise: 
\begin{itemize}
\item Bei Anwendungen (außerhalb der reinen Logik) wird oft die Grundmenke $X$ mit angegeben: \\
$\forall x \in X\; p(x)$ usw.
\item Falls sich Quantoren auf eine Teilmenge $M$ von $X$ beziehen sollen, dann können folgende Schreibweisen verwendet werden:\\
$a = \forall x \in M \; p(x)$, $b=\exists x\in M \; p(x)$.
\item Die Schreibweisen in der formalen Logik sind dann:\\
$a = \forall x \;(x \in M \Rightarrow p(x))$
\end{itemize}

\paragraph{Rechenregeln:}\parskp
$\boxed{\overline{\forall x \; p(x)} \equiv \exists x \; \overline{p(x)}}$\\
$\boxed{\overline{\exists x \; p(x)} \equiv \forall x \; \overline{p(x)}}$

\paragraph{Mehrstellige Aussagefunktionen}

\begin{itemize}
\item $p(x_1, x_2,\: ..., x_n)$,\quad $x_1 \in X_1, x_2 \in X_2,\: ... , x_n \in X_n$\\
Die Grundmengen $X_i$ können, müssen aber nicht für jede Stelle gleich sein.
\item Wird ein Quantor auf eine n-stellige Aussagefunktion angewandt, so entsteht eine (n-1)-stellige Aussagefunktion (eine 0-stellige Aussagefunktion ist eine Aussage)\\
z.B.: $\exists y \; p(x,y,z)=: q(x,z)$, die Variable $y$ wird durch den Quantor $\exists$ gebunden ($y$… gebundene Variable). Wichtig ist der Platz, nicht der Name der Variable.\\
$x, z$ … freie Variable, können durch weitere Quantoren gebunden werden.
\end{itemize}

\subparagraph{Bsp. 6:} \parskp
Ein Dorf bestehe aus 2 Teilen (Ober- und Unterdorf). Es sei $M$ die Menge aller Bewohner des Dorfes. $M_1$ bzw. $M_2$ seien die Teilmengen von $M$, die dem Ober- bzw. Unterdorf entsprechen. 

Wir betrachten folgende zweistellige Aussagefunktionen:\\
$k(x,y)$… Person $x$ (aus $M$) kennt Person $y$ (aus $M$)
\begin{enumerate} [label=\alph*)]
\item $a(x):= \forall y \; k(x,y)$… Person $x$ kennt jeden ($\Rightarrow$ „Für alle $y$ gilt: $x$ kennt $y$“)\\
$b(y):= \exists x \; k(x,y)$ … es gibt jemanden, der $y$ kennt\\
$c := \forall x \forall y \; k(x,y)$ … jeder kennt jeden\\
$d := \forall y \exists x \: k(x,y)$ … jeder wird von wenigstens einer Person gekannt\\
$e := \exists x \forall y \; k(x,y)$ … es gibt mindestens eine Person, die alle Personen kennt\\
Man beachte: \begin{itemize}
\item $d$ und $e$ sind nicht das Gleiche: Die Reihenfolge unterschiedlicher Quantoren muss beachtet werden. Bei $d$ kann für jedes $y$ ein anderes $x$ mit $k(x,y)$ existieren. Diese Abhängigkeit von $y$ wird manchmal in Anwendungen durch $\forall y \: \exists x(y) \; k(x,y)$ ausgedrückt.
\item Es gilt aber $e \Rightarrow d$ (stets wahr: Tautologie). Der Wahrheitsgehalt von z.B. $c, d, e$ kann dagegen nicht mit logischen Mitteln bestimmt werden.
\end{itemize}
\item Negation der Aussagen  bzw. Aussageformen aus a).\\
$\overline{a(x)}\equiv \exists y \; \overline{k(x,y)}$… $x$ kennt wenigstens eine Person nicht\\
$\overline{b(x)} \equiv \forall x \; \overline{k(x,y)}$ … keiner kennt $y$\\
$\overline{c} \equiv \exists x \; \overline{\forall y \; k(x,y)}\equiv \exists x \; \exists y \; \overline{k(x,y)}$ … es gibt jemanden der wenigstens eine Person nicht kennt (jemanden, der nicht alle kennt)\\
$\overline{d} \equiv \exists y \; \forall x \; \overline{k(x,y)}$… es gibt jemanden, der von keiner Person gekannt wird
$\overline{e}\equiv \forall x \; \exists y \; \overline{k(x,y)}$ … jeder kennt wenigstens eine Person nicht.
\item Folgende Aussagen sind mit Hilfe von Quantoren auszudrücken:\\
$f$… jeder aus dem Oberdorf kennt wenigstens eine Person aus dem Unterdorf.\\
$g$… es gibt jemanden im Unterdorf, der alle Personen des Oberdorfs kennt.
\begin{align*}f&=\forall x \in M_1 \; \exists y \in M_2 \; k(x,y)\\
&=\forall x \; (x \in M_1 \Rightarrow \exists y \; (y \in M_2 \wedge k(x,y)))\end{align*}
\begin{align*}
g&= \exists x \in M_2 \; \forall y \in M_1 \; k(x,y)\\
&= \exists x \; (x\in M_2 \wedge \forall y \; (y \in M_1 \Rightarrow k(x,y)))
\end{align*}
\end{enumerate}