%
% API Documentation for API Documentation
% Module doc_example.test
%
% Generated by epydoc 3.0.1
% [Tue May 30 13:03:05 2017]
%

%%%%%%%%%%%%%%%%%%%%%%%%%%%%%%%%%%%%%%%%%%%%%%%%%%%%%%%%%%%%%%%%%%%%%%%%%%%
%%                          Module Description                           %%
%%%%%%%%%%%%%%%%%%%%%%%%%%%%%%%%%%%%%%%%%%%%%%%%%%%%%%%%%%%%%%%%%%%%%%%%%%%

    \index{doc\_example \textit{(package)}!doc\_example.test \textit{(module)}|(}
\section{Module doc\_example.test}

    \label{doc_example:test}
Created on May 29, 2017

\textbf{Author:} jonatan




%%%%%%%%%%%%%%%%%%%%%%%%%%%%%%%%%%%%%%%%%%%%%%%%%%%%%%%%%%%%%%%%%%%%%%%%%%%
%%                               Functions                               %%
%%%%%%%%%%%%%%%%%%%%%%%%%%%%%%%%%%%%%%%%%%%%%%%%%%%%%%%%%%%%%%%%%%%%%%%%%%%

  \subsection{Functions}

    \label{doc_example:test:printme}
    \index{doc\_example \textit{(package)}!doc\_example.test \textit{(module)}!doc\_example.test.printme \textit{(function)}}

    \vspace{0.5ex}

\hspace{.8\funcindent}\begin{boxedminipage}{\funcwidth}

    \raggedright \textbf{printme}(\textit{stri})

    \vspace{-1.5ex}

    \rule{\textwidth}{0.5\fboxrule}
\setlength{\parskip}{2ex}
    Hier wird ein String ausgegeben. Das Selbe wie print() lol :-)

    Insgesamt wird folgendes geleistet:

    \begin{itemize}
    \setlength{\parskip}{0.6ex}
      \item Der String wird auf der Console ausgegeben.

        \begin{enumerate}

        \setlength{\parskip}{0.5ex}
          \item Das ist schoen (Umlaute sind es hier in der doc nicht -- 
            vielleicht sollte man lieber in englisch dokumentieren?).

          \item und elegant!

          \item Nachdem der Header für UTF-8 ergänzt wurde, ist auch utf8 
            erlaubt. Cool, odä?

        \end{enumerate}

      \item Der String wird zurueck gegeben.

    \end{itemize}

    Im Prinzip ist die Funktion bloss:

\begin{alltt}
\pysrcprompt{{\textgreater}{\textgreater}{\textgreater} }\pysrckeyword{print} (stri)
\pysrcoutput{return stri;}\end{alltt}
    Toll, oder?

\setlength{\parskip}{1ex}
      \textbf{Parameters}
      \vspace{-1ex}

      \begin{quote}
        \begin{Ventry}{xxxx}

          \item[stri]

          Der auszugebenende String.

            {\it (type=String)}

        \end{Ventry}

      \end{quote}

      \textbf{Return Value}
    \vspace{-1ex}

      \begin{quote}
      Der eingegebene String.

      {\it (type=String)}

      \end{quote}

    \end{boxedminipage}


%%%%%%%%%%%%%%%%%%%%%%%%%%%%%%%%%%%%%%%%%%%%%%%%%%%%%%%%%%%%%%%%%%%%%%%%%%%
%%                               Variables                               %%
%%%%%%%%%%%%%%%%%%%%%%%%%%%%%%%%%%%%%%%%%%%%%%%%%%%%%%%%%%%%%%%%%%%%%%%%%%%

  \subsection{Variables}

    \vspace{-1cm}
\hspace{\varindent}\begin{longtable}{|p{\varnamewidth}|p{\vardescrwidth}|l}
\cline{1-2}
\cline{1-2} \centering \textbf{Name} & \centering \textbf{Description}& \\
\cline{1-2}
\endhead\cline{1-2}\multicolumn{3}{r}{\small\textit{continued on next page}}\\\endfoot\cline{1-2}
\endlastfoot\raggedright \_\-\_\-p\-a\-c\-k\-a\-g\-e\-\_\-\_\- & \raggedright \textbf{Value:} 
{\tt None}&\\
\cline{1-2}
\end{longtable}

    \index{doc\_example \textit{(package)}!doc\_example.test \textit{(module)}|)}
