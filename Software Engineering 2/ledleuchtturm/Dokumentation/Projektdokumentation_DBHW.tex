\subsection{Aufgabe}

Die Aufgabe des Verantwortlichen für Datenbank und Hardware war namensgebend die Datenbank und die Hardware: Für die Datenbank musste ein Trigger erstellt werden, der den Status einer Maschine an die zu entwickelnde Software weiter gibt. Die Schaltung der Hardware (die Lampen, die an den Raspberry Pi angeschlossen sind) muss entworfen werden und schließlich auf diese Signale reagieren.

\subsection{Ablauf}

\subsubsection{Hardware}
Die Aufgabe des Projektes war es ein auf einem Raspberry Pi laufendes System zu entwickeln, das ein Signal mit LEDs anzeigt. Im Rahmen des Projekts wurden der Gruppe vom Kunden die nötigen Einzelteile der Hardware, bzw. entsprechende finanzielle Mittel, zur Verfügung gestellt:
 
Der gestellte Raspberry Pi wurde mit dem Betriebssystem \texttt{Raspbian} eingerichtet und mit einem WLAN-Stick erweitert. Für das System wurden LED-Lampen in den Farben rot, gelb und grün sowie die entsprechenden Vorwiderstände erworben. Zusammen mit einer Steckplatine wurde so die Schaltung für die LED-Ampel entworfen und mit einem einfachen Python-Skript getestet. 

Insgesamt war der Hardware-Teil der Aufgabe wenig aufwändig.
%One of the project's task was to create a LED signaling system that would be run on a Raspberry Pi device. The team was given necessary parts including the Raspberry Pi and means to purchase pieces for the LED system. For the LED system, necessary red, yellow and green LED's with corresponding resistors were bought from the local Conrad electronics store with the given money. Nextly, the Raspberry Pi was installed in a plastic case and the Raspberry Pi was configured with an operating system "Raspbian". Furthermore, a WLAN USB-Adapter was plugged in to the Raspi to enable WLAN connection. Lastly, a breadboard with the LED-system was created. Afterwards, the LED-system was tested to see if it was correctly built using python libraries and commands.

%The side of the hardware in this project was relatively simple as it involved setting up the Raspberry Pi with an operating system and then creating a simple LED-system. Nevertheless, this was a good oppurtunity to learn the Raspberry Pi computer, it's properties, installing an OS on it and creating a simple electronical circuit.

\subsubsection{Datenbank}
Der Aspekt der Datenbank war wahrscheinlich der ausschlaggebendste Teil des Projekts: Von ihr bzw. dem zu entwickelndem Trigger hing ab, ob das Softwaresystem funktioniert oder nicht. Bevor die Überlegungen zu dem Trigger angestellt werden konnten, musste erst die PostgreSQL-Datenbank mithilfe des vom Kunden zur Verfügung gestellten Dumps eingerichtet werden.
%The database side was perhaps the most crucial side in this project since it depended on the data from the database how the LED's should be working. The type of the database used was PostgreSQL. The team was given a dump of the actual database, where it should get the information from.

Mit der eingerichteten Datenbank wurde ermittelt, wie sie aus dem lokalen Netz mithilfe von Python angesteuert werden kann: Mit dem Python-Paket \texttt{psycopg2}. Des Weiteren musste ein Mittel gefunden werden, mit dem die Datenbank einen Status an die Software senden kann, da der Kunde festgelegt hatte, dass die Stati nicht über polling abgefragt werden dürfen.

Dementsprechend wurde geforscht, wie ein Trigger einzurichten ist, der den Status an das Softwaresystem sendet. PostgreSQL bietet nativ keine Möglichkeit aus einem Trigger eine Nachricht ins Netzwerk zu senden. Die Möglichkeit wurde genutzt aus einem Trigger heraus ein (Python-)Skript aufrufen zu können. Mit diesen Voraussetzungen konnte also eine Funktion geschrieben werden, die durch den Trigger aufgerufen wird und den Status über einen Socket ins Netzwerk sendet.

Für die Gruppe wurde eine kleine Anleitung erstellt, wie die Datenbank und der Trigger einzurichten ist. Diese Anleitung wurde letztendlich für die Benutzerdokumentation verfeinert.
%Firstly, with the given data dump, a local database was created in one of the team member server. Then it was analyzed, which tables were important for the project and how the database could be reached from the Raspberry. A PostgreSQL adapter psycopg2 was found and used to establish connection to the database. Nextly, the means of getting the data from the database to the Raspberry were researched. This was so because the project requirements stated that the final program must not be a "polling" program, meaning that it repeatedly asks the database for information. Instead, the database should inform the Raspberry Pi device when any relevant changes have encurred.
%This meant that the database must be installed with a trigger that would run a function to somehow inform the device. The means for connection were researched, since PostgreSQL does not have a native way to connect to external devices. Through research was found that PostgreSQL supports executing scripts in other languages and sockets could be used to achieve the connection. Then test triggers were created to test the trigger functionality. After that the real triggers were created, which would execute a Python script and send the data to the device. 
%A small documentation was made for the team members on creating a local database with the dump, installing the triggers and about receiving data from the database.

\subsection{Herausforderungen}
Die größte Herausforderung war die Entwicklung wie die Statusinformation durch einen Trigger ins Netzwerk gesendet werden könnte. Die Lösung mit dem Python-Skript im Trigger wurde erst nach längeren Nachforschungen gefunden.

Das Öffnen eines Sockets in diesem Skript eröffnete gleichzeitig auch ein neues Problem: Da Sockets immer zwischen genau zwei Stellen kommunizieren, musste eine Lösung gefunden werden, dass die Datenbank Statusänderungen auch an mehrere Geräte (bzw. Instanz des Softwaressytems auf einem Gerät) senden kann.

Die Lösung dieses Problems waren separate Trigger für jede Maschine in der Datenbank. So konnte jede Maschine ein eigenes Socket zur entsprechenden Softwareinstanz öffnen. Als Alternative wurde noch MQTT in Betracht gezogen, welches viele Vorteile geboten hätte. Letztendlich wurde in einem Gruppentreffen allerdings beschlossen, wie im Pflichtenheft beschrieben, Sockets zu verwenden. 
%The biggest challenge for the database side was how to get the info from the database. It was already seen in the design stage of the project that a trigger is needed. But the work tool to send data was unknown. Through researching the solution was found, using triggers to execute a python script which would create a socket for data sending.
%But the solution of this challenge led to the next one: since sockets are usually a connection between two devices, a solution for multiple devices was needed. 
%It was decided that for every program there would be a separate trigger in the database. Another connection solution was proposed using MQTT which would require running a separate server. This method would have been really great for the project, but was turned down in a team meeting in favor of sockets due to being more simpler, since the program was built has light as possible.

\subsection{Persönliches Fazit}
Dieses Projekt hat mir als Verantwortlichen für Datenbank und Hardware dabei geholfen, die Grundlagen von Kommunikation von und zu Datenbanken zu lernen. Während es einfach war sich mit Python mit einer Datenbank zu verbinden, brauchte es einige extra Schritte, um von der Datenbank ein Signal zum Softwaressytem zu senden -- inklusive der Einrichtung eines Triggers. Zusätzlich zum Entwickeln der Datenbank wurden gelernt das Erarbeitete auch in Form der Benutzerdokumentation zu dokumentieren. Dies war nicht nur für den Kunden, sondern auch für die Gruppenmitglieder wichtig. Zusätzlich dazu wurde ich durch die Aufgaben mit dem einrichten eines Raspberry Pi und dem Nutzen der Hardwareschnittstelle und den LEDs vertraut.
%This project helped to learn the basics of communication with a database in both directions. While it was easy to connect to the database using a python library, the connection from the database back needed extra steps and installing a trigger. This proved usual as we now know that PostgreSQL enables the user to execute custom scripts in many languages which makes the database very flexible. In addition to database communication, documentating the connection and the scripts for the installation of triggers was learned. Since the database handler did not have much to do with the implementation of the program itself, a good documentation and instructions were required so the other team members could continue their work. Finally, a basic knowledge of Raspberry Pi hardware and software installation was obtained, since everything needs to be run on that machine and the controlling of the LED-Pins were made through the device.