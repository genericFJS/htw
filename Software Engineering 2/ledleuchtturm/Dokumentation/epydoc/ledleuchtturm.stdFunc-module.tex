%
% API Documentation for API Documentation
% Module ledleuchtturm.stdFunc
%
% Generated by epydoc 3.0.1
% [Tue Jul 18 00:55:24 2017]
%

%%%%%%%%%%%%%%%%%%%%%%%%%%%%%%%%%%%%%%%%%%%%%%%%%%%%%%%%%%%%%%%%%%%%%%%%%%%
%%                          Module Description                           %%
%%%%%%%%%%%%%%%%%%%%%%%%%%%%%%%%%%%%%%%%%%%%%%%%%%%%%%%%%%%%%%%%%%%%%%%%%%%

    \index{ledleuchtturm \textit{(package)}!ledleuchtturm.stdFunc \textit{(module)}|(}
\section{Modul ledleuchtturm.stdFunc}

    \label{ledleuchtturm:stdFunc}
Funktionen, die beim Entwicklungsprozess benutzt werden.

\textbf{Autor:} Simon Retsch




%%%%%%%%%%%%%%%%%%%%%%%%%%%%%%%%%%%%%%%%%%%%%%%%%%%%%%%%%%%%%%%%%%%%%%%%%%%
%%                               Functions                               %%
%%%%%%%%%%%%%%%%%%%%%%%%%%%%%%%%%%%%%%%%%%%%%%%%%%%%%%%%%%%%%%%%%%%%%%%%%%%

  \subsection{Funktionen}

    \label{ledleuchtturm:stdFunc:printConfigSettings}
    \index{ledleuchtturm \textit{(package)}!ledleuchtturm.stdFunc \textit{(module)}!ledleuchtturm.stdFunc.printConfigSettings \textit{(function)}}

    \vspace{0.5ex}

\hspace{.8\funcindent}\begin{boxedminipage}{\funcwidth}

    \raggedright \textbf{printConfigSettings}(\textit{jsonFileObject})

    \vspace{-1.5ex}

    \rule{\textwidth}{0.5\fboxrule}
\setlength{\parskip}{2ex}
    Gibt den Inhalt des jsonFileObjects auf der Konsole aus.

\setlength{\parskip}{1ex}
      \textbf{Parameter}
      \vspace{-1ex}

      \begin{quote}
        \begin{Ventry}{xxxxxxxxxxxxxx}

          \item[jsonFileObject]

          Objekt welches Daten der Konfigurationsdatei enthält.

            {\it (type=Json Object)}

        \end{Ventry}

      \end{quote}

\textbf{Autor:} Simon Retsch



    \end{boxedminipage}

    \label{ledleuchtturm:stdFunc:debug}
    \index{ledleuchtturm \textit{(package)}!ledleuchtturm.stdFunc \textit{(module)}!ledleuchtturm.stdFunc.debug \textit{(function)}}

    \vspace{0.5ex}

\hspace{.8\funcindent}\begin{boxedminipage}{\funcwidth}

    \raggedright \textbf{debug}(\textit{string}, \textit{DEBUG})

    \vspace{-1.5ex}

    \rule{\textwidth}{0.5\fboxrule}
\setlength{\parskip}{2ex}
    Schreibt die Nachricht string auf der Konsole, falls DEBUG auf 1 
    gesetzt ist.

\setlength{\parskip}{1ex}
      \textbf{Parameter}
      \vspace{-1ex}

      \begin{quote}
        \begin{Ventry}{xxxxxx}

          \item[string]

          Nachricht die auf der Konsole erscheinen soll.

            {\it (type=string)}

          \item[DEBUG]

          Abhängigkeitsvariable.

            {\it (type=int)}

        \end{Ventry}

      \end{quote}

\textbf{Autor:} Simon Retsch



    \end{boxedminipage}


%%%%%%%%%%%%%%%%%%%%%%%%%%%%%%%%%%%%%%%%%%%%%%%%%%%%%%%%%%%%%%%%%%%%%%%%%%%
%%                               Variables                               %%
%%%%%%%%%%%%%%%%%%%%%%%%%%%%%%%%%%%%%%%%%%%%%%%%%%%%%%%%%%%%%%%%%%%%%%%%%%%

  \subsection{Variablen}

    \vspace{-1cm}
\hspace{\varindent}\begin{longtable}{|p{\varnamewidth}|p{\vardescrwidth}|l}
\cline{1-2}
\cline{1-2} \centering \textbf{Name} & \centering \textbf{Beschreibung}& \\
\cline{1-2}
\endhead\cline{1-2}\multicolumn{3}{r}{\small\ldots}\\\endfoot\cline{1-2}
\endlastfoot\raggedright \_\-\_\-p\-a\-c\-k\-a\-g\-e\-\_\-\_\- & \raggedright \textbf{Wert:} 
{\tt None}&\\
\cline{1-2}
\end{longtable}

    \index{ledleuchtturm \textit{(package)}!ledleuchtturm.stdFunc \textit{(module)}|)}
