\section{Einführung}
\subsection{Schnittstelle auf einem Host zur Datenübertragung an andere Prozesse}
\slides{10-sockets_print}{3}
\subsection{Socketschnittstelle: Abgrenzung}
\slides{10-sockets_print}{4}
\subsection{IP-Adressen und Ports}
\slides{10-sockets_print}{5}
\section{Demultiplexing}
\slides{10-sockets_print}{6}
\subsection{Verbindungsloses Demultiplexing (UDP)}
\slides{10-sockets_print}{7}
\paragraph{Testfrage:} Benötigte UDP-Sockets: 2 Sockets (beliebig viele Partner können auf einen Port/Socket kommen, dafür muss darauf geachtet werden, dass das Paket an den richtigen Absender zurück geschickt wird)
\subsection{Verbindungsorientiertes Demultiplexing (TCP)}
\slides{10-sockets_print}{8}
\paragraph{Testfrage:} Benötigte TCP-Sockets: 3 Sockets (jeder Verbindung ist eigenständig und vom Rechner verwaltet, es muss sich nicht drum gekümmert werden, dass es auch an den richtigen Partner zurück geschickt wird)
\slides{10-sockets_print}{9}
\section{Socket-Programmierung}
\slides{10-sockets_print}{10}
\subsection{UDP-Sockets}
\slides{10-sockets_print}{11}
\slides{10-sockets_print}{12}
\subsubsection{UDP-Demo Datenströme}
Datenströme müssen explizit in Pakete gewandelt werden:\\
UDP ist kein Stream $\to$ Konvertierung IO-Stream $\Leftrightarrow$ UDP.
\slides{10-sockets_print}{13}
\subsubsection{UDP-Demo Client}
\slides{10-sockets_print}{14}
\slides{10-sockets_print}{15}
Achtung: Wenn Paket verloren geht, würde Programm stehen bleiben! Dafür gibt es in der Funktion \emph{.recieve()} auch die Option eines Timeouts.\\
Hinweis nebenbei: sendData-Array hätte gar nicht initialisiert werden müssen, weil er in der Zuweisung später sowieso einen neuen Zeiger bekommt.
\subsubsection{UDP-Demo Server}
\slides{10-sockets_print}{16}
\slides{10-sockets_print}{17}
\slides{10-sockets_print}{18}
UDP vor allem bei Anfragen sinnvoll, die in ein Paket passen. Nicht so sinnvoll für mehrere Pakete/Streams.

\subsection{TCP-Sockets}
\slides{10-sockets_print}{19}
\slides{10-sockets_print}{20}
\slides{10-sockets_print}{21}
\subsubsection*{Dreiwege-Handshake}
\slides{10-sockets_print}{22}
\slides{10-sockets_print}{23}
\subsubsection{TCP-Demo Datenströme}
Eingabestrom: Quelle - Tastatur\\
Ausgabestrom: Senke - Socket
\slides{10-sockets_print}{24}
\slides{10-sockets_print}{25}
\subsubsection{TCP-Demo Client}
\slides{10-sockets_print}{26}
\slides{10-sockets_print}{27}
\subsubsection{TCP-Demo Server}
\slides{10-sockets_print}{28}
\slides{10-sockets_print}{29}

\section{Werkzeug Netcat}
\slides{10-sockets_print}{30}

\section*{Zusammenfassung}
\slides{10-sockets_print}{31}