\section{Aufgaben der Transportschicht}
\slides{15-transportschicht_print}{3}
\subsection{Transportschichten im Internet}
\slides{15-transportschicht_print}{4}
\subsection{Multiplexing / Demultiplexing}
\slides{15-transportschicht_print}{5}
\subsection{Verbindungsloses Multiplexing}
\slides{15-transportschicht_print}{6}
\section{UDP}
\subsection{UDP-Header}
(RFC 768)
\slides{15-transportschicht_print}{7}
\subsection{UDP-Lite}
(RFC 3828, UDP für geringe Verzögerungen $\to$ Mediendaten)
\slides{15-transportschicht_print}{8}

\subsection{Netzwerk-Datenformat}
\slides{15-transportschicht_print}{9}
\section{TCP}
\subsection{Verbindungsorientiertes Multiplexing}
\slides{15-transportschicht_print}{10}
\slides{15-transportschicht_print}{11}
\subsection{TCP-Nutzung}
\slides{15-transportschicht_print}{12}
\subsection{TCP-Eigenschaften}
\slides{15-transportschicht_print}{13}
\subsection{TCP-Header}
\slides{15-transportschicht_print}{14}
\slides{15-transportschicht_print}{15}

\subsection{TCP Verbindungsstart}
(Drei-Wege-Handshake)
\slides{15-transportschicht_print}{16}
\subsubsection{Sequenznummern}
\slides{15-transportschicht_print}{17}
\subsection{Freigabe von Verbindungen}
\slides{15-transportschicht_print}{18}
\subsubsection{Zwei-Armeen-Problem}
\slides{15-transportschicht_print}{19}

\subsection{TCP Verbindungsende}
\slides{15-transportschicht_print}{20}
\subsubsection{TCP Zustände des Servers}
\slides{15-transportschicht_print}{21}
\subsubsection{TCP Zustände des Clients}
\slides{15-transportschicht_print}{22}

\subsection{TCP Datenübertragung}
\slides{15-transportschicht_print}{23}
\subsubsection{Beispiel Datenübertragung}
\slides{15-transportschicht_print}{24}
\subsubsection{Beispiel Segmentverlust}
\slides{15-transportschicht_print}{25}
\subsubsection{Beispiel zu knapper Timeout}
\slides{15-transportschicht_print}{26}
\subsubsection{Beispiel Kumulative ACKs}
\slides{15-transportschicht_print}{27}

\subsubsection{Test TCP-Segmenterzeugung}
\slides{15-transportschicht_print}{28}
(potentielle Prüfungsaufgabe)

\subsection{TCP-ACK-Erzeugung}
(RFC 1122, RFC 2581)
\slides{15-transportschicht_print}{29}
\subsubsection{Beispiel TCP Fast Retransmit}
\slides{15-transportschicht_print}{30}

\subsection{TCP Flusskontrolle}
Angleichen der Sende-/Empfangsgeschwindigkeit
\slides{15-transportschicht_print}{31}
\subsubsection{TCP-RcvWindow}
\slides{15-transportschicht_print}{32}
\subsubsection{Beispiel TCP Verbindungsaufbau}
Paketsniffer Wireshark
\slides{15-transportschicht_print}{33}
\slides{15-transportschicht_print}{34}
\subsubsection{Retransmission-Timer}
\slides{15-transportschicht_print}{35}
\subsubsection*{RTT Bestimmung}
\slides{15-transportschicht_print}{36}
\subsubsection*{RTO Bestimmung}
\slides{15-transportschicht_print}{37}
Achtung: Startwerte und Minimum RTO sind hier sehr konservativ angegeben!\\
Hinweis: RTO Formel nützlich für Beleg. Startwerte kürzer setzen und Minimum nicht nötig.

\subsection{TCP Überlastkontrolle}
\slides{15-transportschicht_print}{38}
1. Bild: Flusskontrolle, 2. Bild: Überlastkontrolle\\
Daran stellt man Überlast fest:
\begin{itemize}
\item Paketverlust
\end{itemize}
\subsubsection{Ansätze zur Überlastkontrolle}
\slides{15-transportschicht_print}{39}
\subsubsection{Überlastkontrolle bei TCP}
\slides{15-transportschicht_print}{40}
\subsubsection{TCP Überlastkontrolle durch variable Fenstergröße}
Erhöhung der Fenstergröße bis Verlust eintritt
\slides{15-transportschicht_print}{41}
\subsubsection{TCP Slow Start}
\slides{15-transportschicht_print}{42}
\slides{15-transportschicht_print}{43}
\subsubsection{Fast Recovery}
\slides{15-transportschicht_print}{44}
\subsubsection{Optimale TCP Buffergröße}
Sende- und Empfangspuffer des Sockets
\slides{15-transportschicht_print}{45}
\subsubsection{Zusammenfassung TCP Überlastkontrolle}
\slides{15-transportschicht_print}{46}
\subsubsection{Verfahren zur Verstopfungsvermeidung}
Forschungsthema
\slides{15-transportschicht_print}{47}

\subsection{TCP Reset}
\slides{15-transportschicht_print}{48}
Verfahren beim Empfang in Abhängigkeit vom Zustand:
\slides{15-transportschicht_print}{49}

\subsection{Keepalive}
\slides{15-transportschicht_print}{50}

\section{Zusammenfassung Transportschicht}
\slides{15-transportschicht_print}{51}

\section{Ergänzungen}
\subsection{Erweiterung Zeitstempel}
\slides{15-transportschicht_print}{52}
\subsection{Wireshark}
Erkenntnisgewinn durch Paketanalyse
\slides{15-transportschicht_print}{53}
\subsection{TCP-Fairness}
\slides{15-transportschicht_print}{54}
\subsection{TCP offload engine (TOE)}
\slides{15-transportschicht_print}{55}
\subsection{TCP Zustandsmaschine}
\slides{15-transportschicht_print}{56}