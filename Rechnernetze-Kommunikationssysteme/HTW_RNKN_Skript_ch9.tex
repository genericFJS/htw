\section{Einführung}
\slides{33-vielfachzugriff_print}{3}
\subsection*{Mehrfachzugriffskanäle}
\slides{33-vielfachzugriff_print}{4}
\subsection*{Verzögerung in paketorientierten Netzen}
\slides{33-vielfachzugriff_print}{5}
\subsubsection*{Beispiel}
\slides{33-vielfachzugriff_print}{6}
\subsection*{Gemeinsame Nutzung eines Kanals}
\slides{33-vielfachzugriff_print}{7}
\subsection*{Einteilung}
\slides{33-vielfachzugriff_print}{8}

\section{ALOHA-Protokoll}
\slides{33-vielfachzugriff_print}{9}
\subsection*{ALOHA-Algorithmus}
\slides{33-vielfachzugriff_print}{10}
\subsection{Annahmen zur Berechnung von ALOHA}
\slides{33-vielfachzugriff_print}{11}
\slides{33-vielfachzugriff_print}{12}
\subsubsection{Variablen und Parameter zur Berechnung}
\slides{33-vielfachzugriff_print}{16}
Kanalzugriffsrate $\not =$ Datenrate (es gibt auch Zugriffe ohne erfolgreichen Datentransfer)
\subsection{Wiederholung Wahrscheinlichkeit}
\subsubsection{Binomialverteilung}
\slides{33-vielfachzugriff_print}{13}
\subsubsection{Poisson-Verteilung}
\slides{33-vielfachzugriff_print}{14}
\subsubsection*{Poissonsche Annahmen}
\slides{33-vielfachzugriff_print}{15}

\subsection{Modell des ALOHA-Zugriffsverfahrens}
\slides{33-vielfachzugriff_print}{17}

\subsubsection{Unslotted/Ungetacktetes ALOHA}
\subsubsection*{Prinzip}
\slides{33-vielfachzugriff_print}{18}
\subsubsection*{Effizienz}
\slides{33-vielfachzugriff_print}{19}
\subsubsection*{Durchsatz}
\slides{33-vielfachzugriff_print}{20}
\subsubsection*{Beispiel}
\slides{33-vielfachzugriff_print}{21}

\subsubsection{Slotted/Getacktetes ALOHA}
\slides{33-vielfachzugriff_print}{22}
\subsubsection*{Vergleich ungetacktetes ALOHA}
\slides{33-vielfachzugriff_print}{23}
\subsubsection*{Beispiel}
\slides{33-vielfachzugriff_print}{24}
\slides{33-vielfachzugriff_print}{25}
\subsubsection*{Stabilität}
\slides{33-vielfachzugriff_print}{26}









