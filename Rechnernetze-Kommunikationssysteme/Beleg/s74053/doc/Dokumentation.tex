\newcommand{\customDir}{}
\RequirePackage{ifthen,xifthen}

% Input inkl. Umlaute, Silbentrennung
\RequirePackage[T1]{fontenc}
\RequirePackage[utf8]{inputenc}

% Arbeitsordner (in Abhängigkeit vom Master) Standard: .LateX_master Ordner liegt im Eltern-Ordner
\providecommand{\customDir}{../}
\newcommand{\setCustomDir}[1]{\renewcommand{\customDir}{#1}}
%%% alle Optionen:
% Doppelseitig (mit Rand an der Innenseite)
\newboolean{twosided}
\setboolean{twosided}{false}
% Eigene Dokument-Klasse (alle KOMA möglich; cheatsheet für Spicker [3 Spalten pro Seite, alles kleiner])
\newcommand{\customDocumentClass}{scrreprt}
\newcommand{\setCustomDocumentClass}[1]{\renewcommand{\customDocumentClass}{#1}}
% Unterscheidung verschiedener Designs: htw, fjs
\newcommand{\customDesign}{htw}
\newcommand{\setCustomDesign}[1]{\renewcommand{\customDesign}{#1}}
% Dokumenten Metadaten
\newcommand{\customTitle}{}
\newcommand{\setCustomTitle}[1]{\renewcommand{\customTitle}{#1}}
\newcommand{\customSubtitle}{}
\newcommand{\setCustomSubtitle}[1]{\renewcommand{\customSubtitle}{#1}}
\newcommand{\customAuthor}{}
\newcommand{\setCustomAuthor}[1]{\renewcommand{\customAuthor}{#1}}
%	Notiz auf der Titelseite (A: vor Autor, B: nach Autor)
\newcommand{\customNoteA}{}
\newcommand{\setCustomNoteA}[1]{\renewcommand{\customNoteA}{#1}}
\newcommand{\customNoteB}{}
\newcommand{\setCustomNoteB}[1]{\renewcommand{\customNoteB}{#1}}
% Format der Signatur in Fußzeile:
\newcommand{\customSignature}{\ifthenelse{\equal{\customAuthor}{}} {} {\footnotesize{\textcolor{darkgray}{Mitschrift von\\ \customAuthor}}}}
\newcommand{\setCustomSignature}[1]{\renewcommand{\customSignature}{#1}}
% Format des Autors auf dem Titelblatt:
\newcommand{\customTitleAuthor}[1]{\textcolor{darkgray}{Mitschrift von #1}}
\newcommand{\setCustomTitleAuthor}[1]{\renewcommand{\customTitleAuthor}{#1}}
% Standard Sprache
\newcommand{\customDefaultLanguage}[1]{}
\newcommand{\setCustomDefaultLanguage}[1]{\renewcommand{\customDefaultLanguage}{#1}}
% Folien-Pfad (inkl. Dateiname ohne Endung und ggf. ohne Nummerierung)
\newcommand{\customSlidePath}{}
\newcommand{\setCustomSlidePath}[1]{\renewcommand{\customSlidePath}{#1}}
% Folien Eigenschaften
\newcommand{\customSlideScale}{0.5}
\newcommand{\setCustomSlideScale}[1]{\renewcommand{\customSlideScale}{#1}}
\newcommand{\customSlideHeight}{9.63cm}
\newcommand{\setCustomSlideHeight}[1]{\renewcommand{\customSlideHeight}{#1}}
\newcommand{\customSlideWidth}{12.8cm}
\newcommand{\setCustomSlideWidth}[1]{\renewcommand{\customSlideWidth}{#1}}

%\setboolean{twosided}{true}
%\setCustomDocumentClass{scrartcl}
%\setCustomDesign{htw}
%\setCustomSlidePath{Folien}

\setCustomTitle{Beleg UDP-Dateiübertragung}
\setCustomSubtitle{\texorpdfstring{Rechnernetze /\\ Kommunikationssysteme}{Rechnernetze / Kommunikationssysteme}}
\setCustomAuthor{Falk-Jonatan Strube}
%\setCustomNoteA{TitlepageNoteBeforeAuthor}
\setCustomNoteB{\textcolor{darkgray}{Vorlesung von Prof. Dr.-Ing. Vogt}}

\setCustomSignature{\textcolor{darkgray}{\customAuthor{}\\ s74053}}	% Formatierung der Signatur in der Fußzeile
\setCustomTitleAuthor{Beleg von \customAuthor{} (s74053)}	% Formatierung des Autors auf dem Titelblatt

%-- Prüfen, ob Beamer
\ifthenelse{\equal{\customDocumentClass}{beamer}}{
%%% TODO: andere Layouts für Beamer außer HTW
	\documentclass[ignorenonframetext, 11pt, table]{beamer}
	
	\usenavigationsymbolstemplate{}
	\setbeamercolor{author in head/foot}{fg=black}
	\setbeamercolor{title}{fg=black}
	\setbeamercolor{bibliography entry author}{fg=htworange!70}
	%\setbeamercolor{bibliography entry title}{fg=blue} 
	\setbeamercolor{bibliography entry location}{fg=htworange!60} 
	\setbeamercolor{bibliography entry note}{fg=htworange!60}  
	
	\setbeamertemplate{itemize item}{\color{black}$\bullet$}
	\setbeamertemplate{itemize subitem}{\color{black}--}
	\setbeamertemplate{itemize subsubitem}{\color{black}$\bullet$}
	\makeatother
	\setbeamertemplate{footline}
	{
	\leavevmode
	\def\arraystretch{1.2}
	\arrayrulecolor{gray}
	\begin{tabular}{ p{0.167\textwidth} | p{0.491\textwidth} | p{0.089\textwidth} | p{0.103\textwidth}}
	\hline
	\strut\insertshortauthor & \insertshorttitle & Slide \insertframenumber{}% / \inserttotalframenumber{}
	 & May 4, 2016\\
	\end{tabular}
	}
	\setbeamertemplate{headline}
	{
	\leavevmode
	\setlength{\arrayrulewidth}{1pt}
	\hspace*{2em}	
	\begin{tabular}{p{0.63\textwidth}}
	\rule{0pt}{3em}\normalsize{\textbf{\insertsection\strut}}\\
	\arrayrulecolor{htworange}
	\hline
	\end{tabular}
	\begin{tabular}{l}
	\rule{0pt}{4em}\includegraphics[width=3.25cm]{\customDir .LaTeX_master/HTW_GESAMTLOGO_CMYK.eps}\\
	\end{tabular}
	}
	\makeatletter	
}{	
	%-- Für Spicker einiges anders:
	\ifthenelse{\equal{\customDocumentClass}{cheatsheet}}{
		\documentclass[a4paper,10pt,landscape]{scrartcl}
		\usepackage{geometry}
		\geometry{top=2mm, bottom=2mm, headsep=0mm, footskip=0mm, left=2mm, right=2mm}
		
		% Für Spicker \spsection für Section, zur Strukturierung \HRule oder \HDRule Linie einsetzen
		\usepackage{multicol}
		\newcommand{\spsection}[1]{\textbf{#1}}	% Platzsparende "section" für Spicker
	}{	%-- Ende Spicker-Unterscheidung-if
		%-- Unterscheidung Doppelseitig
		\ifthenelse{\boolean{twosided}}{
			\documentclass[a4paper,11pt, footheight=26pt,twoside]{\customDocumentClass}
			\usepackage[head=23pt]{geometry}	% head=23pt umgeht Fehlerwarnung, dafür größeres "top" in geometry
			\geometry{top=30mm, bottom=22mm, headsep=10mm, footskip=12mm, inner=27mm, outer=13mm}
		}{
			\documentclass[a4paper,11pt, footheight=26pt]{\customDocumentClass}
			\usepackage[head=23pt]{geometry}	% head=23pt umgeht Fehlerwarnung, dafür größeres "top" in geometry
			\geometry{top=30mm, bottom=22mm, headsep=10mm, footskip=12mm, left=20mm, right=20mm}
		}
		%-- Nummerierung bis Subsubsection für Report
		\ifthenelse{\equal{\customDocumentClass}{report} \OR \equal{\customDocumentClass}{scrreprt}}{
			\setcounter{secnumdepth}{3}	% zählt auch subsubsection
			\setcounter{tocdepth}{3}	% Inhaltsverzeichnis bis in subsubsection
		}{}
	}%-- Ende Spicker-Unterscheidung-else
	
	\usepackage{scrlayer-scrpage}	% Kopf-/Fußzeile
	\renewcommand*{\thefootnote}{\fnsymbol{footnote}}	% Fußnoten-Symbole anstatt Zahlen
	\renewcommand*{\titlepagestyle}{empty} % Keine Seitennummer auf Titelseite
	\usepackage[perpage]{footmisc}	% Fußnotenzählung Seitenweit, nicht Dokumentenweit
}

% Input inkl. Umlaute, Silbentrennung
\RequirePackage[T1]{fontenc}
\RequirePackage[utf8]{inputenc}
\usepackage[english,ngerman]{babel}
\usepackage{csquotes}	% Anführungszeichen
\RequirePackage{marvosym}
\usepackage{eurosym}

% Style-Aufhübschung
\usepackage{soul, color}	% Kapitälchen, Unterstrichen, Durchgestrichen usw. im Text
%\usepackage{titleref}
\usepackage[breakwords, fit]{truncate}	% Abschneiden von Sätzen
\renewcommand{\TruncateMarker}{\,…}

% Mathe usw.
\usepackage{amssymb}
\usepackage{amsthm}
\ifthenelse{\equal{\customDocumentClass}{beamer}}{}{
\usepackage[fleqn,intlimits]{amsmath}	% fleqn: align-Umgebung rechtsbündig; intlimits: Integralgrenzen immer ober-/unterhalb
}
%\usepackage{mathtools} % u.a. schönere underbraces
\usepackage{xcolor}
\usepackage{esint}	% Schönere Integrale, \oiint vorhanden
\everymath=\expandafter{\the\everymath\displaystyle}	% Mathe Inhalte werden weniger verkleinert
\usepackage{wasysym}	% mehr Symbole, bspw \lightning
%\renewcommand{\int}{\int\limits}
%\usepackage{xfrac}	% mehr fracs: sfrac{}{}
\let\oldemptyset\emptyset	% schöneres emptyset
\let\emptyset\varnothing
%\RequirePackage{mathabx}	% mehr Symbole
\mathchardef\mhyphen="2D	% Hyphen in Math

% tikz usw.
\usepackage{tikz}
\usepackage{pgfplots}
\pgfplotsset{compat=1.11}	% Umgeht Fehlermeldung
\usetikzlibrary{graphs}
%\usetikzlibrary{through}	% ???
\usetikzlibrary{arrows}
\usetikzlibrary{arrows.meta}	% Pfeile verändern / vergrößern: \draw[-{>[scale=1.5]}] (-3,5) -> (-3,3);
\usetikzlibrary{automata,positioning} % Zeilenumbruch im Node node[align=center] {Text\\nächste Zeile} automata für Graphen
\usetikzlibrary{matrix}
\usetikzlibrary{patterns}	% Schraffierte Füllung
\usetikzlibrary{shapes.geometric}	% Polygon usw.
\tikzstyle{reverseclip}=[insert path={	% Inverser Clip \clip
	(current page.north east) --
	(current page.south east) --
	(current page.south west) --
	(current page.north west) --
	(current page.north east)}
% Nutzen: 
%\begin{tikzpicture}[remember picture]
%\begin{scope}
%\begin{pgfinterruptboundingbox}
%\draw [clip] DIE FLÄCHE, IN DER OBJEKT NICHT ERSCHEINEN SOLL [reverseclip];
%\end{pgfinterruptboundingbox}
%\draw DAS OBJEKT;
%\end{scope}
%\end{tikzpicture}
]	% Achtung: dafür muss doppelt kompliert werden!
\usepackage{graphpap}	% Grid für Graphen
\tikzset{every state/.style={inner sep=2pt, minimum size=2em}}
\usetikzlibrary{mindmap, backgrounds}
%\usepackage{tikz-uml}	% braucht Dateien: http://perso.ensta-paristech.fr/~kielbasi/tikzuml/

% Tabular
\usepackage{longtable}	% Große Tabellen über mehrere Seiten
\usepackage{multirow}	% Multirow/-column: \multirow{2[Anzahl der Zeilen]}{*[Format]}{Test[Inhalt]} oder \multicolumn{7[Anzahl der Reihen]}{|c|[Format]}{Test2[Inhalt]}
\renewcommand{\arraystretch}{1.3} % Tabellenlinien nicht zu dicht
\usepackage{colortbl}
\arrayrulecolor{gray}	% heller Tabellenlinien
\usepackage{array}	% für folgende 3 Zeilen (für Spalten fester breite mit entsprechender Ausrichtung):
\newcolumntype{L}[1]{>{\raggedright\let\newline\\\arraybackslash\hspace{0pt}}m{\dimexpr#1\columnwidth-2\tabcolsep-1.5\arrayrulewidth}}
\newcolumntype{C}[1]{>{\centering\let\newline\\\arraybackslash\hspace{0pt}}m{\dimexpr#1\columnwidth-2\tabcolsep-1.5\arrayrulewidth}}
\newcolumntype{R}[1]{>{\raggedleft\let\newline\\\arraybackslash\hspace{0pt}}m{\dimexpr#1\columnwidth-2\tabcolsep-1.5\arrayrulewidth}}
\usepackage{caption}	% Um auch unbeschriftete Captions mit \caption* zu machen

% Nützliches
\usepackage{verbatim}	% u.a. zum auskommentieren via \begin{comment} \end{comment}
\usepackage{tabto}	% Tabs: /tab zum nächsten Tab oder /tabto{.5 \CurrentLineWidth} zur Stelle in der Linie
\NumTabs{6}	% Anzahl von Tabs pro Zeile zum springen
\usepackage{listings} % Source-Code mit Tabs
\usepackage{lstautogobble} 
\ifthenelse{\equal{\customDocumentClass}{beamer}}{}{
\usepackage{enumitem}	% Anpassung der enumerates
%\setlist[enumerate,1]{label=(\arabic*)}	% global andere Enum-Items
\renewcommand{\labelitemiii}{$\scriptscriptstyle ^\blacklozenge$} % global andere 3. Item-Aufzählungszeichen
}
\usepackage{letltxmacro} % neue Definiton von Grundbefehlen
% Nutzen:
%\LetLtxMacro{\oldemph}{\emph}
%\renewcommand{\emph}[1]{\oldemph{#1}}
\RequirePackage{xpatch}	% ua. Konkatenieren von Strings/Variablen (etoolbox)
\usepackage{xstring}	% String Operationen
\usepackage{minibox}	% Minibox anstatt \fbox{} für Boxen mit Zeilenumbruch


% Einrichtung von lst
\lstset{
basicstyle=\ttfamily, 
%mathescape=true, 
%escapeinside=^^, 
autogobble, 
tabsize=2,
basicstyle=\footnotesize\sffamily\color{black},
frame=single,
rulecolor=\color{lightgray},
numbers=left,
numbersep=5pt,
numberstyle=\tiny\color{gray},
commentstyle=\color{gray},
keywordstyle=\color{green},
stringstyle=\color{orange},
morecomment=[l][\color{magenta}]{\#}
showspaces=false,
showstringspaces=false,
breaklines=true,
literate=%
    {Ö}{{\"O}}1
    {Ä}{{\"A}}1
    {Ü}{{\"U}}1
    {ß}{{\ss}}1
    {ü}{{\"u}}1
    {ä}{{\"a}}1
    {ö}{{\"o}}1
    {~}{{\textasciitilde}}1
}
\usepackage{scrhack} % Fehler umgehen
\def\ContinueLineNumber{\lstset{firstnumber=last}} % vor lstlisting. Zum wechsel zum nicht-kontinuierlichen muss wieder \StartLineAt1 eingegeben werden
\def\StartLineAt#1{\lstset{firstnumber=#1}} % vor lstlisting \StartLineAt30 eingeben, um bei Zeile 30 zu starten
\let\numberLineAt\StartLineAt

% BibTeX
\usepackage[bibencoding=ascii,
%backend=bibtex8,
%style=authortitle, citestyle=authortitle-ibid,
%doi=false,
%isbn=false,
%url=false
]{biblatex}	% BibTeX
\usepackage{makeidx}
%\makeglossary
%\makeindex

% Grafiken
\usepackage{graphicx}
\usepackage{epstopdf}	% eps-Vektorgrafiken einfügen
\usepackage{transparent}	% transparent nutzen: {\transparent{0.4} ...}
%\epstopdfsetup{outdir=\customDir}
% Prüft, ob Grafik existiert (mit \ifvalidimage{}{}) [Quelle: https://tex.stackexchange.com/a/99176]:
\makeatletter
\newif\ifgraphicexist
\catcode`\*=11
\newcommand\ifvalidimage[1]{%
    \begingroup
    \global\graphicexisttrue
    \let\input@path\Ginput@path
    \filename@parse{#1}%
    \ifx\filename@ext\relax
    \@for\Gin@temp:=\Gin@extensions\do{%
        \ifx\Gin@ext\relax
        \Gin@getbase\Gin@temp
        \fi}%
    \else
    \Gin@getbase{\Gin@sepdefault\filename@ext}%
    \ifx\Gin@ext\relax
    \global\graphicexistfalse
    \def\Gin@base{\filename@area\filename@base}%
    \edef\Gin@ext{\Gin@sepdefault\filename@ext}%
    \fi
    \fi
    \ifx\Gin@ext\relax
    \global\graphicexistfalse
    \else 
    \@ifundefined{Gin@rule@\Gin@ext}%
    {\global\graphicexistfalse}%
    {}%
    \fi  
    \ifx\Gin@ext\relax 
    \gdef\imageextension{unknown}%
    \else
    \xdef\imageextension{\Gin@ext}%
    \fi 
    \endgroup 
    \ifgraphicexist
    \expandafter \@firstoftwo
    \else
    \expandafter \@secondoftwo
    \fi 
} 
\catcode`\*=12
\makeatother
\usepackage{letltxmacro}	% Latex-Befehle unter anderem Namen neu definieren
\LetLtxMacro{\forceincludegraphics}{\includegraphics}	% neuer Befehl für includegraphics
\renewcommand{\includegraphics}[2][]{	% altes includegraphics neu definieren, damit es auch nicht vorhandene einfügt
\ifvalidimage{#2}{
\forceincludegraphics[#1]{#2}
}{
\message{Achtung: Grafik wurde nicht gefunden: '#2'}
\minibox[frame]{
\textbf{\StrSubstitute{#2}{_}{\_}}  \ifthenelse{\isempty{#1}}{}{\\\textit{#1}}}
}}

% pdf-Setup
\usepackage{pdfpages}
\ifthenelse{\equal{\customDocumentClass}{beamer}}{}{
\usepackage[bookmarks,%
bookmarksopen=false,% Klappt die Bookmarks in Acrobat aus
colorlinks=true,%
linkcolor=black,%
citecolor=red,%
urlcolor=green,%
]{hyperref}
}

%-- Unterscheidung des Stils
\newcommand{\customLogo}{}
\newcommand{\customPreamble}{}
\ifthenelse{\equal{\customDesign}{htw}}{
	% HTW Corporate Design: Arial (Helvetica)
	\usepackage{helvet}
	\renewcommand{\familydefault}{\sfdefault}
	\renewcommand{\customLogo}{HTW-Logo}
	\renewcommand{\customPreamble}{HTW Dresden}
}{
% \renewcommand{\customLogo}{HTW-Logo.eps}
}

% Nach Dokumentenbeginn ausführen:
\AtBeginDocument{
	% Autor und Titel für pdf-Eigenschaften festlegen, falls noch nicht geschehen
	\providecommand{\pdfAuthor}{John Doe}
	\ifdefempty{\customAuthor} {} {\renewcommand{\pdfAuthor}{\customAuthor}}
	\providecommand{\pdfTitle}{}
	\providecommand{\pdfTitleA}{}
	\providecommand{\pdfTitleB}{}
	\providecommand{\pdfTitleC}{}	
	\ifdefempty{\pdfTitle}{
		\ifdefempty{\customPreamble} {} {\renewcommand{\pdfTitleA}{\customPreamble{} | }}
		\ifdefempty{\customTitle} {\renewcommand{\pdfTitleB}{No Title}} {\renewcommand{\pdfTitleB}{\customTitle}}
		\ifdefempty{\customSubtitle} {} {\renewcommand{\pdfTitleC}{ - \customSubtitle}}
	}{}
	
	\newcommand{\customLogoLocation}{\customDir .LaTeX_master/\customLogo}
	\hypersetup{
		pdfauthor={\pdfAuthor},
		pdftitle={\pdfTitleA\pdfTitleB\pdfTitleC},
	}
	\ifthenelse{\equal{\customDocumentClass}{beamer}}{
		\title{\customTitle}
		\author{\customAuthor}
	}{
		\automark[section]{section}
		\automark*[subsection]{subsection}
		\pagestyle{scrheadings}
		\ifthenelse{\equal{\customDocumentClass}{report} \OR \equal{\customDocumentClass}{scrreprt}}{
		\renewcommand*{\chapterpagestyle}{scrheadings}
		}{}
		%\renewcommand*{\titlepagestyle}{scrheadings}
		\ihead{\includegraphics[height=1.7em]{\customLogoLocation}}
		%\ohead{\truncate{4cm}{\customTitle}}
		\chead{\truncate{.5\textwidth}{\headmark}}
		\ohead{\customTitle}
		\cfoot{\pagemark}
		\ofoot{\customSignature}
		% Titelseite
		\title{
		\includegraphics[width=0.35\textwidth]{\customDir .LaTeX_master/\customLogo}\\\vspace{0.5em}
		\Huge\textbf{\customTitle}
		\ifdefempty{\customSubtitle} {} {\\\vspace*{0.7em}\Large \customSubtitle}
		\\\vspace*{5em}}
		\author{
		\ifdefempty{\customNoteA} {} {\customNoteA \vspace*{1em}}\\ 
		\ifdefempty{\customAuthor} {} {\customTitleAuthor}
		\ifdefempty{\customNoteB}{}{\vspace*{1em}\\\customNoteB}
		}
		
		\ifthenelse{\equal{\customDocumentClass}{cheatsheet}}{
			\pagestyle{empty}
			\setlist{nolistsep}
	%		\usepackage{parskip}	% Aufzählung Abstand
	%		\setlength{\parskip}{0em}
			\lstset{
	    belowcaptionskip=0pt,
	    belowskip=0pt,
	    aboveskip=0pt,
			tabsize=2,
			frame=none,
			numbers=none,
			showspaces=false,
			showstringspaces=false,
			breaklines=true,
			}
		}{}
	}
}

% Unterabschnitte
%\newtheorem{example}{Beispiel}%[section]
%\newtheorem{definition}{Definition}[section]
%\newtheorem{discussion}{Diskussion}[section]
%\newtheorem{remark}{Bemerkung}[section]
%\newtheorem{proof}{Beweis}[section]
%\newtheorem{notation}{Schreibweise}[section]
% LaTeX master Datei(en) zusammengestellt von Falk-Jonatan Strube zur Nutzung an der Hochschule für Technik und Wirtschaft Dresden: https://github.com/genericFJS/htw
\RequirePackage{xcolor}
\RequirePackage{amsmath}
\RequirePackage{letltxmacro}

% Horizontale Linie:
\newcommand{\HRule}[1][\medskipamount]{\par
  \vspace*{\dimexpr-\parskip-\baselineskip+#1}
  \noindent\rule[0.2ex]{\linewidth}{0.2mm}\par
  \vspace*{\dimexpr-\parskip-.5\baselineskip+#1}}
% Gestrichelte horizontale Linie:
\RequirePackage{dashrule}
\newcommand{\HDRule}[1][\medskipamount]{\par
  \vspace*{\dimexpr-\parskip-\baselineskip+#1}
  \noindent\hdashrule[0.2ex]{\linewidth}{0.2mm}{1mm} \par
  \vspace*{\dimexpr-\parskip-.5\baselineskip+#1}}
% Mathe in Anführungszeichen:
\newsavebox{\mathbox}\newsavebox{\mathquote}
\makeatletter
\newcommand{\mq}[1]{% \mathquotes{<stuff>}
  \savebox{\mathquote}{\text{"}}% Save quotes
  \savebox{\mathbox}{$\displaystyle #1$}% Save <stuff>
  \raisebox{\dimexpr\ht\mathbox-\ht\mathquote\relax}{"}#1\raisebox{\dimexpr\ht\mathbox-\ht\mathquote\relax}{''}
}
\makeatother

% Paragraph mit Zähler (Section-Weise)
\newcounter{cparagraphC}
\newcommand{\cparagraph}[1]{
\stepcounter{cparagraphC}
\paragraph{\thesection{}-\thecparagraphC{} #1}
%\addcontentsline{toc}{subsubsection}{\thesection{}-\thecparagraphC{} #1}
\label{\thesection-\thecparagraphC}
}
\makeatletter
\@addtoreset{cparagraphC}{section}
\makeatother


% (Vorlesungs-)Folien einbinden:
% Folien von einer Datei skaliert
\newcommand{\slide}[2][\customSlideScale]{\slides[#1]{}{#2}}
\newcommand{\slideTrim}[6][\customSlideScale]{\slides[#1 , clip,  trim = #5cm #4cm #6cm #3cm]{}{#2}}
% Folien von mehreren nummerierten Dateien skaliert
\newcommand{\slides}[3][\customSlideScale]{\begin{center}
\includegraphics[page=#3, scale=#1]{\customSlidePath #2.pdf}
\end{center}}

% \emph{} anders definieren
\makeatletter
\DeclareRobustCommand{\em}{%
  \@nomath\em \if b\expandafter\@car\f@series\@nil
  \normalfont \else \scshape \fi}
\makeatother

% unwichtiges
\newcommand{\unimptnt}[1]{{\transparent{0.5}#1}}

% alph. enumerate
\newenvironment{anumerate}{\begin{enumerate}[label=(\alph*)]}{\end{enumerate}} % Alphabetische Aufzählung

% Hanging parameters
\newcommand{\hangpara}[1]{\par\noindent\hangindent+2em\hangafter=1 #1\par\noindent}

%% EINFACHE BEFEHLE

% Abkürzungen Mathe
\newcommand{\EE}{\mathbb{E}}
\newcommand{\QQ}{\mathbb{Q}}
\newcommand{\RR}{\mathbb{R}}
\newcommand{\CC}{\mathbb{C}}
\newcommand{\NN}{\mathbb{N}}
\newcommand{\ZZ}{\mathbb{Z}}
\newcommand{\PP}{\mathbb{P}}
\renewcommand{\SS}{\mathbb{S}}
\newcommand{\cA}{\mathcal{A}}
\newcommand{\cB}{\mathcal{B}}
\newcommand{\cC}{\mathcal{C}}
\newcommand{\cD}{\mathcal{D}}
\newcommand{\cE}{\mathcal{E}}
\newcommand{\cF}{\mathcal{F}}
\newcommand{\cG}{\mathcal{G}}
\newcommand{\cH}{\mathcal{H}}
\newcommand{\cI}{\mathcal{I}}
\newcommand{\cJ}{\mathcal{J}}
\newcommand{\cM}{\mathcal{M}}
\newcommand{\cN}{\mathcal{N}}
\newcommand{\cP}{\mathcal{P}}
\newcommand{\cR}{\mathcal{R}}
\newcommand{\cS}{\mathcal{S}}
\newcommand{\cZ}{\mathcal{Z}}
\newcommand{\cL}{\mathcal{L}}
\newcommand{\cT}{\mathcal{T}}
\newcommand{\cU}{\mathcal{U}}
\newcommand{\cX}{\mathcal{X}}
\newcommand{\cV}{\mathcal{V}}
\renewcommand{\phi}{\varphi}
\renewcommand{\epsilon}{\varepsilon}
\renewcommand{\theta}{\vartheta}

% Verschiedene als Mathe-Operatoren
\DeclareMathOperator{\arccot}{arccot}
\DeclareMathOperator{\arccosh}{arccosh}
\DeclareMathOperator{\arcsinh}{arcsinh}
\DeclareMathOperator{\arctanh}{arctanh}
\DeclareMathOperator{\arccoth}{arccoth} 
\DeclareMathOperator{\var}{Var} % Varianz 
\DeclareMathOperator{\cov}{Cov} % Co-Varianz 

% Farbdefinitionen
\definecolor{red}{RGB}{180,0,0}
\definecolor{green}{RGB}{75,160,0}
\definecolor{blue}{RGB}{0,75,200}
\definecolor{orange}{RGB}{255,128,0}
\definecolor{yellow}{RGB}{255,245,0}
\definecolor{purple}{RGB}{75,0,160}
\definecolor{cyan}{RGB}{0,160,160}
\definecolor{brown}{RGB}{120,60,10}

\definecolor{itteny}{RGB}{244,229,0}
\definecolor{ittenyo}{RGB}{253,198,11}
\definecolor{itteno}{RGB}{241,142,28}
\definecolor{ittenor}{RGB}{234,98,31}
\definecolor{ittenr}{RGB}{227,35,34}
\definecolor{ittenrp}{RGB}{196,3,125}
\definecolor{ittenp}{RGB}{109,57,139}
\definecolor{ittenpb}{RGB}{68,78,153}
\definecolor{ittenb}{RGB}{42,113,176}
\definecolor{ittenbg}{RGB}{6,150,187}
\definecolor{itteng}{RGB}{0,142,91}
\definecolor{ittengy}{RGB}{140,187,38}

\definecolor{htworange}{RGB}{249,155,28}

% Textfarbe ändern
\newcommand{\tred}[1]{\textcolor{red}{#1}}
\newcommand{\tgreen}[1]{\textcolor{green}{#1}}
\newcommand{\tblue}[1]{\textcolor{blue}{#1}}
\newcommand{\torange}[1]{\textcolor{orange}{#1}}
\newcommand{\tyellow}[1]{\textcolor{yellow}{#1}}
\newcommand{\tpurple}[1]{\textcolor{purple}{#1}}
\newcommand{\tcyan}[1]{\textcolor{cyan}{#1}}
\newcommand{\tbrown}[1]{\textcolor{brown}{#1}}

% Umstellen der Tabellen Definition
\newcommand{\mpb}[1][.3]{\begin{minipage}{#1\textwidth}\vspace*{3pt}}
\newcommand{\mpe}{\vspace*{3pt}\end{minipage}}

\newcommand{\resultul}[1]{\underline{\underline{#1}}}
\newcommand{\parskp}{$ $\\}	% new line after paragraph
\newcommand{\corr}{\;\widehat{=}\;}
\newcommand{\mdeg}{^{\circ}}

\newcommand{\nok}[2]{\binom{#1}{#2}}	% n über k BESSER: \binom{n}{k}
\newcommand{\mtr}[1]{\begin{pmatrix}#1\end{pmatrix}}	% Matrix
\newcommand{\dtr}[1]{\begin{vmatrix}#1\end{vmatrix}}	% Determinante (Betragsmatrix)
\LetLtxMacro{\originalVec}{\vec}
\renewcommand{\vec}[1]{\underline{#1}}	% Vektorschreibweise
\newcommand{\imptnt}[1]{\colorbox{red!30}{#1}}	% Wichtiges
\newcommand{\intd}[1]{\,\mathrm{d}#1}
\newcommand{\diffd}[1]{\mathrm{d}#1}
% für Module-Rechnung: \pmod{}
\newcommand{\unit}[1]{\,\mathrm{#1}}
\LetLtxMacro{\ntilde}{\tilde}
\renewcommand{\tilde}{\widetilde}
\newcommand{\gdw}{genau dann wenn}
\newcommand{\lecdate}[1]{\begin{flushright}\textcolor{gray}{Vorlesung am #1}\end{flushright}}

%\bibliography{\customDir _Literatur/HTW_Literatur.bib}
%\setlength{\headheight}{10mm}	% default: ca. 8mm
\setlength{\footheight}{10mm}	% default: ca. 8mm


\usepackage{tikz-uml}

% TeXDoclet Compatibility:
\makeatletter
\DeclareOldFontCommand{\rm}{\normalfont\rmfamily}{\mathrm}
\DeclareOldFontCommand{\sf}{\normalfont\sffamily}{\mathsf}
\DeclareOldFontCommand{\tt}{\normalfont\ttfamily}{\mathtt}
\DeclareOldFontCommand{\bf}{\normalfont\bfseries}{\mathbf}
\DeclareOldFontCommand{\it}{\normalfont\itshape}{\mathit}
\DeclareOldFontCommand{\sl}{\normalfont\slshape}{\@nomath\sl}
\DeclareOldFontCommand{\sc}{\normalfont\scshape}{\@nomath\sc}
\makeatother

% TeXDoclet Preamble:
%\usepackage{color}
\usepackage{ifthen}
\usepackage{ifpdf}
\usepackage[headings]{fullpage}
\usepackage{listings}
\lstset{language=Java,breaklines=true}
\ifpdf \usepackage[pdftex, pdfpagemode={UseOutlines},bookmarks,colorlinks,linkcolor={blue},plainpages=false,pdfpagelabels,citecolor={red},breaklinks=true]{hyperref}
  \usepackage[pdftex]{graphicx}
  \pdfcompresslevel=9
  \DeclareGraphicsRule{*}{mps}{*}{}
\else
  \usepackage[dvips]{graphicx}
\fi

\newcommand{\entityintro}[3]{%
  \hbox to \hsize{%
    \vbox{%
      \hbox to .2in{}%
    }%
    {\bf  #1}%
    \dotfill\pageref{#2}%
  }
  \makebox[\hsize]{%
    \parbox{.4in}{}%
    \parbox[l]{5in}{%
      \vspace{1mm}%
      #3%
      \vspace{1mm}%
    }%
  }%
}
\newcommand{\refdefined}[1]{
\expandafter\ifx\csname r@#1\endcsname\relax
\relax\else
{$($in \ref{#1}, page \pageref{#1}$)$}\fi}
\date{null}
\chardef\textbackslash=`\\

\usepackage{ifpdf}
\lstset{language=Java,breaklines=true}

\newcommand{\entityintro}[3]{%
  \hbox to \hsize{%
    \vbox{%
      \hbox to .2in{}%
    }%
    {\bf  #1}%
    \dotfill\pageref{#2}%
  }
  \makebox[\hsize]{%
    \parbox{.4in}{}%
    \parbox[l]{5in}{%
      \vspace{1mm}%
      #3%
      \vspace{1mm}%
    }%
  }%
}
\newcommand{\refdefined}[1]{
\expandafter\ifx\csname r@#1\endcsname\relax
\relax\else
{$($in \ref{#1}, page \pageref{#1}$)$}\fi}
\chardef\textbackslash=`\\

\begin{document}

%\selectlanguage{english}
\maketitle
\newpage
\tableofcontents
\newpage

\chapter{Dokumentation}

\section{Rahmenaufgaben}

Es gibt folgende Rahmenaufgaben, die abseits des Programmierens gelöst werden sollen:
\begin{enumerate}
\setcounter{enumi}{7}
\item Errechneter maximal erzielbarer Durchsatz beim SW-Protokoll bei $10\%$ Paketverlust und $=10\unit{ms}$ Verzögerung:\\
Datenlänge: $1400 \unit{Byte} = 11\,200\unit{Bit}$\\
$R=1$ (Vereinfachung)\\
$r_b$: Geschwindigkeit\\
$P_{CS}=P_{SC}=0,1$\\
$T_{CS}=T_{SC}=10\unit{Bit}/r_b$\\
$T_P=(16+8)\unit{Bit}/r_b+11\,200\unit{Bit}/r_b=11\,224\unit{Bit}/r_b$ \\
$T_W=T_{CS}+T_{SC}+T_{ACK}=10\unit{s}+10\unit{s}+(16+8)\unit{Bit}/r_b=20\unit{s}+24\unit{Bit}/r_b$
\begin{align*}
\eta_{SW}&=\frac{T_P}{T_P+ T_W}\cdot (1-P_{CS})\cdot(1-P_{SC}) \cdot R \\
&= \frac{11\,224\unit{Bit}/r_b}{11\,224\unit{Bit}/r_b+20\unit{s}+24\unit{Bit}/r_b}\cdot 0,9^2\\
&=\frac{11\,224\unit{Bit}/r_b}{11\,248\unit{Bit}/r_b+20\unit{s}}\cdot 0,9^2
\end{align*}
Tatsächlich erzielter Durchsatz:
$$deutlich\;geringer$$
Der Unterschied ist damit zu erklären, dass vor allem das Programm nicht optimiert geschrieben ist.
\item Dokumentation der Funktionen unter Nutzung von \LaTeX{}.\\
Siehe vor allem Abschnitt \ref{stated} für die Zustandsdiagramme, Abschnitt \ref{programing} für Details zur Implementierung (dabei ergeben sich Probleme/Limitierungen/Verbesserungsvorschläge implizit aus dem Unterabschnitt \ref{abgrenz})  und Kapitel \ref{javadoc} für die Funktionen. Zugunsten der besseren Lesbarkeit im Quellcode wurde die Dokumentation der Funktionen in Englisch gehalten.
\end{enumerate}

\section{Programmierung}
\label{programing}

\subsection{Anforderungen}
Die Programmierung wurde nach den Anforderungen der Aufgabenstellung gelöst. Diese lauten wie folgt:
\begin{itemize}
\item Client:
\begin{itemize}
\item Über Konsole mit Parametern „Zieladresse“, „Portnummer“ und „Dateiname“ aufrufbar (zum debuggen auch „Paketverlust“ und „mittlere Verzögerung“)
\item Zeigt während der Übertragung jede Sekunde die Datenrate an
\item Zeigt nach der Übertragung die Datenrate an
\item Korrigiert Fehler bei verlorenen oder vertauschten Paketen:\\
verlorenes ACK: Datenpaket erneut senden, vertauschtes ACK: vorhergehendes Datenpaket senden.
\end{itemize}
\item Server:
\begin{itemize}
\item Über Konsole mit dem Parameter „Portnummer“ aufrufbar (zum debuggen auch „Paketverlust“ und „mittlere Verzögerung“)
\item Speichert Datei in seinem Pfad unter dem korrekten Dateiname (Zeichen „1“ wird angehängt, wenn Datei bereits existiert)
\item Korrigiert Fehler bei verlorenen oder vertauschten Paketen:\\
verlorenes Datenpaket: kein ACK senden, vertauschtes Datenpaket: ACK des vorhergehenden Datenpakets senden.
\end{itemize}
\end{itemize}
\subsection{Eigenschaften}
Folgende Eigenschaften wurden im Rahmen der Aufgabenstellung angepasst:
\begin{itemize}
\item Größe der Datenpakete: $1400 \unit{Byte}\;(+\text{Sessionnr., Paketnr. und ggf. CRC})$\footnote{Basierend auf dem TCP Standard von maxim $1500 \unit{Byte}$.}.
\item Als der „Pfad des Servers“ wurde der Pfad zur \texttt{server-udp}-Datei (Ordner \texttt{s74053}) angenommen, nicht der \texttt{/bin}-Pfad.
\end{itemize}

\subsection{Abgrenzungen}
\label{abgrenz}
In der Aufgabenstellung wurde über einige Merkmale keine Aussage getroffen. Es wurde daher angenommen, dass sie nicht relevant sind. Die Merkmale, die nicht enthalten sind (aber ggf. zur Funktionalität beitragen würden), sind hier aufgeführt.
\begin{itemize}
\item Der Server kann nicht mehrere Clients auf einmal bedienen.
\item Das Programm ist nicht „portabel“. Das heißt, dass es nur bei genau der Unterordneranordnung des Ordners \texttt{s74053} funktioniert.
\item Es wird sich darauf verlassen, dass kein Dateiname übergeben wird, der länger als 255 Byte lang ist.
\item Der Client kann nicht wisse, ob die Datei tatsächlich fertig übertragen wurde, wenn das letzte ACK-Paket vom Server verloren geht.
\end{itemize}

\section{Zustandsdiagramme}
\label{stated}
\subsection{Client}
\begin{center}
\begin{tikzpicture}
\umlstateinitial[y=3,name=start]
\umlbasicstate[y=0,name=startupload, fill=white]{Initialisiere Upload}
\umltrans{start}{startupload}
\umlstatefinal[x=6,y=-2,name=finalerror]
\umlHVtrans[arg={Fehler},pos=.5, anchor2=90]{startupload}{finalerror}
%\umlbasicstate[y=-4,name=sendfirst, fill=white]{Sende erstes Paket}
%\umltrans[arg={Ok},left, pos=.5]{startupload}{sendfirst}
%\umltrans[arg={Timeout},pos=.5]{sendfirst}{finalerror}
\umlbasicstate[y=-4,name=sendpacket, fill=white]{Sende Paket}
\umltrans[arg={Ok},left, pos=.5]{startupload}{sendpacket}
\umlHVtrans[arg={Timeout mit max. Anzahl Versuche},pos=.1, below right, anchor2=-90]{sendpacket}{finalerror}
\umltrans[recursive=160|200|2.5cm, recursive direction = right to right, left, arg={ACK empfangen || Timeout},pos=1.5]{sendpacket}{sendpacket}
\umlstatefinal[y=-7,name=finalok]
\umltrans[arg={ACK zu letztem Paket empfangen}, pos=.5, left]{sendpacket}{finalok}
\end{tikzpicture}
\end{center}

\subsection{Server}
\begin{center}
\begin{tikzpicture}
\umlstateinitial[y=3,name=start]
\umlbasicstate[y=0,name=waitpacket, fill=white]{Warte auf Paket}
\umltrans{start}{startupload}
%\umlbasicstate[y=-4,name=sendfirst, fill=white]{Sende erstes Paket}
%\umltrans[arg={Ok},left, pos=.5]{startupload}{sendfirst}
%\umltrans[arg={Timeout},pos=.5]{sendfirst}{finalerror}
\umlbasicstate[y=-4,name=processpacket, fill=white]{Verarbeite Paket}
\umltrans[arg={Paket empfangen},left, anchor1=-110, anchor2=110, pos=.5]{startupload}{processpacket}
\umlbasicstate[y=-8,name=sendack, fill=white]{Sende ACK}
\umltrans[arg={Ok || Falsches Paket}, left, pos=.5]{processpacket}{sendack}
\umltrans[arg={Fehler},right, anchor1=70, anchor2=-70, pos=.5]{processpacket}{waitpacket}
\umlHVHtrans[right, arg={},pos=1.5,anchor1=180, anchor2=180, arm1=-2.5cm]{sendack}{startupload}
\umlbasicstate[y=-12,name=processfile, fill=white]{Verarbeite Datei}
\umltrans[arg={Paket hat Datei-CRC}, left, pos=.5, anchor1=-110, anchor2=110]{sendack}{processfile}
\umlHVHtrans[right, arg={Fehler},pos=1.5,anchor1=0, anchor2=0, arm1=1.5cm]{processfile}{startupload}
\umltrans[arg={Ok},right, anchor1=70, anchor2=-70, pos=.5]{processfile}{sendack}
\end{tikzpicture}
\end{center}

\chapter{JavaDoc}
\selectlanguage{english}
\label{javadoc}
\section*{Class Hierarchy}{
%\thispagestyle{empty}
\markboth{Class Hierarchy}{Class Hierarchy}
\addcontentsline{toc}{section}{Class Hierarchy}
\subsection*{Classes}
{\raggedright
\hspace{0.0cm} $\bullet$ java.lang.Object {\tiny \refdefined{java.lang.Object}} \\
\hspace{1.0cm} $\bullet$ fjs.filetransfer.udp.FileTransfer {\tiny \refdefined{fjs.filetransfer.udp.FileTransfer}} \\
\hspace{2.0cm} $\bullet$ fjs.filetransfer.udp.Client {\tiny \refdefined{fjs.filetransfer.udp.Client}} \\
\hspace{2.0cm} $\bullet$ fjs.filetransfer.udp.Server {\tiny \refdefined{fjs.filetransfer.udp.Server}} \\
\hspace{1.0cm} $\bullet$ fjs.filetransfer.udp.packets.Packet {\tiny \refdefined{fjs.filetransfer.udp.packets.Packet}} \\
\hspace{2.0cm} $\bullet$ fjs.filetransfer.udp.packets.AckPacket {\tiny \refdefined{fjs.filetransfer.udp.packets.AckPacket}} \\
\hspace{2.0cm} $\bullet$ fjs.filetransfer.udp.packets.DataPacket {\tiny \refdefined{fjs.filetransfer.udp.packets.DataPacket}} \\
\hspace{2.0cm} $\bullet$ fjs.filetransfer.udp.packets.FirstPacket {\tiny \refdefined{fjs.filetransfer.udp.packets.FirstPacket}} \\
}
}
\section{Package fjs.filetransfer.udp}{
\label{fjs.filetransfer.udp}\hypertarget{fjs.filetransfer.udp}{}
\subsection{\label{fjs.filetransfer.udp.Client}Class Client}{
\hypertarget{fjs.filetransfer.udp.Client}{}\vskip .1in 
Client UDP\vskip .1in 
\subsubsection{Declaration}{
\begin{lstlisting}[frame=none]
public class Client
 extends fjs.filetransfer.udp.FileTransfer\end{lstlisting}
\subsubsection{Constructor summary}{
\begin{verse}
\hyperlink{fjs.filetransfer.udp.Client(java.lang.String[])}{{\bf Client(String\lbrack \rbrack )}} \\
\end{verse}
}
\subsubsection{Method summary}{
\begin{verse}
\hyperlink{fjs.filetransfer.udp.Client.getFileName()}{{\bf getFileName()}} \\
\hyperlink{fjs.filetransfer.udp.Client.getPacketDelay()}{{\bf getPacketDelay()}} \\
\hyperlink{fjs.filetransfer.udp.Client.getPacketLossRate()}{{\bf getPacketLossRate()}} \\
\hyperlink{fjs.filetransfer.udp.Client.getPort()}{{\bf getPort()}} \\
\hyperlink{fjs.filetransfer.udp.Client.getServer()}{{\bf getServer()}} \\
\hyperlink{fjs.filetransfer.udp.Client.isDebug()}{{\bf isDebug()}} \\
\hyperlink{fjs.filetransfer.udp.Client.main(java.lang.String[])}{{\bf main(String\lbrack \rbrack )}} \\
\hyperlink{fjs.filetransfer.udp.Client.setDebug(boolean)}{{\bf setDebug(boolean)}} \\
\hyperlink{fjs.filetransfer.udp.Client.setFileName(java.lang.String)}{{\bf setFileName(String)}} \\
\hyperlink{fjs.filetransfer.udp.Client.setPacketDelay(int)}{{\bf setPacketDelay(int)}} \\
\hyperlink{fjs.filetransfer.udp.Client.setPacketLossRate(float)}{{\bf setPacketLossRate(float)}} \\
\hyperlink{fjs.filetransfer.udp.Client.setPort(int)}{{\bf setPort(int)}} \\
\hyperlink{fjs.filetransfer.udp.Client.setServer(java.lang.String)}{{\bf setServer(String)}} \\
\end{verse}
}
\subsubsection{Constructors}{
\vskip -2em
\begin{itemize}
\item{ 
\index{Client(String\lbrack \rbrack )}
\hypertarget{fjs.filetransfer.udp.Client(java.lang.String[])}{{\bf  Client}\\}
\begin{lstlisting}[frame=none]
public Client(java.lang.String[] args)\end{lstlisting} %end signature
}%end item
\end{itemize}
}
\subsubsection{Methods}{
\vskip -2em
\begin{itemize}
\item{ 
\index{getFileName()}
\hypertarget{fjs.filetransfer.udp.Client.getFileName()}{{\bf  getFileName}\\}
\begin{lstlisting}[frame=none]
public java.lang.String getFileName()\end{lstlisting} %end signature
}%end item
\item{ 
\index{getPacketDelay()}
\hypertarget{fjs.filetransfer.udp.Client.getPacketDelay()}{{\bf  getPacketDelay}\\}
\begin{lstlisting}[frame=none]
public int getPacketDelay()\end{lstlisting} %end signature
}%end item
\item{ 
\index{getPacketLossRate()}
\hypertarget{fjs.filetransfer.udp.Client.getPacketLossRate()}{{\bf  getPacketLossRate}\\}
\begin{lstlisting}[frame=none]
public float getPacketLossRate()\end{lstlisting} %end signature
}%end item
\item{ 
\index{getPort()}
\hypertarget{fjs.filetransfer.udp.Client.getPort()}{{\bf  getPort}\\}
\begin{lstlisting}[frame=none]
public int getPort()\end{lstlisting} %end signature
}%end item
\item{ 
\index{getServer()}
\hypertarget{fjs.filetransfer.udp.Client.getServer()}{{\bf  getServer}\\}
\begin{lstlisting}[frame=none]
public java.lang.String getServer()\end{lstlisting} %end signature
}%end item
\item{ 
\index{isDebug()}
\hypertarget{fjs.filetransfer.udp.Client.isDebug()}{{\bf  isDebug}\\}
\begin{lstlisting}[frame=none]
public boolean isDebug()\end{lstlisting} %end signature
}%end item
\item{ 
\index{main(String\lbrack \rbrack )}
\hypertarget{fjs.filetransfer.udp.Client.main(java.lang.String[])}{{\bf  main}\\}
\begin{lstlisting}[frame=none]
public static void main(java.lang.String[] args)\end{lstlisting} %end signature
}%end item
\item{ 
\index{setDebug(boolean)}
\hypertarget{fjs.filetransfer.udp.Client.setDebug(boolean)}{{\bf  setDebug}\\}
\begin{lstlisting}[frame=none]
public void setDebug(boolean debug)\end{lstlisting} %end signature
}%end item
\item{ 
\index{setFileName(String)}
\hypertarget{fjs.filetransfer.udp.Client.setFileName(java.lang.String)}{{\bf  setFileName}\\}
\begin{lstlisting}[frame=none]
public void setFileName(java.lang.String fileName)\end{lstlisting} %end signature
}%end item
\item{ 
\index{setPacketDelay(int)}
\hypertarget{fjs.filetransfer.udp.Client.setPacketDelay(int)}{{\bf  setPacketDelay}\\}
\begin{lstlisting}[frame=none]
public void setPacketDelay(int packetDelay)\end{lstlisting} %end signature
}%end item
\item{ 
\index{setPacketLossRate(float)}
\hypertarget{fjs.filetransfer.udp.Client.setPacketLossRate(float)}{{\bf  setPacketLossRate}\\}
\begin{lstlisting}[frame=none]
public void setPacketLossRate(float packetLossRate)\end{lstlisting} %end signature
}%end item
\item{ 
\index{setPort(int)}
\hypertarget{fjs.filetransfer.udp.Client.setPort(int)}{{\bf  setPort}\\}
\begin{lstlisting}[frame=none]
public void setPort(int port)\end{lstlisting} %end signature
}%end item
\item{ 
\index{setServer(String)}
\hypertarget{fjs.filetransfer.udp.Client.setServer(java.lang.String)}{{\bf  setServer}\\}
\begin{lstlisting}[frame=none]
public void setServer(java.lang.String server)\end{lstlisting} %end signature
}%end item
\end{itemize}
}
}
\subsection{\label{fjs.filetransfer.udp.FileTransfer}Class FileTransfer}{
\hypertarget{fjs.filetransfer.udp.FileTransfer}{}\vskip .1in 
\subsubsection{Declaration}{
\begin{lstlisting}[frame=none]
public class FileTransfer
 extends java.lang.Object\end{lstlisting}
\subsubsection{All known subclasses}{Server\small{\refdefined{fjs.filetransfer.udp.Server}}, Client\small{\refdefined{fjs.filetransfer.udp.Client}}}
\subsubsection{Constructor summary}{
\begin{verse}
\hyperlink{fjs.filetransfer.udp.FileTransfer()}{{\bf FileTransfer()}} \\
\end{verse}
}
\subsubsection{Constructors}{
\vskip -2em
\begin{itemize}
\item{ 
\index{FileTransfer()}
\hypertarget{fjs.filetransfer.udp.FileTransfer()}{{\bf  FileTransfer}\\}
\begin{lstlisting}[frame=none]
public FileTransfer()\end{lstlisting} %end signature
}%end item
\end{itemize}
}
}
\subsection{\label{fjs.filetransfer.udp.Server}Class Server}{
\hypertarget{fjs.filetransfer.udp.Server}{}\vskip .1in 
\subsubsection{Declaration}{
\begin{lstlisting}[frame=none]
public class Server
 extends fjs.filetransfer.udp.FileTransfer\end{lstlisting}
\subsubsection{Constructor summary}{
\begin{verse}
\hyperlink{fjs.filetransfer.udp.Server()}{{\bf Server()}} \\
\end{verse}
}
\subsubsection{Constructors}{
\vskip -2em
\begin{itemize}
\item{ 
\index{Server()}
\hypertarget{fjs.filetransfer.udp.Server()}{{\bf  Server}\\}
\begin{lstlisting}[frame=none]
public Server()\end{lstlisting} %end signature
}%end item
\end{itemize}
}
}
}
\section{Package fjs.filetransfer.udp.packets}{
\label{fjs.filetransfer.udp.packets}\hypertarget{fjs.filetransfer.udp.packets}{}
\subsection{\label{fjs.filetransfer.udp.packets.AckPacket}Class AckPacket}{
\hypertarget{fjs.filetransfer.udp.packets.AckPacket}{}\vskip .1in 
\subsubsection{Declaration}{
\begin{lstlisting}[frame=none]
public class AckPacket
 extends fjs.filetransfer.udp.packets.Packet\end{lstlisting}
\subsubsection{Constructor summary}{
\begin{verse}
\hyperlink{fjs.filetransfer.udp.packets.AckPacket()}{{\bf AckPacket()}} \\
\end{verse}
}
\subsubsection{Constructors}{
\vskip -2em
\begin{itemize}
\item{ 
\index{AckPacket()}
\hypertarget{fjs.filetransfer.udp.packets.AckPacket()}{{\bf  AckPacket}\\}
\begin{lstlisting}[frame=none]
public AckPacket()\end{lstlisting} %end signature
}%end item
\end{itemize}
}
}
\subsection{\label{fjs.filetransfer.udp.packets.DataPacket}Class DataPacket}{
\hypertarget{fjs.filetransfer.udp.packets.DataPacket}{}\vskip .1in 
\subsubsection{Declaration}{
\begin{lstlisting}[frame=none]
public class DataPacket
 extends fjs.filetransfer.udp.packets.Packet\end{lstlisting}
\subsubsection{Constructor summary}{
\begin{verse}
\hyperlink{fjs.filetransfer.udp.packets.DataPacket()}{{\bf DataPacket()}} \\
\end{verse}
}
\subsubsection{Constructors}{
\vskip -2em
\begin{itemize}
\item{ 
\index{DataPacket()}
\hypertarget{fjs.filetransfer.udp.packets.DataPacket()}{{\bf  DataPacket}\\}
\begin{lstlisting}[frame=none]
public DataPacket()\end{lstlisting} %end signature
}%end item
\end{itemize}
}
}
\subsection{\label{fjs.filetransfer.udp.packets.FirstPacket}Class FirstPacket}{
\hypertarget{fjs.filetransfer.udp.packets.FirstPacket}{}\vskip .1in 
\subsubsection{Declaration}{
\begin{lstlisting}[frame=none]
public class FirstPacket
 extends fjs.filetransfer.udp.packets.Packet\end{lstlisting}
\subsubsection{Constructor summary}{
\begin{verse}
\hyperlink{fjs.filetransfer.udp.packets.FirstPacket()}{{\bf FirstPacket()}} \\
\end{verse}
}
\subsubsection{Constructors}{
\vskip -2em
\begin{itemize}
\item{ 
\index{FirstPacket()}
\hypertarget{fjs.filetransfer.udp.packets.FirstPacket()}{{\bf  FirstPacket}\\}
\begin{lstlisting}[frame=none]
public FirstPacket()\end{lstlisting} %end signature
}%end item
\end{itemize}
}
}
\subsection{\label{fjs.filetransfer.udp.packets.Packet}Class Packet}{
\hypertarget{fjs.filetransfer.udp.packets.Packet}{}\vskip .1in 
\subsubsection{Declaration}{
\begin{lstlisting}[frame=none]
public class Packet
 extends java.lang.Object\end{lstlisting}
\subsubsection{All known subclasses}{FirstPacket\small{\refdefined{fjs.filetransfer.udp.packets.FirstPacket}}, DataPacket\small{\refdefined{fjs.filetransfer.udp.packets.DataPacket}}, AckPacket\small{\refdefined{fjs.filetransfer.udp.packets.AckPacket}}}
\subsubsection{Constructor summary}{
\begin{verse}
\hyperlink{fjs.filetransfer.udp.packets.Packet()}{{\bf Packet()}} \\
\end{verse}
}
\subsubsection{Constructors}{
\vskip -2em
\begin{itemize}
\item{ 
\index{Packet()}
\hypertarget{fjs.filetransfer.udp.packets.Packet()}{{\bf  Packet}\\}
\begin{lstlisting}[frame=none]
public Packet()\end{lstlisting} %end signature
}%end item
\end{itemize}
}
}
}


%\newpage
%\printbibliography

\end{document}