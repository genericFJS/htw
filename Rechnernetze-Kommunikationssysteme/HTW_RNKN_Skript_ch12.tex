\section{Einführung}
\subsubsection*{Codierungsverfahren}
\slides{35-fehlerschutz_print}{3}
\subsubsection*{Theorie und Praxis}
\slides{35-fehlerschutz_print}{4}

\subsection{Kanalmodelle}
\subsubsection{BSC und BEC}
\slides{35-fehlerschutz_print}{5}

\subsubsection{AWGN}
(additive white Gaussian noise -- Additiver Gaußkanal)
\slides{35-fehlerschutz_print}{6}
\slides{35-fehlerschutz_print}{7}

\subsubsection{Kanalkapazitäten}
\slides{35-fehlerschutz_print}{8}

\subsubsection*{Beispiel BSC}
Kanalkapazität in Abhängigkeit von der Bitfehlerwahrscheinlichkeit
\slides{35-fehlerschutz_print}{9}
\subsubsection*{Beispiel AWGN}
Annahme eines bestimmten Kanals
\slides{35-fehlerschutz_print}{10}

\subsection{Blockcodierung}
\slides{35-fehlerschutz_print}{11}

\subsubsection{Systematische und nichtsystematische Codierung}
\slides{35-fehlerschutz_print}{12}

\subsection{Hamming-Distanz und -Gewicht}
\slides{35-fehlerschutz_print}{13}
\slides{35-fehlerschutz_print}{14}

\subsubsection{Fehlererkunnung}
\slides{35-fehlerschutz_print}{15}

\subsubsection{Fehlerkorrektur}
\slides{35-fehlerschutz_print}{16}

\subsubsection{Fehlererkennung und -korrektur}
\slides{35-fehlerschutz_print}{17}

\subsubsection{Auslöschungskorrektur}
\slides{35-fehlerschutz_print}{18}

\subsubsection{Hamming-Codes}
\slides{35-fehlerschutz_print}{19}

\subsubsection{Notation von Block-Codes}
\slides{35-fehlerschutz_print}{20}

\subsubsection{Probleme der Kanalcodierung}
\slides{35-fehlerschutz_print}{21}
