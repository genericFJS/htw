\subsection{CPU Entwicklung}
\begin{center}
\href{https://www.karlrupp.net/2015/06/40-years-of-microprocessor-trend-data/}{ 
\includegraphics[width=.9\textwidth]{Vorlesung/a130ParallelEinleitung-img001.png} }
\end{center}

\begin{itemize}
\item Moore's law: alle 2 Jahre verdoppelt sich die Anzahl von Transistoren
\item früher ging damit verdopplung der Taktfrequenz einher
\item heute Taktfrequenz stagniert weil höhere Frequenz zu viel Hitze in der CPU erzeugt
\item Performancesteigerung nur durch mehr Kerne möglich
\item Neue Technologien wie Intel Hyper-Threading bilden auf jedem physikalischen Core noch 2 oder mehr virtuelle Cores
\item Anzahl von physikalischen und virtuellen Kernen steigt rasant
\end{itemize}

\subsection{Traditionelle Sequentielle Programmierung}

\begin{itemize}
\item Software läuft sequentiell (nicht parallel) ab
\item ein Algorithmus besteht aus einer Kette von Anweisungen, die nacheinander ablaufen
\item ein einzelner CPU verarbeitet die Befehle
\item mit mehreren Prozessoren wird die Software nicht schneller oder besser, dazu ist parallele Programmierung nötig
\end{itemize}

\subsection{Parallele Programmierung}

\begin{itemize}
\item mehrere Befehle werden parallel (simultan) abgearbeitet
\item durch mehrere Threads, mehrere Prozessoren oder mehrere Rechner
\item Parallel Programmierung

\begin{itemize}
\item beschleunigt moderne Anwendungen oder ermöglicht sie erst
\item ermöglicht simultane Ausführung mehrer Anwendungsteile
\end{itemize}
\item ist komplexer als sequentielle Programmierung
\end{itemize}
\begin{center}
\href{http://journal.frontiersin.org/article/10.3389/fnhum.2014.00641/full}{ \includegraphics[scale=.6]{Vorlesung/a130ParallelEinleitung-img002.jpg} }
\end{center}

\subsection{Anwendungen paralleler Programmierung}

\begin{itemize}
\item Bildverarbeitung, Computergraphik
\item Aufwendige Berechnungen (Data-Mining)
\item Webroboter (Indizierung des Internets)
\item Künstliche Intelligenz
\item Simulationen (Wetterprognose, Physikalische Simulationen)
\item … 
\end{itemize}
\begin{center}
\includegraphics[scale=.6]{Vorlesung/a130ParallelEinleitung-img003.png}  \includegraphics[scale=.6]{Vorlesung/a130ParallelEinleitung-img004.jpg}  \includegraphics[scale=.6]{Vorlesung/a130ParallelEinleitung-img005.png}  \includegraphics[scale=.6]{Vorlesung/a130ParallelEinleitung-img006.png}  \includegraphics[scale=.6]{Vorlesung/a130ParallelEinleitung-img007.jpg}  \includegraphics[scale=.6]{Vorlesung/a130ParallelEinleitung-img008.jpg} 
\end{center}

\subsection{Moderne supercomputer}

\begin{center}
\includegraphics[scale=.6]{Vorlesung/a130ParallelEinleitung-img009.jpg}\\
Aus \url{https://de.wikipedia.org/wiki/Supercomputer}
\end{center}

\begin{center}
\includegraphics[width=.9\textwidth]{Vorlesung/a130ParallelEinleitung-img010.jpg}\\
Aus \url{http://www.digitaleng.news/de/wp-content/uploads/2016/11/BtN5.jpg} 
\end{center}


\subsection{Deutsche Supercomputer}
%\resizebox{.99\textwidth}{!}
\begin{center}
{\tiny
\begin{tabular}{L{.12} L{.15} L{.1} L{.2} L{.07} L{.13} L{.22}}
Name &
Standort &
TeraFLOPS &
Konfiguration &
TB RAM &
Energiebedarf &
Zweck\\\hline
\href{https://de.wikipedia.org/w/index.php?title=Hazel_Hen&action=edit&redlink=1}{Hazel Hen}&
\href{https://de.wikipedia.org/wiki/H%C3%B6chstleistungsrechenzentrum_Stuttgart}{Höchstleistungsrechenzentrum Stuttgart} &
\raggedleft 7.420,00 &
Cray Aries Netzwerk; 7712 Nodes mit je 24 Kernen (Intel Xeon E5-2680 v3, 30M Cache, 2,50 GHz) &
\raggedleft 987 &
\~{}3.200 kW &
~\\
\href{https://de.wikipedia.org/wiki/JUQUEEN}{JUQUEEN} &
\href{https://de.wikipedia.org/wiki/Forschungszentrum_J%C3%BClich}{Forschungszentrum Jülich} &
\raggedleft 5.900,00 &
IBM BlueGene/Q, 28.672 Power BQC-Prozessoren (16 Kerne, 1,60 GHz) &
\raggedleft 448 &
2.301 kW &
Materialwissenschaften, theoretische Chemie, Elementarteilchenphysik, Umwelt, Astrophysik\\
\href{https://de.wikipedia.org/wiki/SuperMUC}{SuperMUC } &
\href{https://de.wikipedia.org/wiki/Leibniz-Rechenzentrum}{Leibniz-Rechenzentrum} (LRZ) (Garching bei München) &
\raggedleft 2.897,00 &
18.432 \href{https://de.wikipedia.org/wiki/Xeon}{Xeon} E5-2680 CPUs (8 Kerne, 2,7 GHz), 820 Xeon E7-4870 CPUs (10 Kerne, 2,4 GHz) &
\raggedleft 340 &
3.423 kW &
Kosmologie über die Entstehung des Universums, Seismologie und Erdbebenvorhersage\\
HRSK-II &
\href{https://de.wikipedia.org/wiki/Zentrum_f%C3%BCr_Informationsdienste_und_Hochleistungsrechnen}{ZIH}, \href{https://de.wikipedia.org/wiki/TU_Dresden}{TU Dresden} &
\raggedleft 1.600,00 &
43.866 CPU Kerne, Intel Haswell-EP-CPUs (Xeon E5 2680v3), 216 Nvidia Tesla-GPUs &
\raggedleft 130 &
~ &
Wissenschaftliche Anwendungen\\
\href{https://de.wikipedia.org/wiki/HLRE-3}{HLRE-3 "Mistral"} &
\href{https://de.wikipedia.org/wiki/Deutsches_Klimarechenzentrum}{Deutsches Klimarechenzentrum} Hamburg &
\raggedleft 1.400,00 &
1. Ausbaustufe (Juli 2015): 1.500 Knoten bullx B700 DLC, 36.000 Kerne (Haswell), 20 PB Lustre-Festplattensystem, 12 Visualisierungsknoten (je 2 Nvidia Tesla K80 GPUs), Warmwasserkühlung &
\raggedleft 120 &
~ &
Klimamodellierung\\
Cray XC40 &
\href{https://de.wikipedia.org/wiki/Deutscher_Wetterdienst}{Deutscher Wetterdienst} (Offenbach) &
\raggedleft 1.100,00 &
Cray Aries Netzwerk; 1.952 CPUs Intel Xeon E5-2680v3/E5-2695v4 &
\raggedleft 122 &
407 kW &
Numerische Wettervorhersage und Klimasimulationen\\
\end{tabular}
}
\end{center}
Aus \url{https://de.wikipedia.org/wiki/Supercomputer} 

\subsection{Supercomputer nach Ländern}

\begin{center}
\includegraphics[width=.98\textwidth]{Vorlesung/a130ParallelEinleitung-img011.png} 
\end{center}

