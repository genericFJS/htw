\slideScale{.48}
\section{Nutzen von Link}
\slides{Vorlesung/16_LINQ_Studentenversion}{2}
\slides{Vorlesung/16_LINQ_Studentenversion}{3}
\slides{Vorlesung/16_LINQ_Studentenversion}{4}

\section{Beispielszenario}
\slides{Vorlesung/16_LINQ_Studentenversion}{5}
\slides{Vorlesung/16_LINQ_Studentenversion}{6}
\subsection{Ohne LINQ}
\slides{Vorlesung/16_LINQ_Studentenversion}{7}
\subsection{Mit LINQ}
\slides{Vorlesung/16_LINQ_Studentenversion}{8}

\section{LINQ Abfrage}
\slides{Vorlesung/16_LINQ_Studentenversion}{9}

\subsection{Syntax}
\slides{Vorlesung/16_LINQ_Studentenversion}{10}
\subsection{Gegenüberstellung}
\slides{Vorlesung/16_LINQ_Studentenversion}{11}
\slides{Vorlesung/16_LINQ_Studentenversion}{12}
\subsection{Schlüsselwärter und (Standard-)Abfrageoperatoren}
\slides{Vorlesung/16_LINQ_Studentenversion}{13}
\subsection{Struktur}
\slides{Vorlesung/16_LINQ_Studentenversion}{14}
\subsection{from und select -- Datenquelle und Auswahl}
\slides{Vorlesung/16_LINQ_Studentenversion}{15}
\slides{Vorlesung/16_LINQ_Studentenversion}{16}
Selektor: \lstinline`person => person` (Delegate) entspricht hier \lstinline`SELECT *`

\subsection{Fragen}
\slides{Vorlesung/16_LINQ_Studentenversion}{17}
\begin{itemize}
\item Wenn im Vornherein die Datenquelle bekannt ist, kann IntelliSense entsprechend der Quelle Vorschläge machen.
\item Ergebnis ist vom Typ \lstinline`IEnumerable<Person>`
\end{itemize}




