Problem: Mehrere Threads können auf gleiche Daten zugreifen

\subsection{Beispiel} 2 Threads, jeder soll die Variable V um 1 erhöhen

\begin{tabular}{l l}
Thread A &
Thread B\\
1A: Read variable V &
1B: Read variable V\\
2A: Add 1 to variable V &
2B: Add 1 to variable V\\
3A: Write back to variable V &
3B: Write back to variable V\\
\end{tabular}
Sei anfänglich V = 0. Was passiert bei folgenden Abläufen?
\begin{itemize}
\item 1A, 2A, 3A, 1B, 2B, 3B
\item 1A, 1B, 2A, 2B, 3A, 3B
\item 1A, 2A, 1B, 3A, 2B, 3B
\end{itemize}
Es kann als Wert "V+1" statt "V+2" raus kommen, wenn beide den Wert "V" auslesen und dann darauf addieren.

\subsection{Datenabhängigkeiten}

Einige Berechnungen beruhen auf vorher berechneten Daten $\to$ können erst ausgeführt werden wenn diese Daten berechnet sind.\bigskip\\
Bernstein conditions: 2 Prozesse $P_i$, $P_j$ sind unabhängig und können parallel ausgeführt werden, wenn alle Input (I) und Output (O) Variablen die folgenden Bedingungen erfüllen:

\begin{itemize}
\item $I_j \cap O_i = \emptyset$
\item $O_j \cap I_i = \emptyset$
\item $O_j \cap O_i = \emptyset$
\end{itemize}
Bsp: Können diese beiden Zeilen parallel ausgeführt werden? 
\begin{lstlisting}[language={[Sharp]C}]
c = func1(a, b) { ... };  d = func2(a, b) { ... };	// Ausgabevariablen unabhängig: ok
c = func1(a, b) { ... };  a = func2(a, b) { ... };	// Eingabevariable der nächsten Zeile ist abhängig von der Ausgabe der vorherigen: nicht ok
c = func1(a, b) { ... };  c = func2(a, b) { ... };	// Ausgabevariable gleich: nicht ok
\end{lstlisting}

\subsection{Race Condition (Wettlaufsituation)}
Das Ergebnis der Anwendung ist abhängig vom Timing der Threads $\to$ zufälliges, falsches Ergebnis\\
Treten auf, wenn datenabhängige Threads gleichzeitig ausgeführt werden\\
Fun Fact: Gibt es Studien zufolge auch im menschlichen Gehirn … Grund für Zufälligkeiten 

\subsection{Race Conditions vermeiden:} Der kritische Codeblock darf nur betreten werden wenn kein anderer Prozess darauf zugreift. (= mutual exclusion)\bigskip\\
Funktioniert das vielleicht wie folgt?\\
Was passiert wenn 2 Threads (fast) gleichzeitig \lstinline$addOne()$ aufrufen? 

\begin{lstlisting}[language={[Sharp]C}]
static int flag = 0; 
static int V = 0; 
// thread-safe execution of V = V + 1

public void addOne() { 
	do { 
		if (flag == 0) { 
			// lock the code section by setting a flag 
			flag = Thread.CurrentThread.ManagedThreadId;
			V = V + 1; 
			flag = 0; 
			return;  
		} 
	} while (true); 
}
\end{lstlisting}
Es können immer noch zwei Threads gleichzeitig in den kritischen Bereich kommen, wenn sie beide gleichzeitig das flag lesen, bevor der andere es geschrieben hat.

\subsection{Atomare Operationen}

\begin{itemize}
\item unzerteilbare Operation die nicht durch Ende der Zeitscheibe eines Threads auf der CPU unterbrochen werden können. 
\item können nur vollständig oder gar nicht stattfinden kann
\item kleinste atomare Operation: Maschinenbefehl
\item Hochsprachenbefehle sind nicht atomar, Bsp i++ (lesen, inkrementieren, schreiben)
\item Bsp: Was wird ausgegeben? 
\end{itemize}
\begin{lstlisting}[language={[Sharp]C}]
static void Main(string[] args) {
	long counter = 0;
	Thread a = new Thread( () =>  {
		for (long i = 0; i < 100000; i++) 
			counter++;
	});
	Thread b = new Thread( () =>  {
		for (long i = 0; i < 100; i++) 
			counter++;
	});
	a.Start(); b.Start();
	a.Join(); b.Join();
	Console.WriteLine("{0}", counter);

 }
\end{lstlisting}
Ausgabe ist zufällig zwischen $100\,001$ und $100\,100$.

\subsection{Mutual exclusion (Mutex)}
= gegenseitiger Ausschluss\\
Race conditions werden vermieden dadurch dass zu jedem Zeitpunkt nur 1 Thread eine kritischen Codeabschnitt abarbeiten darf, der andere Thread muss solange warten bis der eine Thread fertig ist.\\
Realisierungen:
\begin{itemize}
\item lock
\item semaphore
\item monitor
\item message passing
\end{itemize}

\subsection{Monitor}
\begin{itemize}
\item Definiert kritische Codeabschnitte, die nur von einem Thread betreten werden dürfen. 
\item Ein kritischer Codeabschnitt wird durch eine Speicheradresse identifiziert, dazu kann ein Referenztyp (z.B. ein Object) verwendet werden, z.B. das object, wessen Zugriff kritisch ist.
\item \lstinline$Monitor.Enter(object)$ = Beginn des kritischen Abschnitts.
\item \lstinline$Monitor.Exit(object)$  = Ende des kritischen Abschnitts. 
\end{itemize}

\subsection{Beispiel} Was passiert nun wenn addOne() von 2 Threads gleichzeitig aufgerufen wird? 
\begin{lstlisting}[language={[Sharp]C}]
static object lockObject = new object(); 
static int V = 0; 
// thread-safe execution of V = V + 1 

public void addOne() { 
	Monitor.Enter(lockObject); 
	V = V + 1; 
	Monitor.Exit(lockObject); 
}
\end{lstlisting}

\subsection{Lock}
\begin{lstlisting}[language={[Sharp]C}]
lock (x){
	// ...
}
\end{lstlisting}
ist äquivalent zu 
\begin{lstlisting}[language={[Sharp]C}]
Monitor.Enter(x);
try {
	//...
} finally {
	Monitor.Exit(x);
}
\end{lstlisting}