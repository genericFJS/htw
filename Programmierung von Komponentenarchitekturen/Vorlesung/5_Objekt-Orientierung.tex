\section{Vererbung}
Alle Member der Basisklasse werden quasi vom Compiler in die Subklasse kopiert.

\begin{center}
\href{https://commons.wikimedia.org/w/index.php?curid=6326887}{\includegraphics[width=.9\textwidth]{Vorlesung/a5ObjektOrientierung-img001.png}}
\end{center}

\begin{itemize}
\item In C\# nur einfach Vererbung (genau eine Mutterklasse)
\item dafür beliebige Anzahl von Schnittstellen
\item \emph{base}-Schlüsselwort zum Aufrufen der überschriebenen Basisklassenmember von abgeleiteten Klassen
\item \emph{sealed}-Schlüsselwort verhindert Vererbung von dieser Klasse
\end{itemize}

\subsubsection*{Beispiel}
\begin{lstlisting}[language={[Sharp] C}]
class Fahrzeug{
	int Fahrzeugnummer;
	...
}

class Fahrrad : Fahrzeug {
	int Rahmenhöhe;
	int Fahrzeugnummer;
	// Zugriff auf Fahrzeugnummer Fahrzeugnummer der Oberklasse bspw.: base.Fahrzeugnummer
}

Fahrzeug[] meineFahrzeuge = new Fahrzeug[3];
meineFahrzeuge[0] = new Fahrrad();	// möglich, da Fahrrad Fahrzeug ist
meineFahrzeuge[1] = new Fahrrad();
meineFahrzeuge[2] = new PKW();	// auch möglich, da PKW auch Fahrzeug ist
\end{lstlisting}

\section{Der Konstruktor}

\begin{itemize}
\item spezielle Methode die automatisch bei Speicherallokation mit \emph{new} aufgerufen wird
\item kann nicht explizit aufgerufen werden
\item wenn kein Konstruktor angegeben wird ein Standard-Konstruktor zur Verfügung gestellt 
\item spezielle Syntax: kein Rückgabewert, Name = Klassenname
\end{itemize}\bigskip
\begin{itemize}
\item sind mehrere Konstruktoren vorhanden kann man diese nutzen mit \emph{this}
\item auf Konstruktoren der Mutterklasse kann mit \emph{base} zugegriffen werden
\end{itemize}

\paragraph{Beispiel} Was tun die 3 Konstruktoren von Employee? 
\begin{lstlisting}[language={[Sharp]C}]
class Person {
	public string name;
	public Person(string name) {
		// perform some checks here 
		this.name = name;
	}
}

class Employee : Person {
	public int id;
	public Employee() : this("unbekannt", 0) { }	// ruft den 3. Konstruktor mit dem Namen "unbekannt" und ID 0 auf.
	public Employee(int id) : this("unbekannt", id) { }  	// ruft den 3. Konstruktor mit dem Namen "unbekannt" und der übergebenen ID auf
	public Employee(string name, int id) : base(name) {
		// perform some checks here
		this.id = id;
	} // ruft den Konstruktor des Parents mit dem übergebenen Namen auf (der den Namen zuweist) und weist dann die übergebene ID zu
}
\end{lstlisting}

\section{Polymorphie}
Auf einer Referenz der Basisklasse kann die spezifische Methode der Sub-Klasse aufgerufen werden.\\
Funktioniert in C\# durch die Schlüsselwörter \lstinline$virtual$ und \lstinline$override$. 
\begin{lstlisting}[language={[Sharp]C}]
class Basisklasse { 
	public virtual void test() { 
		Console.WriteLine("Basis"); 
	} 
} 

class Subklasse : Basisklasse { 
	public override void test() { 
		Console.WriteLine("Sub"); 
	} 
} 

// Aufruf:
Basisklasse b = new Subklasse; 
b.test(); // Ausgabe: Sub
\end{lstlisting}

\section{Schlüsselwort-Optionen für Klassenmethoden}

\begin{itemize}
\item \emph{virtual}: Die Methode kann Inhalt enthalten und kann in einer Abgeleiteten Klasse polymorph überschrieben werden
\item \emph{override}: Muss bei einer überschriebenen Methode angegeben werden um eine virtual Methode der Basisklasse zu überschreiben
\item \emph{sealed}: Eine überschriebene virtuelle Methode darf in Tochterklassen nicht weiter überschrieben werden
\item \emph{abstract}: Die Methode ist leer und muss überschrieben werden
\item \emph{new}: Wenn die zu überschreibende Methode nicht virtual in der Basisklasse ist kann override nicht verwendet werden $\to$ new nötig, Polymorphismus funktioniert dann nicht
\item \emph{static}: Methode muss von der Klasse aus aufgerufen werden, nicht von einer Instanz
\item \emph{const}: Der Rückgabewert kann nicht verändert werden (Aufruf function()=1 geht nicht)
\end{itemize}

\subsubsection*{Beispiel}
\begin{lstlisting}[language={[Sharp]C}]
class A {
	public virtual void test() {
		Console.WriteLine("A");
	}
}

class B : A {
	public sealed override void test() {
		Console.WriteLine("B");
	}
}

class C : B {
	public override void test() {	// führt zu Fehler, da test() von B "sealed" ist!
		Console.WriteLine("C");
	}
}
\end{lstlisting}

\section{Optionen für Klassendeklaration}

\begin{itemize}
\item \emph{static}: Klasse darf nur statische Member enthalten und kann nicht instanziiert werden 
\item \emph{abstract}: wenn eine abstracte Methode enthalten ist muss die Klasse als abstract markiert werden, verhindert Instanziierung
\end{itemize}

\section{Optionen für Variablen}
\begin{itemize}
\item \emph{static}: die Variable oder Klasse existiert einmalig für die Funktion oder Klasse und nicht für eine Instanz)
\item \emph{const}: (die Variable ändert sich nicht)
\begin{itemize}
\item const vor Parametern: Die Parameter können nicht verändert werden
\item const vor Membervariable: Variablenwert muss sofort zugewiesen werden, Variable ist dann automatisch auch static, beide Einschränkungen sind aufgehoben wenn \emph{readonly} statt \emph{const} verwendet wird 
\end{itemize}
\end{itemize}

\section{Zugriffsmodifizierer für Sichtbarkeit}

\begin{tabular}{L{.15} | L{.15} L{.15} L{.15} L{.4}}
& Überall & Bibliothek & abgeleitete Klassen & \\\hline
private & ~ & ~ & ~ & $\to$ Standard in einem Typ\newline(z.B. in einer Klasse)\\
protected & ~ & ~ & X & ~\\
internal & ~ & X & X & ~\\
public & X & X & X & ~\\
\end{tabular}

\section{Casting von Klassen}
\begin{itemize}
\item implicit cast in eine Basisklasse:
\begin{lstlisting}[language={[Sharp]C}]
object peter = new Person("Peter");
\end{lstlisting}
\item explicit cast in Subklasse:
\begin{lstlisting}[language={[Sharp]C}]
Person p = (Person) peter; // braucht cast, da peter erst mal nur object ist.
\end{lstlisting} 
\item \emph{as} keyword: 
\begin{lstlisting}[language={[Sharp]C}]
string s = peter as string;	// "as-cast" peter auf string möglich, führt nicht zu Laufzeitfehler
if (s == null) // wenn peter kein string ist, bleibt s null
	Console.WriteLine("Peter ist kein String");
\end{lstlisting}
\item \emph{is} keyword: 
\begin{lstlisting}[language={[Sharp]C}]
if (peter is string)	// ähnlich wie "peter as string" mit Abfrage, ob null
	Console.WriteLine("Peter ist kein String");
\end{lstlisting}
\end{itemize}
\subsubsection*{Beispiel}
\lecdate{07.04.2017}
\begin{lstlisting}[language={[Sharp]C}]
class MyClass{
	public virtual void Write(string text){
		Console.Write(" du bis, {0}", text);
	}
}
class MySubClass : MyClass{
	public override void Write(string text){
		Console.Write("{0} Fisch", text);
	}
}

public static void Main(string[] args){
	MyClass a = new MyClass();
	MyClass b = new MySubClass();
	b.Write("Egal wie");
	a.Write("Helene ist");
	((MyClass) b).Write("");
	Console.Write("er:");
	// Text bis hier: "Egal wie Fisch du bist, Helene ist Fischer"
	((MySubClass) a).Write("Ende");	// funktioniert nicht! Programm stürzt ab.
}
\end{lstlisting}

\section{Properties (Eigenschaft)}
Property = Member-Variable die in einem Objekt definiert ist und auf die von außen zugegriffen werden soll.\\
Zugriff auf Eigenschaften: \\
\begin{tabular}{L{.5} L{.5}}
C++ Style: &  C\# Style:\\
\vspace*{-1.5em}
\begin{lstlisting}[language={[Sharp]C}]
private int myVariable;
public int GetMyVariable() {
	return myVariable;
}
private void SetMyVariable(int value) {
	myVariable = value;
}
public void AccessVariables(int a) {
	int b = GetMyVariable(); SetMyVariable(a);
} 
\end{lstlisting} &
\vspace*{-1.5em}
\begin{lstlisting}[language={[Sharp]C}]
private int myVariable;

public int MyVariable{
	get {
		return myVariable;
	}
	private set {
		myVariable = value;
	}
}

int b = MyVariable;
MyVariable = a;
\end{lstlisting}
\end{tabular}\vspace*{-1.5em}
\begin{itemize}
\item Schreibaufwand reduziert
\item Direkter Zugriff im Aufrufer
\end{itemize}

\subsection{Auto-generated properties}
Will man nur den schreibenden Zugriff auf eine Variable einschränken, kann man noch kürzer schreiben:
\begin{lstlisting}[language={[Sharp]C}]
public int MyVariable {get; private set;}
// Variable myVariable wird intern angelegt.
\end{lstlisting}
\begin{itemize}
\item Dabei wird die zugrundeliegende Variable (vorher \lstinline$myVariable$) nicht mehr benutzt
\item Intern wird eine eigene Variable angelegt auf die nun kein direkter Zugriff mehr möglich ist
\end{itemize}

\section{Object initialization Syntax}

Objekte / Felder einer Klasseninstanz können direkt zugewiesen werden, ohne benutzerdefinierten Konstruktor.
\begin{lstlisting}[language={[Sharp]C}]
class Person { 
	public string Nachname { get; set;} 
	public string Vorname { get; set;} 
	public Person() {} 
}

...

Person p1 = new Person() {Nachname = "Hook"}; 
Person p2 = new Person() {Vorname = "Wendy"}; 
Person p3 = new Person() {Vorname = "Peter", Nachname =  "Pan"};
\end{lstlisting}

\section{Anonyme Objekte}

\begin{itemize}
\item Klasse ohne Bezeichner-Namen
\item zur schnellen Gruppierung von Elementen 
\item Der Typ dieser Objekte ist kompliziert $\to$ var keyword benutzen
\end{itemize}

\begin{lstlisting}[language={[Sharp]C}]
var anonymous = new { Name = "Florian", Alter = 28 };
Console.WriteLine(anonymous.Name);
\end{lstlisting}