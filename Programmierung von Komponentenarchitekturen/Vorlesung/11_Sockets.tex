\section{Netzwerkprogrammierung mit der Klasse Socket}
\subsection{Kommunikation mit Sockets}
\slides{Vorlesung/11_Sockets}{1}
\subsection{Funktionalität}
\slides{Vorlesung/11_Sockets}{2}
\slides{Vorlesung/11_Sockets}{3}

\section{Arbeiten mit Socket-Objekten}
\subsection{Synchrone und Asynchrone Socketfunktionen}
\slides{Vorlesung/11_Sockets}{4}
\subsection{Portwahl bei Client und Server}
\slides{Vorlesung/11_Sockets}{5}
\slides{Vorlesung/11_Sockets}{6}
\subsection{Interprozesskommunikation}
\slides{Vorlesung/11_Sockets}{7}
\subsection{Socketanzahl}
\slides{Vorlesung/11_Sockets}{8}
\slides{Vorlesung/11_Sockets}{9}

\section{Synchroner Datenaustausch}
\slides{Vorlesung/11_Sockets}{10}
\section{Synchron VS Asynchron}
\slides{Vorlesung/11_Sockets}{11}
\subsection{Beispiele}
\slides{Vorlesung/11_Sockets}{12}
\slides{Vorlesung/11_Sockets}{13}

\section{Sockets trennen und beenden}
\slides{Vorlesung/11_Sockets}{14}








