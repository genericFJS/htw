Windows Forms ist ein GUI Toolkit für .NET.\\
Alternative GUI Schnittstelle: Windows Presentation Foundation (schwieriger)

\section{Windows Forms Anwendung erstellen}

New $\to$ Project $\to$ C\# $\to$ Windows Forms Application
\begin{center}
\includegraphics[width=.9\textwidth]{Vorlesung/a10WinForms-img001.png} 
\end{center}

\subsection{Form.cs}
\begin{lstlisting}[language={[Sharp]C}]
using System;
using System.Drawing;
using System.Windows.Forms;

namespace MyFirstForm {
	public partial class Form1 : Form {	// partial: Definition von Klasse Forme1 ist auf mehrere .cs-Dateien verteilt.
		public Form1() {
			InitializeComponent();
		}
	}
}
\end{lstlisting}

\subsection{Form.Designer.cs}
\begin{lstlisting}[language={[Sharp]C}]
namespace MyFirstForm {
	partial class Form1 {	// partial: hier geht's weiter
		// ...
		private void InitializeComponent() {
			this.button1 = new System.Windows.Forms.Button();
			this.SuspendLayout();
			// 
			// button1
			// 
			this.button1.Location = new System.Drawing.Point(84, 95);
			this.button1.Name = "button1";
			this.button1.Size = new System.Drawing.Size(75, 23);
			this.button1.TabIndex = 0;
			this.button1.Text = "button1";
			this.button1.UseVisualStyleBackColor = true;
			// ...
		}
	}
}
\end{lstlisting}

\subsection{Program.cs}
\begin{lstlisting}[language={[Sharp]C}]
using System;
using System.Windows.Forms;

namespace MyFirstForm {
	static class Program {
		[STAThread]
		static void Main() {
				Application.EnableVisualStyles();
				Application.SetCompatibleTextRenderingDefault(false);
				Application.Run(new Form1());
		}
	}
}
\end{lstlisting}

\section{Ein neues Control generieren}

\begin{lstlisting}[language={[Sharp]C}]
public Form1() {
	InitializeComponent();

	var myLabel = new Label();
	//Set some properties
	myLabel.Text = "Some example text";
	myLabel.Location = new Point(100, 60);
	myLabel.ForeColor = Color.Red;
	// Without the following line we would see nothing
	this.Controls.Add(myLabel);
}
\end{lstlisting}

\section{Hosting}

\begin{itemize}
\item alle Controls (außer Form) werden in einem anderen Control gehostet
\item Sichtbarkeit und Position des Controls sind dann fest mit dem Host verbunden
\end{itemize}

\section{Auf Events reagieren}
Beispiele für Events: Click, DragDrop, MouseMove, KeyDown\\
Beim Auftreten dieser Events können EventHandler aufgerufen werden die Code ausführen. \\
EventHandler können für jedes Control in den Properties definiert werden

\begin{center}
\includegraphics[scale=.5]{Vorlesung/a10WinForms-img002.png} 
\end{center}

\section{Benutzerdefiniertes Zeichnen}

Zentrales Object: \lstinline$System.Drawing.Graphics$
\begin{lstlisting}[language={[Sharp]C}]
System.Drawing.Graphics graphics = this.CreateGraphics();
\end{lstlisting}
Enthält viele Befehle die mit Draw oder Fill anfangen.\bigskip\\
Draw-Befehle benötigen einen Pen (definiert Zeichen-Stil und Farbe):
\begin{lstlisting}[language={[Sharp]C}]
Pens.Red // einfacher vordefinierter roter Stift
new Pen(Color.Red, 4f) // Farbe Rot, Stiftdicke 4
\end{lstlisting}\bigskip
Fill-Befehle benötigen einen Brush (z.B. SolidBrush, TextureBrush, …):
\begin{lstlisting}[language={[Sharp]C}]
Brushes.Yellow // Vordefinierter gelber SolidBrush
\end{lstlisting}

\subsubsection*{Beispiel}
\begin{lstlisting}[language={[Sharp]C}]
System.Drawing.Rectangle rectangle = new System.Drawing.Rectangle(100, 100, 200, 200);
graphics.DrawEllipse(Pens.Black, rectangle);
graphics.DrawRectangle(Pens.Red, rectangle);
graphics.FillRectangle(Brushes.Yellow, new Rectangle(200,200, 100, 100));
\end{lstlisting}
\begin{center}
\includegraphics[scale=.5]{Vorlesung/a10WinForms-img003.png}
\end{center} 

\section{Transformation von Graphikobjekten}
Graphikobjekte können
\begin{itemize}
\item skaliert  (\lstinline$graphics.ScaleTransform(x_factor, y_factor);$)
\item verschoben (\lstinline$graphics.TranslateTransform(x, y);$)
\item rotiert  (\lstinline$graphics.RotateTransform(angle);$)
\end{itemize}
werden.

\section{Persistenz}
Graphikobjekte, die mit einfach bei einer Aktion gemalt werden, können wieder verschwinden, wenn das Fenster verkleinert und wieder vergrößert wird, das Fenster aus dem Bildschirm raus und wieder rein geschoben wird usw.\\
Damit sie persistent bleiben, müssen sie in der Funktion \lstinline`OnPoint()` gezeichnet werden.