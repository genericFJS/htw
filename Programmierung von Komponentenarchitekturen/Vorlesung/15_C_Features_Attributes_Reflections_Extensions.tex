\lecdate{09.06.2017}
\section{Attribute}

\begin{itemize}
\item sind Zusatzinformationen (Metadaten) zu Klassen, Methoden und Eigenschaften
\item können zur Laufzeit ausgelesen werden mit einem Verfahren, dass Reflection heißt
\item Syntax:\\
\lstinline`[Attributname]`\\
\lstinline`Typ, Klasse, Methode, Eigenschaft`
\end{itemize}
Beispiel für Attribute:
\begin{itemize}
\item \lstinline`[Obsolete]`
\item \lstinline`[Obsolete("Nutzen Sie die neue Methode newMethod()")]`
\end{itemize}
\subsection{Attribut definieren}
\begin{lstlisting}[language={[Sharp]C}]
public class Author : System.Attribute { 	
	public string Name { get; set; } 
	public Author(string name = "anonymous") {
		Name = name; 
	} 
} 
\end{lstlisting}

\subsection{Attribut nutzen}

\begin{lstlisting}[language={[Sharp]C}]
[Author("S.Aland")] 
class MyClass { 
	// ...
}
\end{lstlisting}

\subsection{Reflection (Attribute abfragen)}
\begin{lstlisting}[language={[Sharp]C}]
Attribute[] attrs = Attribute.GetCustomAttributes(typeof(MyClass));
foreach (System.Attribute attr in attrs) { 
	if (attr is Author) { 
		Author a = (Author)attr; 
		Console.WriteLine("Author: " + a.Name); 
	} 
}
\end{lstlisting}

\section{Extension methods}
fügen Methoden zu bestehenden Typen hinzu
\paragraph{Beispiel} Sie möchten die Reihenfolge der Einträge einer Liste umkehren können, also eine neue Methode \lstinline`Invert()` hinzufügen.
\subsection{Verfahren}
\begin{enumerate}
\item Hilfsklasse erstellen, Liste übergeben
\begin{lstlisting}[language={[Sharp]C}]
List list;

Helpers.Invert(list);
\end{lstlisting}
Bewertung: umständliche Hilfsklasse
\item Von der Klasse erben
\begin{lstlisting}[language={[Sharp]C}]
class MyList : List

MyList list = ...
list.Invert();
\end{lstlisting}
Bewertung: umständliche Vererbung, teils Vererbung nicht möglich (bei versiegelten Klassen, von denen nicht geerbt werden darf, bspw. \lstinline`string`)
\item Methode direkt in Klasse List implementieren
\begin{lstlisting}[language={[Sharp]C}]
List list;

list.Invert();
\end{lstlisting}
unschön vorgegebene Klasse zu ändern!
\item $\Rightarrow$ Extensions Methods nutzen, um \lstinline`Invert()` zur Klasse List hinzuzufügen ohne List direkt zu ändern
\begin{lstlisting}[language={[Sharp]C}]
List list;

list.Invert();
\end{lstlisting}
Bewertung: gute Lösung
\end{enumerate}
Am „schönsten“ ist also die letzte Methode mit einer Extension Methode\\
Aufruf:
\begin{lstlisting}[language={[Sharp]C}]
List<int> myList = new List<int>() { 1, 2, 7, 8 }; 
myList.Invert(); 

for (int i = 0; i < tlist.Count; i++)  
	Console.WriteLine(tlist[i]); // writes 8 7 2 1 
// ...
\end{lstlisting}
Funktioniert durch Hinzufügen von
\begin{lstlisting}[language={[Sharp]C}]
// um eine Extension Method zu finden, werden alle statischen Klassen mit statischen Funktionen durchsucht und geguckt, was passt
static class Extension { // statische Klasse mit statischer Methode! Name egal.
	public static void Invert<T>(this List<T> list){	// this als erstes Argument: Indikator, dass es sich um eine Extension Method handelt
		for (int i = 0; i < list.Count / 2; i++) { 
			T temp = list[i]; 
			list[i] = list[list.Count {}- i {}- 1];  
			list[list.Count {}- i {}- 1] = temp; 
		} 
	}
}
\end{lstlisting}


\subsection{Definition}

\begin{itemize}
\item statische Methode in einer statischen Klasse
\item erstes Argument: this … gibt den Typ an, auf den die Extension Methode aufgerufen wird
\item Name der Klasse die die Extension method definiert ist egal, die Klasse wird nie benutzt
\end{itemize}

\paragraph{Beispiel} eine Zahl auf jedes Element eines Arrays addieren\\
Aufruf: 
\begin{lstlisting}[language={[Sharp]C}]
int[] a = new int[] { 1, 2, 3 }; 
a = a.Add(2); 
for (int i=0; i<a.Length; i++) 
	Console.WriteLine(a[i]); // writes: 3 4 5 
\end{lstlisting}
Funktioniert durch Hinzufügen von: 
\begin{lstlisting}[language={[Sharp]C}]
public static class Extension { 
	public static int[] Add(this int[] a, int b) { 
		for (int i = 0; i < a.Length; i++) 
		a[i] += b; 
		return a; 
	}
} 
\end{lstlisting}
Fazit: Cooles Feature, Hauptvorteil: Lesbarkeit, Klarheit