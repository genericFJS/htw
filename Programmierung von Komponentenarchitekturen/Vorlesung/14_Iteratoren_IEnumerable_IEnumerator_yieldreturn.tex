Ziel: Durchlaufen von Elementen mit einer foreach Schleife\\
Im Programm soll aber eine einfache foreach Schleife definiert werden können bei der irgendeine Sammlung von Daten durchlaufen wird. 


\subsubsection*{Beispiel}
\begin{lstlisting}[language={[Sharp]C}]
foreach (string s in StringList) { 
	Console.WriteLine(s); 
} 
\end{lstlisting}

Die Sammlung der Daten die zu Durchlaufen sind, soll in StringList enthalten sein.\\
Welche möglichen Datentypen für StringList kennen Sie bis jetzt?\\
Nun wollen wir das erweitern: StringList soll z.B. 
\begin{itemize}
\item aus einer Datenbank 
\item aus einer Datei
\item aus dem Internet
\end{itemize}
lesen, und zwar dynamisch (erst bei Abfrage des nächsten Wertes)\\
Vorteil, sie brauchen das Auslesen nicht in ihrer Schleife implementieren, sondern in StringList (Trennung von Daten und Iteration)
\bigskip\\
Frage: Was muss StringList erfüllen damit das funktioniert?\\
Antwort: Es muss das Interface IEnumerable implementieren!
\section{IEnumerable/IEnumerator}
IEnumerable liefert die Methode GetEnumerator, die benutzt werden kann um eine Auflistung zu durchlaufen
\begin{lstlisting}[language={[Sharp]C}]
public interface IEnumerable {
	IEnumerator GetEnumerator();
}
\end{lstlisting}

IEnumerator steuert das durchlaufen einer Aufzählung (entspricht einem Iterator in C++/Java)

\begin{lstlisting}[language={[Sharp]C}]
public interface IEnumerator {
	T Current { get; }  
	// Ruft das aktuelle Element in der Auflistung ab.
	bool MoveNext();
	// Setzt den Enumerator auf das nächste Element der Auflistung.
	// Rückgabewerte:
	// true, wenn der Enumerator erfolgreich auf das nächste Element gesetzt wurde,
	// false, wenn der Enumerator das Ende der Auflistung überschritten hat.

	void Reset();
	// Setzt den Enumerator auf seine anfängliche Position vor dem ersten Element
	// in der Auflistung.
}
\end{lstlisting}

Wie bekommt man so einen Iterator (IEnumerator)? 
\begin{itemize}
\item Man schreib ihn selbst
\item Man nutzt yield return
\end{itemize}

\section{yield return}
yield return funktioniert ähnlich wie return in einer Methode
\begin{itemize}
\item Wird verwendet um jedes Element einer Auflistung einzeln zurück zu geben
\item Der Rückgabewert der Methode ist damit automatisch ein Iterator (IEnumerator)
\item Wenn eine yield return-Anweisung erreicht wird, wird ein Wert zurückgegeben, und die aktuelle Position im Code wird gespeichert. Wenn die Iteratorfunktion das nächste Mal aufgerufen wird, wird die Ausführung von dieser Position fortgesetzt
\end{itemize}

Beispiel: 
\begin{lstlisting}[language={[Sharp]C}]
static void Main(string[] args) { 
	DayList daylist = new DayList(); 
	foreach (string s in daylist) { Console.WriteLine(s); } 
}

public class DayList { 
	public IEnumerator GetEnumerator() { 
		yield return "Montag"; 
		yield return "Dienstag"; 
		yield return "Mittwoch"; 
		yield return "Donnerstag"; 
		yield return "Freitag"; 
	} 
}

// oder äquivalent dazu:
public class DayList { 
	string[] tage = { "Montag", "Dienstag", "Mittwoch", "Donnerstag", "Freitag" };
	public IEnumerator GetEnumerator() { 
		for (int i = 0; i < tage.Length; i++) 
		yield return tage[i]; 
	} 
} 
\end{lstlisting}
Verständnistest: Wie würde man die Klasse DayList ohne yield return definieren? \\
$\to$ Enumerator muss selbst erstellt werden:
\begin{lstlisting}[language={[Sharp]C}]
public class Daylist : IEnumerator{
	int i = -1;
	string[] tage = { "Montag", "Dienstag", "Mittwoch", "Donnerstag", "Freitag" };
	
	public object Current { 
		get {
			return tage[i]; 
		} 
	}
	
	public bool MoveNext() {
		i++; 
		if (i>4) 
			return false; 
		return true;
	}
	
	public void Reset() {
		i = -1;
	}
	
	public IEnumerator GetEnumerator(){
		return this;
	}
}

static void Main(string[] args){
	Daylist daylist = new Daylist();
	foreach (string s in daylist) Console.WritLine(s);
}
\end{lstlisting}
\subsection{Generischer IEnumerator}
IEnumerator und IEnumerable gibt es auch als generische Typen \lstinline`IEnumerator<T>, IEnumerable<T>`

\subsubsection*{Beispiel}
Sie wollen eine Datei in einer foreach-Schleife Zeile für Zeile durchlaufen.\\
Dabei sollen nur Zeilen mit Inhalt (Length>1) berücksichtigt werden
\begin{lstlisting}[language={[Sharp]C}]
int i = 0; 
foreach (string line in new ExtractLinesFromFile(@"C:\eula.1028.txt"))
	Console.WriteLine("Zeile " + i++ +": " + line + "\n"); 
\end{lstlisting}
