\begin{itemize}
\item Oft müssen ähnliche Operationen / Funktionalitäten für unterschiedliche Datentypen bereitgestellt werden
\item Um nicht gleichen Code mehrfach zu schreiben nutzt man Platzhalter (template) für Datentypen
\item Beispiel: List<T>
\end{itemize}

\section{Generische Klasse definieren}

\begin{lstlisting}[language={[Sharp]C}]
class GenericClass<T> {
	T myVariable;
	public GenericClass(T myVar){
		myVariable = myVar;	
	}
	public T MyVariable{
		get {
			return myVariable;		
		}
	}
}
\end{lstlisting}

\section{Generische Klasse benutzen}

\begin{lstlisting}[language={[Sharp]C}]
GenericClass<int> a = new GenericClass<int>(5);
\end{lstlisting}

\begin{center}
\includegraphics{Vorlesung/a7Generik-img001.png}
\end{center}

Ähnlich für: 
\begin{itemize}
\item struct 
\item interfaces
\item delegates
\end{itemize}
~

\section{Generische Methoden}
Generik funktioniert genauso für Methoden
\begin{lstlisting}[language={[Sharp]C}]
class MyClass {
	public static void Swap<T>(ref T l, ref T r) { 
	// auch möglich mit mehreren Generiks:
	// public static void Swap<T1, T2>(ref T1 l, ref T2 r) { 
		T temp = r;
		r = l;
		l = temp;   
	}
}

int a = 3;
int b = 4;
MyClass.Swap(ref a, ref b);
\end{lstlisting}

\section{Template Constraints}
Constraints = Vorgaben an die Templates

\subsection{Interface Constraint}

\begin{lstlisting}[language={[Sharp]C}]
public static void Swap<T>(ref T l, ref T r) where T : IComparable{
	if (l.CompareTo(r) < 0)  {
		T temp = r;
		r = l;
		l = temp;
	}
}
\end{lstlisting}

\subsection{Klassen Constraint}
\begin{lstlisting}[language={[Sharp]C}]
class MyGenericClass<T> where T : IComparable{

}
\end{lstlisting}
\subsection{Vererbungs Constraints}

\begin{lstlisting}[language={[Sharp]C}]
class MyGenericClass<T> where T : Random

class MyGenericClass<T, V> where T : List<V>

class MyGenericClass<T, V> where T : List<V> where V : Random
\end{lstlisting}

\subsection{Instanzierbarkeits Constraint}
\begin{lstlisting}[language={[Sharp]C}]
T CreateNew<T>() where T : new() {
	return new T(); // geht nur wegen " : new()", sonst: Compilerfehler
}
\end{lstlisting}

\section{Beispiel}
\subsection{Aufgabe}

Zur Verwaltung der Fahrzeuge eines Konzerns existiert die Klasse \lstinline$Car$, die alle Daten eines Fahrzeuges enthält.
\begin{lstlisting}[language={[Sharp]C}]
public class Car {
	private int gewicht;		// in kg
	public int Gewicht{
		get {return gewicht; }
		private set {gewicht = value;}
	}
	// ...
}
\end{lstlisting}

Sie möchten nun alle Fahrzeuge des Konzerns in einer Liste Sammeln. C\# stellt dafür den generischen Typ \lstinline$List<T>$ zur Verfügung. Auf einem Objekt des Typs \lstinline$List$ kann die Methode \lstinline$Add(T element)$ aufgerufen werden um Objekte hinzuzufügen. Außerdem wollen Sie alle Fahrzeuge der Liste automatisch nach ihrem Gewicht sortieren lassen, ohne etwas an der sensiblen Klasse \lstinline$Car$ zu ändern. Sie finden dazu die Methode \lstinline$Sort$ der Klasse \lstinline$List$:
\begin{lstlisting}[language={[Sharp]C}]
public void Sort(IComparer<T> comparer);
\end{lstlisting}
Sie suchen nach dem Interface IComparer und finden folgende Definition:
\begin{lstlisting}[language={[Sharp]C}]
public interface IComparer<in T> {
	// Returns:	1		if y>x
	//					-1	if y<x 
	int Compare(T x, T y);
}
\end{lstlisting}
Was müssen sie schreiben um…
\begin{enumerate}
\item … eine Liste anzulegen und die bereits erzeugten Autos \lstinline$ferrari, trabbi, tesla$ einzufügen?
\item … die Liste automatisch sortieren zu lassen?
\item Was ist der Vorteil der Generik in \lstinline$List<T>$? Was wäre schlecht daran, wenn die \lstinline$List$ nicht generisch wäre?
\end{enumerate}

\subsection{Antwort}
\begin{enumerate}
\item $ $
\begin{lstlisting}[language={[Sharp]C}]
List<Car> list = new List<Car>();
list.Add(ferarri);
list.Add(trabbi);
list.Add(tesla);
\end{lstlisting}
\item $ $
\begin{lstlisting}[language={[Sharp]C}]
class Comparer : IComparer<Car> {
	int Compare(Car x, Car y){
		if (x.Gewicht < y.Gewicht)
			return 1;
		else
			return -1;
	}
}
list.Sort(new Comparer());
\end{lstlisting}
\item Die Liste weiß, was sie für einen Datentyp enthält. Sonst wäre es nur eine Liste von Objekten. Damit könnten auch „fremde“ Objekte (Tiere, Fahrräder, Strings, Zahl, …) eingefügt werden. Diese könnten beispielsweise nicht mehr so einfach verglichen werden.
\end{enumerate}