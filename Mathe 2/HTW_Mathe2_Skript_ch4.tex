\section{Der Integralbegriff}
\subsection{Das bestimmte Integral}
\paragraph{Problem:} \parskp
\emph{Gegeben:} Kurve $y=f(x), \;x\in [a,b]$ und $f(x) \geq 0$.\\
\emph{Gesucht:} Flächeninhalt $I$ unter der Kurve
\begin{center}
\includegraphics[scale=.75]{Vorlesung/ABB80}
\end{center}
\emph{Vorgehen:} 
\begin{itemize}
\item Zerlegung $Z$ des Intervalls $[a,b]$:\\
$a=x_0<x_1<x_2<x_3<\dotsb<x_{n-1}<n_n=b$
\item In jedem Teilintervall Zwischenstelle $\xi_i\in [x_{in}, x_i]$ wählen. Dies ergibt die Zerlegung $Z^*$ ($Z$ mit Zwischenstellen).
\item $\Delta (Z^*):= \max_{i=1,\dots, n}(x_i-x_{i-1}=$ … Länge des größten Teilintervalls
\item Approximation von $I$ durch die Summe von Rechteckflächen: $$S(Z^*,f):=\sum_{i=1}^n f(\xi_i)(x_i-x_{i-1})$$
$S(Z^*,f)$ heißt Riemann-Summe. Sie ist abhängig von der Zerlegung $Z^*$.
\end{itemize}
\begin{center}
\includegraphics[scale=.75]{Vorlesung/ABB81}
\end{center}
\paragraph{Def. 1} Die Funtkion $f$ heißt (Riemann-)integrierbar über $[a,b]$ falls für jede Zerlegungsfolge $Z^*_\mu$ von $[a,b]$ mit $\lim_{\mu\to \infty}\Delta (Z^*_\mu)=0$ gilt: $\lim_{\mu\to\infty}S(Z^*_\mu, f) = I$. Die Zahl $I$ heißt dann bestimmtes Integral von $f$ über $[a,b]$. Bezeichnung: $i=\int_a^bf(x)\intd{x}$.
\subparagraph{Diskussion:}
\begin{itemize}
\item Def. 1 basiert auf der Forderung $f(x)\geq 0$. Falls $f(x)<0$ für alle $x\in [a,b]$, so gilt im Falle der Integrierbarkeit $\int_a^bf(x)\intd{x}<0$:
\begin{center}
\includegraphics[scale=.75]{Vorlesung/ABB82}
\end{center}
$\Rightarrow$ Flächeninhalt $F=\int_a^b|f(x)|\intd{x}=-\int_a^bf(x)\intd{x}$.
\item Man definiert:\\
$\int_a^af(x)\intd{x}:=0$\\
$\int_b^af(x)\intd{x}:= - \int_a^bf(x)\intd{x} \quad (b>a)$
\item Eigenschaften des bestimmten Integrals:\\
$\int_a^bf(x)\intd{x}=\int_a^cf(x)\intd{x}+\int_c^bf(x)\intd{x}$\\
für beliebige $a, b, c \in \RR$.
\item $\int_a^bc_1 u(x) + c_2 v(x) \intd{x} = c_1 \int_a^bu(x)\intd{x}+c_2 \int_a^b v(x) \intd{x}$ für $c_1, c_2 \in \RR$
\end{itemize}
\paragraph{Satz 1:} Es sei $f: [a,b] \to \RR$ stetig. Dann ist $f$ auf $[a,b]$ integrierbar. 
\subparagraph{Diskussion:} 
\begin{itemize}
\item Falls $f$ stückweise stetig ist, mit endlich vielen Sprungstellen, so ist $f$ ebenfalls integrierbar (Integration von Sprungstelle zu Sprungstelle).
\begin{center}
\includegraphics[scale=.75]{Vorlesung/ABB93}
\end{center}
\item Nicht integrierbar ist bspw. $f: [0,1] \to \RR, \; f(x) = \begin{cases}
1 & x \text{ irrational}\\
0 & x \text{ rational}
\end{cases}$
\end{itemize}
\subsection{Stammfunktion und unbestimmtes Integral}
\paragraph{Satz 2:} (Mittelwertsatz der Integralrechnung)\\
Sei $f: [a,b]\to \RR$ stetig. Dann existiert (mindestens) ein $\xi \in (a,b)$ mit: 
\[\int_a^b f(x)\intd{x}= f(\xi) (b-a)\]
Anschaulich:
\begin{center}
\includegraphics[scale=.75]{Vorlesung/ABB94}
\end{center}
Wir nennen $m=\frac{1}{b-a}\int_a^b f(x) \intd{x}$ den Integralmittelwert von $f$ auf $[a,b]$.\\
\emph{Integral mit variabler oberer Grenze:}\\
Wir betrachten $\int_a^x f(t) \intd{t} =: F(x)$
\begin{center}
\includegraphics[scale=.75]{Vorlesung/ABB95}
\end{center}
\paragraph{Satz 3:} Sei $f:[a,b]\to \RR$ stetig. Dann ist $F(x)=\int_a^x f(t)\intd{t}$ auf $[a,b]$ differenzierbar und es gilt:
\[F'(x)=f(x)\]
Beweis:\\
$\frac{F(x+h)-F(x)}{h}=\frac{\int_x^{x+h} f(t) \intd{t}}{h}\overset{\substack{\text{Satz 2}\\\text{(mit }\xi\in (x,x+h))}}{=}\frac{f(\xi)\cdot (x+h-x)}{h}=f(\xi)\overset{h\to 0}{\longrightarrow} f(x)$ da $f$ stetig.\\
$\Rightarrow F'(x) = f(x)$
\paragraph{Def. 2:} Die Funktion $F$ heißt \emph{Stammfunktion} von $f$ (auf $[a,b]$), wenn gilt $F'(x)=f(x)$.
\subparagraph{Diskussion:} Ist $F$ eine Stammfunktion, so ist auch $\tilde{F}$ mit $\tilde{F}(x)=F(x)+C$ eine Stammfunktion.
\paragraph{Def. 3:} Die Menge $\{F(x)+C|C\in \RR$ aller Stammfunktionen von $f$, wobei $F$ beliebige Stammfunktion von $f$ ist, heißt unbestimmtes Integral von $f$.\\
Bezeichunung: $\int f(x) \intd{x}=F(x)+C$
\subsection{Hauptsatz der Differential- und Integralrechnung (HDI)}
\paragraph{Satz 4:} Sei $f:[a,b]\to \RR$ stetig und $F$ beliebige Stammfunktion von $f$.
\[\int_a^b f(x) \intd{x}=\big[F(x)\big]_a^b=F(b)-F(a)\]
Beweis: Satz 3 liefert $F_1(x):=\int_a^x f(t) \intd{t}$ ist Stammfunktion von $f$. Also gilt $F(x)=F_1(x)+k$\\
$\Rightarrow F(b)-F(a)=F_1(b)+k-\underbrace{F_1(a)}_{=0}-k = \int_a^bf(t)\intd{t}$
\subparagraph{Diskussion:}
\begin{enumerate}
\item $\underbrace{\int_a^b f(x) \intd{x}}_{\substack{\text{Flächeninhaltsproblem,}\\\text{ Integralrechnung}}}=\underbrace{F(b)-F(a)}_{\substack{\text{Stammfunktion,}\\\text{Umkehrung der Differentialrechnung}}}$\\
Dieser Term ist also der Zusammenhang zwischen der Differential- und der Integralrechnung.
\item Symbolik: $\frac{\mathrm{d}F(x)}{\mathrm{d}x}=f(x) \Leftrightarrow \underbrace{\int \intd{F(x)}}_{F(x)+C}=\int f(x) \intd{x}$
\item Aus Tabellen zur Differentiation lassen sich Integrationsregeln ableiten.
\end{enumerate}
\subparagraph{Beispiele:}
\begin{anumerate}
\item $\frac{\intd{}}{\intd{x}}\cos x = - \sin x \\
\Leftrightarrow \int -\sin x \intd{x}=\cos x + C^* \quad |\cdot (-1)\\
\int \sin x \intd{x}= - \cos x + \underbrace{C}_{=-C^*}$
\item $\frac{\intd{}}{\intd{x}}x^{\alpha+1}=(\alpha+1)x^\alpha\\
\Leftrightarrow \int x^\alpha \intd{x} = \frac{1}{\alpha +1} x^{\alpha +1} + C$ (falls $\alpha \not = -1$)
\end{anumerate}
\section{Integrationsmethoden}
\subsection{Substitution}
Zu berechnen ist $\int f\big(g(x)\big)\cdot g'(x) \intd{x}$. Bekannt sei dabei die Stammfunktion $F$ von $f$. Dann gilt:
\[\int f\big( g(x) \big) g'(x) \intd{x}\overset{\substack{\text{Subst.}\\ u=g(x)}}{=}\int f(u) \intd{u}=F(u) + C =\overset{u=g(x)}{=}F\big(g(x)\big)+C \]
Substitution $u=g(x)$ impliziert $\frac{\intd{u}}{\intd{x}}=g'(x) \Rightarrow \intd{u}=g'(x) \intd{x}$.\\
\emph{Merke:} Anwendung dieser Methode ist zweckmäßig, wenn der Integrand das Produkt eine Verknüpfung zweier Funktionen mit der Ableitung der inneren Funktion ist und eine Stammfunktion für die äußere Funktion bekannt ist.
\paragraph{Bsp. 1:} $\int \frac{1}{x}\sqrt[3]{\ln x}\intd{x}\overset{\substack{u=\ln x\\\intd{x}=\tfrac{1}{x}\intd{u}}}{=}\int \underbrace{\sqrt[3]{u}}_{u^{\tfrac{1}{3}}}\intd{u}=\frac{3}{4}u^{\frac{4}{3}}+C=\frac{3}{4}(\ln x)^{\tfrac{4}{3}}+C$
\paragraph{Bsp. 2:}$\int x e^{-x^2}\intd{x}\overset{\substack{u=-x^2\\\intd{x}=-\tfrac{\intd{u}}{2x}}}{=}\int x e^u\frac{\intd{u}}{-2x}=-\frac{1}{2}\int e^u \intd{u}=-\frac{1}{2}e^u+C=-\frac{1}{2}e^{-x^2}+C$
\paragraph{Bsp. 3:} (Substitution bei bestimmten Integral)
\begin{itemize}
\item 1. Variante: Grenzen ersetzen\\
$I=\int_0^{\sqrt{8}}x\sqrt{1+x^2}\intd{x}\overset{\substack{u=1+x^2\\dx=\tfrac{\intd{u}}{2x}}}{=}\int_1^9 x \sqrt{u} \frac{\intd{u}}{2x}=\frac{1}{2}\int u^{\tfrac{1}{2}}\intd{u}= \left[\frac{1}{3}u^{\tfrac{3}{2}}\right]_1^9=\frac{1}{3}(27-1)=\frac{26}{3}$\\
Grenzen in Substitution einsetzen $u=1+x^2\Rightarrow u_{unt}=1+0^2=1 \quad u_{ob}=1+\sqrt{8}^2=9$
\item 2. Variante: Erst unbestimmtes Integral  lösen\\
$I=\int_0^{\sqrt{8}}x\sqrt{1+x^2}\intd{x}=\frac{1}{3}(1+x^2)^{\tfrac{3}{2}}+C$\\
Dann Grenzen einsetzen:\\
$I=\left[\frac{1}{3}(1+x^2)^{\tfrac{3}{2}}\right]_0^{\sqrt{8}}=\frac{1}{3}(27-1)=\frac{26}{3}$
\end{itemize}
\subparagraph{Bsp. 4:} $\int \frac{f'(x)}{f(x)}\intd{x}=\ln |f(x)|+C$\\
(Zähler = Ableitung des Nenners)\\
Nutze dazu die Substitution $u=f(x), \; \intd{x}=\frac{\intd{u}}{f'(x)}$\\
$\Rightarrow \int \dots = \int \frac{1}{u} \intd{u}=\ln | u| + C = \ln | f(x) | + C$
\paragraph{Bsp. 5:}(lineare Substitution)\\
Allgemein: $\int f(ax+b) \intd{x} \overset{\substack{u=ax+b\\\tfrac{du}{dx}=a}}{=} \int f(u) \frac{\intd{u}}{a}\overset{\substack{F: \text{ Stammfkt.}\\ \text{von } f}}{=}\frac{1}{a}\cdot F(u) + C$
\begin{anumerate}
\item $\int \cos (3x) = \frac{1}{3}\sin (3x) + C$
\item $\int e^{-2x} \intd{x}=\frac{1}{-2}e^{-2x}+C$
\item $\int (3x+4)^6 \intd{x}=\frac{1}{3}\cdot \frac{1}{7}(3x+4)^7+C$
\item $\int \sin \left( \frac{x}{2}+\pi\right) \intd{x}=2 \cdot  -\cos\left(\frac{x}{2}+\pi\right)+C$
\end{anumerate}

\subparagraph{Diskussion:} Neben diesen „natürlichen“ und leicht erkennbaren Substitutionen sind weiter Substitutionen durch die Einführung von „künstlichen“ Variablen möglich:\\
$\int f(x) \intd{x}\overset{\substack{x=f(t)\\\tfrac{\diffd{x}}{\diffd{t}}=\dot{\varphi}(t)}}{=}\int f(\varphi(t)) \cdot \dot{\varphi}(t) \intd{t}$\\
Dies entsprecht der Substitutionsregel, von rechts nach links gelesen. Falls die rechte Seite davon integrierbar ist (mit Stammfunktion $H$), dann:\\
$\int f(x) \intd{x}=H(t) + C = H\big(\varphi^{-1}(t)\big) + C$ \qquad (falls $\varphi^{-1}$ existiert)
\subparagraph{Bsp. 6:} \parskp
$\int \frac{1}{\sqrt{1+x^2}}\intd{x}\overset{\substack{x=\sinh(t)\\\tfrac{\diffd{x}}{\diffd{t}}=\cosh(t)\\\cosh^2(t)-\sinh^2(t)=1}}{=}\int \frac{1}{\cosh(t)}\cosh(t)\intd{t}=\int \intd{t}=t+C = \arcsinh (x) + C$\\
Für weitere geeignete Substitutionen siehe Integrationstabelle.

\subsection{Partielle Integration}
Produktregel der Differentiation:\\
$\frac{\diffd{}}{\diffd{x}}\big(u(x)\cdot v(x)\big) = u'(x)\cdot v(x)+u(x)\cdot v'(x)$\\
$\Rightarrow u(x) v(x) \int u'(x) v(x) \intd{x} + \int u(x) v'(x) \intd{x}$\\
$\Rightarrow \boxed{\int u(x)v'(x)\intd{x}=u(x)v(x)- \int u'(x) v(x) \intd{x} }$

\subparagraph{Bsp. 7:}
\begin{anumerate}
\item $\int \underbrace{x}_{u(x)} \underbrace{\sin(2x)}_{v'(x)} \intd{x}=\underbrace{x}_{u}\cdot \underbrace{-\frac{1}{2}\cos(2x)}_{v} - \int \underbrace{1}_{u'}\cdot \underbrace{-\frac{1}{2}\cos (2x)}_{v}\intd{x}=-\frac{x}{2}\cos(2x)+\frac{1}{4}\sin(2x) + C $\\
$u'(x) = 1 \qquad v(x) = -\frac{1}{2}\cos (2x)$
\item $\int \underbrace{x^3}_{v'}\underbrace{\ln x}_{u} = \frac{1}{4}x^4 \ln x - \int \frac{1}{x}\cdot \frac{1}{4}x^4 \intd{x} = \frac{1}{4} x^4 \ln x - \frac{1}{16} x^4 + C = \frac{1}{4}x^4 \left(\ln x - \frac{1}{4} \right) ++C$
\end{anumerate}
\emph{Merke:} Typische Anwendungsfälle für partielle Integration (mit $p(x)$ jeweils als $u$):
\begin{itemize}
\item $\int p(x) e^{ax} \intd{x}$
\item $\int p(x) \cos (ax) \intd{x}$
\item $\int p(x) \sin (ax) \intd{x}$
\end{itemize}
aber (mit $\ln(x)$ jeweils als $u$):
\begin{itemize}
\item $\int p(x) \cdot \ln (x) \intd{x}$
\item $\int x^\alpha \cdot \ln(x) \intd{x}$
\end{itemize}
\subparagraph{Bsp. 8:} \parskp
$\int \arctan (x ) \intd{x} \overset{\substack{u=\arctan(x)\\ v'=1}}{=}x \cdot \arctan
(x) - \int x \cdot \frac{1}{1+x^2}\intd{x}=x \cdot \arctan (x) - \frac{1}{2}\ln(|x^2+1|)+C$\\
$u'=\frac{1}{1+x^2} \qquad v = x$
\subsection{Integration gebrochen rationaler Funktionen}
Gegeben: Gebrochen rationale Funktion $f(x)=\frac{p(x)}{q(x)}$\\
Integration erfolgt in 5 Schritten:
\begin{enumerate}
\item Falls $f$ unecht gebrochen: Polynomdivision erhalten dann $f(x) =\underbrace{a(x)}_{\text{Polynom}} + \underbrace{\frac{r(x)}{q(x)}}_{\text{echt gebrochen}}$
\item Nullstellen von $q$ ermitteln. Dann Zerlegung $q$: \\
$q(x) = (x-\alpha_1)^{k_1}\cdot (x-\alpha_2)^{k_2}\cdot \dots \cdot (x^2+p_1x+q_1)^{m_1}\cdot (x^2+p_2x+q_2)^{m_2}\cdot \dots$\\
$k_i$: reelle Nullstellen \qquad $m_i$: nicht reell zerlegbar\\
Dabei kürzt man eventuelle gemeinsame Faktoren in $r$ und $q$ heraus.
\item \emph{Ansatz für die Partialbruchzerlegung}\\
$\frac{r(x)}{q(x)}= $Summe von Partialbrüchen\\
Jeden Faktor der Form $\begin{cases}
(x-\alpha)^k\\
(x^2+px+q)^m
\end{cases}$ der Gleichung entspricht der Anteil \\
$\begin{cases}
\frac{A_1}{x-\alpha}+\frac{A_2}{(x-\alpha)^2}+\dots + \frac{A_2}{(x-\alpha)^k}\\
\frac{B_1x+C_1}{x^2+px+q}+\frac{B_2 x + C_2}{(x^2+px+q)^2}+\dots + \frac{B_m x + C_m}{(x^2+px+q)^m}
\end{cases}$ in dieser Summe.
\subparagraph{Bsp. 9:}
\begin{align*}
f(x) &= \frac{x^2+4}{(x-1)^3(x+5)(x^2+2x+2)^2}\\
&=\frac{A}{x-1}+\frac{B}{(x-1)^2}+\frac{C}{(x-1)^3}+ \frac{D}{x+5}+ \frac{Ex+F}{x^2+2x+2}+\frac{Gx+H}{(x^2+2x+2)^2}
\end{align*}
Beachte: $x^2+2x+2$ ist reell nicht weiter zerlegbar, Nullstelle: $1\pm i$.
\item Ermittlung der Koeffizienten durch 
\begin{itemize}
\item Multiplikation des Ansatzes der Partialbruchzerlegung mit $q(x)$
\item Kombination der folgenden beiden Methoden
\begin{anumerate}
\item Einsetzen der reellen Nullstellen
\item Koeffizientenvergleich
\end{anumerate}
(falls $q$ nur reelle Nullstellen hat, recht Methode a.)
\end{itemize}
\item Integration der Partialbrüche
\begin{anumerate}
\item $\int \frac{1}{(x-\alpha)^j} \intd{x}=\begin{cases}
\ln(|x-\alpha|)+C & j=1\\
\frac{1}{1-j}(x-\alpha)^{1-j}+C & j=2,3,4,\dots
\end{cases}$
\item $\int \frac{3x+C}{(x^2+px+q)^j}\intd{x}=\int \frac{\tfrac{B}{2}(2x+q)}{(x^2+px+q)^j}+\frac{C-\tfrac{Bp}{2}}{(x^2+px+q)^j}\intd{x}$
\begin{itemize}
\item $\int \frac{2x+p}{(x^2+px+q)^2}\intd{x}$: Nutze Substitution.
\item $\int \frac{1}{(x^2+px+q)^j}\overset{\substack{\text{quadratische}\\\text{Ergänzung}}}{=}\int \frac{1}{\left(\left(x+\frac{p}{2}\right)^2+q-\frac{p^2}{4}\right)^j}\intd{x}\overset{u=x+\frac{p}{2}}{=}\int \frac{1}{\left(u^2+a^2\right)^j}\intd{x}$\\
$\begin{cases}
j=1 & \text{Stammfunktion siehe Merkblatt}\\
j>1 & \text{siehe weitere Formelsammlung (selten)}
\end{cases}$
\end{itemize}
\end{anumerate}
\end{enumerate}
\subparagraph{Bsp. 10:} $I=\int \frac{3x+4}{x^2+2x-3}\intd{x}$
\begin{itemize}
\item echt gebrochen
\item Nullstellen des Nenners: $x_1=-3, \; x_2 = 1$\\
$\Rightarrow x^2+2x-3=(x+3)(x-1)$
\end{itemize}
Ansatz für PBZ:
\begin{align*}
\frac{3x+4}{(x+3)(x-1)}&=\frac{A}{x+3}+\frac{B}{x-1} \qquad |\cdot (x+3)(x-1)\\
3x+4 &= A(x-1)+B(x+3)
\end{align*}
Einsetzen der NS: \\
$x_1: \qquad -5 = A\cdot (-4) \Rightarrow A = \frac{5}{4}$\\
$x_2: \qquad 7 = B \cdot 4 \Rightarrow B=\frac{7}{4}$
\begin{align*}
\Rightarrow I &= \int \frac{\frac{5}{4}}{x+3}\intd{x}+\int \frac{\frac{7}{4}}{x-1}\intd{x}\\
 &= \frac{5}{4} \ln (|x+3|) + \frac{7}{4}\ln(|x-1|)+C
\end{align*}
\subparagraph{Bsp. 11:} $I=\int \frac{7x^2-10x+37}{(x+1)(x^2-4x+13)}\intd{x}$
\begin{itemize}
\item echt gebrochen
\item Nenner reell nicht weiter zerlegbar (denn Nullstellen von $x^2-4x+13$ sind $x_{1/2}=2\pm3i$)
\item Partialbruchzerlegung:
\begin{align*}
\frac{7x^2-10x+37}{(x+1)(x^2-4x+13)}&=\frac{A}{x+1}+\frac{Bx+C}{x^2-4x+13} \quad |\cdot \text{Nenner}\\
7x^2-10x+37&=A(x^2-4x+13)+(Bx+C)(x+1)\\
&= (A+B)x^2+(-4A+B+C)x+(13A+C)
\end{align*}
\begin{tabular}{l l l}
Einsetzen der Nullstelle $x=-1$: & $54=18 A$ & $\Rightarrow A=3$\\
Koeffizientenvergleich $x^2$: & $7=A+B $ & $\Rightarrow B=4$\\
 $x^0$ & $13A+C)$ & $\Rightarrow C=-2$ 
\end{tabular}\\
$\Rightarrow \int \frac{7x^2-10x+37}{(x+1)(x^2-4x+13)}\intd{x}=\int \underbrace{\frac{3}{x+1}}_{a}+ \underbrace{\frac{4x-2}{x^2-4x+13}}_{b}\intd{x} $\\
$a: \; \int \frac{3}{x+1} \intd{x} = 3 \ln|x+1| +C_1$\\
$b: \; \int \frac{4x-2}{x^2-4x+13}\intd{x}=\int\frac{2(2x-4)-2+8}{x^2-4x+13}=2\underbrace{\int \frac{2x-4}{x^2-4x+13}}_{I_1}+\underbrace{\frac{6}{x^2-4x+13}}_{I_2}\intd{x}$\\
$I_1 = \int \frac{2x-4}{x^2-4x+13}\intd{x}\overset{\text{Subst.}}{=}\ln|x^2-4x+13| + C_2$\\
$I_2 = 6\int \frac{\intd{x}}{(x-2)^2+9}=6\int \frac{\intd{u}}{u^2+3^2}=6\cdot \frac{1}{3}\arctan\left(\frac{u}{3}\right)+C_3=2\cdot\arctan\left(\frac{x-2}{3}\right)+C_3$\medskip\\
$\Rightarrow I = 3\ln|x+1| + 2 \ln\left(x^2-4x+13\right)+2\arctan\left(\frac{x-2}{3}\right)+C_4$
\end{itemize}

\subsection{Integration von Potenzreihen}
\paragraph{Satz 1:} Es sei $f:(x_0-r, x_0+r)\to \RR$, $f(x)=\sum_{n=0}^\infty a_n (x-x_0)^n$ die Grenzfunktion der Potenzreihe (mit Konvergenzradius $r$). Dann ist $F: (x_0-r,x_0+r)\to \RR$, $F(x)=\sum_{n=0}^\infty \frac{a_n}{n+1}(x-x_0)^{n+1}$ eine Stammfunktion von $f$ (gliedweises Integrieren im Konvergenzintervall).
\subparagraph{Bsp. 12:} \parskp
$f(x)=\frac{1}{1+x^2}=1-x^2+x^4-x^6+\dots \quad |x|<1$ (siehe Übung, nutze geometrische Reihe)\\
$\Rightarrow \int \frac{1}{1+x^2}\intd{x}=x-\frac{x^3}{3}+\frac{x^5}{5}-\frac{x^7}{7}+\dots + C_1$\\
Wir wissen aber auch: $\int \frac{1}{1+x^2}\intd{x}=\arctan (x) + C_2$\\
$\Rightarrow \arctan(x) = x-\frac{x^3}{3}+\frac{x^5}{5}-\frac{x^7}{7}+\dots + C_3$ \quad mit $C_3=0$ (setze $x=0$ ein) und $|x|<1$.
\subparagraph{Bsp. 13:} Gesucht ist Stammfunktion zu $f(x)=e^{-x^2}$
\begin{align*}
F(x)&=\int_0^x e^{-t^2}\intd{t}=\int_0^x\left(1-t^2+\frac{(t^2)^2}{2!}-\frac{(t^2)^3}{3!}+\dots\right)\intd{t}\\
&= x - \frac{x^3}{3}+\frac{x^5}{5\cdot 2!}-\frac{x^7}{7\cdot 3!}+\frac{x^9}{9\cdot 4!}-\cdots 
\quad (x\in \RR)
\end{align*}
\subparagraph{Diskussion:} 
\begin{itemize}
\item $\int e^{-t^2}\intd{t}$ nicht geschlossen auswertbar.
\item Für nicht zu große $x$ ist die Reihendarstellung zur Auswertung von $F$ gut geeignet.\\
z.B. $\int_0^1 e^{-x^2}\intd{x}=\underbrace{1-\frac{1}{3}+\frac{1}{5\cdot 2!}-\frac{1}{7\cdot 3!}+\frac{1}{9\cdot 4!}-\frac{1}{11\cdot 5!}+\frac{1}{13\cdot 6!}-\frac{1}{15\cdot 7!}+\frac{1}{17\cdot 8!}}_{0,746824\cdots}-\dots$\\
$\left|\text{Fehler}\right| < \frac{1}{19\cdot 9!}=1,4504\cdot 10^{-7}$ (vgl. Satz über Leibnitz-Kriterium)
\end{itemize}

\section{Numerische Integration}
\emph{Ziel:} Berechne $I=\int_a^b f(x) \intd{x}$ falls Stammfunktion „kompliziert“ oder nicht elementar angebbar.\\
\emph{Prinzip:} 
\begin{anumerate}
\item Zerlegung von $[a,b]$ in $n$ gleichlange Teilintervalle der Länge $h=\frac{1}{n}(b-a)$\\
$\Rightarrow$ Teilpunkte sind $x_k=a+k\cdot h \quad (k=0,1,\dots,n)$, $y_k=f(x_k)$
\begin{center}
\includegraphics[scale=.75]{Vorlesung/ABB101}
\end{center}
\item Ersetze $f(x)$ über den Teilintervallen durch einfachere Funktionen.\\
z.B.: \begin{itemize}
\item lineare Funktionen \quad $\rightsquigarrow$ Trapez Regel
\item quadratische Funktionen \quad$\rightsquigarrow$ SIMPSON-Regel
\end{itemize}
\begin{center}
\includegraphics[scale=.75]{Vorlesung/ABB102}
\includegraphics[scale=.75]{Vorlesung/ABB103}
\end{center}
Als Näherung für $I$ ergibt sich für die Simpson-Regel: \\
$I \approx S_n(h)=\frac{h}{3}\left(\tblue{(y_0+y_n)}+4\tgreen{(y_1+y_3+\dots+y_{n-1})}+2\tred{(y_2+y_4+\dots+y_{n-2})}\right)$\\
falls $n$ gerade ist.
\end{anumerate}
\subparagraph{Diskussion:}
\begin{enumerate}
\item Fehlerabschätzung:\\
$I=S_n(h)-\frac{h^4(b-a)}{180}f^{(4)}(\xi) \quad a < \xi < b$\\
(falls $f^{(4)}$ stetig in $[a,b]$)
\item Simpson-Regel ist für Polynome einschließlich Grad 3 exakt.
\item Praktische Durchführung: Schrittweitenhalbierung\\
Startwert: $S^{(1)}=S_n(h)$ für geeignetes $h$. $S^{(2)}=S_{2n}\left(\frac{h}{2}\right)$, $S^{(3)}=S_{4n}\left(\frac{h}{4}\right)$, usw. bis sich die Ziffern in gewünschter Genauigkeit nicht mehr ändern.
\end{enumerate}
\subparagraph{Bsp.:} $\int_0^1 e^{-x^2}\intd{x}$\\
$n=4, \; h=0,25$\\
\begin{tabular}{l l l l l}
$k$ & $x_k$ & \tblue{$y_0, y_n$} & \tgreen{$y_{2j+1}$} & \tred{$y_{2j}$}\\
\hline
0 & 0		& 1	&\\
1 & 0,25& 	&0,939413\\
2 & 0,5	&		&					& 0,778801\\
3 & 0,75&		&0,569783\\
4 & 1		& 0,367879\\
\hline 
 & &		1,367879 & 1,509196 & 0,778801
\end{tabular}\\
$S_4(0,25)=\frac{0,25}{3}\left(1,367879 + 4\cdot 1,509196 + 2\cdot 0,778801\right)$

\section{Uneigentliche Integrale}
\begin{itemize}
\item Vorbetrachtung:\\
Bisher $\int_a^b f(x) \intd{x}$ wobei $[a,b]$ endliches Integral auf $f$ stückweise stetig auf $[a,b]$ (daher beschränkt)
\item 2 Erweiterungen:
\begin{enumerate}
\item unendliches Intervall $(-\infty, b]$, $[a,\infty)$ oder $(-\infty, \infty)$
\item Funktion $f$ unbeschränkt (Unendlichkeits- bzw. Polstellen)
\end{enumerate}
\end{itemize}
\paragraph{Unendliches Intervall} (zu 1.)
\begin{anumerate}
\item $\int_{-\infty}^b f(x) \intd{x}:= \lim_{A\to \infty} \int_A^b f(x) \intd{x}$\\
analog:\\
$\int_a^{\infty} f(x) \intd{x} := \lim_{B \to \infty} \int_a^{B} f(x) \intd{x}$
\item $\int_{-\infty}^{\infty} f(x) \intd{x} := \int_{-\infty}^c f(x) \intd{x} + \int_c^{\infty} f(x) \intd{x}$ für beliebiges $c \in \RR$ (bspw. $c=0$).
\end{anumerate}
\subparagraph{Diskussion:}
\begin{enumerate}
\item Falls die Grenzwerte existieren, so heißt das Integral konvergent, sonst divergent.
\item Ein berühmtes Beispiel ist die $\Gamma$-Funktion:\\
$\Gamma (x) = \int_0^{\infty} e^{-t} t^{x-1} \intd{t} \qquad (x >0 )$\\
Eigenschaft: $\Gamma (n) = (n-1)!$ falls $n \in \NN$
\end{enumerate}
\subparagraph{Bsp. 1:} 
\begin{anumerate}
\item $\int_0^{\infty} e^{-x} \intd{x} = \lim_{A \to \infty} \int_0^A e^{-x} \intd{x} = \lim_{A \to \infty} \left[ -e^{-x}\right]_0^A = \lim_{A \to \infty} \left( -e^{-A}+e^0\right) =1$
\item $\int_0^{\infty} \cos x \intd{x}= \lim_{A\to \infty} \int_0^A \cos x \intd{x} = \lim_{a\to \infty} \left[ \sin (x) \right]_0^A = \lim_{A\to \infty} \left( \sin A - \underbrace{\sin 0}_{=0}\right)$\\
Grenzwert existiert nicht $\Rightarrow$ Integral unbestimmt divergent.
\item $\int_1^{\infty} = \lim_{A\to \infty} \int_1^A \frac{1}{x} \intd{x} = \lim_{A\to \infty} \left[ \ln | x| \right]_1^A = \lim_{A\to \infty} \left( \ln A - \underbrace{\ln 1}_{0}\right) = \infty$\\
$\Rightarrow$ bestimmt divergent
\end{anumerate}
\paragraph{Unbeschränkter Integrand} (zu 2.)
\begin{anumerate}
\item bspw. Unendlichkeitsstellen bei $b$:\\
$\int_a^b f(x) \intd{x} := \lim_{\varepsilon \searrow 0} \int_a^{b-\varepsilon} f(x) \intd{x}$
\begin{center}
\includegraphics[scale=.75]{Vorlesung/ABB104}
\end{center}
\item falls Unendlichkeitsstelle $x_0$ im Inneren von $[a,b]$ liegt:\\
$\int_a^b f(x) \intd{x} := \int_a^{x_0} f(x) \intd{x} + \int_{x_0}^b f(x) \intd{x}$\\
… und nutzen nun a.):\\
$\lim_{\varepsilon\searrow 0} \int_a^{x_0-\varepsilon} f(x) \intd{x} + \lim_{\varepsilon\searrow 0} \int_{x_0 + \varepsilon}^b f(x) \intd{x}$
\end{anumerate}
\subparagraph{Bsp. 2:} \parskp
$\int_0^4 \frac{1}{\sqrt{x}}\intd{x} = \lim_{\varepsilon\searrow0} \left[2\sqrt{x}\right]_{\varepsilon}^4 = \lim_{\varepsilon \searrow 0} \left( 2 \sqrt{4}-2 \sqrt{\varepsilon}\right) = 4$
\paragraph{Unendliches Intervall und unbeschränkter Integrand} (Kombination von 1. und 2.)
\subparagraph{Bsp. 3} \parskp
$I = \int_1^{\infty} \frac{\intd{x}}{x \sqrt{x-1}}= \int_1^2 \frac{\intd{x}}{x\sqrt{x-1}}+ \int_2^{\infty} \frac{\intd{x}}{x\sqrt{x-1}}$\\
mit $\int \frac{\intd{x}}{x\sqrt{x-1}}= \underbrace{\dots}_{\text{Subst. }t=\sqrt{x-1}} = 2 \arctan \sqrt{x-1} + C$:
\begin{align*}
I &= \lim_{\varepsilon \searrow 0} \int_{1+\varepsilon}^2 \frac{\intd{x}}{x \sqrt{x-1}}+ \lim_{a\to \infty} \int_2^A \frac{\intd{x}}{x\sqrt{x-1}}\\
&= \lim_{\varepsilon\searrow0} \left[ 2\arctan \sqrt{x-1}\right]_{1+\varepsilon}^2 + \lim_{A\to \infty} \left[ 2 \arctan\sqrt{x-1}\right]_2^A\\
&= \lim_{\varepsilon\searrow 0} ( 2 \arctan 1 - 2 \arctan \sqrt{\varepsilon}) + \lim_{A\to\infty} (2 \arctan \sqrt{A-1}-2\arctan 1 )\\
&= \underbrace{\lim_{A \to \infty} 2 \arctan \sqrt{A-1}}_{\pi} - \underbrace{\lim_{\varepsilon \searrow} 2 \arctan \sqrt{\varepsilon}}_{0}\\
&= \pi
\end{align*}

\section{Anwendungen}
\subsection{Geometrische Anwendungen}
\subsubsection{Inhalte ebener Flächen}
\begin{itemize}
\item \begin{center}
\includegraphics[scale=.75]{Vorlesung/ABB105}
\end{center}
mit $a<b$ und $f(x) \geq 0$:\\
$F=\int_a^b f(x) \intd{x}$
\item \begin{center}
\includegraphics[scale=.75]{Vorlesung/ABB106}
\end{center}
mit $a<b<c$:\\
$F=\int_a^c |f(x)| \intd{x}=\left| \int_a^b f(x) \intd{x}\right| + \left| \int_b^c f(x) \intd{x}\right|= \int_a^b f(x) \intd{x} + \int_c^b f(x) \intd{x}$
\item \begin{center}
\includegraphics[scale=.75]{Vorlesung/ABB107}
\end{center}
mit $f(x)$: obere Funktion und $g(x)$: untere Funktion:\\
$F=\int_{x_1}^{x_2} f(x) - g(x) \intd{x}$
\end{itemize}
\subparagraph{Bsp. 1:} Gesucht ist der Flächeninhalt $F$ des von der Ellipse $\frac{x^2}{a^2}+\frac{y^2}{b^2}=1 \quad (a>0, \; b>0)$ begrenzten Bereichs.
\begin{center}
\includegraphics[scale=.75]{Vorlesung/ABB108}
\end{center}
$y= \pm b\sqrt{a-\frac{x^2}{a2}}$
\begin{align*}
F&=4 \cdot \int_0^a b \sqrt{1-\frac{x^2}{a}}\intd{x}=\frac{4b}{a}\int_0^a\sqrt{a^2-x^2}\intd{x}\overset{\substack{\text{Subst.}\\x=a\sin t}}{=}\frac{4b}{a}\left[\frac{1}{2}\left(x\sqrt{a^2-x^2}+a^2 \arcsin\frac{x}{a}\right)\right]_0^a\\
&=\frac{4b}{a}\frac{1}{2}a^2\underbrace{\arcsin 1}_{\frac{\pi}{2}}=\pi \cdot ab
\end{align*}

\subsubsection{Bogenlänge}
\paragraph{Bogenlänge ebener Kurven}
\begin{center}
\includegraphics[scale=.75]{Vorlesung/ABB109}
\end{center}
Kurve $K$ mit Parameterdarstellung. $x=x(t) \quad y = y(t) \quad t \in [\alpha, \beta]$
\begin{itemize}
\item Vorgehen: Approximieren durch Streckenzug, dann Verfeinerung\\
Länge des Streckenzugs: $\sum_{i=1}^n \overline{P_{i-1}P_i}=\sum_{i=1}^n \sqrt{(\Delta x_i)^2+(\Delta y_i)^2}$
\begin{center}
\includegraphics[scale=.75]{Vorlesung/ABB110}
\end{center}
$\Rightarrow \text{(Mittelwertsatz der Differentialrechnung} \sum_{i=1}^n \overline{P_{i-1}P_i}= \sum_{i=1}^n \sqrt{(\dot{x}(u_i))^2 + (\dot{y}(v_i))^2}\cdot \Delta t_i$\\
mit $u_i, v_i\in (t_{i-1},t_i)$\\
Verfeinerung:\\
$\int_{\alpha}^{\beta} \sqrt{(\dot{x}(t))^2 + (\dot{y}(t))^2}\intd{t}$
\end{itemize}
\subparagraph{Diskussion:}
\begin{itemize}
\item Bogenlänge der Kurve $\vec{r} = \vec{r}(t) = \mtr{x(t) \\ y(t)}$ zwischen $\alpha$ und $t$ ist\\
$s=\int_{\alpha}^t \underbrace{\sqrt{(\dot{x}(u))^2 + (\dot{y}(u))^2}}_{|\dot{r}(u)|}\intd{u} =: f(t)$\\
$\Rightarrow \frac{\diffd{s}}{\diffd{s}}= |\dot{r}(t)| \Rightarrow \diffd{s}=|\dot{r}(t)| \intd{t}=\sqrt{\dot{x}^2+\dot{y}^2}\intd{t}$ (heißt Bogenelement)
\item Tabelle (Bogenlänge ebener Kurven)\\
\begin{tabular}{l | l}
Kurvendarstellung & Bogenlänge $s$, Bogenelement $\diffd{s}$\\ 
\hline
$x=x(t),\; y=y(t),\; t \in [\alpha,\beta]$ & $s= \int_{\alpha}^{\beta} \underbrace{\sqrt{(\dot{x}(t))^2+(\dot{y}(t))^2}\intd{t}}_{\diffd{s}}$\\
$y =f(x),\; x\in [a,b]$ & $s=\int_a^b\underbrace{\sqrt{1+(f'(x))^2}\intd{x}}_{\diffd{s}}$\\
$x= g(y), \; y \in [c,d] $ & $s=\int_c^d\underbrace{\sqrt{1+(g'(y))^2}\intd{y}}_{\diffd{s}}$\\
$r=r(\varphi), \; \varphi\in [\alpha, \beta]$ & $s=\int_{\alpha}^{\beta}\underbrace{\sqrt{(r(\varphi))^2+(r'(\varphi))^2}\intd{\varphi}}_{\diffd{s}}$
\end{tabular}
\end{itemize}
\paragraph{Bogenlänge von Raumkurven} \parskp
Gegeben sei Kurve $K $mit Parameterdarstellung $x=x(t), \; y=y(t), \; z=z(t), \; t \in [\alpha
, \beta]$.\\
Die Bogenlänge berechnet sich dann mittels\\
$s=\int_{\alpha}^{\beta} \sqrt{(\dot{x}(t))^2+(\dot{y}(t))^2+(\dot{z}(t))^2}\intd{t}$
\subparagraph{Bsp. 2} (Schraubenlinie)\\
$\vec{r}= \vec{r}(t)=\mtr{a \cos t \\ a \sin t \\ \frac{h}{2\pi}t}, \; t \in [0,2\pi]$\\
$\dot{r}(t) = \mtr{-a \sin t \\ a \cos t \\ \frac{h}{2\pi}}$
\begin{align*}
s&= \int_0^{2\pi} \sqrt{\dot{x}^2+\dot{y}^2+\dot{z}^2}\intd{t}\\
&= \int_0^{2\pi} \sqrt{\underbrace{a^2 \sin^2t - a^2 \cos^2t}_{a^2}+\frac{h^2}{(2\pi)^2}}\intd{t}\\
&= 2\pi \sqrt{a^2+\frac{h^2}{4\pi^2}}\\
&= \sqrt{4\pi^2a^2+h^2}
\end{align*}
\subsubsection{Volumen von Rotationskörpern}
\begin{center}
\includegraphics[scale=.75]{Vorlesung/ABB118}
\end{center}
\begin{anumerate}
\item Gegeben:
\begin{itemize}
\item Kurve $K$ mit $y=f(x), \; x \in [a,b]$
\item Das Flächenstück $F_x$ zwischen Kurve und $x$-Achse rotiere um die $x$-Achse.
\end{itemize}
Gesucht:
\begin{itemize}
\item Volumen $V_x$ des dabei erzeugten Körpers
\end{itemize}
Es gilt: $V_x = \pi \cdot \int_a^b(f(x))^2\intd{x}$\\
(Idee: Approximation von $F_x$ durch Rechteckflächen \\
$\overset{Rotation}{\rightsquigarrow}$ Zylinderscheiben $V_{\alpha}\approx \sum_i \pi (f(\xi_i))^2\Delta x_i$\\
Grenzübergang: $V_x=\pi \int_a^b (f(x))^2\intd{x}$)
\subparagraph{Diskussion:}
\begin{enumerate}
\item Allgemein gilt $V_x=\pi \int_a^b y^2\intd{x} \quad (a<b)$
\item Falls K in Parameterform gegeben $x=x(t),\; y=y(t), \; \alpha ... t ... \beta$, dann ergibt sich \\
$V_x=\pi \int_{\alpha}^{\beta} (y(t))^2\cdot \underbrace{\dot{x}(t) \intd{t}}_{\diffd{x}}$, wobei $...$ (die Orientierung) so zu wählen ist, dass $a:= x(\alpha) < x(\beta) = b$ gilt. Unter Umständen kann $\alpha<\beta$ sein.
\end{enumerate}
\item Gegeben: Kurve $x=g(y), \; y\in [c,d]$
\begin{center}
\includegraphics[scale=.75]{Vorlesung/ABB119}
\end{center}
Gesucht: Volumen $V_y$ bei Rotation von $F_y$ um $y$-Achse.\\
$V_y=\pi \int_c^d x^2\intd{y} = \pi \int_c^d (g(y))^2\intd{y}$\\
Für Parameterdarstellung:\\
$V_y=\pi \int_{\alpha}^{\beta}(x(t))^2\cdot \underbrace{\dot{y}(t) \intd{t}}_{\diffd{y}}$, wobei $c=y(\alpha)<y(\beta)=d$
\subparagraph{Bsp. 3:} Gesucht ist das Volumen eines Rotationsparaboloids der Höhe $h$ und mit Basisradius $R$.
\begin{center}
\includegraphics[scale=.75]{Vorlesung/ABB120}
\end{center}
Kurve: $y=ax^2, \; h=aR^2 \Rightarrow a =\frac{h}{R^2}$
\begin{align*}
V_y&=\pi\int_0^h x^2\intd{y}=\pi\int_0^h \frac{y}{a}\intd{y} = \pi\int_0^h \frac{R^2}{h}y  \intd{y}\\
&= \frac{\pi R^2}{h}\int_0^h y \intd{y}=\frac{\pi R^2}{h}\left[\frac{1}{2}y^2\right]_0^h\\
&= \frac{\pi R^2}{h}\cdot \frac{1}{2}h^2=\frac{\pi R^2 h}{2}
\end{align*}
\end{anumerate}

\subsubsection{Mantelflächen von Rotationskörpern}
\begin{anumerate}
\item Gegeben: Kurve $K$ mit $y=f(x)\geq 0, \; x \in [a,b]$\\
Gesucht: Die von der Kurve $K$, bei Rotation um $x$-Achse erzeugte Rotationsfläche $M_x$.
\begin{center}
\includegraphics[scale=.75]{Vorlesung/ABB121}
\end{center}
$M_x=2\pi \int_a^b f(x) \sqrt{1+(f'(x))^2}\intd{x}$\\
(Approximation der Kurve $K$ durch Polygonzug\\
$\overset{Rotation}{\rightsquigarrow}$ Kegelstumpffläche\\
$M_x\approx \sum_i 2\pi f(\xi_i)\Delta s_i\\
\to 2\pi \int_a^b f(x)\underbrace{\sqrt{1+(f'(x))^2}\intd{x}}_{\diffd{s}}$ (Bogenelement))\\
Allgemein gilt: $M_x=2\pi \int_a^by\intd{s}$
\item Gegeben: Kurve $x=g(y) \geq 0, \; y\in [c,d]$\\
Gesucht: Mantelfläche bei Rotation um die $y$-Achse.\\
$M_y=2\pi \int_K x \intd{s}=2\pi \int_c^d g(y) \sqrt{1+(g'(y))^2}\intd{y}$\\
Für Parameterdarstellung:\\
$M_y=2\pi \int_{\alpha}^{\beta} x(t) \sqrt{(\dot{x}(t))^2+(\dot{y}(t))^2}\intd{t}$ mit $\alpha \leq t \leq \beta$
\subparagraph{Bsp. 4:} Kugeloberfläche
\begin{center}
\includegraphics[scale=.75]{Vorlesung/ABB122}
\end{center}
Halbkreis $K$ soll um $x$-Achse rotiert werden.\\
$x=R\cdot \cos t\\
y = R \cdot \sin t\\
t \in [0,\pi]$\\
$\dot{x}(t) = -R \sin t \qquad \dot{y}(t) = R \cos t$
\begin{align*}
M_x&= 2\pi \int_K y\intd{s}=2\pi \int_0^{\pi} y(t) \sqrt{\dot{x}^2+\dot{y}^2}\intd{t}\\
&= 2\pi \int_0^{\pi} R \sin t \cdot R \intd{t} = 2\pi R^2 \left[-\cos t\right]_0^{\pi}\\
&= 4\pi R^2
\end{align*}
\end{anumerate}
\subsubsection{Fourier-Reihen}
Gegeben: Funktion $f(x), \; x \in [0,T]$
\begin{center}
\includegraphics[scale=.75]{Vorlesung/ABB123}
\end{center}
Gesucht: Reihendarstellung mit trigonometrischen Funktionen der Periode $T, \; \frac{T}{2}, \; \frac{T}{3}, \dots$\\
d.h. (mit $\omega= \frac{2\pi}{T}$):\\
$\cos (\omega x), \; \cos (2 \omega x) ,\; \cos (3 \omega x) , \dots$\\
$\sin (\omega x), \; \sin (2 \omega x) ,\; \sin (3 \omega x), \dots$\\
Ansatz ist daher:\\
$f(x) = \frac{a_0}{2}+\sum_{k=1}^a \left( a_k \cos (k\omega x) + b_k \sin (k\omega x) \right)$, wobei die Koeffizienten $a_k, \; b_k,\; k\geq 0$ zu ermitteln sind.\\
Motivation ist: Approximation von $f$ zum Zwecke der Speicherplatzreduzierung (Abgespeichert werden i.A. nur wenige der Koeffizienten $a_k$ und $b_k$. Das gilt auch dann, wenn $f$ in diskreter Form vorliegt, d.h. in Form von Messwerten $y_k$ an vielen Messstellen $x_k$.)\\
Vorgehensweise:
\begin{enumerate}
\item Betrachte zunächst die endliche Reihe $f_n(x):= \frac{a_0}{2}+\sum_{k=1}^n a_k \cos (k\omega x) + b_k \sin (k \omega x )$.\\
Ziel ist: $a_k$ und $b_k$ so wählen, dass $f_n$ möglichst gute Approximation von $f$ ist.
\item Approximation in Vektorräumen mit Skalarprodukt
\begin{itemize}
\item Es sei $V$ ein Vektorraum. Das Skalarprodukt $(f,g)$ ist eine Abbildung von $V\times V$ nach $\RR$ mit den Eigenschaften 
\begin{anumerate}
\item $(f,f) > 0$ für alle $f \not = 0$
\item $(f,g) = (g,f)$ für alle $f,g \in V$ (Symmetrie)
\item $(\alpha f+\beta g,h) = \alpha (f,h) + \beta (g.h.)$ für $f,g,h \in V, \; \alpha, \beta \in \RR$ (Linearität)
\end{anumerate}
\item Die Norm von $f$ (oder Betrag von $f$) ist $\|f\|:= \sqrt{(f,f)}$ für $f\in V$
\item $f$ und $g$ orthogonal $:\Leftrightarrow$ $(f,g) = 0$
\item Beispiele:
\begin{anumerate}
\item $V=\RR^n, \; (\vec{x}, \vec{y}) = \sum_{i=1}^n x_i y_i$
\item $C(a,b)$ … Menge der auf $[a,b]$ stetigen reellwertigen Funktionen mit Skalarprodukt: $(f,g) = \int_a^b f(x) g(x) \intd{x}$
\end{anumerate} 
\item Aufgabe: Approximation von $f\in V$ durch $f^*\in V^*$ wobei $V^*\subseteq V$ ein $m$-dimensionaler Unterraum ist, mit orthogonaler Basis $e_1,\dots,e_m$ (d.h. $(e_i, e_j) = 0 $ falls $i\not = j$). Gesucht ist also dasjenige $f^* \in V^*$, für welches $\| f - f^* \|$ minimal wird.
\begin{center}
\includegraphics[scale=.75]{Vorlesung/ABB126}
\end{center}
d.h. $f^*$ ist die orthogonale Projektion von $f$ auf $V^*$
\paragraph{Satz 1:} Die orthogonale Projektion $f^*$ von $f$ auf $V^*$ ist gegeben durch: $f^* = \sum_{i=1}^m \alpha_i e_i$, wobei $\alpha_i=\frac{(f,e_i)}{(e_i,e_i)}, \; i=1,..,n$
\end{itemize}
\item Übertragung auf $C(O,T), \; T=\frac{2\pi}{\omega} \to \omega = \frac{2\pi}{T}$
\begin{itemize}
\item Man kann zeigen, dass die Funktion\\
$\underbrace{1}_{g_0},\; \underbrace{\cos(\omega x)}_{g_1},\; \underbrace{\cos(2\omega x)}_{g_2},\; \cos (2\omega x) ,\; \dots ,\; \underbrace{\sin (\omega x)}_{h_1},\; \underbrace{\sin(2\omega x)}_{h_2},\; \sin(2\omega x ),\; \dots$\\
in $C(O,T)$ orthogonal sind.\\
z.B. $(g_0, g_1) = \int_0^T 1 \cdot \cos (\omega x) \intd{x}=\left[ \frac{1}{\omega }\sin (\omega x) \right]_0^T=0$
\item Außerdem gilt $(g_0, g_0) = T ,\; (g_k, g_k) = \frac{T}{2}=(h_k, h_k)$ mit $k\in \NN$
\end{itemize}
\item Damit ergibt die Projektion von $f \in C(0,T)$ in $V^*=L(g_0,g_1, \dots, g_n, h_1, \dots, h_n)$ folgende Koeffizienten:\\
$a_k = \frac{(f, g_k)}{(g_k, g_k)}=\frac{2}{T}\int_0^T f(x) \cos(k \omega x) \intd{x}$\\
$b_k = \frac{(f, h_k)}{(h_k, h_k)}=\frac{2}{T}\int_0^T f(x) \sin(k \omega x) \intd{x}$\\
(bei der Approximation in der Ausgangsgleichung $f_n$)
\item Frage: $f_n\to f$ falls $n\to \infty$?
\paragraph{Satz 2:} Seien $f$ und $f'$ auf $[0,T]$ stückweise stetig mit höchstens endlich vielen, endlichen Sprungquellen, dann gilt an allen Stetigkeitsstellen von $f$: \\
$f(x) = \underbrace{\frac{a_0}{2}+\sum_{k=1}^{\infty} (a_k \cos (k\omega x) + b_k \sin (k \omega x) )}_{\lim_{n\to\infty} f_n(x)}$ mit\\
 $a_0 = \frac{2}{T}\int_0^T f(x) \intd{x},\\
a_k = \frac{2}{T}\int_0^T f(x) \cos (k \omega x) \intd{x}, \\
b_k = \frac{2}{T} \int_0^T f(x) \sin (k \omega x) \intd{x}$ \qquad ($k\in \NN$)\\
Für die Sprungstellen $x_S$ gilt $\lim_{n\to\infty} f_n(x_S) = \frac{1}{2}\left( \lim_{x\nearrow x_s} f(x) + \lim_{x \searrow x_S} f(x)\right)$
\begin{center}
\includegraphics[scale=.75]{Vorlesung/ABB127}
\end{center}
Also: Ja, $f_n\to f$ falls $n\to \infty$.
\subparagraph{Bemerkung:} 
\begin{enumerate}
\item Die Vorstehenden Ausführungen gelten automatisch für die periodische Fortsetzung $\tilde f$ einer zunächst auf $[0,T]$ definierten Funktion $f$
\begin{center}
\includegraphics[scale=.75]{Vorlesung/ABB128}
\end{center}
\item Im Fall der periodischen Fortsetzung kann das Integrationsintervall $[0,T]$ durch beliebiges Intervall der Länge $T$ ersetzt werden, z.B. $\left[-\frac{T}{2}, \frac{T}{2}\right]$
\item Ist $f$ symmetrisch gilt (vereinfachend):\\
\begin{tabular}{l | l l l }
$f$ gerade & $a_0 = \frac{4}{T}\int_0^{\frac{T}{2}}f(x) \intd{x}$, $ a_k = \frac{4}{T}\int_0^{\frac{T}{2}}f(x) \cos (k\omega x) \intd{x}$, $ b_k =0$\\
\hline
$f$ ungerade & $a_0=0$, $a_k = 0$, $b_k = \frac{4}{T}\int_0^{\frac{T}{2}}f(x) \sin (k\omega x) \intd{x}$
\end{tabular}  
\item Amplitudenspektrum\\
$A_k := \sqrt{a_k^2+b_k^2}\quad k=1,2,3,\dots$\\
… Amplituden der Schwingungen, die sich durch Zusammenfassung der Sinus- und Cosinusanteile gleicher Frequenz ergeben.\\
(Möglichkeit der Verstärkung/Dämpfung der Schwingung)\\
Die Fourier Reihe lässt sich mit diesem $A_k$ auf folgende Form bringen:\\
$f(x)=\frac{A_0}{2}+\sum_{k=1}^{\infty} A_k \cos (k x - \varphi_k)$ mit $A_0=a_0$ und $\varphi_k = \arccos \left( \frac{a_k}{A_k}\right) \quad k\geq 1$
\end{enumerate}
\end{enumerate}
\subparagraph{Bsp. 6:} $f(x) = \begin{cases}
0 & -1 \leq x < 0\\
1 & 0 \leq x <1
\end{cases}$, $\tilde f$ periodische Fortsetzung
\begin{center}
\includegraphics[scale=.75]{Vorlesung/ABB129}
\end{center}
Gesucht: Fourier Reihe zu $f$ mit $T=2, \; \omega = \frac{2\pi}{2}= \pi$\\
$a_0 = \frac{2}{2} \int_{-1}^1 f(x) \intd{x} = \int_0^1 1 \intd{x} = 1$\\
$a_k = \frac{2}{2} \int_{-1}^1 f(x) \cos (k \pi x) \intd{x} = \int_0^1 \cos (k \pi x) = \frac{1}{k\pi}\left[\sin (k\pi x) \right]_0^1=0$\\
$b_k = \frac{2}{2} \int_{-1}^1 f(x) \sin (k\pi x) \intd{x} = \int_0^1 \sin (k\pi x) \intd{x} = -\frac{1}{k\pi} \left[ \cos (k \pi x)\right]_0^1 = -\frac{1}{k\pi}(cos(k\pi)-1) \\
\Rightarrow b_k=\begin{cases}
0 & k \text{ gerade}\\
\frac{2}{k\pi} & k \text{ ungerade}\\
\end{cases}$\\
$\tilde f ( x) = \frac{1}{2}+ \frac{2}{\pi} \left(\sin (\pi x) + \frac{1}{3} \sin (3\pi x) + \frac{1}{5} \sin (5 \pi x) + \dots\right) \quad x \in RR\setminus \ZZ$
\begin{center}
\includegraphics[scale=.75]{Vorlesung/ABB130}
\end{center}

\paragraph{Diskreter Fall:} Gegeben $y=f(x), \; x\in [0,T]$ ausgewertet (gemessen) an $N$ Messstellen (wobei $N$ gerade) z.B. Messung alle $\frac{1}{100}\mathrm{sec}=\text{Samplingrate }100\; \mathrm{Hz}$ (bei Audio CDs $44,1\; \mathrm{kHz}$
\begin{center}
\includegraphics[scale=.75]{Vorlesung/ABB131}
\end{center}
$x_j = j\cdot h = j \cdot \frac{T}{N}$ für $j=0, \dots, N-1$\\
$y_j = f(x_j)$
\begin{itemize}
\item Mit Ansatz von stetigen Funktionen für $n\leq \frac{N}{2}$ führt im Vektorraum $\RR^N$ mittels Satz 1 auf: \\
$a_0=\frac{2}{N}\sum_{j=0}^{N-1} y_j, \; \\
a_k = \frac{2}{N}\sum_{j=0}^{N-1} y_j \cdot \cos (k \omega x_j), \; \\
b_k = \frac{2}{N} \sum_{j=0}^{N-1} y_j \cdot \sin (k \omega x_j)$
\item Im Fall $2n = N$ ergibt sich keine Approximation, sondern eine exakte Darstellung des Vektors $\vec{y}=\left(y_j\right)_{j=0}^{N-1}$ mit einer anderen Basis.\\
Mit Hilfe der Eulerschen Formel $e^{i\varphi}=\cos \varphi + i \sin \varphi$ lässt sich der Ansatz mit komplexen Koeffizienten $c_k$ umschreiben:\\
$f(x)=\sum_{k=0}^{N-1} c_k e^{i\omega k x}$ und damit $y_j=f(x_j) = \sum_{k=0}^{N-1} c-k e^{i\frac{2\pi}{N}j\cdot k} \quad (j=0,1,\dots, N-1)$\\
In Matrix-Schreibweise:\\
Sei $ w:= e^{i\frac{2\pi}{N}}s$ (eine von $N$ Lösungen der Kreisgleichung $Z^N=1$)\\
$\vec{F}=\left( w^{jk}\right)_{j,k=0}^{N-1}$\\
$\Rightarrow \vec{y}=\vec{F}\, \vec{c}$ (inverse diskrete Fourier Transformation \emph{IDFT})
\item Man kann zeigen, dass $\vec{F}$ regulär ist mit $\vec{F}^{-1}=\frac{1}{N}F^*$, wobei $F^*=\left(\overline{w}^{kj}\right)_{k,j=0}^{N-1}$ und $\overline{w}=e^{i\frac{2\pi}{N}}$ (konjugiert komplex). Damit $\vec{c}=\frac{1}{N}F^*\vec{y}$. D.h. $\boxed{c_k = \frac{1}{N}\sum_{j=0}^{N-1} y_j e^{-i\frac{2\pi}{N}k\cdot j}}$ (diskrete Fourier Transformation \emph{DFT})
\item Fast Fourier Transformation \emph{FFT}\\
Wenn $N=2^m$, lässt sich die Rechenzeit stark verkürzen.
\item Diskrete Cosinus Transformation \emph{DCT}\\
$y=f(x) , \; x \in [0,T], \; N$ gleiche Intervalle (meist $N=2^m$). Die Abtastpunkte sind hier die \emph{Mitten der Intervalle}.
\begin{center}
\includegraphics[scale=.75]{Vorlesung/ABB132}
\end{center}
$x_j=\frac{h}{2}+j\cdot h=T\cdot \frac{2j+1}{2N} \quad (j=0,\dots, N-1)$\\
Ansatz: $y=\sum_{k=0}^n a_k \cos (k\frac{\omega}{k}x)$ (stetiger Fall)\\
$y_j=\sum_{k=0}^{N-1} a_k \cos\left(\frac{k\pi (2j+1)}{2N}\right)$ (diskreter Fall)\\
Matrixform: $\vec{y}=\vec{C}\vec{a}$\\
Die Spalte von $\vec{C}$ bilden orthogonale Basis von $\RR^N$ (entsprechende Normierung führt zu \emph{orthonormaler Basis}). Daher können wir wie bei der Fourier-Transformation vorgehen:\\
$\vec{a}=\vec{C}^{-1}y$ … DCT\\
$\vec{y}=C \vec{a}$ … IDCT\\
Anwendung: Audio und Videokompression (MP3, JPEG). Die Komprimierung erfolgt dabei im Frequenzbereich. kleine Amplituden $a_k\to 0$ $\rightsquigarrow$ Datenreduktion
\end{itemize}