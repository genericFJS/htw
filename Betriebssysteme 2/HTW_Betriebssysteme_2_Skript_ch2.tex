\lecdate{26.04.2017}
Mechanismen:
\begin{itemize}
\item Message Passing
\item Shared Memory Segment
\item Datei
\item Signale
\item Pipe (unbenannt)
\begin{itemize}
\item unidirektional
\item nur zwischen verwandten Prozessen möglich
\end{itemize}
\item Pipe (benannt)
\item Socket
\end{itemize}
\section{Message Passing}
\slides{03-ipc}{2}

\subsection{Synchronisation mit Nachrichten}
\slides{03-ipc}{3}
\subsection{Adressierung}
\slides{03-ipc}{4}
\subsection{Praxisbeispiele}
\subsubsection{Message Passing Interface (MPI)}
\slides{03-ipc}{5}
\subsubsection*{Beispiel 2 Knoten, FORTRAN}
\slides{03-ipc}{6}
\slides{03-ipc}{7}
\subsubsection{Mikrokern L4}
\slides{03-ipc}{8}
\subsubsection{Pascal-FC}
\slides{03-ipc}{9}
\slides{03-ipc}{10}
\subsubsection{Nachrichtenwarteschlangen in der System-V-IPC}
System-V-IPC:
\begin{itemize}
\item Semaphore
\item Message Queues
\item Shared-Memory-Signale
\end{itemize}
\slides{03-ipc}{11}
Details: \lstinline$msgserver.c$ und \lstinline$msgclient.c$.

\section{Benannte Pipes}
\begin{itemize}
\item „sieht aus“ wie eine Datei, d.h. hat einen Eintrag in Dateisystem
\item syscall: \lstinline$mkfifo()$ -- legt FIFO (benutzt Pipe) in Dateisystem an
\item zugehöriges Unix-Kommando: \lstinline$mkfifo$
\item Kommandos: \lstinline$open(), read(), write(), close()$ (kein \lstinline$lseek$!)
\item Löschen mittels \lstinline$rmlink()$ bzw \lstinline$rm$-Kommando
\end{itemize}

\section{Shared Memory}
\lecdate{03.05.2017}
\slides{03-ipc}{12}
\subsection{Systemrufe in der System-V-API}
\slides{03-ipc}{13}
\slides{03-ipc}{14}
\subsection{POSIX-API}
\slides{03-ipc}{15}




























