\lecdate{24.05.2017}
%\section{Überblick}
%\slides{06-filesystems}{2}

\section{Implementierungen von Dateisystemen}
\subsection{Kontinuierliche Allokation}
\slides{06-filesystems}{3}

\subsection{Verkettete Liste}
\slides{06-filesystems}{4}
Beispiel ist nicht so schnell, da gesprungen werden muss (0015$\to$FFFF$\to$0014).\\
\subsubsection{Nachteile}
\slides{06-filesystems}{5}
Wichtiger konzeptioneller Nachteil: Dadurch, dass in jedem Nutzdatenblock eine Verwaltungsinformation gespeichert ist, ist der Netto-Nutzdatenblock keine 2er-Potenz mehr!
\subsubsection{Liste mit Zuordnungstabelle}
\slides{06-filesystems}{6}
Größe bspw. FAT16: $2^{16}\cdot 32 \unit{KiB}$ ($32\unit{KiB}$: Clustergröße)

\subsection{Indizierte Speicherung}
\slides{06-filesystems}{7}
\subsubsection{Speicherung mit variablen Indexblocks}
\slides{06-filesystems}{8}
\subsubsection{Indirekt-indizierte Speicherung}
\slides{06-filesystems}{9}
\slides{06-filesystems}{10}

%\subsubsection{Beispiel ISO 9660 (1988)}
%\slides{06-filesystems}{11}
%\slides{06-filesystems}{12}

\subsubsection{Beispiel Unix Dateisystem}
\slides{06-filesystems}{13}
\subsubsection*{Dateiadressierung mittels Inodes}
\slides{06-filesystems}{14}
(graue Datenblöcke sind gleich groß wie die weiß gezeichneten Datenblöcke!)
\slides{06-filesystems}{15}

\subsection{Journaling}

\section{I/O-Scheduling}
\subsection{FCFS, SSTF}

\subsection{SCAN (Elevator) und Varianten}

\subsection{Shortest Access Time First (SATF)}

\subsection{Verfahren im Linux-Kernel}




