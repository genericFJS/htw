\lecdate{28.06.2017}

\section{Grundbegriff}
\subsection{Ziele der Systemsicherheit}
\slides{07-security}{3}
\subsection{Bedrohungen}
\slides{07-security}{4}

\section{Bösartige Software}
\subsection{Überblick}
\slides{07-security}{5}
\begin{itemize}
\item lokaler Angriff (insider Angriff): Nutzer hat bereits Zugang zu System und versucht bspw. \lstinline`root` zu werden.
\item entfernter Angriff: Zugriff von außen (kein Account auf angegriffenen System vorhanden).
\item on-line-Angriff: Hacker ist verbunden ist mit angegriffenen System und „hackt“ (selten) [auch lokal möglich].
\item off-line-Angriff: bspw. durchforsten von Passwort-Dateien (die erklaut wurden).
\end{itemize}

\subsection{Logische Bomben}
\slides{07-security}{6}
\subsubsection*{Ausschnitt McAfee Aktivierungskalender}
\slides{07-security}{7}

\subsection{Hintertüren (Backdoors)}
\slides{07-security}{8}
\subsubsection*{Beispiel 1}
\slides{07-security}{9}
\subsubsection*{Beispiel 2}
\slides{07-security}{10}
Bei \lstinline`current->uid = 0` wird nicht verglichen, sondern \lstinline`uid` gesetzt! Wenn also die angegeben Option-Flags gesetzte werden, wird der ausführende Nutzer \lstinline`root`.
\slides{07-security}{11}

\subsection{Trojanisches Pferd}
\slides{07-security}{12}
\subsubsection*{Beispiel Unix}
\slides{07-security}{17}
\lstinline`u+s`: Setzen des Sticky-Bits (Programm läuft unter rechten des Eigentümers der Datei, nicht unter den rechten des Ausführenden)

\subsection{(Computer-)Viren}
\slides{07-security}{18}
\subsubsection*{Einfaches Beispiel}
\slides{07-security}{23}
\subsubsection*{Erweitertes Beispiel}
\slides{07-security}{24}
$\to$ keine Mehrfachinfektion mehr

\subsection{Würmer}
\slides{07-security}{25}
\subsubsection*{Komponenten eines Wurms}
\slides{07-security}{26}
\slides{07-security}{27}
Vokabeln:
\begin{itemize}
\item Vulnerability: Schwachstelle
\item Exploit: Ausnutzen einer Schwachstelle
\end{itemize}

\subsection{Rootkits}
\slides{07-security}{28}
Hinweis: Rootkit ist nicht dafür da, root zu werden, sondern das Schadprogramm zu verbergen.
\subsubsection*{Arten von Rootkits}
\slides{07-security}{29}
\subsubsection*{Gegenmaßnahmen}
\slides{07-security}{30}

\section{Authentifizierungsmechanismen}

\section{Angriffstechniken}

\subsection{Buffer Overflow}

\subsection{Return-into-Libc}

\subsection{Format String Exploit}

\section{Angriffscode (Shell)}


% Folie 44:
\lstinline`strcopy` interessiert sich nicht für Buffergröße $\to$ gesamtes \lstinline`argv[1]` wird in Speicher geschrieben. Dadurch kann die Rücksprungadresse überschrieben werden. Damit kann Schadcode ausgeführt werden (das wird von modernen Betriebssystemen allerdings verhindert).