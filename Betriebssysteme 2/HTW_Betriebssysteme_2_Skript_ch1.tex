\lecdate{22.03.2017}
\section{Beispiel Bankautomat}
\slides{00-synchronisation}{2}
\subsection*{Möglicher (typischer) Ablauf}
\slides{00-synchronisation}{3}
\section{Race Condition}
\slides{00-synchronisation}{4}
\section{Kritischer Abschnitt}
\slides{00-synchronisation}{5}
$\to$ Lost-Update-Problem!
\begin{framed}
Zugriffsoperationen zur gemeinsam genutzten Variable
bilden einen so genannten \emph{kritischen Abschnitt} (critical
section).
\end{framed}
Zur Erinnerung: Scheduling (kooperativ $\to$ syscalls [$\Rightarrow$ Semaphor], …)
\section{Steuerung des kritischen Abschnitts}
\slides{00-synchronisation}{6}
\subsection*{Steuerung durch klammernde Funktionen}
\slides{00-synchronisation}{7}
$\to$ dezentrale Steuerung\\
Was müssen die Funktionen \lstinline$enter_cs()$ und \lstinline$leave_cs()$ tun?
\slides{00-synchronisation}{8}
\section{Realisierungsvarianten}
\slides{00-synchronisation}{9}
\section{Dezentrale Lösungsversuche}
\subsection{Naiver Versuch 1: „Ping-Pong“}
\slides{00-synchronisation}{11}
$\Rightarrow$ funktioniert daher nicht vernünftig.
\subsection{Versuch 2: Variable zeigt freien kritischen Abschnitt an}
\slides{00-synchronisation}{12}
\subsubsection*{Bewertung}
\slides{00-synchronisation}{13}
$\Rightarrow$ funktioniert gar nicht, da kritischer Abschnitt von beiden betreten werden kann!
\subsection{Versuch 3: Mehrere Variablen zeigen freien kritischen Abschnitt an}
\slides{00-synchronisation}{14}
\subsubsection*{Bewertung}
\slides{00-synchronisation}{15}
$\Rightarrow$ funktioniert auch nicht, aus selben Grund wie Versuch 2!
\subsection{Versuch 4: Mehrere Variablen zeigen Wunsch zum Eintreten in kritischen Abschnitt an}
\slides{00-synchronisation}{16}
\subsubsection*{Bewertung}
\slides{00-synchronisation}{17}
$\Rightarrow$ funktioniert nicht vernünftig, da Deadlock-Risiko.
\subsection{Versuch 5: Mehrere Variablen mit Eintrittswunsch und weiterer Prüfung}
\slides{00-synchronisation}{18}
\subsubsection*{Bewertung}
\slides{00-synchronisation}{19}
$\Rightarrow$ durch zufällige Verzögerung ist diese Lösung schon relativ gut. Statistisch kann trotzdem (bei Mehrprozessorsystemen) noch ein Livelock eintreten.
\subsection{Algorithmus von Peterson}
Vergleich auch: Algorithmus von Dekker und Dijkstra.
\slides{00-synchronisation}{20}
\subsubsection*{Bewertung}
\slides{00-synchronisation}{21}
Anmerkung: Bei Implementation kann es passieren, dass es nicht funktioniert $\to$ Der Algorithmus geht davon aus, dass Wertänderungen (bspw. von \lstinline$turn$) sofort von dem anderen Prozess gesehen wird. Tatsächlich passiert das (bei typischen Multiprozessorsystemen) erst zeitverzögert. Dafür müssten Schreibbarrieren eingeführt werden.
\subsection{Fazit}
\lecdate{29.03.2017}
\slides{00-synchronisation}{22}
„busy waiting“: Prozess verbraucht noch immer Ressourcen (durch \lstinline$while$-Schleife), obwohl er nur warten soll.

\section{Zentrale Lösung: Semaphore}
\subsubsection*{Motivation}
\slides{00-synchronisation}{23}

\subsection{Zustände eines Semaphors}
\slides{00-synchronisation}{24}

\subsection{Operationen eines Semaphors}
\slides{00-synchronisation}{25}
\slides{00-synchronisation}{26}

\subsubsection*{Implementierung der P()-Funktion}
\slides{00-synchronisation}{27}

\subsection{API für Semaphore}
\subsubsection*{Unter Unix}
\slides{00-synchronisation}{28}
\subsubsection*{Weitere}
\slides{00-synchronisation}{29}

\subsection{Anwendung: Zeitliche Synchronisation von Prozessen}
\slides{00-synchronisation}{31}
\subsubsection*{Rendezvous}
\slides{00-synchronisation}{34}
Das Vertauschen würde zu einem Deadlock führen.
\subsection{Weitere Aspekte}
\slides{00-synchronisation}{35}
Bekannte anschauliche Probleme:
\begin{itemize}
\item Leser-Schreiber-Problem (gleichzeitiges Lesen, nur einer darf immer schreiben)
\item Erzeuger-Verbraucher-Problem (beschränkter Puffer)
\item Problem des schlafenden Friseurs (verschieden exklusive Ressourcen)-
\item Problem der dinierenden Philosophen (Verteilung von begrenzten Ressourcen ohne Verhungern/Deadlock)
\end{itemize}

\subsubsection{Problem der dinierenden Philosophen}
\slides{00-synchronisation}{36}
\subsubsection*{Naive Lösung}
\begin{lstlisting}[language=C]
/* Code für den n-ten Philosophen, naive Loesung */
void philosopher(int n){
  while(1) {
    /* denken */
    take_fork(n);             /* rechte Gabel nehmen */
    take_fork((n+1)%5);       /* linke Gabel nehmen */
    /* essen */
    put_fork((n+1)%5);        /* linke Gabel weglegen */
    put_fork(n);              /* rechte Gabel weglegen */
  }
}

\end{lstlisting}
Problem: jeder greift nach seiner rechten Gabel $\to$ keiner kann mehr nach der linken greifen (Deadlock)!
\subsubsection*{Vorsichtiges Greifen nach 2. Gabel}
\begin{lstlisting}[language=C]
/* Code für den n-ten Philosophen, 'vorsichtiges' Greifen nach 2. Gabel */
void philosopher(int n){
  while(1) {
    /* denken */
  again:
    take_fork(n);             /* rechte Gabel nehmen */
    if (!available(fork((n+1)%5)) { /* falls linke Gabel nicht verfuegbar ... */
      put_fork(n);            /* ... rechte Gabel zuruecklegen ... */ 
      sleep(10);              /* ... eine Weile warten ... */ 
      goto again;             /* ... und von vorn probieren. */
    }
    take_fork((n+1)%5);       /* linke Gabel nehmen */
    /* essen */
    put_fork((n+1)%5);        /* linke Gabel weglegen */
    put_fork(n);              /* rechte Gabel weglegen */
  }
}
\end{lstlisting}
Problem: Kann zu einem Livelock führen (besser: sleep-Zeit zufällig auswählen).

\subsubsection*{1 Semaphor}
\begin{lstlisting}[language=C]
/* Code für den n-ten Philosophen, 1 Semaphor */
semaphore eat = 1;        /* Init: offen */
void philosopher(int n){
  while(1) {
    /* denken */
    P(eat);                   /* Erlaubnis zum Aufnehmen der Gabeln einholen */
    take_fork(n);             /* rechte Gabel nehmen */
    take_fork((n+1)%5);       /* linke Gabel nehmen */
    /* essen */
    put_fork((n+1)%5);        /* linke Gabel weglegen */
    put_fork(n);              /* rechte Gabel weglegen */
    V(eat);                   /* Erlaubnis zurückgeben */
  }
}

\end{lstlisting}
Problem: Maximale Anzahl an essenden Philosophen wird auf 1 reduziert!

\subsubsection*{Lösung nach Tanenbaum}
\begin{lstlisting}[language=C]
/* Loesung des Philosophenproblems nach Tanenbaum */
#define N 5
#define RIGHT(i) (((i)+1) % N)
#define LEFT(i) (((i)==N) ? 0: (i)+1)

enum phil_state {THINKING, HUNGRY, EATING};
enum phil_state state[N];     /* geeignet initialisiert */

semaphore mutex = 1;  
semaphore s[N];               /* müssen geschlossen initialisiert werden */

void test(int i){
  if ((state[i] == HUNGRY) &&
      (state[LEFT(i)] != EATING) &&
      (state[RIGHT(i)] != EATING)) {
    state[i] = EATING;
    V(s[i]);
  }
}

void get_forks(int i){
  P(mutex);
  state[i] = HUNGRY;
  test(i);
  V(mutex);
  P(s[i]);                /* hungrig blockieren oder verlassen, d.h., essen */
}

void put_forks(int i){
  P(mutex);
  state[i] = THINKING;
  test(LEFT(i));          /* linken Nachbarn ggf. wecken */
  test(RIGHT(i));         /* rechten Nachbarn ggf. wecken */
}

void philosopher(int n){
  while(1) {
    /* denken */
    get_forks(n);
    /* essen */
    put_forks(n); 
  }
}
\end{lstlisting}
Achtung: Programme mit vielen Semaphoren sind unübersichtlich!

\subsubsection*{Einfache pragmatische Lösung}
\begin{lstlisting}[language=C]
/* alternative Loesung des Philosophenproblems */
#define N 5
#define RIGHT(i) (((i)+1) % N)
#define LEFT(i) (((i)==N) ? 0: (i)+1)

semaphore free = N-1;                    /* nicht-binaerer Semaphor */  
semaphore fork[N] = { 1, 1, 1, 1, 1};    /* offen initialisiert */

void philosopher(int n){
  while(1) {
    /* denken */
    P(free);
    P(fork[RIGHT(n)]);
    P(fork[LEFT(n)]);
    /* essen */
    V(fork[LEFT(n)]);
    V(fork[RIGHT(n)]);
    V(free);
  }
}
\end{lstlisting}

\section{Fazit}
\slides{00-synchronisation}{38}