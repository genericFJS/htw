\newcommand{\customDir}{../}
\RequirePackage{ifthen,xifthen}

% Input inkl. Umlaute, Silbentrennung
\RequirePackage[T1]{fontenc}
\RequirePackage[utf8]{inputenc}

% Arbeitsordner (in Abhängigkeit vom Master) Standard: .LateX_master Ordner liegt im Eltern-Ordner
\providecommand{\customDir}{../}
\newcommand{\setCustomDir}[1]{\renewcommand{\customDir}{#1}}
%%% alle Optionen:
% Doppelseitig (mit Rand an der Innenseite)
\newboolean{twosided}
\setboolean{twosided}{false}
% Eigene Dokument-Klasse (alle KOMA möglich; cheatsheet für Spicker [3 Spalten pro Seite, alles kleiner])
\newcommand{\customDocumentClass}{scrreprt}
\newcommand{\setCustomDocumentClass}[1]{\renewcommand{\customDocumentClass}{#1}}
% Unterscheidung verschiedener Designs: htw, fjs
\newcommand{\customDesign}{htw}
\newcommand{\setCustomDesign}[1]{\renewcommand{\customDesign}{#1}}
% Dokumenten Metadaten
\newcommand{\customTitle}{}
\newcommand{\setCustomTitle}[1]{\renewcommand{\customTitle}{#1}}
\newcommand{\customSubtitle}{}
\newcommand{\setCustomSubtitle}[1]{\renewcommand{\customSubtitle}{#1}}
\newcommand{\customAuthor}{}
\newcommand{\setCustomAuthor}[1]{\renewcommand{\customAuthor}{#1}}
%	Notiz auf der Titelseite (A: vor Autor, B: nach Autor)
\newcommand{\customNoteA}{}
\newcommand{\setCustomNoteA}[1]{\renewcommand{\customNoteA}{#1}}
\newcommand{\customNoteB}{}
\newcommand{\setCustomNoteB}[1]{\renewcommand{\customNoteB}{#1}}
% Format der Signatur in Fußzeile:
\newcommand{\customSignature}{\ifthenelse{\equal{\customAuthor}{}} {} {\footnotesize{\textcolor{darkgray}{Mitschrift von\\ \customAuthor}}}}
\newcommand{\setCustomSignature}[1]{\renewcommand{\customSignature}{#1}}
% Format des Autors auf dem Titelblatt:
\newcommand{\customTitleAuthor}[1]{\textcolor{darkgray}{Mitschrift von #1}}
\newcommand{\setCustomTitleAuthor}[1]{\renewcommand{\customTitleAuthor}{#1}}
% Standard Sprache
\newcommand{\customDefaultLanguage}[1]{}
\newcommand{\setCustomDefaultLanguage}[1]{\renewcommand{\customDefaultLanguage}{#1}}
% Folien-Pfad (inkl. Dateiname ohne Endung und ggf. ohne Nummerierung)
\newcommand{\customSlidePath}{}
\newcommand{\setCustomSlidePath}[1]{\renewcommand{\customSlidePath}{#1}}
% Folien Eigenschaften
\newcommand{\customSlideScale}{0.5}
\newcommand{\setCustomSlideScale}[1]{\renewcommand{\customSlideScale}{#1}}
\newcommand{\customSlideHeight}{9.63cm}
\newcommand{\setCustomSlideHeight}[1]{\renewcommand{\customSlideHeight}{#1}}
\newcommand{\customSlideWidth}{12.8cm}
\newcommand{\setCustomSlideWidth}[1]{\renewcommand{\customSlideWidth}{#1}}

%\setboolean{twosided}{true}
\setCustomDocumentClass{cheatsheet}
%\setCustomDesign{htw}
%\setCustomSlidePath{Folien}

\setCustomTitle{Programmierung I}
\setCustomSubtitle{Spicker}
\setCustomAuthor{Falk-Jonatan Strube}
%\setCustomNoteA{TitlepageNoteBeforeAuthor}
\setCustomNoteB{Vorlesung von Prof. Dr.-Ing. Beck}

%\setcustomSignature{\footnotesize{\textcolor{darkgray}{Mitschrift von\\ \customAuthor}}	% Formatierung der Signatur in der Fußzeile
%\setcustomTitleAuthor{\textcolor{darkgray}{Mitschrift von #1}}	% Formatierung des Autors auf dem Titelblatt

%-- Prüfen, ob Beamer
\ifthenelse{\equal{\customDocumentClass}{beamer}}{
%%% TODO: andere Layouts für Beamer außer HTW
	\documentclass[ignorenonframetext, 11pt, table]{beamer}
	
	\usenavigationsymbolstemplate{}
	\setbeamercolor{author in head/foot}{fg=black}
	\setbeamercolor{title}{fg=black}
	\setbeamercolor{bibliography entry author}{fg=htworange!70}
	%\setbeamercolor{bibliography entry title}{fg=blue} 
	\setbeamercolor{bibliography entry location}{fg=htworange!60} 
	\setbeamercolor{bibliography entry note}{fg=htworange!60}  
	
	\setbeamertemplate{itemize item}{\color{black}$\bullet$}
	\setbeamertemplate{itemize subitem}{\color{black}--}
	\setbeamertemplate{itemize subsubitem}{\color{black}$\bullet$}
	\makeatother
	\setbeamertemplate{footline}
	{
	\leavevmode
	\def\arraystretch{1.2}
	\arrayrulecolor{gray}
	\begin{tabular}{ p{0.167\textwidth} | p{0.491\textwidth} | p{0.089\textwidth} | p{0.103\textwidth}}
	\hline
	\strut\insertshortauthor & \insertshorttitle & Slide \insertframenumber{}% / \inserttotalframenumber{}
	 & May 4, 2016\\
	\end{tabular}
	}
	\setbeamertemplate{headline}
	{
	\leavevmode
	\setlength{\arrayrulewidth}{1pt}
	\hspace*{2em}	
	\begin{tabular}{p{0.63\textwidth}}
	\rule{0pt}{3em}\normalsize{\textbf{\insertsection\strut}}\\
	\arrayrulecolor{htworange}
	\hline
	\end{tabular}
	\begin{tabular}{l}
	\rule{0pt}{4em}\includegraphics[width=3.25cm]{\customDir .LaTeX_master/HTW_GESAMTLOGO_CMYK.eps}\\
	\end{tabular}
	}
	\makeatletter	
}{	
	%-- Für Spicker einiges anders:
	\ifthenelse{\equal{\customDocumentClass}{cheatsheet}}{
		\documentclass[a4paper,10pt,landscape]{scrartcl}
		\usepackage{geometry}
		\geometry{top=2mm, bottom=2mm, headsep=0mm, footskip=0mm, left=2mm, right=2mm}
		
		% Für Spicker \spsection für Section, zur Strukturierung \HRule oder \HDRule Linie einsetzen
		\usepackage{multicol}
		\newcommand{\spsection}[1]{\textbf{#1}}	% Platzsparende "section" für Spicker
	}{	%-- Ende Spicker-Unterscheidung-if
		%-- Unterscheidung Doppelseitig
		\ifthenelse{\boolean{twosided}}{
			\documentclass[a4paper,11pt, footheight=26pt,twoside]{\customDocumentClass}
			\usepackage[head=23pt]{geometry}	% head=23pt umgeht Fehlerwarnung, dafür größeres "top" in geometry
			\geometry{top=30mm, bottom=22mm, headsep=10mm, footskip=12mm, inner=27mm, outer=13mm}
		}{
			\documentclass[a4paper,11pt, footheight=26pt]{\customDocumentClass}
			\usepackage[head=23pt]{geometry}	% head=23pt umgeht Fehlerwarnung, dafür größeres "top" in geometry
			\geometry{top=30mm, bottom=22mm, headsep=10mm, footskip=12mm, left=20mm, right=20mm}
		}
		%-- Nummerierung bis Subsubsection für Report
		\ifthenelse{\equal{\customDocumentClass}{report} \OR \equal{\customDocumentClass}{scrreprt}}{
			\setcounter{secnumdepth}{3}	% zählt auch subsubsection
			\setcounter{tocdepth}{3}	% Inhaltsverzeichnis bis in subsubsection
		}{}
	}%-- Ende Spicker-Unterscheidung-else
	
	\usepackage{scrlayer-scrpage}	% Kopf-/Fußzeile
	\renewcommand*{\thefootnote}{\fnsymbol{footnote}}	% Fußnoten-Symbole anstatt Zahlen
	\renewcommand*{\titlepagestyle}{empty} % Keine Seitennummer auf Titelseite
	\usepackage[perpage]{footmisc}	% Fußnotenzählung Seitenweit, nicht Dokumentenweit
}

% Input inkl. Umlaute, Silbentrennung
\RequirePackage[T1]{fontenc}
\RequirePackage[utf8]{inputenc}
\usepackage[english,ngerman]{babel}
\usepackage{csquotes}	% Anführungszeichen
\RequirePackage{marvosym}
\usepackage{eurosym}

% Style-Aufhübschung
\usepackage{soul, color}	% Kapitälchen, Unterstrichen, Durchgestrichen usw. im Text
%\usepackage{titleref}
\usepackage[breakwords, fit]{truncate}	% Abschneiden von Sätzen
\renewcommand{\TruncateMarker}{\,…}

% Mathe usw.
\usepackage{amssymb}
\usepackage{amsthm}
\ifthenelse{\equal{\customDocumentClass}{beamer}}{}{
\usepackage[fleqn,intlimits]{amsmath}	% fleqn: align-Umgebung rechtsbündig; intlimits: Integralgrenzen immer ober-/unterhalb
}
%\usepackage{mathtools} % u.a. schönere underbraces
\usepackage{xcolor}
\usepackage{esint}	% Schönere Integrale, \oiint vorhanden
\everymath=\expandafter{\the\everymath\displaystyle}	% Mathe Inhalte werden weniger verkleinert
\usepackage{wasysym}	% mehr Symbole, bspw \lightning
%\renewcommand{\int}{\int\limits}
%\usepackage{xfrac}	% mehr fracs: sfrac{}{}
\let\oldemptyset\emptyset	% schöneres emptyset
\let\emptyset\varnothing
%\RequirePackage{mathabx}	% mehr Symbole
\mathchardef\mhyphen="2D	% Hyphen in Math

% tikz usw.
\usepackage{tikz}
\usepackage{pgfplots}
\pgfplotsset{compat=1.11}	% Umgeht Fehlermeldung
\usetikzlibrary{graphs}
%\usetikzlibrary{through}	% ???
\usetikzlibrary{arrows}
\usetikzlibrary{arrows.meta}	% Pfeile verändern / vergrößern: \draw[-{>[scale=1.5]}] (-3,5) -> (-3,3);
\usetikzlibrary{automata,positioning} % Zeilenumbruch im Node node[align=center] {Text\\nächste Zeile} automata für Graphen
\usetikzlibrary{matrix}
\usetikzlibrary{patterns}	% Schraffierte Füllung
\usetikzlibrary{shapes.geometric}	% Polygon usw.
\tikzstyle{reverseclip}=[insert path={	% Inverser Clip \clip
	(current page.north east) --
	(current page.south east) --
	(current page.south west) --
	(current page.north west) --
	(current page.north east)}
% Nutzen: 
%\begin{tikzpicture}[remember picture]
%\begin{scope}
%\begin{pgfinterruptboundingbox}
%\draw [clip] DIE FLÄCHE, IN DER OBJEKT NICHT ERSCHEINEN SOLL [reverseclip];
%\end{pgfinterruptboundingbox}
%\draw DAS OBJEKT;
%\end{scope}
%\end{tikzpicture}
]	% Achtung: dafür muss doppelt kompliert werden!
\usepackage{graphpap}	% Grid für Graphen
\tikzset{every state/.style={inner sep=2pt, minimum size=2em}}
\usetikzlibrary{mindmap, backgrounds}
%\usepackage{tikz-uml}	% braucht Dateien: http://perso.ensta-paristech.fr/~kielbasi/tikzuml/

% Tabular
\usepackage{longtable}	% Große Tabellen über mehrere Seiten
\usepackage{multirow}	% Multirow/-column: \multirow{2[Anzahl der Zeilen]}{*[Format]}{Test[Inhalt]} oder \multicolumn{7[Anzahl der Reihen]}{|c|[Format]}{Test2[Inhalt]}
\renewcommand{\arraystretch}{1.3} % Tabellenlinien nicht zu dicht
\usepackage{colortbl}
\arrayrulecolor{gray}	% heller Tabellenlinien
\usepackage{array}	% für folgende 3 Zeilen (für Spalten fester breite mit entsprechender Ausrichtung):
\newcolumntype{L}[1]{>{\raggedright\let\newline\\\arraybackslash\hspace{0pt}}m{\dimexpr#1\columnwidth-2\tabcolsep-1.5\arrayrulewidth}}
\newcolumntype{C}[1]{>{\centering\let\newline\\\arraybackslash\hspace{0pt}}m{\dimexpr#1\columnwidth-2\tabcolsep-1.5\arrayrulewidth}}
\newcolumntype{R}[1]{>{\raggedleft\let\newline\\\arraybackslash\hspace{0pt}}m{\dimexpr#1\columnwidth-2\tabcolsep-1.5\arrayrulewidth}}
\usepackage{caption}	% Um auch unbeschriftete Captions mit \caption* zu machen

% Nützliches
\usepackage{verbatim}	% u.a. zum auskommentieren via \begin{comment} \end{comment}
\usepackage{tabto}	% Tabs: /tab zum nächsten Tab oder /tabto{.5 \CurrentLineWidth} zur Stelle in der Linie
\NumTabs{6}	% Anzahl von Tabs pro Zeile zum springen
\usepackage{listings} % Source-Code mit Tabs
\usepackage{lstautogobble} 
\ifthenelse{\equal{\customDocumentClass}{beamer}}{}{
\usepackage{enumitem}	% Anpassung der enumerates
%\setlist[enumerate,1]{label=(\arabic*)}	% global andere Enum-Items
\renewcommand{\labelitemiii}{$\scriptscriptstyle ^\blacklozenge$} % global andere 3. Item-Aufzählungszeichen
}
\usepackage{letltxmacro} % neue Definiton von Grundbefehlen
% Nutzen:
%\LetLtxMacro{\oldemph}{\emph}
%\renewcommand{\emph}[1]{\oldemph{#1}}
\RequirePackage{xpatch}	% ua. Konkatenieren von Strings/Variablen (etoolbox)
\usepackage{xstring}	% String Operationen
\usepackage{minibox}	% Minibox anstatt \fbox{} für Boxen mit Zeilenumbruch


% Einrichtung von lst
\lstset{
basicstyle=\ttfamily, 
%mathescape=true, 
%escapeinside=^^, 
autogobble, 
tabsize=2,
basicstyle=\footnotesize\sffamily\color{black},
frame=single,
rulecolor=\color{lightgray},
numbers=left,
numbersep=5pt,
numberstyle=\tiny\color{gray},
commentstyle=\color{gray},
keywordstyle=\color{green},
stringstyle=\color{orange},
morecomment=[l][\color{magenta}]{\#}
showspaces=false,
showstringspaces=false,
breaklines=true,
literate=%
    {Ö}{{\"O}}1
    {Ä}{{\"A}}1
    {Ü}{{\"U}}1
    {ß}{{\ss}}1
    {ü}{{\"u}}1
    {ä}{{\"a}}1
    {ö}{{\"o}}1
    {~}{{\textasciitilde}}1
}
\usepackage{scrhack} % Fehler umgehen
\def\ContinueLineNumber{\lstset{firstnumber=last}} % vor lstlisting. Zum wechsel zum nicht-kontinuierlichen muss wieder \StartLineAt1 eingegeben werden
\def\StartLineAt#1{\lstset{firstnumber=#1}} % vor lstlisting \StartLineAt30 eingeben, um bei Zeile 30 zu starten
\let\numberLineAt\StartLineAt

% BibTeX
\usepackage[bibencoding=ascii,
%backend=bibtex8,
%style=authortitle, citestyle=authortitle-ibid,
%doi=false,
%isbn=false,
%url=false
]{biblatex}	% BibTeX
\usepackage{makeidx}
%\makeglossary
%\makeindex

% Grafiken
\usepackage{graphicx}
\usepackage{epstopdf}	% eps-Vektorgrafiken einfügen
\usepackage{transparent}	% transparent nutzen: {\transparent{0.4} ...}
%\epstopdfsetup{outdir=\customDir}
% Prüft, ob Grafik existiert (mit \ifvalidimage{}{}) [Quelle: https://tex.stackexchange.com/a/99176]:
\makeatletter
\newif\ifgraphicexist
\catcode`\*=11
\newcommand\ifvalidimage[1]{%
    \begingroup
    \global\graphicexisttrue
    \let\input@path\Ginput@path
    \filename@parse{#1}%
    \ifx\filename@ext\relax
    \@for\Gin@temp:=\Gin@extensions\do{%
        \ifx\Gin@ext\relax
        \Gin@getbase\Gin@temp
        \fi}%
    \else
    \Gin@getbase{\Gin@sepdefault\filename@ext}%
    \ifx\Gin@ext\relax
    \global\graphicexistfalse
    \def\Gin@base{\filename@area\filename@base}%
    \edef\Gin@ext{\Gin@sepdefault\filename@ext}%
    \fi
    \fi
    \ifx\Gin@ext\relax
    \global\graphicexistfalse
    \else 
    \@ifundefined{Gin@rule@\Gin@ext}%
    {\global\graphicexistfalse}%
    {}%
    \fi  
    \ifx\Gin@ext\relax 
    \gdef\imageextension{unknown}%
    \else
    \xdef\imageextension{\Gin@ext}%
    \fi 
    \endgroup 
    \ifgraphicexist
    \expandafter \@firstoftwo
    \else
    \expandafter \@secondoftwo
    \fi 
} 
\catcode`\*=12
\makeatother
\usepackage{letltxmacro}	% Latex-Befehle unter anderem Namen neu definieren
\LetLtxMacro{\forceincludegraphics}{\includegraphics}	% neuer Befehl für includegraphics
\renewcommand{\includegraphics}[2][]{	% altes includegraphics neu definieren, damit es auch nicht vorhandene einfügt
\ifvalidimage{#2}{
\forceincludegraphics[#1]{#2}
}{
\message{Achtung: Grafik wurde nicht gefunden: '#2'}
\minibox[frame]{
\textbf{\StrSubstitute{#2}{_}{\_}}  \ifthenelse{\isempty{#1}}{}{\\\textit{#1}}}
}}

% pdf-Setup
\usepackage{pdfpages}
\ifthenelse{\equal{\customDocumentClass}{beamer}}{}{
\usepackage[bookmarks,%
bookmarksopen=false,% Klappt die Bookmarks in Acrobat aus
colorlinks=true,%
linkcolor=black,%
citecolor=red,%
urlcolor=green,%
]{hyperref}
}

%-- Unterscheidung des Stils
\newcommand{\customLogo}{}
\newcommand{\customPreamble}{}
\ifthenelse{\equal{\customDesign}{htw}}{
	% HTW Corporate Design: Arial (Helvetica)
	\usepackage{helvet}
	\renewcommand{\familydefault}{\sfdefault}
	\renewcommand{\customLogo}{HTW-Logo}
	\renewcommand{\customPreamble}{HTW Dresden}
}{
% \renewcommand{\customLogo}{HTW-Logo.eps}
}

% Nach Dokumentenbeginn ausführen:
\AtBeginDocument{
	% Autor und Titel für pdf-Eigenschaften festlegen, falls noch nicht geschehen
	\providecommand{\pdfAuthor}{John Doe}
	\ifdefempty{\customAuthor} {} {\renewcommand{\pdfAuthor}{\customAuthor}}
	\providecommand{\pdfTitle}{}
	\providecommand{\pdfTitleA}{}
	\providecommand{\pdfTitleB}{}
	\providecommand{\pdfTitleC}{}	
	\ifdefempty{\pdfTitle}{
		\ifdefempty{\customPreamble} {} {\renewcommand{\pdfTitleA}{\customPreamble{} | }}
		\ifdefempty{\customTitle} {\renewcommand{\pdfTitleB}{No Title}} {\renewcommand{\pdfTitleB}{\customTitle}}
		\ifdefempty{\customSubtitle} {} {\renewcommand{\pdfTitleC}{ - \customSubtitle}}
	}{}
	
	\newcommand{\customLogoLocation}{\customDir .LaTeX_master/\customLogo}
	\hypersetup{
		pdfauthor={\pdfAuthor},
		pdftitle={\pdfTitleA\pdfTitleB\pdfTitleC},
	}
	\ifthenelse{\equal{\customDocumentClass}{beamer}}{
		\title{\customTitle}
		\author{\customAuthor}
	}{
		\automark[section]{section}
		\automark*[subsection]{subsection}
		\pagestyle{scrheadings}
		\ifthenelse{\equal{\customDocumentClass}{report} \OR \equal{\customDocumentClass}{scrreprt}}{
		\renewcommand*{\chapterpagestyle}{scrheadings}
		}{}
		%\renewcommand*{\titlepagestyle}{scrheadings}
		\ihead{\includegraphics[height=1.7em]{\customLogoLocation}}
		%\ohead{\truncate{4cm}{\customTitle}}
		\chead{\truncate{.5\textwidth}{\headmark}}
		\ohead{\customTitle}
		\cfoot{\pagemark}
		\ofoot{\customSignature}
		% Titelseite
		\title{
		\includegraphics[width=0.35\textwidth]{\customDir .LaTeX_master/\customLogo}\\\vspace{0.5em}
		\Huge\textbf{\customTitle}
		\ifdefempty{\customSubtitle} {} {\\\vspace*{0.7em}\Large \customSubtitle}
		\\\vspace*{5em}}
		\author{
		\ifdefempty{\customNoteA} {} {\customNoteA \vspace*{1em}}\\ 
		\ifdefempty{\customAuthor} {} {\customTitleAuthor}
		\ifdefempty{\customNoteB}{}{\vspace*{1em}\\\customNoteB}
		}
		
		\ifthenelse{\equal{\customDocumentClass}{cheatsheet}}{
			\pagestyle{empty}
			\setlist{nolistsep}
	%		\usepackage{parskip}	% Aufzählung Abstand
	%		\setlength{\parskip}{0em}
			\lstset{
	    belowcaptionskip=0pt,
	    belowskip=0pt,
	    aboveskip=0pt,
			tabsize=2,
			frame=none,
			numbers=none,
			showspaces=false,
			showstringspaces=false,
			breaklines=true,
			}
		}{}
	}
}

% Unterabschnitte
%\newtheorem{example}{Beispiel}%[section]
%\newtheorem{definition}{Definition}[section]
%\newtheorem{discussion}{Diskussion}[section]
%\newtheorem{remark}{Bemerkung}[section]
%\newtheorem{proof}{Beweis}[section]
%\newtheorem{notation}{Schreibweise}[section]
% LaTeX master Datei(en) zusammengestellt von Falk-Jonatan Strube zur Nutzung an der Hochschule für Technik und Wirtschaft Dresden: https://github.com/genericFJS/htw
\RequirePackage{xcolor}
\RequirePackage{amsmath}
\RequirePackage{letltxmacro}

% Horizontale Linie:
\newcommand{\HRule}[1][\medskipamount]{\par
  \vspace*{\dimexpr-\parskip-\baselineskip+#1}
  \noindent\rule[0.2ex]{\linewidth}{0.2mm}\par
  \vspace*{\dimexpr-\parskip-.5\baselineskip+#1}}
% Gestrichelte horizontale Linie:
\RequirePackage{dashrule}
\newcommand{\HDRule}[1][\medskipamount]{\par
  \vspace*{\dimexpr-\parskip-\baselineskip+#1}
  \noindent\hdashrule[0.2ex]{\linewidth}{0.2mm}{1mm} \par
  \vspace*{\dimexpr-\parskip-.5\baselineskip+#1}}
% Mathe in Anführungszeichen:
\newsavebox{\mathbox}\newsavebox{\mathquote}
\makeatletter
\newcommand{\mq}[1]{% \mathquotes{<stuff>}
  \savebox{\mathquote}{\text{"}}% Save quotes
  \savebox{\mathbox}{$\displaystyle #1$}% Save <stuff>
  \raisebox{\dimexpr\ht\mathbox-\ht\mathquote\relax}{"}#1\raisebox{\dimexpr\ht\mathbox-\ht\mathquote\relax}{''}
}
\makeatother

% Paragraph mit Zähler (Section-Weise)
\newcounter{cparagraphC}
\newcommand{\cparagraph}[1]{
\stepcounter{cparagraphC}
\paragraph{\thesection{}-\thecparagraphC{} #1}
%\addcontentsline{toc}{subsubsection}{\thesection{}-\thecparagraphC{} #1}
\label{\thesection-\thecparagraphC}
}
\makeatletter
\@addtoreset{cparagraphC}{section}
\makeatother


% (Vorlesungs-)Folien einbinden:
% Folien von einer Datei skaliert
\newcommand{\slide}[2][\customSlideScale]{\slides[#1]{}{#2}}
\newcommand{\slideTrim}[6][\customSlideScale]{\slides[#1 , clip,  trim = #5cm #4cm #6cm #3cm]{}{#2}}
% Folien von mehreren nummerierten Dateien skaliert
\newcommand{\slides}[3][\customSlideScale]{\begin{center}
\includegraphics[page=#3, scale=#1]{\customSlidePath #2.pdf}
\end{center}}

% \emph{} anders definieren
\makeatletter
\DeclareRobustCommand{\em}{%
  \@nomath\em \if b\expandafter\@car\f@series\@nil
  \normalfont \else \scshape \fi}
\makeatother

% unwichtiges
\newcommand{\unimptnt}[1]{{\transparent{0.5}#1}}

% alph. enumerate
\newenvironment{anumerate}{\begin{enumerate}[label=(\alph*)]}{\end{enumerate}} % Alphabetische Aufzählung

% Hanging parameters
\newcommand{\hangpara}[1]{\par\noindent\hangindent+2em\hangafter=1 #1\par\noindent}

%% EINFACHE BEFEHLE

% Abkürzungen Mathe
\newcommand{\EE}{\mathbb{E}}
\newcommand{\QQ}{\mathbb{Q}}
\newcommand{\RR}{\mathbb{R}}
\newcommand{\CC}{\mathbb{C}}
\newcommand{\NN}{\mathbb{N}}
\newcommand{\ZZ}{\mathbb{Z}}
\newcommand{\PP}{\mathbb{P}}
\renewcommand{\SS}{\mathbb{S}}
\newcommand{\cA}{\mathcal{A}}
\newcommand{\cB}{\mathcal{B}}
\newcommand{\cC}{\mathcal{C}}
\newcommand{\cD}{\mathcal{D}}
\newcommand{\cE}{\mathcal{E}}
\newcommand{\cF}{\mathcal{F}}
\newcommand{\cG}{\mathcal{G}}
\newcommand{\cH}{\mathcal{H}}
\newcommand{\cI}{\mathcal{I}}
\newcommand{\cJ}{\mathcal{J}}
\newcommand{\cM}{\mathcal{M}}
\newcommand{\cN}{\mathcal{N}}
\newcommand{\cP}{\mathcal{P}}
\newcommand{\cR}{\mathcal{R}}
\newcommand{\cS}{\mathcal{S}}
\newcommand{\cZ}{\mathcal{Z}}
\newcommand{\cL}{\mathcal{L}}
\newcommand{\cT}{\mathcal{T}}
\newcommand{\cU}{\mathcal{U}}
\newcommand{\cX}{\mathcal{X}}
\newcommand{\cV}{\mathcal{V}}
\renewcommand{\phi}{\varphi}
\renewcommand{\epsilon}{\varepsilon}
\renewcommand{\theta}{\vartheta}

% Verschiedene als Mathe-Operatoren
\DeclareMathOperator{\arccot}{arccot}
\DeclareMathOperator{\arccosh}{arccosh}
\DeclareMathOperator{\arcsinh}{arcsinh}
\DeclareMathOperator{\arctanh}{arctanh}
\DeclareMathOperator{\arccoth}{arccoth} 
\DeclareMathOperator{\var}{Var} % Varianz 
\DeclareMathOperator{\cov}{Cov} % Co-Varianz 

% Farbdefinitionen
\definecolor{red}{RGB}{180,0,0}
\definecolor{green}{RGB}{75,160,0}
\definecolor{blue}{RGB}{0,75,200}
\definecolor{orange}{RGB}{255,128,0}
\definecolor{yellow}{RGB}{255,245,0}
\definecolor{purple}{RGB}{75,0,160}
\definecolor{cyan}{RGB}{0,160,160}
\definecolor{brown}{RGB}{120,60,10}

\definecolor{itteny}{RGB}{244,229,0}
\definecolor{ittenyo}{RGB}{253,198,11}
\definecolor{itteno}{RGB}{241,142,28}
\definecolor{ittenor}{RGB}{234,98,31}
\definecolor{ittenr}{RGB}{227,35,34}
\definecolor{ittenrp}{RGB}{196,3,125}
\definecolor{ittenp}{RGB}{109,57,139}
\definecolor{ittenpb}{RGB}{68,78,153}
\definecolor{ittenb}{RGB}{42,113,176}
\definecolor{ittenbg}{RGB}{6,150,187}
\definecolor{itteng}{RGB}{0,142,91}
\definecolor{ittengy}{RGB}{140,187,38}

\definecolor{htworange}{RGB}{249,155,28}

% Textfarbe ändern
\newcommand{\tred}[1]{\textcolor{red}{#1}}
\newcommand{\tgreen}[1]{\textcolor{green}{#1}}
\newcommand{\tblue}[1]{\textcolor{blue}{#1}}
\newcommand{\torange}[1]{\textcolor{orange}{#1}}
\newcommand{\tyellow}[1]{\textcolor{yellow}{#1}}
\newcommand{\tpurple}[1]{\textcolor{purple}{#1}}
\newcommand{\tcyan}[1]{\textcolor{cyan}{#1}}
\newcommand{\tbrown}[1]{\textcolor{brown}{#1}}

% Umstellen der Tabellen Definition
\newcommand{\mpb}[1][.3]{\begin{minipage}{#1\textwidth}\vspace*{3pt}}
\newcommand{\mpe}{\vspace*{3pt}\end{minipage}}

\newcommand{\resultul}[1]{\underline{\underline{#1}}}
\newcommand{\parskp}{$ $\\}	% new line after paragraph
\newcommand{\corr}{\;\widehat{=}\;}
\newcommand{\mdeg}{^{\circ}}

\newcommand{\nok}[2]{\binom{#1}{#2}}	% n über k BESSER: \binom{n}{k}
\newcommand{\mtr}[1]{\begin{pmatrix}#1\end{pmatrix}}	% Matrix
\newcommand{\dtr}[1]{\begin{vmatrix}#1\end{vmatrix}}	% Determinante (Betragsmatrix)
\LetLtxMacro{\originalVec}{\vec}
\renewcommand{\vec}[1]{\underline{#1}}	% Vektorschreibweise
\newcommand{\imptnt}[1]{\colorbox{red!30}{#1}}	% Wichtiges
\newcommand{\intd}[1]{\,\mathrm{d}#1}
\newcommand{\diffd}[1]{\mathrm{d}#1}
% für Module-Rechnung: \pmod{}
\newcommand{\unit}[1]{\,\mathrm{#1}}
\LetLtxMacro{\ntilde}{\tilde}
\renewcommand{\tilde}{\widetilde}
\newcommand{\gdw}{genau dann wenn}
\newcommand{\lecdate}[1]{\begin{flushright}\textcolor{gray}{Vorlesung am #1}\end{flushright}}

%\bibliography{\customDir _Literatur/HTW_Literatur.bib}

\begin{document}
\begin{multicols*}{3}

\spsection{gcc}
Ablauf für eine „hello.c“ Datei.
\begin{enumerate}
\item Pre-Prozessor (Zeilen im Quelltext mit \emph{\#} werden hier interpretiert): hello.c $\rightarrow$ hello.e \\
\emph{gcc -E hello.c > hello.e}
\item Compiler: hello.e $\rightarrow$ hello.o\\
\emph{gcc -c hello.c}
\item Linker (Bindet Objekt-Datei (xxx.o) mit Librarys zusammen): hello.o $\rightarrow$ a.out / hello\\
\emph{gcc hello.c [-o hello]}
\end{enumerate}

Alle komplieren:\\
gcc *.c (dafür braucht es die Header Datei file.h, die alle Funktionsdeklarationen enthält [außer main]) $\Rightarrow$ alle Dateien werden in eine kompiliert \smallskip\\
Math:\\
\emph{gcc code.c \textbf{-lm}}
\HRule[4pt]
%\spsection{main}
\begin{lstlisting}[language=C]
int main(int argc, char* argv[]){ ... } 
// argc: #Parameter		argv[i]: Parameter (argv[0]: Programmname)
\end{lstlisting}
\HDRule[4pt]
%\spsection{Einlesen von Eingabe}
\begin{lstlisting}[language=C]
char vBuf[128];
fgets(vBuf,128,stdin); 
myInt=atoi(vBuf);	// ganze Zahlen
myFloat=atof(vBuf); // Gleitkomma
\end{lstlisting}
\HDRule[4pt]
%\spsection{Ausgabe}
\begin{lstlisting}[language=C]
printf("%04d",i);	// integer mit führenden Nullen
%d Dezimalwert
%p Adresswert (braucht &i)
%.5f Float mit 5 Nachkommastellen
%-30s String bzw. char-Array der linksbündig max bis zur 30. Stelle im Terminal ausgegeben wird
\end{lstlisting}
\HDRule[4pt]
%\spsection{Gebrochen rechnen ohne Gleitkomma} $ $
\begin{lstlisting}[language=C]
int a=31,b=3;
a = a*10/b;	// a = 310/3 = 103.33333
printf("%d.%d\n",a/10,a%10); // Ausgabe 3.3 (anstatt nur 3)
\end{lstlisting}
\HDRule[4pt]
%\spsection{Summenberechnung} Nutzung des vorhergehenden Summanden.
\begin{lstlisting}[language=C]
	int i=1;
	double x = 5.0, y=1.0, summand = 1.0;
	while (summand>0,00005){	// e^x=1+x/1!+x*x/2!+...
		summand = summand *x/i;
		y += summand;
		i++;
	}
}
\end{lstlisting}
\HDRule[4pt]
%\spsection{Eingabe als Abbruchbedingung} $ $
\begin{lstlisting}[language=C]
	while (operator != toupper('q')){	
	// prüft 1. Zeichen der Eingabe
		fgets(buf, 128, stdin);
		operator = buf[0]; }
		
	while (1){
		fgets(buf, 128, stdin);
		if (!strncmp(buf, "qq", 2)) break; }	
		// prüft 1. zwei Zeichen
\end{lstlisting}
\HDRule[4pt]
\begin{lstlisting}[language=C]
type myArr[] = {...};
for (i=0; i<sizeof(myArr)/sizeof(type); i++){
	myArr[i]=...
}	// durch Array iterieren
\end{lstlisting}
\HRule[4pt]
\spsection{Eingebaute Datentypen}\\
Float: kann einige Zahlen (ganzzahlig) nicht darstellen (bspw. 2), hat Probleme sehr große und sehr kleine Zahlen miteinander zu addieren (durch Normierung der Exponenten kann die kleine zu 0 werden, oder Nachkommastellen verloren gehen)\\
Achtung: boolean kein Datentyp in C: $\curvearrowright$ Abfrage von true/false durch int: 
$int=0\corr false$ \;
$int \not =0 \corr true$

\spsection{2er Komplement}
positive Zahl: $0110 \: 1100$\\
Negation: $1001\: 0011$\; $+1$\\
$\Rightarrow$Komplement: $1001 \: 0100 = -108 = 0x94$

\spsection{Variable}
4 Kennzeichen einer Variable:
\begin{itemize}
\item Adresse im Speicher (Ort)
\item Datentyp (Verarbeitungsbreite)
\item Bitkombination (Wert)
\item Symbolischer Name
\end{itemize}
Ein Vektor fasst mehrere Variablen gleichen Datentyps unter einer zusammen.\smallskip\\
Bei der Initialisierung hat die Variable einen Ausgangswert:
\begin{itemize}
\item Initialisierung innerhalb einer Funktion: zufälliger Wert (alte Speicherbelegung)
\item Init. außerhalb einer Funktion: 0
\end{itemize}
\begin{lstlisting}[language=C]
char c='c';	// 'c'=99 (ASCII)
char c=99;
\end{lstlisting}
\begin{lstlisting}[language=C]
int i=8, j=5, k;
char c=99, d='d';
float x=0.005, y=-0.01, z;
z=i/j;	// i/j wird in int gerechnet, also 8/5=1 und nicht 1.6! -  z ist dann trotzdem float (Wert: 0.000...)
z=k=x;	// k=x=0 (wird abgeschnitten), also z=0.00000...
k=j=5?i:j;	// das selbe wie k=((j==5)?i:j); ist j=5? Wenn ja, dann k=i. Wenn nein, dann k=j.
printf("%d\n", i=!6 );	// !6 entspricht !(ungleich Null)=0
printf("%d\n", i&j );	// i bitweise mit j Verknüpft (ge-UND-et):
//  00001000
// &00000101
// =00000000
\end{lstlisting}
\HRule[4pt]
Simple Sort
\begin{lstlisting}[language=C]
int data[] = {7,3,9,2,5};
int main(){
	int ige, iro;
	for (irt=0; irt<(5-1); irt++={
		for (ige = irt+1, ige<5, ige++){
			if (data[ige] < data[irt]){
				int tmp = data [ige];
				data[ige] = data[irt];
				data[irt] = tmp;
			}		}		}		}
\end{lstlisting}
Alphabetische Sortierung:\\
Jede Zeile bzw. jeden Array-Eintrag mit den folgenden vergleichen und vertauschen, wenn kleiner (jedes char durchgehen, 
\begin{lstlisting}[language=C]
#define N 10 // Länge der Zeichenkette
char data[][N] = {"Max", "Moritz", ...};

void printArr(char arr[][N]){
	int i,j;
	for(i = 0; i <  sizeof(data)/N; i++) {
    for(j = 0; j < N; j++) { printf("%c", arr[i][j]);  }
    printf("\n");
	}	}

main(){	
	int rowA, rowB, x;
	for (rowA = 0; rowA<(N-1); rowA++){
		for (rowB = rowA+1; rowB<N; rowB++){
			x = 0;
			for (x=0; data[rowA][x] == data[rowB][x] && data[rowA][x]!=0; x++){
				;			
			}
			if (data[rowA][x] > data[rowB][x]){
				char tmp;
				for ( ; x<N; x++){
					tmp = data[rowA][x];
					data[rowA][x] = data[rowB][x];
					data[rowB][x] = tmp;
				}		}		}		}
	printArr(data);
}
\end{lstlisting}
\HRule[4pt]
\spsection{Ausdrücke}
Unäre Operatoren (bspw. $-$ (negativ-Zeichen), $++$ (Inkrementierung) oder Klammern(cast))\\
Binäre Operatoren (bspw. $+$, $-$ (Rechenzeichen), $<=$ usw.)
\begin{lstlisting}[language=C]
int i;
long d;
i=(int)d;	// cast: Typwandlung
i++; // Postfixoperator (wird im Rahmen eines groesseren Ausdrucks als letztes ausgefuehrt)
i=1;
j=6;
k=j+i++; // k=7, i=2
++i; // Praefixoperator (wird im Rahmen eines groesseren 	Ausdrucks als erstes ausgefuehrt)
i=1;
j=6;
k=j+ ++i; // k=8, i=2
\end{lstlisting}
Andere Zeichen: $\wedge$ = XOR, $\sim$ = Bit-weise Negation, <\! < = shift (nach links) (Bsp. i=4; i= i <\! < 2; $\Rightarrow$ i wird 16: 00000100 <\! < 2 $\Rightarrow$ 00010000)
\HRule[4pt]
\begin{lstlisting}[language=C]
	while (x<5){ ... }
\end{lstlisting}
\HDRule[4pt]
\begin{lstlisting}[language=C]
	do{	...	} while (x<5);
\end{lstlisting}
Abbrechen der Schleife: break
\begin{lstlisting}[language=C]
	while (1){	...
		if (x<5) break; }
\end{lstlisting}
Abbrechen der aktuelle Iteration (reset der Schleife): continue
\begin{lstlisting}[language=C]
	while (x<5){	...	
		if (x<4) continue;
		printf(...); }	// printf wird nur bei x>=4 ausgeführt
\end{lstlisting}
\HDRule[4pt]
\begin{lstlisting}[language=C]
	for (i=1 ; x < 5 ; i++){ ... }
\end{lstlisting}
\HDRule[4pt]
\begin{lstlisting}[language=C]
switch (i){ // i ist ganzzahliger Ausdruck
	case 1:	// wenn 1
		... break;
	case 2 ... 5: // zwischen 2 und 5
		... break;
	default:
		...
}
\end{lstlisting}
\HRule[4pt]
\spsection{Zeichenketten}
\begin{lstlisting} [language=C]
	fgets(buf, 128,stdin);
	buf[strlen(buf)-1]=0;	// an der Stelle strlen(buf) liegt die terminierende Null,  an strlen(buf)-1 die return-Taste der Eingabe
	puts(buf);	// puts gibt gesamten String aus, printf muss drüber iterieren
	while (buf[i]!=0)
		printf("%c", buf[i++]);
\end{lstlisting}
\HRule[4pt]
\spsection{Funktionen}
Wenn kein return\_type gewählt wurde, dann default: \emph{int}.\\
Wenn kein return\_type gebraucht wird, gibt man \emph{void} an.
\spsection{Speicherklassen:}\\
\emph{auto} (automatische Variable): wird vom Stack erzeugt (Kellerspeicher)\\
lokale Varibalen\\
\emph{extern}: Variable, die in einem anderen Kontext vereinbart ist\\
\emph{static}: leben bis zum Programmende, global-statische Variabeln werden nicht exportiert, immer initialiesiert, default 0\\
\emph{register}: Variablen werden nach Möglichkeit in ein Prozessorregister gelegt (schnell)\\
\emph{volatile}: Variabeln werden immer im Hauptspeicher abgelegt (Gegenteil von register)
\begin{lstlisting}[language=C]
long fakult(int x);	//Funktionsdeklaration (Prototyp)
\end{lstlisting}
\spsection{Header-File}
\begin{lstlisting}[language=C]
#include "fe.h" // wie bspw. stdio.h: eigene Datei in anderen QT
\end{lstlisting}
\HRule[4pt]
\spsection{Pointer}
\begin{lstlisting}[language=C]
	*x = &i // x verweist auf die Adresse von i. Ausgabe von x gibt nur Adresse. *x (oder x[0]) (Dereferenzierung) ergibt Wert von i
\end{lstlisting}
Achtung: Bei Übergaben von Arrays (bspw. in Funktionen) wird nur Pointer auf das erste Element übergeben. Somit ist daraus auch nicht die Länge berechenbar. Des weiteren wird beim modifizieren der Daten im Array das original-Array überschrieben (da es nicht als Kopie, sondern als Verweis übergeben wurde)!

Berechnen Array Länge in Funktion (wenn nicht übergeben):
\begin{lstlisting}[language=C]
int mystrlen1(char *p){
	int i;
	for (i=0; p[i]!=0; i++);
	return i;
}
// oder genau so gültig (mit Pointer gerechnet):
int mystrlen2(char *p){
	int count;
	while (*p++)
		count++;
	return count;
}
\end{lstlisting}

\spsection{Rekursion}

\begin{itemize}
\item Rechtsrekursion (erst etwas rechnen, dann in die Rekursion gehen $\rightarrow$ fakultr)
\item Linksrekursion (erst Rekursiv aufrufen, dann etwas ausführen $\rightarrow$ printu)
\end{itemize}
Eine Links- oder Rechtsrekursion lässt sich auch Iterativ darstellen.
\begin{itemize}
\item Zentralrekursion
\end{itemize}
Eine Zentralrekursion lässt sich nicht Iterativ darstellen.\smallskip\\
Problem Rekursion: Es lässt sich nicht vorhersehen, wie viel Speicher benötigt wird. Da ist die Schleife leichter überschaubar.

\spsection{Benutzerdefinierte Datentypen}

Enum:

Aufzählungstyp (festgesetzte Bezeichnungen auf einen integer-Wert).\medskip\\
Struct:

Zusammenfassung von mehreren Komponenten (unterschiedliche eingebaute Dateitypen als un- intialisierte Variablen), die durch einen Namen beschrieben werden. Verwendung zur Modellierung eines Sachverhalts (wie im Beispiel Student mit seinen Eigenschaften).\medskip\\
Typedef:

Es wird ein synonymer Typname für einen existierenden Typnamen erstellt. So kann die Variableininitialisierung verkürzt werden (im Skript: struct tStudent$\rightarrow$tStud ).\medskip\\
Union:

Datensätze werden im Vergleich zum Struct übereinander geschrieben (Sinnvoll, wenn Unterstrukturen gleiche und auch ungleiche Eigenschaften haben. Beispielsweise Diplom- und Bachelor-Studenten).
\HRule[4pt]
\spsection{Struct}
Struktur: alle Elemente liegen hintereinander (nicht zwangsläufig unmittelbar hintereinander) im Speicher
\begin{lstlisting}[language=C]
typedef int myint; 
struct myStruc{
	char name[30];
	int Nummer;
} 
typedef struct { ... } myStruc2;
sizeof(struct myStruc);
sizeof(myStruc2);
void putStr(struct myStruc s){
	puts(s.name);
}
void putStrP(myStruc *s){
	puts(s->name); // oder: puts( (*s).name );
}
struct myStru stru = {"String", 123};
\end{lstlisting}

\spsection{enum}

\begin{lstlisting}[language=C]
enum tWoTa{
	Montag=1, // bei Montag=1, fängt's bei 1 an zu zählen (anstatt 0)
	Dienstag, 
	Mittwoch, 
	Donnerstag,
	Freitag,
	Samstag=Freitag+1+0x10,	// verändert das Wochenende
	Sonntag};
\end{lstlisting}
\HRule[4pt]
\spsection{memory-allocation}
\begin{lstlisting}[language=C]
	int i;
	tStud s;
	int anz = 0;
	tStud *ps = NULL, *psx;
	while (weiter == 'y'){
		s = getStud();
		if(ps==NULL){	// Wenn noch kein Speicher freigegeben
			ps=malloc(sizeof(tStud));	
			if (ps) {exit(-1);}
		} else {	// Sonst Speicher erweitern
			psx=realloc(ps,(anz+1)*sizeof(tStud));
			if (psx){
				ps=psx;
			} else { exit(-1);}
		}		
		*(ps+anz) = s;	// Adresse vom freigegebenen Speicher
		anz++;
	}
	// am Ende Speicher wieder freigeben. Achtung: psx ist nur Zeiger auf ps
	free (ps);
\end{lstlisting}
Verwendung von malloc/realloc:\\
Speicher nach Bedarf aus dem heap.\\
malloc hat als Ausgabe void* (generischer Pointer). Dieser ist nicht derefernzierbar und zuweisungskompatibel zu jedem getypten Pointer. Man kann mit ihm ebenfalls nicht rechnen (keine Arithmetik).\\
malloc: Speicher für Variable frei machen\\
realloc: freigemachten Speicher erweitern\\
free: Speicher wieder freigeben
\HRule[4pt]
\spsection{Listen}

Ringliste:
\begin{lstlisting}[language=C]
// Strukturtyp für Konnektor (Element mit Inhalt):
typedef struct TCNCT{
	struct TCNCT* next;	// tCnct geht noch nicht innerhalb!
	void *pItem;	// void für generische Daten
}tCnct;
typedef struct{
	tCnct* pFirst;
	tCnct* pLast;
	tCnct* tCurr;
}tList;
\end{lstlisting}
Listenimplementation:
\begin{lstlisting}[language=C]
// erzeugt leere Liste:
tList *CreateList(void){
	tList* ptmp;
	ptmp=malloc(sizeof(tList));
	if(ptmp!=NULL){	// offene Liste: anfängliches tList hat nur NULL-Pointer
		ptmp->pFirst=ptmp->pLast=ptmp->pCurr=NULL;
	}
	return ptmp;
}
// hinten einfügen:
int InsertTail (tList* pList, void *pItemIns){
	// Verschieden Situationen: Anfügen an leere oder schon vorhandene Liste
	tCnct *ptmp = malloc(sizeof(tCnct));
	ptmp->next=NULL;
	if(ptmp){
		ptmp->pItem = pItemIns;	// Connector mit Inhalt füllen
		if (pList->pFirst!=NULL){	// Liste Leer
			pList->pFirst=pList->pLast = ptmp;
		} else {	// Liste enthält schon Konnektoren
			pList->pLast->next=ptmp;	// Das vorher letzte Element zeigt nun auf das eingefügte und damit neue letzte Element
			pList->pLast = ptmp;			// das neue letzte Element
		}
		pList->pCurr=ptmp;	// Das Element, mit dem zuletzt hantiert wurde ist pCurr
	}
	return (int)ptmp;
}
\end{lstlisting}

Unterschied: Offene Liste und Ringliste. Offene Liste startet mit NULL-Zeigern.

Oder: doppelt verkettete Ringliste. Vorteil: Jedes Element hat einen Vorgänger und einen Nachfolger. Dadurch reicht eine Funktion, die nach einem Element ein neues einfügen kann. Das kann an beliebiger Stelle passieren.

\spsection{Bäume}
\begin{lstlisting}[language=C]
typedef struct{
	void *pdata;
	struct TNODE* px[2];
} tNode
tNode treeInit={};
char* data[]={"moritz", "paul",  NULL);
tNode *pTree;
int mycmp(void*p1, void*p2){
	return (strcmp((char*)p1, (char*)p2>0)?0:1;
}
void addToTree(tNode *pt, void* pdata,  int (*fcmp)(void*, void*)){
	int i;
	if(pt->pdata == NULL){
		pt->pdata=pdata;
	} else {
		i=fcmp(pt->pdata, pdata);
			if(pt->px[i] == NULL){
				pt->px[i] = malloc(sizeof(tNode));
				*(pt->px[i]) = treeInit;
			}
			attToTree(pt->px[i],pdata, mycmp);
	}
}
	char *ptmp;
	char **p=data;
	// erstes Node erstellen (leer):
	pTree = malloc(sizeof(tNode));
	*pTree = treeInit;
	while  (*p){
		addToTree(pTree, *p, mycmp); 
		p++;
	}
	while (1){
		fgets(buf,128,stdin);
		buf[strlen(buf)-1]=0;
		ptmp=malloc(strlen(buf)+1);
		strcpy(ptmp,buf);
		addToTree(pTree, ptmp, mycmp);
}
\end{lstlisting}

\HRule[4pt]
\spsection{Dateiarbeit in C}
\begin{lstlisting}[language=C]
FILE * pf;
pf=fopen("myFile.txt","rt"); // ("Dateiname inkl. Pfad", "Modus")
\end{lstlisting}
\begin{tabular}{l | l}
b/t& Texdatei ([t]) / Binärdatei(b)\\
\hline 
r&Lesen öffen\\
w&Scheiben (überschreiben), ggf. erzeugen\\
a&Schreiben am Dateiende, ggf. erzeugen\\
r+&Lesen und Schreiben(ändern)\\
w+&Lesen und Schreiben(überschreiben), ggf. erzeugen\\
a+&Anfügen, Lesen, Erzeugen, ggf. erzeugen\\
\end{tabular}

\spsection{Als Binärdatei speichern}

\begin{lstlisting}[language=C]
...
FILE *pf;
...
	pf = fopen("Studs.bin","rb");
	if(pf){
		// Dateigröße ermitteln:
		fseek(pf,0,SEEK_END);
		anz=ftell(pf)/sizeof(tStud);
		rewind(pf);		
		// Daten lesen
		for (i=0; i<anz; i++){
			if(ps==NULL){
				ps=malloc(sizeof(tStud));
				if (ps==NULL){puts("malloc hat nicht geklappt"); exit(-1);}
				*ps=readStud(pf);
			} else {	
				psx=realloc(ps,(anz+1)*sizeof(tStud));
				if (psx){
					ps=psx;
					*(ps+i)=readStud(pf);
				} else { puts("realloc hat nicht geklappt."); exit(-1);}
			}
		}
		fclose(pf);
	}
...
	// Daten speichern
	pf=fopen("Studs.bin","wb");
	if(pf==NULL){puts("fopen (write) hat nicht geklappt");exit(-1);
	for (i=0; i<anz; i++){
		writeStud(pf,ps+i);
	}
	fclose(pf);
...
\end{lstlisting}
\begin{lstlisting}[language=C]
tStud readStud(FILE* f){
	tStud s;
	fread(&s, sizeof(tStud),1,f);
	return s;
}
void write Stud(File* f, tStud* pStud){
	fwrite(pStud, sizeof(tStud),1,f);
}
\end{lstlisting}
\spsection{Als Textdatei speichern} (alternativ und ergänzend zum obigen Beispiel)
\begin{lstlisting}[language=C]
int getAnz(FILE* pf);
tStud readStud(FILE* pf);
void writeStud(FILE* pf, tStud* ps);
\end{lstlisting}
\begin{lstlisting}[language=C]
// Zeilen zählen und durch 4 teilen
int getAnz(FILE *pf){
	char buf[128];
	int n=0;
	while (fgets(buf, 128, pf)) n++;\\
	fseek(pf, 0, SEEK_SET);
	return n/4;
}
tStud readStud(FILE* pf){
	tStud s={};
	char buf[128];
	if(fgets(buf, 128, pf)){
		buf[strlen(buf)-1]=0;
		s.name = malloc(strlen(buf+1));
		if (s.name) strcpy(s.name,buf);
		else fprintf(stderr, "malloc faild in readStud\n");
		fscanf(pf,"%d\n%d\n%f\n",&s.matrNr,&s.belNote,&s.klNote);
	}
	return s;
}
void writeStud(FILE *pf, tStud* ps){
	fprintf(pf,"%s;%d;%d;%f\n",ps->name,ps->matrNr,ps->belNote,ps->klNote);  
}
\end{lstlisting}
\begin{lstlisting}[language=C]
...
	// Einlesen aus Datei
	pf=fopen("Studs.txt", "rt");
	if (pf){
		anz=getAnz(pf);		
...
	// Schreiben
	pf=fopen("Studs.txt", "wt");
\end{lstlisting}

\spsection{Als CSV-Datei speichern}
Ähnlich wie bei .txt, bloß trennt man die Datensätze durch Semicolon und nicht durch neue Zeile.\\
Sinnvolle Funktion zur Zerlegung der Datensätze: 
\begin{lstlisting}[language=C]
strtok(buf,";\n");
\end{lstlisting}

\HRule[4pt]
\spsection{Funktionspointer}
\begin{lstlisting}[language=C]
typedef void f(void);
f* pf;	// Das ist der Funktionspointer
\end{lstlisting}
oder:
\begin{lstlisting}[language=C]
typedef void (*tpf)(void);
tpf pf;	// Das ist auch ein Funktionspointer
\end{lstlisting}
Anwendungsbeispiel:
\begin{lstlisting}[language=C]
void fxyz(void){
	puts("xyz");
}
pf = fxyz;
fxyz();
pf(); // ruft beides fxyz() auf!
\end{lstlisting}
Beispiel:
\begin{lstlisting}[language=C]
typedef void (*tpf)(int i);
tpf pFunc;
	pFunc=printDec;
	pFunc(x);
	pFunc=printHex;
	pFunc(x);
\end{lstlisting}
\HRule[4pt]
\spsection{Preprozessor}
Zeilenverlängerung
\begin{lstlisting} [language=C]
int ma\
in(){
	puts{Spass\
	in c");
}
\end{lstlisting}

include

\begin{lstlisting}[language=C]
#include <...>	// für Systemheaderfiles (zu finden unter usr/include)
#include "..."	// für Applikationsheaderfiles (eigene)
\end{lstlisting}
Mehrfaches Einbinden u.ä. kann unterbunden werden durch:
\begin{lstlisting}[language=C]
#ifdef _H_DEBUG_
#define _H_DEBUG_
#include <stdio.h>
#endif
\end{lstlisting}

\spsection{Symboldefinitionen}

\begin{lstlisting}[language = C]
// Allgemein:
#define SYMBOL Tokensequenz
// Beispiel:
#define N 10
...
int inArray[N];	// Preprozessor ersetzt N mit 10
#define LEN 30 + 1
// KLAMMERN SETZTEN! ->
// #define LEN (30 + 1)
...
char name[LEN*3];	
// Achtung: LEN wird vorm Ausrechnen ersetzt. 
// Also: LEN*3 entspricht 30+1*3 und nicht (30+1)*3
\end{lstlisting}

Darauf ist zu achten:
\begin{itemize}
\item Tokensequenz (bei Zahlen) am besten Klammern.
\item kein Semikolon!
\item define-Konstruktion muss auf einer Zeile stehen.
\end{itemize}

\spsection{Parameterbehaftete Macros}

\begin{lstlisting} [language=C]
#define SYMBOL(<parameterlist>)	Ersatztokenfolge
// Wichtig: Runde Klammer muss unmittelbar hinter SYMBOL stehen
#define SYMBOL(x) "str1" #x "str2"
// Bewirkt Verkettung der Strings mit dem Parameter
#define SYMBOL(x,y) x##y
// Ein neues Token ensteht im C-Quelltext aus x und y
\end{lstlisting}
Beispiel:
\begin{lstlisting}[language=C]
#define PYTHAGORAS(a,b) sqrt(a*a + b*b)
// Klammern wieder wichtig:
// #define PYTHAGORAS(a,b) sqrt((a)*(a) + (b)*(b))
#define STR1(x) "Max " #x " Moritz"
\end{lstlisting}
Darauf ist zu achten:
\begin{itemize}
\item Parameter im Macro klammern
\item keine Seiteneffekte programmieren (Inkrementierung, Funktionsaufruf usw.)
\end{itemize}
Vordefinierten Symbole bspw.:
\begin{lstlisting} [language=C]
__FILE__ // Quelltext-Dateiname
__DATE__ // Datum, zu dem das Programm kompiliert wurde
\end{lstlisting}

\spsection{Bedingte Übersetzung}

\begin{lstlisting}[language=C]
#if <const-expr>
#if defined <symbol>
#ifdef <symbol>
#if !defined <symbol>
#ifndef <symbol>
#else 
#elif <const-expr>
#endif
\end{lstlisting}
Bsp.:
\begin{lstlisting} [language=C]
#ifdef DEBUG
	printf("Debuginformation");
#endif
\end{lstlisting}
Gibt Debug-Information nur bei \emph{gcc progr.c -DDEBUG} aus.

\HRule[4pt]
\spsection{Funktionen mit variabler Argumentliste}

vgl.: printf() mit beliebig vielen Argumenten abhängig von \% d usw. im ersten String.
\begin{enumerate}
\item wenigstens ein fester Parameter
\item dann folgt, … \\
(Es können also weitere Parameter folgen. Typ und Anzahl der Parameter ist unbekannt.)
\end{enumerate}

Macros zum Umgang mit variabler Argumentliste:
\begin{lstlisting}[language=C]
#include <stdarg.h>	// Voraussetzung für variable Argumentlisten

va_start(ap, la);	
// ap: Argument Pointer ( va_list ap; )
// la: Last Argument (letzter Parameter mit Typ und Namen vor ...)
printf("Name: %s, MatrNr: %d \n", Stud.name, Stud.MatrNr); 

x=va_arg(ap, type);
// x: Zielvariable für den Wert des Arguments (muss vom Typ des tatsächlichen Parameters sein)
// ap wird um sizeof(type) erhöht
// dieser Wert wird x zugewiesen

va_end(ap);
\end{lstlisting}
Bsp.:
\begin{lstlisting}[language=C]
#include <stdarg.h>
#include <recout.h>	// Andere my...-Funktionen in Arg.tgz
#define xprintf(x) myprintf x

void myprintf(const char* fmt, ...){ // fmt bspw. ("Programm: %s\n", argv[0])
	va_list ap;
	char		*p;
	char		*pstr;
	int			ival;
	
	va_start(ap,fmt);
	
	for (	p=fmt; // p auf Anfang des Formatsteuerstrings
				*p;
				p++){
		if (*p!='%'){
			myputc(*p);
		} else {
			switch(*++p){
				case 's':
					pstr=va_arg(ap,char*);
					myputs(pstr);
					break;
				case 'd':
					ival=va_arg(ap,int);
					myputd(ival);
					break;
				default:
					myputc(*p);
					break;
			}
		}
	}
}

int main(int argc, char* argv[]){
	int i;
	if (argc>1)
		i = myatoi(argv[1]);
	else
		i = -1;
		
	
	myprintf("Programm: %s\n int Value: %d\n", argv[0], argc);
	myprintf("i: %d, %x, DoubleVal: %f, Char: %c, 
						Adresse argv[0]: %p",i,i,d, argv[0][0], argv[0]);
						
	xprintf( ("Test xprintf: %s\n", argv[0]) );
\end{lstlisting}
\end{multicols*}

\end{document}
