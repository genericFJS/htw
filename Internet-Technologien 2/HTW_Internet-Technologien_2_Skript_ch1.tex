\lecdate{17.10.2017}
\section{Motivation}
\slides{it2-10-echtzeitdienste_print}{3}
\subsection{Beispiel IPTV}
\slides{it2-10-echtzeitdienste_print}{4}
\subsubsection*{Probleme IPTV}
\slides{it2-10-echtzeitdienste_print}{5}
\subsection{Echtzeitsysteme}
\slides{it2-10-echtzeitdienste_print}{6}

\section{Verzögerungen in paketorientierten Netzen}
\slides{it2-10-echtzeitdienste_print}{7}
\slides{it2-10-echtzeitdienste_print}{8}
\subsubsection*{Beispiele}
\slides{it2-10-echtzeitdienste_print}{9}
\subsection{Datenrate-Verzögerungsprodukt}
\slides{it2-10-echtzeitdienste_print}{10}
\subsection{Round-Trip-Time (RTT)}
\slides{it2-10-echtzeitdienste_print}{11}
\subsubsection*{RTT-Bestimmung}
\slides{it2-10-echtzeitdienste_print}{13}
\subsubsection{Retransmission-Timer bei TCP}
\slides{it2-10-echtzeitdienste_print}{12}
\subsubsection*{Bestimmung des RTO bei TCP}
\slides{it2-10-echtzeitdienste_print}{14}
\subsubsection{Latenz von TCP: Fast Retransmit}
\slides{it2-10-echtzeitdienste_print}{15}
\subsection{Anforderungen für VoIP}
\slides{it2-10-echtzeitdienste_print}{16}
\subsubsection*{Verzögerungen}
\slides{it2-10-echtzeitdienste_print}{17}
Häufigster Grund für Verzögerung/Paketverlust: Pufferüberlauf (am Router)!
\subsubsection*{Jitter-Puffer}
\slides{it2-10-echtzeitdienste_print}{18}
\subsubsection*{Protokolle im Umfeld VoIP}
\slides{it2-10-echtzeitdienste_print}{19}

\section{Quality of Service}
\slides{it2-10-echtzeitdienste_print}{22}
\subsection{QoS Schicht 3}
\slides{it2-10-echtzeitdienste_print}{20}
\subsection{QoS Schicht 2}
\slides{it2-10-echtzeitdienste_print}{21}

\section{Fehlerschutz}
\lecdate{24.10.2017}
\slides{it2-10-echtzeitdienste_print}{23}
\subsection{ARQ}
\slides{it2-10-echtzeitdienste_print}{24}
\subsection{FEC}
\slides{it2-10-echtzeitdienste_print}{25}
\subsection{3}
\slides{it2-10-echtzeitdienste_print}{26}

\section{Problemfelder}
\subsection{Unicast vs Multicast}
\slides{it2-10-echtzeitdienste_print}{27}
\subsection{Datenkarussell}
\slides{it2-10-echtzeitdienste_print}{28}
\subsection{Datenübertragung in P2P-Netzen}
\slides{it2-10-echtzeitdienste_print}{29}
\subsection{Mobile Empfänger}
\slides{it2-10-echtzeitdienste_print}{30}
\subsection{IP-Multicast über WLAN}
\slides{it2-10-echtzeitdienste_print}{31}
Als Empfänger hat man nur macht über Schicht 7: Dem entsprechend kann der Fehlerschutz nur dort über FEC passieren.
\subsection{Application-Layer Hybrid-Fehlerschutzverfahren}
\slides{it2-10-echtzeitdienste_print}{32}

\section{Zusammenfassung}
\slides{it2-10-echtzeitdienste_print}{34}
%\subsubsection*{Ziele der Vorlesung}
%\slides{it2-10-echtzeitdienste_print}{35}


