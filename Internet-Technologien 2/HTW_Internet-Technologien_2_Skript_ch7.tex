\section{Motivation}
\subsection{Dilemma der Kanalcodierung}
\slides{it2-35-lincodes_print}{3}
\subsection{Motivation für Lineare Codes}
\slides{it2-35-lincodes_print}{4}
\subsection{Vektorraum}
\slides{it2-35-lincodes_print}{5}
GF: Galois-Feld (endlicher Körper)… (endlicher) Körper: man kann jede Operation ausführen (Multiplikation/Addition).
\subsubsection*{Beispiel Vektorraum mit GF(3)}
\slides{it2-35-lincodes_print}{6}

\section{Definition Linearer Code}
\slides{it2-35-lincodes_print}{7}
(2. Beispiel: man kommt nie auf $111$ $\to$ ist also kein linearer Code)\\
Hinweis: auf $000$ kommt man immer, wenn man einen beliebiges Codewort mit sich selbst addiert.
\subsection{Hamming-Gewicht}
\slides{it2-35-lincodes_print}{8}
Der Abstand zwischen zwei Codewörtern ist wieder ein Codewort. Also muss man nur die einsen Zählen…
\subsection{Gewichtsverteilung}
\slides{it2-35-lincodes_print}{9}
\subsection{Hamming-Codes}
\slides{it2-35-lincodes_print}{10}
Rote Codewörter: „Einheitsvektoren“ (haben jeweils nur eine $1$) $\to$ durch Skalierung, … kann jeder anderer „Vektor“/Codewort dargestellt werden.
\subsection{Singleton-Schranke}
\slides{it2-35-lincodes_print}{11}

\section{Hamming-Schranke}
\slides{it2-35-lincodes_print}{12}
\subsubsection*{Beispiel}
\slides{it2-35-lincodes_print}{13}

\section{Generatormatrix}
\slides{it2-35-lincodes_print}{14}
\subsection{Bildung des Codewortes}
\slides{it2-35-lincodes_print}{15}
\subsection{Systematische Codierung}
\slides{it2-35-lincodes_print}{16}
\subsection{Codierschaltung}
\slides{it2-35-lincodes_print}{17}

\section{Dualer Code}
\slides{it2-35-lincodes_print}{18}

\section{Prüfmatrix}
\slides{it2-35-lincodes_print}{19}
\subsection*{Kanal}
\slides{it2-35-lincodes_print}{20}

\section{Nebenklassenzerlegung}
\slides{it2-35-lincodes_print}{21}
Nebenklassenführer: Der Code der Nebenklasse mit den wenigsten einsen (in Fällen 0-5 eindeutig, bei 6 und 7 nicht eindeutig).
\subsubsection*{Beispiel}
\slides{it2-35-lincodes_print}{22}
Nebenklassencodes sind willkürlich gewählt (solange sie den Voraussetzungen entsprechen).
\subsection{Fehlerkorrektur}
\slides{it2-35-lincodes_print}{23}

\section{Syndromdekodierung}
\lecdate{05.12.2017}
\subsection{Fehlererkennung}
\slides{it2-35-lincodes_print}{24}
\subsubsection*{Beispiele}
\slides{it2-35-lincodes_print}{25}
\subsection{Fehlerkorrektur}
\slides{it2-35-lincodes_print}{26}
Empfänger erhält $y=x+e$
\begin{enumerate}
\item $s=y\cdot H^T$ $\to$ $\hat{e}$
\item $\hat{x}=y+\hat{e}$
\end{enumerate}
\subsubsection*{Syndromtabelle}
\slides{it2-35-lincodes_print}{27}

\section{Hamming Codes}
\subsection{Definition}
\slides{it2-35-lincodes_print}{28}
\subsection{Erzeugung}
\slides{it2-35-lincodes_print}{29}
\subsection{Decodierung}
\slides{it2-35-lincodes_print}{30}
\subsubsection*{Eigenschaften Syndromdecodierung}
\slides{it2-35-lincodes_print}{31}
\subsection{Beispiele}
\subsubsection*{Hamming-Code (63,57,3)}
\slides{it2-35-lincodes_print}{32}
\subsubsection*{BCH-Code (511,259,61)}
\slides{it2-35-lincodes_print}{33}

\section{Zusammenfassung}
\slides{it2-35-lincodes_print}{34}


