% Header aus der Vorlage
\input{../LaTeX_master_HTW_en}
\renewcommand{\workingdir}{../../../}

\bibliography{EngC1_Proposal_Literature_url.bib}

% Definition von Titel, Autor usw.
\DTitel{\textsc{English C1 Written Proposal}}
\DUntertitel{Complete Memorandum}
\DAutor{Falk-Jonatan Strube, Dimitrij Becher}
\DNotiz{}
\renewcommand{\Dokumentensignatur}{}

\begin{document}

%\maketitle
%\newpage
%\tableofcontents
%\newpage

\chapter*{\textsc{Memorandum}}
\begin{tabular}{l l}
\emph{To:} & B.Arch. B.Sc. Darlene Kilian\\
\emph{From:} & Dimitrij Becher and Falk-Jonatan Strube\\
\emph{Date:} & April 20, 2016\\
\emph{Subject:} & Object oriented programming and modelling\\
\end{tabular}\medskip\\
The following represents the written proposal for our \emph{English for Special Purposes Seminar}. The subject of our seminar will be "Object oriented programming and modeling" as agreed upon during class. Our proposal outlines the subject and how it will be presented to at the seminar. 

%This is in response to the proposal assignment due on June 21st of this month. As I had mentioned in the topic planner and proposal bulletin board recently, I intend to provide an informational view of contemporary microchip fabrication and the way in which the photolithography section of the wafer fab affects the manufacture of microchips. The following proposal describes the problem that this project addresses, outlines the information I intend to present, and discusses the time and resources required to complete the study.
\section*{Background}
Although having very diverse backgrounds and fields of study, students in the \emph{English C1} class are expected to have a basic understanding of logic and mathematics -- as is needed in this computer science subject. Taking that into consideration and regarding the circumstance of some students not being fluent in any kind of programming language, the seminar presents the concepts of the topic with simplified \emph{pseudocode} and easy to understand figures.
%Students in the Semiconductor Manufacturing Technology (SMT) program at Austin Community College often hear that the photolithography section of the wafer fabrication facility is the most important part of the fab. However, this aspect of the process in manufacturing semiconductors receives little or no in-depth coverage in any of the SMT courses I am aware of. Students graduating from the SMT program may be at some disadvantage when they seek jobs in industry if they have no understanding of the photolithography process.
\section*{Proposal}
The seminar will teach the students the basic concept of object oriented programming by illustrating the elemental designs via simple \emph{pseudocode} and comprehensible figures. There will be a basic comparison between \emph{generic} and \emph{object oriented programming} to highlight the differences. The \emph{Unified Modeling Language (UML)}, a visual modeling language, will be used to present the most important principles of object oriented programming.
%In the report, I'll present how and why the photolithography section of the wafer fab is so important to the manufacturing of semiconductors (microchips.) This section will also cover the basics of manufacturing microchips in a specific manufacturing process flow. I will not be presenting any information that may be trade secrets to particular companies, such as data about the copper chips that IBM currently has in development, or the steps that IBM is taking to build a 1 GHz chip.
\section*{Benefits}
Our seminar is advantageous to get a deeper understanding of programming by learning the basics of this more advanced way to model and program an application. Even for students studying different fields than computer science, the seminar will be beneficial as almost all fields of study have at least one \emph{informatics} module.


%The primary benefit I see from writing this report will be the educational value—or SMT students and others interested in the semiconductor manufacturing process. To my knowledge, this direction in learning about how the different sections of the wafer fab has never been taken before. Another benefit is that this report will be written in a student's point of view, so that may help others understand the process more effectively. One other benefit is that this project ought to show my interest in the field and the professionalism of my work. I intend to list this project on my resume and have a copy of the report in my portfolio when I interview for jobs in this field.
\section*{Results}
The seminar will be held based on our research and preparation and will last about 45 minutes. As listed below in the tentative outline, the seminar will consist of the presentation, an explanation of the important vocabulary, a language exercise and a discussion.

There will be a complete handout on the afore mentioned parts.

Following the seminar we will write a follow-up report with an evaluation and conclusion to the seminar and the topic itself.
%The end product will consist of at least four single-spaced pages for a written version, and at least four files for the HTML version. It will consist of the microchip fabrication process flow and the explanations of how and why photolithography is important to fab operations. Graphics illustrating the universal process flow and effects of photolithography on the microchip fabrication process will be included to emphasize points presented in the report. To clarify the technical language used in the report, I will append a glossary.
\section*{Projected Schedule}
This is the projected schedule for our seminar:\\
%The following is a tentative schedule for the report:
\begin{tabular}{L{0.15} L{0.7}}
April 20 & Filing of the proposal\\
April 22 & Beginning research\\
April 27 \newline & Participating in seminars by fellow students;\newline 
evaluating the good and the bad for our own seminar\\
April 29 & Completing seminar research\\
May 2 & Completing seminar contents; uploading handouts\\
May 4 & Seminar; starting project portfolio and written report\\
June 10 & Uploading project portfolio\\
June 17 & Uploading written report
\end{tabular}
%June 21	Proposal uploaded; begin research.
%July 7	Complete compiling research from library, Internet, and textbooks.
%July 19	Complete interviews and visits to wafer fabs.
%August 6	Final copy of report uploaded.
%This schedule is subject to change as required, but I do not foresee any problems in maintaining this timeline.
\section*{Our Qualifications}
As computer science students we can offer experience in the general field of programming based on our successful first and continuing second semester.

Our skills do not only include knowledge of the programming language \emph{C} but also abstract subjects such as programming algorithms as taught in our first semester.

With our experience regarding the topic and knowledge of the typesetting language \LaTeX, we can offer a handouts and reports with quality content and a professional design.
 
%I am currently an ACC student pursuing a major in Semiconductor Manufacturing Technology, and I will receive my certification by early August of this year.
%My current grade point average in the program is 4.0 out of 4.0. My studies have included the basics of manufacturing industry operations, the microchip manufacturing process flow, the theories behind the processes in the manufacture of microchips, and basic electronics.
%I am familiar with both PC and Macintosh computers, and can use MS-DOS 6.22, Windows 3.11, Windows 95, and MacOS 8.1. My software knowledge includes Ami Pro 3.1, Microsoft Works 4.0 for Windows 95, and Netscape 4.05.
%I have written personal web pages since 1994, and can utilize HTML 3.2. I have also made some graphics for my web pages, using Microsoft Paint and Jasc's Paint Shop Pro (3.11 to 5.0.)
%With the experience I had mentioned, I can easily write a report for both Works and HTML formats.
\section*{Tentative Outline}
\begin{enumerate} [label=\Roman*.]
\item Introduction
\begin{enumerate} [label=\Alph*.]
\item Generic programming
\item Object oriented programming
\end{enumerate}
\item Object oriented programming
\begin{enumerate} [label=\Alph*.]
\item Inheritance
\item Polymorphism
\end{enumerate}
\item Glossary
\item Language Practice
\item Discussion
\end{enumerate}
\section*{Tentative Bibliography}
\nocite{*}
\printbibliography[heading=none]
\end{document}