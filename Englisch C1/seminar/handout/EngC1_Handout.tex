% Header aus der Vorlage
\input{../LaTeX_master_HTW_en}
\renewcommand{\workingdir}{../../../}

\bibliography{../proposal/EngC1_Proposal_Literature.bib}

% Definition von Titel, Autor usw.
\DTitel{\textsc{Object oriented programming / modeling}}
\DUntertitel{Handout}
\DAutor{Falk-Jonatan Strube}
\DNotiz{}
\renewcommand{\Dokumentensignatur}{}
\renewcommand{\Autorformat}[1]{\textcolor{darkgray}{Seminar by}\\\textcolor{darkgray}{#1}}
\date{May 4, 2016} 

\begin{document}

\maketitle
\newpage
%\tableofcontents
%\newpage

\section*{Seminar outline}
\begin{itemize}
\item Selected glossary
\item Presentation of the topic
\begin{itemize}
\item Generic programming
\item Object oriented programming and modeling
\begin{itemize}
\item Inheritance
\item Polymorphism
\end{itemize}
\item Relevance of the topic
\item Sources
\end{itemize}
\item (Language) practice
\item Discussion
\end{itemize}
\section*{Glossary}
\begin{tabular}{L{0.22} L{0.39}  L{0.39}}
Term & Meaning & In this context\\
\hline
Generic & Relating to a whole group of similar things, rather than to any particular thing & "Normal", nothing special\\
Redundant & Being similar to something else and thereby not needed & Duplicate code\\
Class & A group with certain qualities & Code with a concept\\
Object & A thing & An implemented class \\
Implementation & Putting a plan into action / starting to use something & The concept of a class applied to an object\\
Inheritance & A characteristic passed down from your parents & A shared characteristic between classes\\
Polymorphism & Poly: many, morph: form & Inherited classes may have many different implementations\\
\end{tabular}
\newpage
\section*{Abstract}

With this project the author aims to inform his fellow students about the benefits of object oriented programming and modeling.\medskip

The research was based on items from a diverse bibliography including articles, books and websites. The most important piece of literature is the book \emph{Softwaretechnology for beginners} which is based on a lecture of the TU Dresden.\medskip

After reading the texts it can be stated that object oriented programming is an advanced way of structuring a computer program during development. By modeling dependencies and correlations in classes with the \emph{Unified Modeling Language} (UML), a complex code formation can be made understandable. This does not only benefit team communication and communication with a client, it also improves scalability since the general overview allows specific changes in the model and the program itself.

The key concepts of object oriented programming are inheritance and polymorphism. With inheritance, similar parts of a program may be reused in \emph{child}- and \emph{parent}-objects. Polymorphism allows the modification of a inherited function. Both features add to comprehensibility and scalability.\medskip

It can be concluded that object oriented programming is used in almost every complex computer program. From operating systems such as \emph{Windows} or \emph{Linux} to word processor applications such as \emph{Microsoft Word} -- object oriented programs are everywhere. Though generic programming is more commonly known, since it is easier to grasp at first, object oriented programming is more common overall.

\section*{(Language) practice}
Recall the UML-diagram from the presentation:
\begin{center}
\resizebox{.7\textwidth}{!}{
\begin{tikzpicture}
\tikzumlset{fill class=white!0}
\umlclass{Human}{hunger}{satisfy\_hunger()}
\umlclass[x=-3, y=-3, anchor=north]{Adult}{time\_available}{\strut}
\umlclass[x=3, y=-3, anchor=north]{Student}{motivation}{satisfy\_hunger()}
\umlVHVinherit{Adult}{Human} 
\umlVHVinherit{Student}{Human} 
\tikzumlset{fill class=white!0, text=gray, draw=gray}
\umlclass[x=8, y= -1] {Class name}{properties}{functions}
\end{tikzpicture}
}
\end{center}
\paragraph{Task:} 
\begin{anumerate}
\item Try to fill out the class names of the UML-diagram below based on the following sentences (use the diagram above as a reference on the structure).
\begin{itemize}
\item Students, alumni and employees of an university are persons.
\item Students can be beginners or long-term students.
\item A Professor is an employee.
\end{itemize}
\newpage
\item Now try to fill out the fields for properties and functions ($\corr$ actions).
\begin{itemize}
\item Think of properties or functions that a \emph{parent}-class has (and therefore the child-class too) $\Rightarrow$ inheritance. 
\item What new properties or functions may the child class(es) have, which do not fit to the parent?
\item What properties or functions may be \emph{overloaded} $\Rightarrow$ polymorphism?
\end{itemize}
\end{anumerate}

\begin{center}
\resizebox{.99\textwidth}{!}{
\begin{tikzpicture}
\tikzumlset{fill class=white!0, text=white!0}
\umlclass													 {PersonMMMMMM}{name\\\strut}{eat()\\ work() \\ sleep()}
\umlclass[x=-3, y=-4, anchor=north]{AlumnusMMMMM}{\strut\\\strut}{visitUniversity()\\ drinkBeer()}
\umlclass[x=3, y=-4, anchor=north] {StudentMMMMM}{\strut\\\strut}{visitLecture()\\ drinkBeer()}
\umlclass[x=9, y=-4, anchor=north] {EmployeeMMMM}{salary\\\strut}{\strut\\\strut}
\umlclass[x=9, y=-9, anchor=north] {ProfessorMMM}{\strut\\\strut}{giveLecture()\\\strut}
\umlclass[x=0, y=-9, anchor=north] {BeginnerMMMM}{\strut\\\strut}{visitEverything()\\\strut}
\umlclass[x=5, y=-9, anchor=north] {Long-term student}{salary\\\strut}{ignoreLecture()\\\strut}
\umlVHVinherit{AlumnusMMMMM}{PersonMMMMMM} 
\umlVHVinherit{StudentMMMMM}{PersonMMMMMM} 
\umlVHVinherit{EmployeeMMMM}{PersonMMMMMM} 
\umlVHVinherit{BeginnerMMMM}{StudentMMMMM} 
\umlVHVinherit{Long-term student}{StudentMMMMM}
\umlVHVinherit{ProfessorMMM}{EmployeeMMMM} 
\end{tikzpicture}
}
\end{center}

\section*{Discussion}
\begin{anumerate}
\item What do you think? Is object oriented modeling more comprehensive than the code from generic applications?
\item Have you written a (small) program before? What challenges do you remember? Could they be tackled better with object oriented programming?
\item Do you think the object oriented way of programming is used often in modern applications? Why? Why not?
\end{anumerate}

\section*{Mind map}
\resizebox{0.99\textwidth}{!}{
\begin{tikzpicture}[mindmap, grow cyclic, every node/.style=concept, concept color=black!10, 
    level 1/.append style={level distance=5cm,sibling angle=90},
    level 2/.append style={level distance=3cm,sibling angle=45},
    level 3/.append style={level distance=2.2cm,sibling angle=55},]

\node{Object oriented programming / modelling}
   child { node {Object orientated programming}
        child { node {Im\-ple\-men\-ta\-tion}}
        child { node {Effi\-cien\-cy}}
        child { node {reusabilty}}
        child { node {Pro\-gramm\-ing lang\-uag\-es}
        	child {node{Java}}
        	child {node{C++}}
        	child {node{C\#}}
        }
        child { node (psc) {scope (code)}}
        child { node (pcm) {com\-pre\-hen\-si\-bil\-ity (code)}}
        child { node {Difficult to understand (concept)}}
    }
    child [concept color=black!20] { node {Object oriented modelling}
        child { node {UML}}
        child { node {Com\-mu\-ni\-ca\-tion}
        	child {node{Team}}
        	child {node{Client}}
        }
        child { node {Scalability}}
        child { node (msc) {scope (project)}}
        child { node (mcm) {com\-pre\-hen\-si\-bil\-ity (project)}}
    }
    child [concept color=black!25] { node {Generic programming}
        child { node {Easy to understand (concept)}}
        child { node {Difficult to understand (code)}}
        child { node {Duplicate code}}
        child { node {Pro\-gramm\-ing lang\-uag\-es}
        	child {node{C}}
        	child {node{Haskell}}
        	child {node{SQL}}
        }
        child { node {Lacks scale}}
    }
    child [concept color=black!30] { node {Concepts}
        child [grow=60] { node {In\-heri\-tance}}
        child [grow=0] { node {Poly\-morphism}}
        child [grow=-60] { node {Objects and Classes}}
    };
\end{tikzpicture}
}
\newpage
\section*{Bibliography}
\nocite{*}
\printbibliography[heading=none]
\end{document}