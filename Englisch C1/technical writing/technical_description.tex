\newcommand{\customDir}{}
\RequirePackage{ifthen,xifthen}

% Input inkl. Umlaute, Silbentrennung
\RequirePackage[T1]{fontenc}
\RequirePackage[utf8]{inputenc}

% Arbeitsordner (in Abhängigkeit vom Master) Standard: .LateX_master Ordner liegt im Eltern-Ordner
\providecommand{\customDir}{../}
\newcommand{\setCustomDir}[1]{\renewcommand{\customDir}{#1}}
%%% alle Optionen:
% Doppelseitig (mit Rand an der Innenseite)
\newboolean{twosided}
\setboolean{twosided}{false}
% Eigene Dokument-Klasse (alle KOMA möglich; cheatsheet für Spicker [3 Spalten pro Seite, alles kleiner])
\newcommand{\customDocumentClass}{scrreprt}
\newcommand{\setCustomDocumentClass}[1]{\renewcommand{\customDocumentClass}{#1}}
% Unterscheidung verschiedener Designs: htw, fjs
\newcommand{\customDesign}{htw}
\newcommand{\setCustomDesign}[1]{\renewcommand{\customDesign}{#1}}
% Dokumenten Metadaten
\newcommand{\customTitle}{}
\newcommand{\setCustomTitle}[1]{\renewcommand{\customTitle}{#1}}
\newcommand{\customSubtitle}{}
\newcommand{\setCustomSubtitle}[1]{\renewcommand{\customSubtitle}{#1}}
\newcommand{\customAuthor}{}
\newcommand{\setCustomAuthor}[1]{\renewcommand{\customAuthor}{#1}}
%	Notiz auf der Titelseite (A: vor Autor, B: nach Autor)
\newcommand{\customNoteA}{}
\newcommand{\setCustomNoteA}[1]{\renewcommand{\customNoteA}{#1}}
\newcommand{\customNoteB}{}
\newcommand{\setCustomNoteB}[1]{\renewcommand{\customNoteB}{#1}}
% Format der Signatur in Fußzeile:
\newcommand{\customSignature}{\ifthenelse{\equal{\customAuthor}{}} {} {\footnotesize{\textcolor{darkgray}{Mitschrift von\\ \customAuthor}}}}
\newcommand{\setCustomSignature}[1]{\renewcommand{\customSignature}{#1}}
% Format des Autors auf dem Titelblatt:
\newcommand{\customTitleAuthor}[1]{\textcolor{darkgray}{Mitschrift von #1}}
\newcommand{\setCustomTitleAuthor}[1]{\renewcommand{\customTitleAuthor}{#1}}
% Standard Sprache
\newcommand{\customDefaultLanguage}[1]{}
\newcommand{\setCustomDefaultLanguage}[1]{\renewcommand{\customDefaultLanguage}{#1}}
% Folien-Pfad (inkl. Dateiname ohne Endung und ggf. ohne Nummerierung)
\newcommand{\customSlidePath}{}
\newcommand{\setCustomSlidePath}[1]{\renewcommand{\customSlidePath}{#1}}
% Folien Eigenschaften
\newcommand{\customSlideScale}{0.5}
\newcommand{\setCustomSlideScale}[1]{\renewcommand{\customSlideScale}{#1}}
\newcommand{\customSlideHeight}{9.63cm}
\newcommand{\setCustomSlideHeight}[1]{\renewcommand{\customSlideHeight}{#1}}
\newcommand{\customSlideWidth}{12.8cm}
\newcommand{\setCustomSlideWidth}[1]{\renewcommand{\customSlideWidth}{#1}}

%\setboolean{twosided}{true}
%\setCustomDocumentClass{scrartcl}
%\setCustomDesign{htw}
%\setCustomSlidePath{Folien}
\newcommand{\chapterHead}{1}

\setCustomTitle{\textsc{Technical Description}}
\setCustomSubtitle{Pokémon reality gear and game}
\setCustomAuthor{Falk-Jonatan Strube}
%\setCustomNoteA{TitlepageNoteBeforeAuthor}
%\setCustomNoteB{Vorlesung von}

\setCustomSignature{}	% Formatierung der Signatur in der Fußzeile
\setCustomTitleAuthor{\customAuthor\\and Anxhela Merko}	% Formatierung des Autors auf dem Titelblatt

%-- Prüfen, ob Beamer
\ifthenelse{\equal{\customDocumentClass}{beamer}}{
%%% TODO: andere Layouts für Beamer außer HTW
	\documentclass[ignorenonframetext, 11pt, table]{beamer}
	
	\usenavigationsymbolstemplate{}
	\setbeamercolor{author in head/foot}{fg=black}
	\setbeamercolor{title}{fg=black}
	\setbeamercolor{bibliography entry author}{fg=htworange!70}
	%\setbeamercolor{bibliography entry title}{fg=blue} 
	\setbeamercolor{bibliography entry location}{fg=htworange!60} 
	\setbeamercolor{bibliography entry note}{fg=htworange!60}  
	
	\setbeamertemplate{itemize item}{\color{black}$\bullet$}
	\setbeamertemplate{itemize subitem}{\color{black}--}
	\setbeamertemplate{itemize subsubitem}{\color{black}$\bullet$}
	\makeatother
	\setbeamertemplate{footline}
	{
	\leavevmode
	\def\arraystretch{1.2}
	\arrayrulecolor{gray}
	\begin{tabular}{ p{0.167\textwidth} | p{0.491\textwidth} | p{0.089\textwidth} | p{0.103\textwidth}}
	\hline
	\strut\insertshortauthor & \insertshorttitle & Slide \insertframenumber{}% / \inserttotalframenumber{}
	 & May 4, 2016\\
	\end{tabular}
	}
	\setbeamertemplate{headline}
	{
	\leavevmode
	\setlength{\arrayrulewidth}{1pt}
	\hspace*{2em}	
	\begin{tabular}{p{0.63\textwidth}}
	\rule{0pt}{3em}\normalsize{\textbf{\insertsection\strut}}\\
	\arrayrulecolor{htworange}
	\hline
	\end{tabular}
	\begin{tabular}{l}
	\rule{0pt}{4em}\includegraphics[width=3.25cm]{\customDir .LaTeX_master/HTW_GESAMTLOGO_CMYK.eps}\\
	\end{tabular}
	}
	\makeatletter	
}{	
	%-- Für Spicker einiges anders:
	\ifthenelse{\equal{\customDocumentClass}{cheatsheet}}{
		\documentclass[a4paper,10pt,landscape]{scrartcl}
		\usepackage{geometry}
		\geometry{top=2mm, bottom=2mm, headsep=0mm, footskip=0mm, left=2mm, right=2mm}
		
		% Für Spicker \spsection für Section, zur Strukturierung \HRule oder \HDRule Linie einsetzen
		\usepackage{multicol}
		\newcommand{\spsection}[1]{\textbf{#1}}	% Platzsparende "section" für Spicker
	}{	%-- Ende Spicker-Unterscheidung-if
		%-- Unterscheidung Doppelseitig
		\ifthenelse{\boolean{twosided}}{
			\documentclass[a4paper,11pt, footheight=26pt,twoside]{\customDocumentClass}
			\usepackage[head=23pt]{geometry}	% head=23pt umgeht Fehlerwarnung, dafür größeres "top" in geometry
			\geometry{top=30mm, bottom=22mm, headsep=10mm, footskip=12mm, inner=27mm, outer=13mm}
		}{
			\documentclass[a4paper,11pt, footheight=26pt]{\customDocumentClass}
			\usepackage[head=23pt]{geometry}	% head=23pt umgeht Fehlerwarnung, dafür größeres "top" in geometry
			\geometry{top=30mm, bottom=22mm, headsep=10mm, footskip=12mm, left=20mm, right=20mm}
		}
		%-- Nummerierung bis Subsubsection für Report
		\ifthenelse{\equal{\customDocumentClass}{report} \OR \equal{\customDocumentClass}{scrreprt}}{
			\setcounter{secnumdepth}{3}	% zählt auch subsubsection
			\setcounter{tocdepth}{3}	% Inhaltsverzeichnis bis in subsubsection
		}{}
	}%-- Ende Spicker-Unterscheidung-else
	
	\usepackage{scrlayer-scrpage}	% Kopf-/Fußzeile
	\renewcommand*{\thefootnote}{\fnsymbol{footnote}}	% Fußnoten-Symbole anstatt Zahlen
	\renewcommand*{\titlepagestyle}{empty} % Keine Seitennummer auf Titelseite
	\usepackage[perpage]{footmisc}	% Fußnotenzählung Seitenweit, nicht Dokumentenweit
}

% Input inkl. Umlaute, Silbentrennung
\RequirePackage[T1]{fontenc}
\RequirePackage[utf8]{inputenc}
\usepackage[english,ngerman]{babel}
\usepackage{csquotes}	% Anführungszeichen
\RequirePackage{marvosym}
\usepackage{eurosym}

% Style-Aufhübschung
\usepackage{soul, color}	% Kapitälchen, Unterstrichen, Durchgestrichen usw. im Text
%\usepackage{titleref}
\usepackage[breakwords, fit]{truncate}	% Abschneiden von Sätzen
\renewcommand{\TruncateMarker}{\,…}

% Mathe usw.
\usepackage{amssymb}
\usepackage{amsthm}
\ifthenelse{\equal{\customDocumentClass}{beamer}}{}{
\usepackage[fleqn,intlimits]{amsmath}	% fleqn: align-Umgebung rechtsbündig; intlimits: Integralgrenzen immer ober-/unterhalb
}
%\usepackage{mathtools} % u.a. schönere underbraces
\usepackage{xcolor}
\usepackage{esint}	% Schönere Integrale, \oiint vorhanden
\everymath=\expandafter{\the\everymath\displaystyle}	% Mathe Inhalte werden weniger verkleinert
\usepackage{wasysym}	% mehr Symbole, bspw \lightning
%\renewcommand{\int}{\int\limits}
%\usepackage{xfrac}	% mehr fracs: sfrac{}{}
\let\oldemptyset\emptyset	% schöneres emptyset
\let\emptyset\varnothing
%\RequirePackage{mathabx}	% mehr Symbole
\mathchardef\mhyphen="2D	% Hyphen in Math

% tikz usw.
\usepackage{tikz}
\usepackage{pgfplots}
\pgfplotsset{compat=1.11}	% Umgeht Fehlermeldung
\usetikzlibrary{graphs}
%\usetikzlibrary{through}	% ???
\usetikzlibrary{arrows}
\usetikzlibrary{arrows.meta}	% Pfeile verändern / vergrößern: \draw[-{>[scale=1.5]}] (-3,5) -> (-3,3);
\usetikzlibrary{automata,positioning} % Zeilenumbruch im Node node[align=center] {Text\\nächste Zeile} automata für Graphen
\usetikzlibrary{matrix}
\usetikzlibrary{patterns}	% Schraffierte Füllung
\usetikzlibrary{shapes.geometric}	% Polygon usw.
\tikzstyle{reverseclip}=[insert path={	% Inverser Clip \clip
	(current page.north east) --
	(current page.south east) --
	(current page.south west) --
	(current page.north west) --
	(current page.north east)}
% Nutzen: 
%\begin{tikzpicture}[remember picture]
%\begin{scope}
%\begin{pgfinterruptboundingbox}
%\draw [clip] DIE FLÄCHE, IN DER OBJEKT NICHT ERSCHEINEN SOLL [reverseclip];
%\end{pgfinterruptboundingbox}
%\draw DAS OBJEKT;
%\end{scope}
%\end{tikzpicture}
]	% Achtung: dafür muss doppelt kompliert werden!
\usepackage{graphpap}	% Grid für Graphen
\tikzset{every state/.style={inner sep=2pt, minimum size=2em}}
\usetikzlibrary{mindmap, backgrounds}
%\usepackage{tikz-uml}	% braucht Dateien: http://perso.ensta-paristech.fr/~kielbasi/tikzuml/

% Tabular
\usepackage{longtable}	% Große Tabellen über mehrere Seiten
\usepackage{multirow}	% Multirow/-column: \multirow{2[Anzahl der Zeilen]}{*[Format]}{Test[Inhalt]} oder \multicolumn{7[Anzahl der Reihen]}{|c|[Format]}{Test2[Inhalt]}
\renewcommand{\arraystretch}{1.3} % Tabellenlinien nicht zu dicht
\usepackage{colortbl}
\arrayrulecolor{gray}	% heller Tabellenlinien
\usepackage{array}	% für folgende 3 Zeilen (für Spalten fester breite mit entsprechender Ausrichtung):
\newcolumntype{L}[1]{>{\raggedright\let\newline\\\arraybackslash\hspace{0pt}}m{\dimexpr#1\columnwidth-2\tabcolsep-1.5\arrayrulewidth}}
\newcolumntype{C}[1]{>{\centering\let\newline\\\arraybackslash\hspace{0pt}}m{\dimexpr#1\columnwidth-2\tabcolsep-1.5\arrayrulewidth}}
\newcolumntype{R}[1]{>{\raggedleft\let\newline\\\arraybackslash\hspace{0pt}}m{\dimexpr#1\columnwidth-2\tabcolsep-1.5\arrayrulewidth}}
\usepackage{caption}	% Um auch unbeschriftete Captions mit \caption* zu machen

% Nützliches
\usepackage{verbatim}	% u.a. zum auskommentieren via \begin{comment} \end{comment}
\usepackage{tabto}	% Tabs: /tab zum nächsten Tab oder /tabto{.5 \CurrentLineWidth} zur Stelle in der Linie
\NumTabs{6}	% Anzahl von Tabs pro Zeile zum springen
\usepackage{listings} % Source-Code mit Tabs
\usepackage{lstautogobble} 
\ifthenelse{\equal{\customDocumentClass}{beamer}}{}{
\usepackage{enumitem}	% Anpassung der enumerates
%\setlist[enumerate,1]{label=(\arabic*)}	% global andere Enum-Items
\renewcommand{\labelitemiii}{$\scriptscriptstyle ^\blacklozenge$} % global andere 3. Item-Aufzählungszeichen
}
\usepackage{letltxmacro} % neue Definiton von Grundbefehlen
% Nutzen:
%\LetLtxMacro{\oldemph}{\emph}
%\renewcommand{\emph}[1]{\oldemph{#1}}
\RequirePackage{xpatch}	% ua. Konkatenieren von Strings/Variablen (etoolbox)
\usepackage{xstring}	% String Operationen
\usepackage{minibox}	% Minibox anstatt \fbox{} für Boxen mit Zeilenumbruch


% Einrichtung von lst
\lstset{
basicstyle=\ttfamily, 
%mathescape=true, 
%escapeinside=^^, 
autogobble, 
tabsize=2,
basicstyle=\footnotesize\sffamily\color{black},
frame=single,
rulecolor=\color{lightgray},
numbers=left,
numbersep=5pt,
numberstyle=\tiny\color{gray},
commentstyle=\color{gray},
keywordstyle=\color{green},
stringstyle=\color{orange},
morecomment=[l][\color{magenta}]{\#}
showspaces=false,
showstringspaces=false,
breaklines=true,
literate=%
    {Ö}{{\"O}}1
    {Ä}{{\"A}}1
    {Ü}{{\"U}}1
    {ß}{{\ss}}1
    {ü}{{\"u}}1
    {ä}{{\"a}}1
    {ö}{{\"o}}1
    {~}{{\textasciitilde}}1
}
\usepackage{scrhack} % Fehler umgehen
\def\ContinueLineNumber{\lstset{firstnumber=last}} % vor lstlisting. Zum wechsel zum nicht-kontinuierlichen muss wieder \StartLineAt1 eingegeben werden
\def\StartLineAt#1{\lstset{firstnumber=#1}} % vor lstlisting \StartLineAt30 eingeben, um bei Zeile 30 zu starten
\let\numberLineAt\StartLineAt

% BibTeX
\usepackage[bibencoding=ascii,
%backend=bibtex8,
%style=authortitle, citestyle=authortitle-ibid,
%doi=false,
%isbn=false,
%url=false
]{biblatex}	% BibTeX
\usepackage{makeidx}
%\makeglossary
%\makeindex

% Grafiken
\usepackage{graphicx}
\usepackage{epstopdf}	% eps-Vektorgrafiken einfügen
\usepackage{transparent}	% transparent nutzen: {\transparent{0.4} ...}
%\epstopdfsetup{outdir=\customDir}
% Prüft, ob Grafik existiert (mit \ifvalidimage{}{}) [Quelle: https://tex.stackexchange.com/a/99176]:
\makeatletter
\newif\ifgraphicexist
\catcode`\*=11
\newcommand\ifvalidimage[1]{%
    \begingroup
    \global\graphicexisttrue
    \let\input@path\Ginput@path
    \filename@parse{#1}%
    \ifx\filename@ext\relax
    \@for\Gin@temp:=\Gin@extensions\do{%
        \ifx\Gin@ext\relax
        \Gin@getbase\Gin@temp
        \fi}%
    \else
    \Gin@getbase{\Gin@sepdefault\filename@ext}%
    \ifx\Gin@ext\relax
    \global\graphicexistfalse
    \def\Gin@base{\filename@area\filename@base}%
    \edef\Gin@ext{\Gin@sepdefault\filename@ext}%
    \fi
    \fi
    \ifx\Gin@ext\relax
    \global\graphicexistfalse
    \else 
    \@ifundefined{Gin@rule@\Gin@ext}%
    {\global\graphicexistfalse}%
    {}%
    \fi  
    \ifx\Gin@ext\relax 
    \gdef\imageextension{unknown}%
    \else
    \xdef\imageextension{\Gin@ext}%
    \fi 
    \endgroup 
    \ifgraphicexist
    \expandafter \@firstoftwo
    \else
    \expandafter \@secondoftwo
    \fi 
} 
\catcode`\*=12
\makeatother
\usepackage{letltxmacro}	% Latex-Befehle unter anderem Namen neu definieren
\LetLtxMacro{\forceincludegraphics}{\includegraphics}	% neuer Befehl für includegraphics
\renewcommand{\includegraphics}[2][]{	% altes includegraphics neu definieren, damit es auch nicht vorhandene einfügt
\ifvalidimage{#2}{
\forceincludegraphics[#1]{#2}
}{
\message{Achtung: Grafik wurde nicht gefunden: '#2'}
\minibox[frame]{
\textbf{\StrSubstitute{#2}{_}{\_}}  \ifthenelse{\isempty{#1}}{}{\\\textit{#1}}}
}}

% pdf-Setup
\usepackage{pdfpages}
\ifthenelse{\equal{\customDocumentClass}{beamer}}{}{
\usepackage[bookmarks,%
bookmarksopen=false,% Klappt die Bookmarks in Acrobat aus
colorlinks=true,%
linkcolor=black,%
citecolor=red,%
urlcolor=green,%
]{hyperref}
}

%-- Unterscheidung des Stils
\newcommand{\customLogo}{}
\newcommand{\customPreamble}{}
\ifthenelse{\equal{\customDesign}{htw}}{
	% HTW Corporate Design: Arial (Helvetica)
	\usepackage{helvet}
	\renewcommand{\familydefault}{\sfdefault}
	\renewcommand{\customLogo}{HTW-Logo}
	\renewcommand{\customPreamble}{HTW Dresden}
}{
% \renewcommand{\customLogo}{HTW-Logo.eps}
}

% Nach Dokumentenbeginn ausführen:
\AtBeginDocument{
	% Autor und Titel für pdf-Eigenschaften festlegen, falls noch nicht geschehen
	\providecommand{\pdfAuthor}{John Doe}
	\ifdefempty{\customAuthor} {} {\renewcommand{\pdfAuthor}{\customAuthor}}
	\providecommand{\pdfTitle}{}
	\providecommand{\pdfTitleA}{}
	\providecommand{\pdfTitleB}{}
	\providecommand{\pdfTitleC}{}	
	\ifdefempty{\pdfTitle}{
		\ifdefempty{\customPreamble} {} {\renewcommand{\pdfTitleA}{\customPreamble{} | }}
		\ifdefempty{\customTitle} {\renewcommand{\pdfTitleB}{No Title}} {\renewcommand{\pdfTitleB}{\customTitle}}
		\ifdefempty{\customSubtitle} {} {\renewcommand{\pdfTitleC}{ - \customSubtitle}}
	}{}
	
	\newcommand{\customLogoLocation}{\customDir .LaTeX_master/\customLogo}
	\hypersetup{
		pdfauthor={\pdfAuthor},
		pdftitle={\pdfTitleA\pdfTitleB\pdfTitleC},
	}
	\ifthenelse{\equal{\customDocumentClass}{beamer}}{
		\title{\customTitle}
		\author{\customAuthor}
	}{
		\automark[section]{section}
		\automark*[subsection]{subsection}
		\pagestyle{scrheadings}
		\ifthenelse{\equal{\customDocumentClass}{report} \OR \equal{\customDocumentClass}{scrreprt}}{
		\renewcommand*{\chapterpagestyle}{scrheadings}
		}{}
		%\renewcommand*{\titlepagestyle}{scrheadings}
		\ihead{\includegraphics[height=1.7em]{\customLogoLocation}}
		%\ohead{\truncate{4cm}{\customTitle}}
		\chead{\truncate{.5\textwidth}{\headmark}}
		\ohead{\customTitle}
		\cfoot{\pagemark}
		\ofoot{\customSignature}
		% Titelseite
		\title{
		\includegraphics[width=0.35\textwidth]{\customDir .LaTeX_master/\customLogo}\\\vspace{0.5em}
		\Huge\textbf{\customTitle}
		\ifdefempty{\customSubtitle} {} {\\\vspace*{0.7em}\Large \customSubtitle}
		\\\vspace*{5em}}
		\author{
		\ifdefempty{\customNoteA} {} {\customNoteA \vspace*{1em}}\\ 
		\ifdefempty{\customAuthor} {} {\customTitleAuthor}
		\ifdefempty{\customNoteB}{}{\vspace*{1em}\\\customNoteB}
		}
		
		\ifthenelse{\equal{\customDocumentClass}{cheatsheet}}{
			\pagestyle{empty}
			\setlist{nolistsep}
	%		\usepackage{parskip}	% Aufzählung Abstand
	%		\setlength{\parskip}{0em}
			\lstset{
	    belowcaptionskip=0pt,
	    belowskip=0pt,
	    aboveskip=0pt,
			tabsize=2,
			frame=none,
			numbers=none,
			showspaces=false,
			showstringspaces=false,
			breaklines=true,
			}
		}{}
	}
}

% Unterabschnitte
%\newtheorem{example}{Beispiel}%[section]
%\newtheorem{definition}{Definition}[section]
%\newtheorem{discussion}{Diskussion}[section]
%\newtheorem{remark}{Bemerkung}[section]
%\newtheorem{proof}{Beweis}[section]
%\newtheorem{notation}{Schreibweise}[section]
\RequirePackage{xcolor}
%% EINFACHE BEFEHLE

% Abkürzungen Mathe
\newcommand{\EE}{\mathbb{E}}
\newcommand{\QQ}{\mathbb{Q}}
\newcommand{\RR}{\mathbb{R}}
\newcommand{\CC}{\mathbb{C}}
\newcommand{\NN}{\mathbb{N}}
\newcommand{\ZZ}{\mathbb{Z}}
\newcommand{\PP}{\mathbb{P}}
\renewcommand{\SS}{\mathbb{S}}
\newcommand{\cA}{\mathcal{A}}
\newcommand{\cB}{\mathcal{B}}
\newcommand{\cC}{\mathcal{C}}
\newcommand{\cD}{\mathcal{D}}
\newcommand{\cE}{\mathcal{E}}
\newcommand{\cF}{\mathcal{F}}
\newcommand{\cG}{\mathcal{G}}
\newcommand{\cH}{\mathcal{H}}
\newcommand{\cI}{\mathcal{I}}
\newcommand{\cJ}{\mathcal{J}}
\newcommand{\cM}{\mathcal{M}}
\newcommand{\cN}{\mathcal{N}}
\newcommand{\cP}{\mathcal{P}}
\newcommand{\cR}{\mathcal{R}}
\newcommand{\cS}{\mathcal{S}}
\newcommand{\cZ}{\mathcal{Z}}
\newcommand{\cL}{\mathcal{L}}
\newcommand{\cT}{\mathcal{T}}
\newcommand{\cU}{\mathcal{U}}
\newcommand{\cV}{\mathcal{V}}
\renewcommand{\phi}{\varphi}
\renewcommand{\epsilon}{\varepsilon}

% Farbdefinitionen
\definecolor{red}{RGB}{180,0,0}
\definecolor{green}{RGB}{75,160,0}
\definecolor{blue}{RGB}{0,75,200}
\definecolor{orange}{RGB}{255,128,0}
\definecolor{yellow}{RGB}{255,245,0}
\definecolor{purple}{RGB}{75,0,160}
\definecolor{cyan}{RGB}{0,160,160}
\definecolor{brown}{RGB}{120,60,10}

\definecolor{itteny}{RGB}{244,229,0}
\definecolor{ittenyo}{RGB}{253,198,11}
\definecolor{itteno}{RGB}{241,142,28}
\definecolor{ittenor}{RGB}{234,98,31}
\definecolor{ittenr}{RGB}{227,35,34}
\definecolor{ittenrp}{RGB}{196,3,125}
\definecolor{ittenp}{RGB}{109,57,139}
\definecolor{ittenpb}{RGB}{68,78,153}
\definecolor{ittenb}{RGB}{42,113,176}
\definecolor{ittenbg}{RGB}{6,150,187}
\definecolor{itteng}{RGB}{0,142,91}
\definecolor{ittengy}{RGB}{140,187,38}

% Textfarbe ändern
\newcommand{\tred}[1]{\textcolor{red}{#1}}
\newcommand{\tgreen}[1]{\textcolor{green}{#1}}
\newcommand{\tblue}[1]{\textcolor{blue}{#1}}
\newcommand{\torange}[1]{\textcolor{orange}{#1}}
\newcommand{\tyellow}[1]{\textcolor{yellow}{#1}}
\newcommand{\tpurple}[1]{\textcolor{purple}{#1}}
\newcommand{\tcyan}[1]{\textcolor{cyan}{#1}}
\newcommand{\tbrown}[1]{\textcolor{brown}{#1}}

% Umstellen der Tabellen Definition
\newcommand{\mpb}[1][.3]{\begin{minipage}{#1\textwidth}\vspace*{3pt}}
\newcommand{\mpe}{\vspace*{3pt}\end{minipage}}

\newcommand{\resultul}[1]{\underline{\underline{#1}}}
\newcommand{\parskp}{$ $\\}	% new line after paragraph
\newcommand{\corr}{\;\widehat{=}\;}
\newcommand{\mdeg}{^{\circ}}

\newcommand{\nok}[2]{\begin{pmatrix}#1\\#2\end{pmatrix}}	% n über k BESSER: \binom{n}{k}
\newcommand{\mtr}[1]{\begin{pmatrix}#1\end{pmatrix}}	% Matrix
\newcommand{\dtr}[1]{\begin{vmatrix}#1\end{vmatrix}}	% Determinante (Betragsmatrix)
\renewcommand{\vec}[1]{\underline{#1}}	% Vektorschreibweise
\newcommand{\imptnt}[1]{\colorbox{red!30}{#1}}	% Wichtiges
\newcommand{\intd}[1]{\,\mathrm{d}#1}
\newcommand{\diffd}[1]{\mathrm{d}#1}

%\bibliography{\customDir .Literatur/HTW_Literatur.bib}

% Run texcount on tex-file and write results to a sum-file
\immediate\write18{texcount \jobname.tex -1 -sum -out=\jobname.sum}
\newcommand\wordcount{\input{\jobname.sum}}

\newcommand{\customLogoLocation}{src/logo.jpg}
\newcommand{\customLogoHeadHeight}{3em}
\newcommand{\customLogoWidth}{0.8\linewidth}
\newcommand{\poke}{\,\!\texorpdfstring{\begingroup
\setbox0=\hbox{\includegraphics[width=3.5em]{src/logo_small.jpg}}%
\parbox{\wd0}{\box0}\endgroup%\includegraphics[height=1em]{src/logo_small.jpg}
}{Pokémon}}
\newcommand{\pokeR}{\,\!\texorpdfstring{\begingroup
\setbox0=\hbox{\includegraphics[width=4em]{src/logo.jpg}}%
\parbox{\wd0}{\box0}\endgroup%\includegraphics[height=1em]{src/logo_small.jpg}
}{Pokémon}}
%\newcommand{\poke}{Pokémon}
\newcommand{\pokeT}{Pokémon}

\usepackage{wrapfig}
\usepackage{multicol}

\begin{document}

\selectlanguage{english}
\maketitle
\newpage
\tableofcontents
\vfill
Word Count: \wordcount
\newpage
\chapter{Introduction}

\emph{\poke{} reality game and gear} is an innovative new Soft- and Hardware-Bundle which lets you dive into the Pokémon world. Introducing the \emph{\poke{} reality gear} for an all new leap into immersion in gaming, enabling you to be as close to your favorite \poke{} as never before.\medskip

Description Gear \medskip

The game offers the choice on how exactly you want to experience the \poke{} world. The world and its inhabitants have grown over the years, with each generation adding new regions to explore and hundreds new \poke{} to catch.

This time, for the first time ever, you may choose on which regions you want to explore -- or not -- and which  \poke{} you want to encounter -- or not. In the section \textit{\nameref{software}} you can get an in detail description on what to expect.

So what are you waiting for? Start getting to know your new favorite \poke{} game with the best gear available!

\section{Glossary}
New to \poke{} or virtual reality? Here are some words you might not know:\\
\begin{tabular}{L{.3}L{.7}}
Pokémon\newline & pocket monster (animal-like monsters which live in the wild and are caught and trained by Pokémon trainers)\\
NPC & non-player-characters (computer controlled characters)\\
HP & health points
\end{tabular}
%Glossary?!

\chapter{The Gear}

%Warnings (next to parts which need warnings)?

\section{Hardware Layout}

%How the whole headset looks like (what it is made of)

\section{Important Parts}

%How the most important parts look like (and work)

\chapter{The Game}

\section{Consumer Awareness}
\label{software}
\begin{wrapfigure}{R}{60pt}
\vspace{-18pt}
\includegraphics[scale =1]{src/ratingsymbol_e10}
\vspace{-30pt}
\end{wrapfigure}
\emph{\poke{} reality: the game} is generally suitable for all ages. It contains minimal mild cartoon violence and some mild language.

It is recommended to pause playing the game every hour for at least five minutes. This ensures minimal disorientation after long game sessions.

\section{Scope Of The Game}

The main focus of \emph{\poke{} reality: the game} is the exploration of the game world and the encounters with all kinds of \poke{}. In contrast to previous installments of \poke{} games, \emph{\poke{} reality: the game} lets you chose what regions you want to explore and what \poke{} you want to encounter.

\begin{figure}[!ht]
\begin{center}
{%
\setlength{\fboxsep}{2pt}%
\setlength{\fboxrule}{1pt}%
\fbox{\includegraphics[width=0.7\linewidth]{src/pokemap}}%
}%
\end{center}
\caption[The \emph{\pokeT{} reality game} map with all available 'regions']{The \emph{\poke{} reality game} map with all available 'regions'}
\label{pokemap}
\end{figure}

The \poke{} world is composed of all your favorite regions of the previous games. The following regions are open for you to explore (see Figure \ref{pokemap}):
\begin{multicols}{2}
\begin{itemize}
\item Kanto
\item Johto
\item Hoenn
\item Sinnoh
\item Unova
\item Kalos
\item Alma
\item Orre
\item Oblivia
\item Fiore
\item Orange Islands
\item Sevii Islands
\end{itemize} 
\end{multicols}

All these regions carry unique bioms and moods. There are different kinds of towns and landscapes to explore: From the windy \emph{Orange Islands} with their gusting sea to the steep mountains in the west of \emph{Kanto}.

One of the most important feature of \emph{\poke{} reality: the game} is the fact, that you may pick what \poke{} you want to encounter. 

You want a more retrospective experience? No problem: just deactivate all \poke{} except the first generation and you will only encounter \emph{Kanto-\poke{}}.

You want the full experience? Just keep the settings unchanged and get to know the whole range of \poke{}.

You may choose among the following sets of \poke{}, together over 800 \poke{}:
\begin{multicols}{2}
\begin{itemize}
\item 1. Generation: 151 \emph{Kanto-\pokeT}
\item 2. Generation: 100 \emph{Johto-\pokeT}
\item 3. Generation: 135 \emph{Hoenn-\pokeT}
\item 4. Generation: 107 \emph{Sinnoh-\pokeT}
\item 5. Generation: 156 \emph{Einall-\pokeT}
\item 6. Generation: 72 \emph{Kalos-\pokeT}
\item 7. Generation: 79 \emph{Alola-\pokeT}
\item[]
\end{itemize}
\end{multicols}

\section{Gameplay}

The gameplay consists mainly of three parts:
\begin{itemize}
\item exploring the world
\item fighting and catching new \poke{}
\item battling \poke{}-trainers
\end{itemize}

The final goal in the game is to catch as many \poke{} as you can, become the best \poke{}-trainer and find all the secrets hidden in the world.

\subsection{Exploration}

As you will playing the game with your \emph{\poke{} reality gear}, the exploring itself is pretty much how you explore the real world, too: Walk and wonder (Figure \ref{exploration}).

\begin{figure}[!ht]
\begin{center}
{%
\setlength{\fboxsep}{2pt}%
\setlength{\fboxrule}{1pt}%
\fbox{\includegraphics[width=0.7\linewidth]{src/pokeworld}}%
}%
\end{center}
\caption[Exploring the \pokeT{} world]{Exploring the \poke{} world}
\label{exploration}
\end{figure}

By keeping an eye open for details, you may discover hidden paths which lead to secret areas. There you may find rare \poke{} and unique items (Figure \ref{secrets}).

\begin{figure}[!ht]
\begin{center}
{%
\setlength{\fboxsep}{2pt}%
\setlength{\fboxrule}{1pt}%
\fbox{\includegraphics[width=0.7\linewidth]{src/pokeworld2}}%
}%
\end{center}
\caption[Discovering secrets in the \pokeT{} world]{Discovering secrets in the \poke{} world}
\label{secrets}
\end{figure}

You can manage the following in the game menu:
\begin{itemize}
\item caught \poke{}
\item found secrets
\item your game progress
\end{itemize}
This menu is used intuitively with your \emph{\poke{} reality gear} (Figure \ref{menu}).

\begin{figure}[!ht]
\begin{center}
{%
\setlength{\fboxsep}{2pt}%
\setlength{\fboxrule}{1pt}%
\fbox{\includegraphics[width=0.7\linewidth]{src/interface}}%
}%
\end{center}
\caption[The \emph{\pokeT{} reality game} menu]{The \emph{\poke{} reality game} menu}
\label{menu}
\end{figure}

%\newpage
\subsection[\pokeT{}-battles]{\poke{}\,-battles}
During your exploration of the \poke{} world, you will encounter random \poke{} in the wild. In populated areas you may find other \poke{}-trainers (NPC). Both lead to \poke{} battles between you and your opponent.

A battle has the following sequence of events:
\begin{enumerate}
\item You choose a \poke{} to battle the opponent.
\item The battle proceeds turn-based. In one turn each you may take one of the following actions:
\begin{itemize}
\item choose an offensive or defensive action for your \poke{} to execute
\item use a special item from the inventory to buff the fighting \poke{}
\item exchange the fighting \poke{} for a different \poke{} from your team
\item retreat from the battle
\end{itemize}
\item These turns are repeated until the fight is over. If you lose you will be teleported back to the nearest town. By winning your \poke{} gain experience and you may get some item-rewards.
\end{enumerate}

%\newpage
In Figure \ref{battle} you can see the layout of a battle:
\begin{itemize}
\item Your \poke{} is at the bottom left with the corresponding name and its HP on the right.
\item Your opponent is at the top: the name and HP at the top left and the \poke{} on the right.
\end{itemize}

\begin{figure}[!ht]
\begin{center}
{%
\setlength{\fboxsep}{2pt}%
\setlength{\fboxrule}{1pt}%
\fbox{\includegraphics[width=0.7\linewidth]{src/interface3}}%
}%
\end{center}
\caption[A \pokeT{} battle]{A \poke{} battle}
\label{battle}
\end{figure}

\newpage

\subsection[Catching \pokeT{}]{Catching \poke{}}
During battles with wild \poke{} you have the chance to catch them to add to your team. This takes place during one of your turns: Instead of attacking the opponent, you select a Pokéball from your inventory (Figure \ref{catching}). With some luck, this results in the \poke{} being caught by you. To increase your chances, it is recommended to weaken the \poke{} you are trying to catch before using a Pokéball by attacking it and thus reducing its HP.

\begin{figure}[!ht]
\begin{center}
{%
\setlength{\fboxsep}{2pt}%
\setlength{\fboxrule}{1pt}%
\fbox{\includegraphics[width=0.7\linewidth]{src/interface2}}%
}%
\end{center}
\caption[Catching a \pokeT{} with Pokéballs using the game menu]{Catching a \poke{} with Pokéballs using the game menu}
\label{catching}
\end{figure}

\chapter{Conclusion}

Thank you for reading this technical description of the \emph{\poke{} reality game and gear}. We hope you enjoy our new revolutionary way of experiencing the \poke{} wold.

\bigskip

If you have any feedback and suggestions regarding the \emph{\poke{} reality game and gear} you may contact us over email or send us a letter:\bigskip\\
\begin{tabular}{l L{.5}}
&Merko\& Strube Inc.\\
\Letter & Charmanderstreet 5\newline
D-04359 Squirtle Town\\
\Email & MandS.inc@pokereality.com\\
\end{tabular}

\chapter{Questions}
[not present in final release]
\begin{itemize}
\item Is the level of detail of process description of the game enough or too much? Is something missing?
\end{itemize}

\newpage
\listoffigures
%\newpage
%\printbibliography

\end{document}