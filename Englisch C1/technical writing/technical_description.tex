\newcommand{\customDir}{}
\RequirePackage{ifthen,xifthen}

% Input inkl. Umlaute, Silbentrennung
\RequirePackage[T1]{fontenc}
\RequirePackage[utf8]{inputenc}

% Arbeitsordner (in Abhängigkeit vom Master) Standard: .LateX_master Ordner liegt im Eltern-Ordner
\providecommand{\customDir}{../}
\newcommand{\setCustomDir}[1]{\renewcommand{\customDir}{#1}}
%%% alle Optionen:
% Doppelseitig (mit Rand an der Innenseite)
\newboolean{twosided}
\setboolean{twosided}{false}
% Eigene Dokument-Klasse (alle KOMA möglich; cheatsheet für Spicker [3 Spalten pro Seite, alles kleiner])
\newcommand{\customDocumentClass}{scrreprt}
\newcommand{\setCustomDocumentClass}[1]{\renewcommand{\customDocumentClass}{#1}}
% Unterscheidung verschiedener Designs: htw, fjs
\newcommand{\customDesign}{htw}
\newcommand{\setCustomDesign}[1]{\renewcommand{\customDesign}{#1}}
% Dokumenten Metadaten
\newcommand{\customTitle}{}
\newcommand{\setCustomTitle}[1]{\renewcommand{\customTitle}{#1}}
\newcommand{\customSubtitle}{}
\newcommand{\setCustomSubtitle}[1]{\renewcommand{\customSubtitle}{#1}}
\newcommand{\customAuthor}{}
\newcommand{\setCustomAuthor}[1]{\renewcommand{\customAuthor}{#1}}
%	Notiz auf der Titelseite (A: vor Autor, B: nach Autor)
\newcommand{\customNoteA}{}
\newcommand{\setCustomNoteA}[1]{\renewcommand{\customNoteA}{#1}}
\newcommand{\customNoteB}{}
\newcommand{\setCustomNoteB}[1]{\renewcommand{\customNoteB}{#1}}
% Format der Signatur in Fußzeile:
\newcommand{\customSignature}{\ifthenelse{\equal{\customAuthor}{}} {} {\footnotesize{\textcolor{darkgray}{Mitschrift von\\ \customAuthor}}}}
\newcommand{\setCustomSignature}[1]{\renewcommand{\customSignature}{#1}}
% Format des Autors auf dem Titelblatt:
\newcommand{\customTitleAuthor}[1]{\textcolor{darkgray}{Mitschrift von #1}}
\newcommand{\setCustomTitleAuthor}[1]{\renewcommand{\customTitleAuthor}{#1}}
% Standard Sprache
\newcommand{\customDefaultLanguage}[1]{}
\newcommand{\setCustomDefaultLanguage}[1]{\renewcommand{\customDefaultLanguage}{#1}}
% Folien-Pfad (inkl. Dateiname ohne Endung und ggf. ohne Nummerierung)
\newcommand{\customSlidePath}{}
\newcommand{\setCustomSlidePath}[1]{\renewcommand{\customSlidePath}{#1}}
% Folien Eigenschaften
\newcommand{\customSlideScale}{0.5}
\newcommand{\setCustomSlideScale}[1]{\renewcommand{\customSlideScale}{#1}}
\newcommand{\customSlideHeight}{9.63cm}
\newcommand{\setCustomSlideHeight}[1]{\renewcommand{\customSlideHeight}{#1}}
\newcommand{\customSlideWidth}{12.8cm}
\newcommand{\setCustomSlideWidth}[1]{\renewcommand{\customSlideWidth}{#1}}

%\setboolean{twosided}{true}
%\setCustomDocumentClass{scrartcl}
%\setCustomDesign{htw}
%\setCustomSlidePath{Folien}
\newcommand{\chapterHead}{1}

\setCustomTitle{\textsc{Technical Description}}
\setCustomSubtitle{Pokémon reality gear and game}
\setCustomAuthor{Falk-Jonatan Strube}
%\setCustomNoteA{TitlepageNoteBeforeAuthor}
%\setCustomNoteB{Vorlesung von}

\setCustomSignature{}	% Formatierung der Signatur in der Fußzeile
\setCustomTitleAuthor{\customAuthor\\and Anxhela Merko}	% Formatierung des Autors auf dem Titelblatt

%-- Prüfen, ob Beamer
\ifthenelse{\equal{\customDocumentClass}{beamer}}{
%%% TODO: andere Layouts für Beamer außer HTW
	\documentclass[ignorenonframetext, 11pt, table]{beamer}
	
	\usenavigationsymbolstemplate{}
	\setbeamercolor{author in head/foot}{fg=black}
	\setbeamercolor{title}{fg=black}
	\setbeamercolor{bibliography entry author}{fg=htworange!70}
	%\setbeamercolor{bibliography entry title}{fg=blue} 
	\setbeamercolor{bibliography entry location}{fg=htworange!60} 
	\setbeamercolor{bibliography entry note}{fg=htworange!60}  
	
	\setbeamertemplate{itemize item}{\color{black}$\bullet$}
	\setbeamertemplate{itemize subitem}{\color{black}--}
	\setbeamertemplate{itemize subsubitem}{\color{black}$\bullet$}
	\makeatother
	\setbeamertemplate{footline}
	{
	\leavevmode
	\def\arraystretch{1.2}
	\arrayrulecolor{gray}
	\begin{tabular}{ p{0.167\textwidth} | p{0.491\textwidth} | p{0.089\textwidth} | p{0.103\textwidth}}
	\hline
	\strut\insertshortauthor & \insertshorttitle & Slide \insertframenumber{}% / \inserttotalframenumber{}
	 & May 4, 2016\\
	\end{tabular}
	}
	\setbeamertemplate{headline}
	{
	\leavevmode
	\setlength{\arrayrulewidth}{1pt}
	\hspace*{2em}	
	\begin{tabular}{p{0.63\textwidth}}
	\rule{0pt}{3em}\normalsize{\textbf{\insertsection\strut}}\\
	\arrayrulecolor{htworange}
	\hline
	\end{tabular}
	\begin{tabular}{l}
	\rule{0pt}{4em}\includegraphics[width=3.25cm]{\customDir .LaTeX_master/HTW_GESAMTLOGO_CMYK.eps}\\
	\end{tabular}
	}
	\makeatletter	
}{	
	%-- Für Spicker einiges anders:
	\ifthenelse{\equal{\customDocumentClass}{cheatsheet}}{
		\documentclass[a4paper,10pt,landscape]{scrartcl}
		\usepackage{geometry}
		\geometry{top=2mm, bottom=2mm, headsep=0mm, footskip=0mm, left=2mm, right=2mm}
		
		% Für Spicker \spsection für Section, zur Strukturierung \HRule oder \HDRule Linie einsetzen
		\usepackage{multicol}
		\newcommand{\spsection}[1]{\textbf{#1}}	% Platzsparende "section" für Spicker
	}{	%-- Ende Spicker-Unterscheidung-if
		%-- Unterscheidung Doppelseitig
		\ifthenelse{\boolean{twosided}}{
			\documentclass[a4paper,11pt, footheight=26pt,twoside]{\customDocumentClass}
			\usepackage[head=23pt]{geometry}	% head=23pt umgeht Fehlerwarnung, dafür größeres "top" in geometry
			\geometry{top=30mm, bottom=22mm, headsep=10mm, footskip=12mm, inner=27mm, outer=13mm}
		}{
			\documentclass[a4paper,11pt, footheight=26pt]{\customDocumentClass}
			\usepackage[head=23pt]{geometry}	% head=23pt umgeht Fehlerwarnung, dafür größeres "top" in geometry
			\geometry{top=30mm, bottom=22mm, headsep=10mm, footskip=12mm, left=20mm, right=20mm}
		}
		%-- Nummerierung bis Subsubsection für Report
		\ifthenelse{\equal{\customDocumentClass}{report} \OR \equal{\customDocumentClass}{scrreprt}}{
			\setcounter{secnumdepth}{3}	% zählt auch subsubsection
			\setcounter{tocdepth}{3}	% Inhaltsverzeichnis bis in subsubsection
		}{}
	}%-- Ende Spicker-Unterscheidung-else
	
	\usepackage{scrlayer-scrpage}	% Kopf-/Fußzeile
	\renewcommand*{\thefootnote}{\fnsymbol{footnote}}	% Fußnoten-Symbole anstatt Zahlen
	\renewcommand*{\titlepagestyle}{empty} % Keine Seitennummer auf Titelseite
	\usepackage[perpage]{footmisc}	% Fußnotenzählung Seitenweit, nicht Dokumentenweit
}

% Input inkl. Umlaute, Silbentrennung
\RequirePackage[T1]{fontenc}
\RequirePackage[utf8]{inputenc}
\usepackage[english,ngerman]{babel}
\usepackage{csquotes}	% Anführungszeichen
\RequirePackage{marvosym}
\usepackage{eurosym}

% Style-Aufhübschung
\usepackage{soul, color}	% Kapitälchen, Unterstrichen, Durchgestrichen usw. im Text
%\usepackage{titleref}
\usepackage[breakwords, fit]{truncate}	% Abschneiden von Sätzen
\renewcommand{\TruncateMarker}{\,…}

% Mathe usw.
\usepackage{amssymb}
\usepackage{amsthm}
\ifthenelse{\equal{\customDocumentClass}{beamer}}{}{
\usepackage[fleqn,intlimits]{amsmath}	% fleqn: align-Umgebung rechtsbündig; intlimits: Integralgrenzen immer ober-/unterhalb
}
%\usepackage{mathtools} % u.a. schönere underbraces
\usepackage{xcolor}
\usepackage{esint}	% Schönere Integrale, \oiint vorhanden
\everymath=\expandafter{\the\everymath\displaystyle}	% Mathe Inhalte werden weniger verkleinert
\usepackage{wasysym}	% mehr Symbole, bspw \lightning
%\renewcommand{\int}{\int\limits}
%\usepackage{xfrac}	% mehr fracs: sfrac{}{}
\let\oldemptyset\emptyset	% schöneres emptyset
\let\emptyset\varnothing
%\RequirePackage{mathabx}	% mehr Symbole
\mathchardef\mhyphen="2D	% Hyphen in Math

% tikz usw.
\usepackage{tikz}
\usepackage{pgfplots}
\pgfplotsset{compat=1.11}	% Umgeht Fehlermeldung
\usetikzlibrary{graphs}
%\usetikzlibrary{through}	% ???
\usetikzlibrary{arrows}
\usetikzlibrary{arrows.meta}	% Pfeile verändern / vergrößern: \draw[-{>[scale=1.5]}] (-3,5) -> (-3,3);
\usetikzlibrary{automata,positioning} % Zeilenumbruch im Node node[align=center] {Text\\nächste Zeile} automata für Graphen
\usetikzlibrary{matrix}
\usetikzlibrary{patterns}	% Schraffierte Füllung
\usetikzlibrary{shapes.geometric}	% Polygon usw.
\tikzstyle{reverseclip}=[insert path={	% Inverser Clip \clip
	(current page.north east) --
	(current page.south east) --
	(current page.south west) --
	(current page.north west) --
	(current page.north east)}
% Nutzen: 
%\begin{tikzpicture}[remember picture]
%\begin{scope}
%\begin{pgfinterruptboundingbox}
%\draw [clip] DIE FLÄCHE, IN DER OBJEKT NICHT ERSCHEINEN SOLL [reverseclip];
%\end{pgfinterruptboundingbox}
%\draw DAS OBJEKT;
%\end{scope}
%\end{tikzpicture}
]	% Achtung: dafür muss doppelt kompliert werden!
\usepackage{graphpap}	% Grid für Graphen
\tikzset{every state/.style={inner sep=2pt, minimum size=2em}}
\usetikzlibrary{mindmap, backgrounds}
%\usepackage{tikz-uml}	% braucht Dateien: http://perso.ensta-paristech.fr/~kielbasi/tikzuml/

% Tabular
\usepackage{longtable}	% Große Tabellen über mehrere Seiten
\usepackage{multirow}	% Multirow/-column: \multirow{2[Anzahl der Zeilen]}{*[Format]}{Test[Inhalt]} oder \multicolumn{7[Anzahl der Reihen]}{|c|[Format]}{Test2[Inhalt]}
\renewcommand{\arraystretch}{1.3} % Tabellenlinien nicht zu dicht
\usepackage{colortbl}
\arrayrulecolor{gray}	% heller Tabellenlinien
\usepackage{array}	% für folgende 3 Zeilen (für Spalten fester breite mit entsprechender Ausrichtung):
\newcolumntype{L}[1]{>{\raggedright\let\newline\\\arraybackslash\hspace{0pt}}m{\dimexpr#1\columnwidth-2\tabcolsep-1.5\arrayrulewidth}}
\newcolumntype{C}[1]{>{\centering\let\newline\\\arraybackslash\hspace{0pt}}m{\dimexpr#1\columnwidth-2\tabcolsep-1.5\arrayrulewidth}}
\newcolumntype{R}[1]{>{\raggedleft\let\newline\\\arraybackslash\hspace{0pt}}m{\dimexpr#1\columnwidth-2\tabcolsep-1.5\arrayrulewidth}}
\usepackage{caption}	% Um auch unbeschriftete Captions mit \caption* zu machen

% Nützliches
\usepackage{verbatim}	% u.a. zum auskommentieren via \begin{comment} \end{comment}
\usepackage{tabto}	% Tabs: /tab zum nächsten Tab oder /tabto{.5 \CurrentLineWidth} zur Stelle in der Linie
\NumTabs{6}	% Anzahl von Tabs pro Zeile zum springen
\usepackage{listings} % Source-Code mit Tabs
\usepackage{lstautogobble} 
\ifthenelse{\equal{\customDocumentClass}{beamer}}{}{
\usepackage{enumitem}	% Anpassung der enumerates
%\setlist[enumerate,1]{label=(\arabic*)}	% global andere Enum-Items
\renewcommand{\labelitemiii}{$\scriptscriptstyle ^\blacklozenge$} % global andere 3. Item-Aufzählungszeichen
}
\usepackage{letltxmacro} % neue Definiton von Grundbefehlen
% Nutzen:
%\LetLtxMacro{\oldemph}{\emph}
%\renewcommand{\emph}[1]{\oldemph{#1}}
\RequirePackage{xpatch}	% ua. Konkatenieren von Strings/Variablen (etoolbox)
\usepackage{xstring}	% String Operationen
\usepackage{minibox}	% Minibox anstatt \fbox{} für Boxen mit Zeilenumbruch


% Einrichtung von lst
\lstset{
basicstyle=\ttfamily, 
%mathescape=true, 
%escapeinside=^^, 
autogobble, 
tabsize=2,
basicstyle=\footnotesize\sffamily\color{black},
frame=single,
rulecolor=\color{lightgray},
numbers=left,
numbersep=5pt,
numberstyle=\tiny\color{gray},
commentstyle=\color{gray},
keywordstyle=\color{green},
stringstyle=\color{orange},
morecomment=[l][\color{magenta}]{\#}
showspaces=false,
showstringspaces=false,
breaklines=true,
literate=%
    {Ö}{{\"O}}1
    {Ä}{{\"A}}1
    {Ü}{{\"U}}1
    {ß}{{\ss}}1
    {ü}{{\"u}}1
    {ä}{{\"a}}1
    {ö}{{\"o}}1
    {~}{{\textasciitilde}}1
}
\usepackage{scrhack} % Fehler umgehen
\def\ContinueLineNumber{\lstset{firstnumber=last}} % vor lstlisting. Zum wechsel zum nicht-kontinuierlichen muss wieder \StartLineAt1 eingegeben werden
\def\StartLineAt#1{\lstset{firstnumber=#1}} % vor lstlisting \StartLineAt30 eingeben, um bei Zeile 30 zu starten
\let\numberLineAt\StartLineAt

% BibTeX
\usepackage[bibencoding=ascii,
%backend=bibtex8,
%style=authortitle, citestyle=authortitle-ibid,
%doi=false,
%isbn=false,
%url=false
]{biblatex}	% BibTeX
\usepackage{makeidx}
%\makeglossary
%\makeindex

% Grafiken
\usepackage{graphicx}
\usepackage{epstopdf}	% eps-Vektorgrafiken einfügen
\usepackage{transparent}	% transparent nutzen: {\transparent{0.4} ...}
%\epstopdfsetup{outdir=\customDir}
% Prüft, ob Grafik existiert (mit \ifvalidimage{}{}) [Quelle: https://tex.stackexchange.com/a/99176]:
\makeatletter
\newif\ifgraphicexist
\catcode`\*=11
\newcommand\ifvalidimage[1]{%
    \begingroup
    \global\graphicexisttrue
    \let\input@path\Ginput@path
    \filename@parse{#1}%
    \ifx\filename@ext\relax
    \@for\Gin@temp:=\Gin@extensions\do{%
        \ifx\Gin@ext\relax
        \Gin@getbase\Gin@temp
        \fi}%
    \else
    \Gin@getbase{\Gin@sepdefault\filename@ext}%
    \ifx\Gin@ext\relax
    \global\graphicexistfalse
    \def\Gin@base{\filename@area\filename@base}%
    \edef\Gin@ext{\Gin@sepdefault\filename@ext}%
    \fi
    \fi
    \ifx\Gin@ext\relax
    \global\graphicexistfalse
    \else 
    \@ifundefined{Gin@rule@\Gin@ext}%
    {\global\graphicexistfalse}%
    {}%
    \fi  
    \ifx\Gin@ext\relax 
    \gdef\imageextension{unknown}%
    \else
    \xdef\imageextension{\Gin@ext}%
    \fi 
    \endgroup 
    \ifgraphicexist
    \expandafter \@firstoftwo
    \else
    \expandafter \@secondoftwo
    \fi 
} 
\catcode`\*=12
\makeatother
\usepackage{letltxmacro}	% Latex-Befehle unter anderem Namen neu definieren
\LetLtxMacro{\forceincludegraphics}{\includegraphics}	% neuer Befehl für includegraphics
\renewcommand{\includegraphics}[2][]{	% altes includegraphics neu definieren, damit es auch nicht vorhandene einfügt
\ifvalidimage{#2}{
\forceincludegraphics[#1]{#2}
}{
\message{Achtung: Grafik wurde nicht gefunden: '#2'}
\minibox[frame]{
\textbf{\StrSubstitute{#2}{_}{\_}}  \ifthenelse{\isempty{#1}}{}{\\\textit{#1}}}
}}

% pdf-Setup
\usepackage{pdfpages}
\ifthenelse{\equal{\customDocumentClass}{beamer}}{}{
\usepackage[bookmarks,%
bookmarksopen=false,% Klappt die Bookmarks in Acrobat aus
colorlinks=true,%
linkcolor=black,%
citecolor=red,%
urlcolor=green,%
]{hyperref}
}

%-- Unterscheidung des Stils
\newcommand{\customLogo}{}
\newcommand{\customPreamble}{}
\ifthenelse{\equal{\customDesign}{htw}}{
	% HTW Corporate Design: Arial (Helvetica)
	\usepackage{helvet}
	\renewcommand{\familydefault}{\sfdefault}
	\renewcommand{\customLogo}{HTW-Logo}
	\renewcommand{\customPreamble}{HTW Dresden}
}{
% \renewcommand{\customLogo}{HTW-Logo.eps}
}

% Nach Dokumentenbeginn ausführen:
\AtBeginDocument{
	% Autor und Titel für pdf-Eigenschaften festlegen, falls noch nicht geschehen
	\providecommand{\pdfAuthor}{John Doe}
	\ifdefempty{\customAuthor} {} {\renewcommand{\pdfAuthor}{\customAuthor}}
	\providecommand{\pdfTitle}{}
	\providecommand{\pdfTitleA}{}
	\providecommand{\pdfTitleB}{}
	\providecommand{\pdfTitleC}{}	
	\ifdefempty{\pdfTitle}{
		\ifdefempty{\customPreamble} {} {\renewcommand{\pdfTitleA}{\customPreamble{} | }}
		\ifdefempty{\customTitle} {\renewcommand{\pdfTitleB}{No Title}} {\renewcommand{\pdfTitleB}{\customTitle}}
		\ifdefempty{\customSubtitle} {} {\renewcommand{\pdfTitleC}{ - \customSubtitle}}
	}{}
	
	\newcommand{\customLogoLocation}{\customDir .LaTeX_master/\customLogo}
	\hypersetup{
		pdfauthor={\pdfAuthor},
		pdftitle={\pdfTitleA\pdfTitleB\pdfTitleC},
	}
	\ifthenelse{\equal{\customDocumentClass}{beamer}}{
		\title{\customTitle}
		\author{\customAuthor}
	}{
		\automark[section]{section}
		\automark*[subsection]{subsection}
		\pagestyle{scrheadings}
		\ifthenelse{\equal{\customDocumentClass}{report} \OR \equal{\customDocumentClass}{scrreprt}}{
		\renewcommand*{\chapterpagestyle}{scrheadings}
		}{}
		%\renewcommand*{\titlepagestyle}{scrheadings}
		\ihead{\includegraphics[height=1.7em]{\customLogoLocation}}
		%\ohead{\truncate{4cm}{\customTitle}}
		\chead{\truncate{.5\textwidth}{\headmark}}
		\ohead{\customTitle}
		\cfoot{\pagemark}
		\ofoot{\customSignature}
		% Titelseite
		\title{
		\includegraphics[width=0.35\textwidth]{\customDir .LaTeX_master/\customLogo}\\\vspace{0.5em}
		\Huge\textbf{\customTitle}
		\ifdefempty{\customSubtitle} {} {\\\vspace*{0.7em}\Large \customSubtitle}
		\\\vspace*{5em}}
		\author{
		\ifdefempty{\customNoteA} {} {\customNoteA \vspace*{1em}}\\ 
		\ifdefempty{\customAuthor} {} {\customTitleAuthor}
		\ifdefempty{\customNoteB}{}{\vspace*{1em}\\\customNoteB}
		}
		
		\ifthenelse{\equal{\customDocumentClass}{cheatsheet}}{
			\pagestyle{empty}
			\setlist{nolistsep}
	%		\usepackage{parskip}	% Aufzählung Abstand
	%		\setlength{\parskip}{0em}
			\lstset{
	    belowcaptionskip=0pt,
	    belowskip=0pt,
	    aboveskip=0pt,
			tabsize=2,
			frame=none,
			numbers=none,
			showspaces=false,
			showstringspaces=false,
			breaklines=true,
			}
		}{}
	}
}

% Unterabschnitte
%\newtheorem{example}{Beispiel}%[section]
%\newtheorem{definition}{Definition}[section]
%\newtheorem{discussion}{Diskussion}[section]
%\newtheorem{remark}{Bemerkung}[section]
%\newtheorem{proof}{Beweis}[section]
%\newtheorem{notation}{Schreibweise}[section]
\RequirePackage{xcolor}
%% EINFACHE BEFEHLE

% Abkürzungen Mathe
\newcommand{\EE}{\mathbb{E}}
\newcommand{\QQ}{\mathbb{Q}}
\newcommand{\RR}{\mathbb{R}}
\newcommand{\CC}{\mathbb{C}}
\newcommand{\NN}{\mathbb{N}}
\newcommand{\ZZ}{\mathbb{Z}}
\newcommand{\PP}{\mathbb{P}}
\renewcommand{\SS}{\mathbb{S}}
\newcommand{\cA}{\mathcal{A}}
\newcommand{\cB}{\mathcal{B}}
\newcommand{\cC}{\mathcal{C}}
\newcommand{\cD}{\mathcal{D}}
\newcommand{\cE}{\mathcal{E}}
\newcommand{\cF}{\mathcal{F}}
\newcommand{\cG}{\mathcal{G}}
\newcommand{\cH}{\mathcal{H}}
\newcommand{\cI}{\mathcal{I}}
\newcommand{\cJ}{\mathcal{J}}
\newcommand{\cM}{\mathcal{M}}
\newcommand{\cN}{\mathcal{N}}
\newcommand{\cP}{\mathcal{P}}
\newcommand{\cR}{\mathcal{R}}
\newcommand{\cS}{\mathcal{S}}
\newcommand{\cZ}{\mathcal{Z}}
\newcommand{\cL}{\mathcal{L}}
\newcommand{\cT}{\mathcal{T}}
\newcommand{\cU}{\mathcal{U}}
\newcommand{\cV}{\mathcal{V}}
\renewcommand{\phi}{\varphi}
\renewcommand{\epsilon}{\varepsilon}

% Farbdefinitionen
\definecolor{red}{RGB}{180,0,0}
\definecolor{green}{RGB}{75,160,0}
\definecolor{blue}{RGB}{0,75,200}
\definecolor{orange}{RGB}{255,128,0}
\definecolor{yellow}{RGB}{255,245,0}
\definecolor{purple}{RGB}{75,0,160}
\definecolor{cyan}{RGB}{0,160,160}
\definecolor{brown}{RGB}{120,60,10}

\definecolor{itteny}{RGB}{244,229,0}
\definecolor{ittenyo}{RGB}{253,198,11}
\definecolor{itteno}{RGB}{241,142,28}
\definecolor{ittenor}{RGB}{234,98,31}
\definecolor{ittenr}{RGB}{227,35,34}
\definecolor{ittenrp}{RGB}{196,3,125}
\definecolor{ittenp}{RGB}{109,57,139}
\definecolor{ittenpb}{RGB}{68,78,153}
\definecolor{ittenb}{RGB}{42,113,176}
\definecolor{ittenbg}{RGB}{6,150,187}
\definecolor{itteng}{RGB}{0,142,91}
\definecolor{ittengy}{RGB}{140,187,38}

% Textfarbe ändern
\newcommand{\tred}[1]{\textcolor{red}{#1}}
\newcommand{\tgreen}[1]{\textcolor{green}{#1}}
\newcommand{\tblue}[1]{\textcolor{blue}{#1}}
\newcommand{\torange}[1]{\textcolor{orange}{#1}}
\newcommand{\tyellow}[1]{\textcolor{yellow}{#1}}
\newcommand{\tpurple}[1]{\textcolor{purple}{#1}}
\newcommand{\tcyan}[1]{\textcolor{cyan}{#1}}
\newcommand{\tbrown}[1]{\textcolor{brown}{#1}}

% Umstellen der Tabellen Definition
\newcommand{\mpb}[1][.3]{\begin{minipage}{#1\textwidth}\vspace*{3pt}}
\newcommand{\mpe}{\vspace*{3pt}\end{minipage}}

\newcommand{\resultul}[1]{\underline{\underline{#1}}}
\newcommand{\parskp}{$ $\\}	% new line after paragraph
\newcommand{\corr}{\;\widehat{=}\;}
\newcommand{\mdeg}{^{\circ}}

\newcommand{\nok}[2]{\begin{pmatrix}#1\\#2\end{pmatrix}}	% n über k BESSER: \binom{n}{k}
\newcommand{\mtr}[1]{\begin{pmatrix}#1\end{pmatrix}}	% Matrix
\newcommand{\dtr}[1]{\begin{vmatrix}#1\end{vmatrix}}	% Determinante (Betragsmatrix)
\renewcommand{\vec}[1]{\underline{#1}}	% Vektorschreibweise
\newcommand{\imptnt}[1]{\colorbox{red!30}{#1}}	% Wichtiges
\newcommand{\intd}[1]{\,\mathrm{d}#1}
\newcommand{\diffd}[1]{\mathrm{d}#1}

%\bibliography{\customDir .Literatur/HTW_Literatur.bib}

% Run texcount on tex-file and write results to a sum-file
\immediate\write18{texcount \jobname.tex -1 -sum -out=\jobname.sum}
\newcommand\wordcount{\input{\jobname.sum}}

\newcommand{\customLogoLocation}{src/logo_big.jpg}
\newcommand{\customLogoHeadHeight}{3em}
\newcommand{\customLogoWidth}{0.8\linewidth}

\newcommand{\poke}{\,\!\texorpdfstring{\begingroup
\setbox0=\hbox{\includegraphics[width=3.5em]{src/logo_small.jpg}}%
\parbox{\wd0}{\box0}\endgroup%\includegraphics[height=1em]{src/logo_small.jpg}
}{Pokémon}\,}
%\newcommand{\poke}{Pokémon}
\newcommand{\pokeT}{Pokémon}

\usepackage{wrapfig}
\usepackage{multicol}

\begin{document}

\selectlanguage{english}
\maketitle
\newpage
\tableofcontents
\vfill
%Word Count: \wordcount words
\newpage
\chapter{Introduction}

\emph{\poke{} reality game and gear} is an innovative new Soft- and Hardware-Bundle which lets you dive into the \poke{} world.\medskip

The \emph{\poke{} reality gear} isn't just a visual virtual reality device. This new product has been launched with an innovative new technique: the neuro-stimulator. Signals can be transmitted to the brain and cause specific sensory inputs. This provides a very accurate visual representation of the game and comes with physical sensation options. In combination with the head motion and eye tracking sensors it creates the perfect virtual reality to experience your \poke{} adventure. \medskip

The game offers the choice of how exactly you want to experience the \poke{} world. The world and its inhabitants have grown over the years, with each generation adding new regions to discover and hundreds new \poke{} to catch.

This time, for the first time ever, you may choose which regions you want to explore -- or not -- and which  \poke{} you want to encounter -- or not. In the section \textit{\nameref{software}} you can get an in detail description on what to expect. \medskip

So what are you waiting for? Start getting to know your new favorite \poke{} game with the best gear available!

\section{Glossary}
New to \poke{} or virtual reality? Here are some words you might not know:\\
\begin{tabular}{L{.3}L{.7}}
% Eye tracking sensors & is a device for measuring eye positions and eye movement\\
% Head tracking & the picture displayed in front of the user shifts as one looks up, down and side to side or angles his head\\
Stereoscopic image & Two warped images on each half of the screen\\
% Pixel resolution & the capability of the sensor to observe or measure the smallest object clearly with distinct boundaries\\
Neuron \newline & A cell that transmits information in the brain through electrical and chemical signals\\
Gyroscope & A device that senses angular velocity\\
Accelerometer \newline & A device used to measure the acceleration of a moving or vibrating body\\
% Optogenetics & is a biological technique which involves the use of light to control cells in living tissue, typically neurons\\
% LED-Arrays & assemblies of LED packages or dies that can be built using several methods. Each method hinges on the manner and extent to which the chips themselves are packaged by the LED semiconductor manufacturer \\
Pokémon\newline & Pocket monster (animal-like monsters which live in the wild and are caught and trained by Pokémon trainers)\\
NPC & Non-player-characters (computer controlled characters)\\
HP & Health points\\
\end{tabular}
%Glossary?!

\chapter{The Gear}

The \emph{\poke{} reality gear} is a virtual reality headset that uses different technologies in order to allow you to dive into the \poke{} world (Figure \ref{headset}). This headset creates a virtual environment in front of the user's eyes, while the attached sensors and neuro-stimulator launch you into  your \emph{\poke{} reality}.\medskip

The headset is manufactured in collaboration with \emph{Oculus VR, LCC.}

\begin{figure}[!ht]
\begin{center}
\includegraphics[width=0.7\linewidth]{src/vrheadset}
\end{center}
\caption[The \emph{\pokeT{} reality gear}]{The \emph{\poke{} reality gear}}
\label{headset}
\end{figure}

The \emph{\poke{} reality gear} is going to sit perfectly on your head and is very light with a weight of only 318 grams. In addition it is designed towards you not feeling any pressure anywhere as the weight is evenly distributed on your head. As the straps which keep the device in place are made of a very elastic and adjustable material, the headset is suitable for you regardless of your age and size. Furthermore, the built-in earphones are  intended to fit your ears without causing any pressure, contributing to optimal comfort. Since the the gear is easily connected to any device via plug-and-play, you should be ready to experience your \poke{} world out of the box.

\section{Customer Awareness}

Using the \emph{\poke{} reality gear} for more than three hours without interruption can cause nausea in certain sensitive individuals. It is advised to use the device for a maximum of two hours at a time. 

The visual feedback combined with the neuro-stimulators could make the user lose himself in the virtual world. Children under the age of 12 and psychologically unstable users are advised to disable the neuro-stimulators, as it can easily affect their senses and the perception of the real world. 

If you notice any discomfort during the use of the \emph{\poke{} reality gear} your first step should be to disable the neuro-stimulators.

\newpage
\section{The Headset In Detail}

The \emph{\poke{} reality gear} has six main parts:
\begin{enumerate}
\setlength\itemsep{.05em}
\setstretch{1}
\item Lenses
\item Display
\item Eye and head motion tracking technology
\item Audio
\item Neuro-stimulators 
\item Microphone [not visible]
\end{enumerate}

\begin{figure}[!ht]
\begin{center}
\includegraphics[width=0.9\linewidth]{src/vrcomponents}
\end{center}
\caption[The \emph{\pokeT{} reality gear} in detail]{The \emph{\poke{} reality gear} in detail}
\end{figure}

\subsection{Lenses} 
The lenses (Figure \ref{vrlens}) are made up of concentric prisms of uniform thickness. Thus, a crystal clear, stable vision is guaranteed. They were made to establish a focal point, so that you can perceive the depth of the image, which essentially means that you will be able to gaze beyond the virtual environment of the \emph{\poke{} reality game} and live the world while playing.

\begin{figure}[!ht]
\begin{center}
\includegraphics[width=0.4\linewidth]{src/vrlensa}
\includegraphics[width=0.4\linewidth]{src/vrlensb}
\end{center}
\caption[The lenses of the \emph{\pokeT{} reality gear}]{The lenses of the \emph{\poke{} reality gear}}
\label{vrlens}
\end{figure}

\subsection{Display}
The \emph{\poke{} reality gear} display (Figure \ref{vrdisplay}) features 1920$\times$1200 pixels for each eye with 90 Hz dual split screens. The screen is positioned a few centimeters in front of your eyes and projects a stereoscopic image. When these images are viewed through the lenses, you will feel like standing inside the \emph{\poke{} reality}.

\begin{figure}[!ht]
\begin{center}
\includegraphics[width=0.7\linewidth]{src/vrdispl}
\end{center}
\caption[The display of the \emph{\pokeT{} reality gear}]{The display of the \emph{\poke{} reality gear}}
\label{vrdisplay}
\end{figure}

\subsection{Tracking technology (eye and head motion tracking)}
Being built with multiple sensor inputs, the \emph{\poke{} reality gear} is able to track your head movement due to a gyroscope, an accelerometer and a compass. The visual angle in the virtual world can therefore adapt to the position of your head. Eye data is collected by using an internal eye tracker. This eye tracker includes two common components: a light source and a camera. The camera tracks the reflection of the light source along with visible ocular features such as the pupil. This data is used to extrapolate the orientation of the eye and ultimately the focus points on the display.
\begin{figure}[!ht]
\begin{center}
\includegraphics[width=0.7\linewidth]{src/vrtrack}
\end{center}
\caption[Usage of the \emph{\pokeT{} reality gear} tracking technology]{Usage of the \emph{\poke{} reality gear} tracking technology}
\label{vrtrack}
\end{figure}

\subsection{Audio}
The \emph{\poke{} reality gear} headphones (Figure \ref{vraudio}) are equipped with \emph{Head Transfer Function Technology} which, in combination with the head tracking sensors, offer a 3D audio 'spatialisation'. This method interpolates your head movement and alters the sounds of the game world accordingly to make them sound like they come from the real world.

\begin{figure}[!ht]
\begin{center}
\includegraphics[width=0.5\linewidth]{src/vraudio}
\end{center}
\caption[The \emph{\pokeT{} reality gear} headphones]{The \emph{\poke{} reality gear} headphones}
\label{vraudio}
\end{figure}

\subsection{Neuro-Stimulators}
There are two stimulators, located on each side of the \emph{\poke{} reality gear}, touching the temples when the headset has been put on. These stimulators work based on the \emph{neuron-manipulating wave interference} technique: Both devices send precisely timed pulses of high frequency photons at designated regions in your brain. These pulses influence your neurons and cause you to feel a variety of sensations such as touch and smell. 

This method is widely tested and completely safe for usage. It is a key feature for making the final step into your \poke{} world.

\begin{figure}[!ht]
\begin{center}
\includegraphics[width=0.7\linewidth]{src/vrneuro}
\end{center}
\caption[The neuro-stimulators are almost invisible to the naked eye]{The neuro-stimulators are almost invisible to the naked eye}
\label{vrneuro}
\end{figure}

\subsection{Microphone}
The microphone is enclosed within the \emph{\poke{} reality gear}. With this microphone you are able to talk to NPCs and give commands to your \poke{}. Additionally it offers the ambient noise volume adjustment feature, enabling you to emerge into the \poke{} world even in noisier environments.

\begin{figure}[!ht]
\begin{center}
\includegraphics[width=0.2\linewidth]{src/vrmic}
\end{center}
\caption[The integrated microphone]{The integrated microphone}
\label{vrmic}
\end{figure}

\chapter{The Game}
\label{software}

\section{Consumer Awareness}
\begin{wrapfigure}{R}{60pt}
\vspace{-18pt}
\includegraphics[scale =1]{src/ratingsymbol_e10}
\vspace{-30pt}
\end{wrapfigure}
The \emph{\poke{} reality game} is generally suitable for all ages. It contains minimal mild cartoon violence and some mild language.

To reiterate the warnings concerning \emph{\poke{} reality headset}: It is recommended to pause playing the game every couple of hours for at least five minutes. This ensures minimal disorientation after long game sessions.

\section{Scope Of The Game}

The main focus of the \emph{\poke{} reality game} is the exploration of the \poke{} world and the encounters with all kinds of \poke{}. In contrast to previous installments of \poke{} games, the \emph{\poke{} reality game} lets you choose the regions you want to explore and what \poke{} you want to encounter.

\begin{figure}[!ht]
\begin{center}
\includegraphics[width=0.7\linewidth]{src/pokemap}
\end{center}
\caption[The \emph{\pokeT{} reality game} map with all available 'regions']{The \emph{\poke{} reality game} map with all available 'regions'}
\label{pokemap}
\end{figure}

The \poke{} world is composed of all your favorite regions of the previous games. The following regions are open for you to explore (see Figure \ref{pokemap}):
\begin{multicols}{2}
\begin{itemize}
\item Kanto
\item Johto
\item Hoenn
\item Sinnoh
\item Unova
\item Kalos
\item Alma
\item Orre
\item Oblivia
\item Fiore
\item Orange Islands
\item Sevii Islands
\end{itemize} 
\end{multicols}

All these regions carry unique bioms and moods. There are different kinds of towns and landscapes to explore: From the windy \emph{Orange Islands} with their gusting sea to the steep mountains in the west of \emph{Kanto}.

One of the most important feature of the \emph{\poke{} reality game} is the fact, that you may pick the \poke{} you want to encounter. 

Do you want a more retrospective experience? No problem: Just deactivate all \poke{} except the first generation and you will only encounter \emph{Kanto-\poke{}}.

Do you want the full experience? Just keep the settings unchanged and get to know the whole range of \poke{}.

You may choose among the following sets of \poke{}, together over 800 \poke{}:
\begin{multicols}{2}
\begin{itemize}
\setlength\itemsep{.05em}
\setstretch{1}
\item 1. Generation: 151 \emph{Kanto-\pokeT}
\item 2. Generation: 100 \emph{Johto-\pokeT}
\item 3. Generation: 135 \emph{Hoenn-\pokeT}
\item 4. Generation: 107 \emph{Sinnoh-\pokeT}
\item 5. Generation: 156 \emph{Einall-\pokeT}
\item 6. Generation: 72 \emph{Kalos-\pokeT}
\item 7. Generation: 79 \emph{Alola-\pokeT}
\item[]
\end{itemize}
\end{multicols}

\section{Gameplay}

The gameplay consists mainly of three parts:
\begin{itemize}
\item Exploring the world
\item Fighting and catching new \poke{}
\item Battling \poke{}-trainers
\end{itemize}

\subsection{Exploration}

As you will be playing the game with your \emph{\poke{} reality gear}, the exploring itself is pretty much how you explore the real world, too: Walk and wonder (Figure \ref{exploration}).

\begin{figure}[!ht]
\begin{center}
\includegraphics[width=0.7\linewidth]{src/pokeworld}
\end{center}
\caption[Exploring the \pokeT{} world]{Exploring the \poke{} world}
\label{exploration}
\end{figure}

By keeping an eye open for details, you may discover hidden paths which lead to secret areas. There you may find rare \poke{} and unique items (Figure \ref{secrets}).

\begin{figure}[!ht]
\begin{center}
\includegraphics[width=0.7\linewidth]{src/pokeworld2}
\end{center}
\caption[Discovering secrets in the \pokeT{} world]{Discovering secrets in the \poke{} world}
\label{secrets}
\end{figure}

You can manage the following in the game menu:
\begin{itemize}
\item Caught \poke{}
\item Found secrets
\item Your game progress
\end{itemize}
This menu is designed to be used intuitively with your \emph{\poke{} reality gear} (Figure \ref{menu}).

\begin{figure}[!ht]
\begin{center}
\includegraphics[width=0.7\linewidth]{src/interface}
\end{center}
\caption[The \emph{\pokeT{} reality game} menu]{The \emph{\poke{} reality game} menu}
\label{menu}
\end{figure}

%\newpage
\subsection[\pokeT{}-battles]{\poke{}\,-battles}
During your exploration of the \poke{} world, you will encounter random \poke{} in the wild. In populated areas you may find other \poke{}-trainers (NPC). Both lead to \poke{} battles between you and your opponent.

A battle has the following sequence of events:
\begin{enumerate}
\item You choose a \poke{} to battle the opponent.
\item The battle proceeds turn-based. In one turn each you may take one of the following actions:
\begin{itemize}
\item Choose an offensive or defensive action for your \poke{} to execute
\item Use a special item from the inventory to buff the fighting \poke{}
\item Exchange the fighting \poke{} for a different \poke{} from your team
\item Retreat from the battle
\end{itemize}
\item These turns are repeated until the fight is over. If you lose, you will be teleported back to the nearest town. By winning, your \poke{} gain experience and you may get some item-rewards.
\end{enumerate}

%\newpage
In Figure \ref{battle} you can see the layout of a battle:
\begin{itemize}
\item Your \poke{} is at the bottom left with the corresponding name and its HP on the right.
\item Your opponent is at the top: the name and HP at the top left and the \poke{} on the right.
\end{itemize}

\begin{figure}[!ht]
\begin{center}
\includegraphics[width=0.7\linewidth]{src/interface3}
\end{center}
\caption[A \pokeT{} battle]{A \poke{} battle}
\label{battle}
\end{figure}

\newpage

\subsection[Catching \pokeT{}]{Catching \poke{}}
During battles with wild \poke{} you have the option to add them to your team by catching them. This takes place during one of your turns during the battle: Instead of attacking the opponent, you select a Pokéball from your inventory (Figure \ref{catching}). With some luck, this results in the \poke{} being caught by you. To increase your chances, it is recommended to weaken the \poke{} you are trying to catch before using a Pokéball by attacking it and thus reducing its HP.

\begin{figure}[!ht]
\begin{center}
\includegraphics[width=0.7\linewidth]{src/interface2}
\end{center}
\caption[Catching a \pokeT{} with Pokéballs using the game menu]{Catching a \poke{} with Pokéballs using the game menu}
\label{catching}
\end{figure}

\section{Conclusion}

Your final goal in the game is to become the ultimate \poke{}-master by catching as many \poke{} as you can and finding all the secrets hidden in the world.

By exploring the \poke{} world, catching \poke{} and battling trainers, you will get closer to this dream…

\chapter{Final Notes}

Thank you for reading this technical description of the \emph{\poke{} reality game and gear}. We hope you enjoy our new revolutionary way of experiencing the \poke{} world.

For information regarding the installation and usage of the \emph{\poke{} reality game and gear} please refer to our separately released \emph{technical instruction}.

\bigskip

If you have any feedback and suggestions regarding the \emph{\poke{} reality game and gear} you may contact us over email or send us a letter:\bigskip\\
\begin{tabular}{l L{.5}}
&Merko\& Strube Inc.\\
\Letter & Charmanderstreet 5\newline
D-04359 Squirtle Town\\
\Email & MandS.inc@pokereality.com\\
\end{tabular}

%\newpage
%\listoffigures
%\newpage
%\printbibliography

\end{document}