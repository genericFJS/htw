\section{Merkles Rätsel}
\subsubsection*{Schlüsselübertragung über unsicheren Kanal}
\slides{it1-52-krypto-asym_print}{3}
(1974 vom Studenten Ralph Merkle vorgeschlagen)
\slides{it1-52-krypto-asym_print}{4}

\section{Diffie-Hellman-Schlüsselvereinbarung}
\slides{it1-52-krypto-asym_print}{5}
(Martin Heffman, Whitefield Diffie und Ralph Merkle)
\subsection{Diskreter Logarithmus}
\slides{it1-52-krypto-asym_print}{6}
\subsubsection*{Beispiel}
\slides{it1-52-krypto-asym_print}{7}
\subsection{Algorithmus}
\slides{it1-52-krypto-asym_print}{8}
\subsection{Einschätzung}
\slides{it1-52-krypto-asym_print}{9}
\subsection{Diffie-Hellman-Beispiel}
\slides{it1-52-krypto-asym_print}{10}
\subsection{Fragen}
\slides{it1-52-krypto-asym_print}{11}
\section{Man-in-the-Middle}
\slides{it1-52-krypto-asym_print}{12}
\section{Public-Key-Kryptographie (Asym. Kryptographie)}
\slides{it1-52-krypto-asym_print}{13}

\section{ElGamal-Verschlüsselung}
\slides{it1-52-krypto-asym_print}{14}

\section{RSA}
\slides{it1-52-krypto-asym_print}{15}
Rivest, Shamir, Adleman (1977)
\subsubsection*{Problem: Faktorisierung}
\slides{it1-52-krypto-asym_print}{16}
\subsubsection*{Restklassenring}
\slides{it1-52-krypto-asym_print}{17}
\subsection{Euleresche Phi-Funktion}
\slides{it1-52-krypto-asym_print}{18}
\subsubsection{Satz von Euler und Fermat}
\slides{it1-52-krypto-asym_print}{19}
\slides{it1-52-krypto-asym_print}{20}
\slides{it1-52-krypto-asym_print}{21}
\subsubsection{Sonderfall für RSA}
\slides{it1-52-krypto-asym_print}{22}

\subsection{Schlüsselgenerierung}
\slides{it1-52-krypto-asym_print}{23}

\subsubsection{Ver-/Entschlüsselung}
\slides{it1-52-krypto-asym_print}{24}
\subsubsection*{Zahlenbeispiel}
\slides{it1-52-krypto-asym_print}{25}
\subsubsection{Sage}
\slides{it1-52-krypto-asym_print}{26}
\subsubsection*{Operationen in Sage}
\slides{it1-52-krypto-asym_print}{27}

\subsection{Eigenschaften}
\slides{it1-52-krypto-asym_print}{28}

\subsection{Signatur}
\slides{it1-52-krypto-asym_print}{29}
\subsubsection{Verfügbare Primzahlen}
\slides{it1-52-krypto-asym_print}{30}

\subsection{Implementierungsaspekte}
\slides{it1-52-krypto-asym_print}{31}
\subsubsection*{Modulare Exponentiation}
\slides{it1-52-krypto-asym_print}{32}
\subsubsection{Wahl des öffentlichen Exponents}
\slides{it1-52-krypto-asym_print}{33}
\subsubsection{Sitzungsschlüssel}
\slides{it1-52-krypto-asym_print}{34}

\subsection{Sicherheit}
\slides{it1-52-krypto-asym_print}{35}
\slides{it1-52-krypto-asym_print}{36}
\subsubsection*{Äquivalente Schlüssellängen}
\slides{it1-52-krypto-asym_print}{37}
\subsubsection{Probleme}
\slides{it1-52-krypto-asym_print}{38}

\section{Sicherheitslücken in der Praxis}
\slides{it1-52-krypto-asym_print}{40}

\section{Verschiedenes}
\subsection{Elliptische Kurven (ECC)}
\slides{it1-52-krypto-asym_print}{42}
\slides{it1-52-krypto-asym_print}{43}
\subsection{DH vs ECDH}
\slides{it1-52-krypto-asym_print}{44}

\section{Zusammenfassung}
\slides{it1-52-krypto-asym_print}{41}

\section{Fragen}
\slides{it1-52-krypto-asym_print}{39}










