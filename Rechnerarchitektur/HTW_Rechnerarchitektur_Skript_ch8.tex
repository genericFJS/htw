\section{Einführung}
\subsection{Trends}
\slides{08}{2}

\unimptnt{
\subsection{stete Leistungszunahme}
\slides{08}{3}

\subsection{Bewertung der Leistungsfähigkeit}
\slides{08}{4}
\slides{08}{5}
}

\section{Einfache Metriken: MIPS, FLOPS und Co.}

\slides{08}{6}
\slides{08}{7}

\subsection{Einfache Hardwaremaßzahlen}

\subsubsection{MIPS}
\slides{08}{8}
\unimptnt{
\subsubsection{FLOPS, OPS, TOPS}
\slides{08}{9}
}
\subsubsection{MPS, FLOPS \& Co. in der Kritik}
\unimptnt{
\slides{08}{10}
}
\slides{08}{11}

\section{Auswertung von Hardwaremaßnahmen}

\subsection{Mittlere Operationszeit}
\slides{08}{12}

\subsection{Ausführungszeit (Execution Time)}
\unimptnt{
\slides{08}{13}
}
\slides{08}{14}

\subsection{Parallelitätsprofil}
\slides{08}{15}

\subsection{Einfache Hardwaremaßzahlen:  CPU-Zeit und CPI}
\slides{08}{16}
\slides{08}{17}

\subsection{Befehlsverteilung und der CPI-Wert}
\slides{08}{18}

\subsection{Programmabhängiges Leistungsmodell}
\slides{08}{19}

\subsection{Minimierung der Anzahl ausgeführter Befehle}
\slides{08}{20}
\slides{08}{21}

\subsection{Minimierung der Taktzyklen pro Durchschnittsbefehl}
\slides{08}{22}

\subsection{Minimierung der Taktzykluszeit}
\slides{08}{23}

\subsection{(Normierte) Maximalleistung}
\slides{08}{24}
\slides{08}{25}

\subsection{Prozessor-Speicher-Leistungsbewertung}
\slides{08}{26}

\subsection{Zusammenfassung: Auswertung von Hardwaremaßnahmen}
\slides{08}{27}

\section{Parallelverarbeitung, Amdahl, Gustafson}
\subsection{Speedup}
\slides{08}{28}

\subsection{Effizienz}
\slides{08}{29}

\subsection{Ahmallsches Gesetz bei Parallelverarbeitung}
\slides{08}{30}
\unimptnt{
\slides{08}{31}
}
\slides{08}{32}
\subsubsection{Ahmdalsches Gesetz}
\slides{08}{33}

\subsection{Gustafsons Gesetz}
\subsubsection{Ansatz}
\slides{08}{34}
\subsubsection{Erkenntnisse}
\slides{08}{35}
\unimptnt{
\subsubsection{Herleitung}
\slides{08}{36}
}
\subsection{Ahmdal vs Gustafson}
\unimptnt{
\slides{08}{37}
}
Hauptunterschied Ahmdal vs Gustafson:\\
Ahmdal schließt von der Leistung eines seriellen Systems auf die eines parallelen Systems.\\
Gustafson schließt von der Leistung eines parallelen Systems, wie langsam es auf einem seriellen System laufen würde.\\
Dem entsprechend kommen unterschiedlich Annahmen heraus.

\subsection{Skalierbarkeit eines Parallelsystems}
\slides{08}{38}

\section{Benchmarks}
\subsection{Benchmark-Programme}
\slides{08}{39}
\unimptnt{
\subsection{Benchmark-Typen}
\subsubsection{Reale Applikationen}
\slides{08}{40}
\subsubsection{Toy Benchmarks}
\slides{08}{41}
\subsubsection{Kernels}
\slides{08}{42}
\slides{08}{43}
\subsubsection{Synthetische Benchmarks}
\slides{08}{44}
\subsubsection*{Whetstone / 1976}
\slides{08}{45}
\subsubsection*{Dhrystone / 1984}
\slides{08}{46}
}
\subsection{Standardisierte Benchmarks}
\slides{08}{47}
\unimptnt{
\slides{08}{48}
\subsubsection{SPEC}
\slides{08}{49}
\subsubsection*{CPU2006}
\slides{08}{50}
}

\section{Abschließendes}
\unimptnt{
\subsection{Benchmark-Wahn und seine Probleme}
\slides{08}{51}
\subsection{Leistungsmessung: übliche Trugschlüsse}
\slides{08}{52}
}
\subsection{Leistungsmessung: Zusammenfassung}
\slides{08}{53}












