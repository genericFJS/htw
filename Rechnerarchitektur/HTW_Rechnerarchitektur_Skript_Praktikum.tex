\part{Praktikum}
\slidePath{Praktikum/RA_043_P00D}

\chapter{Einführung}
\section{Mikrocontroller}
\slide{2}
\subsection{Unterscheidung und Architektur}
\slide{3}

\unimptnt{
\subsection{Taktung}
\slide{4}
\subsection{Typische Peripherie}
\slide{5}
}
\subsection{Einsatzbeispiele}
\slide{6}
\subsection*{Computerklassen/-typen}
\begin{itemize}
\item Wegwerfcomputer
\item Microcontroller, embedded Computer
\item Mobile Computer
\item Personalcomputer
\item Server
\item Großrechner
\end{itemize}

\subsection{Speicherbedarf}
\slide{7}
\slide{8}

\unimptnt{
\chapter{Fallbeispiel Atmel AVR}
\section{AVR und AVR-Architektur im Überblick}
\subsection{Allgemeines}
\slide{9}
\subsection{CPU und Archtitektur}
\slide{10}
\slide{11}
\slide{12}
\subsection{CPU zu Statusregister SREG}
\slide{13}
\subsection{Interne Perepherie}
\slide{14}

\section{Speichertypen}
\subsection{Speicher}
\slide{15}
\slide{16}
\subsection{Stack}
\slide{17}

\section{Beispielcontroller}
\subsection{Typenbeispiele}
\slide{18}
\subsection{Blockschaltbild und Besonderheiten}
\slide{19}
\subsection{Gehäuse und Anschlüsse}
\slide{20}

\section{Entwicklungswerkzeuge}
\subsection{Demoplattform AVR Butterfly}
\slide{21}
\subsection{Hardware}
\slide{22}
\subsubsection{Experimentalplattform STK500/501}
\slide{23}
\slide{24}
\subsection{Software}
\slide{25}

\section{Baugruppen und deren Nutzung und Programmierung}
\subsection{System- und Takt-Anschlüsse}
\slide{26}
\subsection{Bidirektionale Ports}
\slide{27}
\slide{28}
\slide{29}
\slide{30}
uint8\_t für 8-Bit unsigned Integer - gut, da AVR 8-Bit ist.
\slide{31}
\slide{32}
\slide{33}
}
Wichtig: Bit-Tricks !!\\
1<<3 : 3. Bit wird auf 1 gesetzt; |= (bitweises Oder) belässt alle Bits gleich, bis auf das 3. (in dem Bsp.), das wird 1.\\
\textasciitilde() (bitweise Negation); \&= belässt alle Bits gleich, bis auf das 3., das wird 1.\\
\^{}= (exklusives Oder) fürs Invertieren
\unimptnt{
\slide{34}
\slide{35}
\slide{36}
\slide{37}
\slide{38}

\subsection{Alternative Nutzung der Universalports}
\slide{39}
\slide{40}
\slide{41}
\slide{42}

\subsection{Interrupts}
\subsubsection{Grundprinzip}
\slide{43}

\subsubsection{Interruptquellen}
\slide{44}
\subsubsection*{Beispiel ATmega128}
\slide{45}
\slide{46}
\subsubsection{Anforderungen an Interrupt-Routinen}
\slide{47}
Interrupt in C: mit globalen Variablen
\subsubsection{Beispiel für Interruptnutzung}
\slide{48}
Hinweis Atmel-Board: 0 ist "1" und 1 ist "0" (im Sinne, dass bei 0 die Lampe leuchtet)
\slide{49}
\slide{50}
\slide{51}

\subsection{Timer/Counter}
\slide{52}
\subsubsection{Prescaler (Vorteiler)}
\slide{53}
\subsubsection{Betriebsmodi}
\slide{54}

\subsubsection{Normal Mode}
\slide{55}
\subsubsection{CTC-Mode}
\slide{56}
\subsubsection{PWM-Modus}
\slide{57}
\subsubsection{Beispiel 1 für Timer/Counter-Nutzung}
\slide{58}
\subsubsection{Beispiel 2 für Timer/Counter-Nutzung}
\slide{59}
\slide{60}
\slide{61}
\slide{62}
\slide{63}
}