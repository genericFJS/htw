\section{Bus: Einführung}
\subsection{Definition}
\slides{09}{2}
\subsection{Generelle Struktur}
\slides{09}{3}
\subsection*{Was definiert einen Bus?}
\slides{09}{4}
\subsection{Vor- und Nachteile}
\slides{09}{5}
\subsection{Terminologie}
\slides{09}{6}
\slides{09}{7}
\slides{09}{8}
\slides{09}{9}
\subsection{Physikalische Struktur: parallel und seriell}
\slides{09}{10}

\section{Hierarchie}
\subsection*{Hierarchische Organisation}
\slides{09}{11}
\subsection{Prozessorbus}
\slides{09}{12}
\subsection{Speicherbus}
\slides{09}{13}
\subsection{Peripheriebus}
\slides{09}{14}
\subsection{Ein-/Ausgabebus}
\slides{09}{15}

\unimptnt{
\subsection{Alternative Einteilung}
\slides{09}{16}
}

\section{Topologie und Kopplung}
\subsection{Topologien}
\slides{09}{17}
\subsubsection{Repeater}
\slides{09}{18}
\subsubsection{Hub (Nabe/Mittelpunkt)}
\slides{09}{19}
\subsubsection{Bridge (Brücke)}
\slides{09}{20}

\section{Dedizierung}
\subsection{Dedizierter Bus}
\slides{09}{21}
\subsection{Nichtdedizierter Bus}
\slides{09}{22}

\section{Partitionierung}
\subsection{Ressourcenpartitionierter Bus}
\slides{09}{23}
\unimptnt{
\subsubsection*{Charakteristika}
\slides{09}{24}
}
\subsection{Funktionspartitionierter Bus}
\slides{09}{25}

\section{Transaktion und Übertragungsarten}
\subsection{Transaktion}
\subsubsection{Überblick}
\slides{09}{26}
\subsubsection*{(Nicht-)multiplexiert}
\slides{09}{27}
\unimptnt{
\subsection{Multiplexbus}
\slides{09}{28}
}

\subsection{Übertragungsarten}
\subsubsection{Überblick}
\slides{09}{29}
\subsubsection{WRITE-Operation}
\slides{09}{30}
\subsubsection{READ-Operation}
\slides{09}{31}
\unimptnt{
\subsubsection{Bus mit Blockübertragung (Burst-Zyklus)}
\slides{09}{32}
}

\section{Adressierung}
\subsection{Einführung}
\slides{09}{33}

\subsection{logisch und geographisch}
\slides{09}{34}

\subsection{Uni-, Multi-, Broadcast}
\slides{09}{35}

\section{Protokolle}
\subsection{Einführung}
\slides{09}{36}

\unimptnt{
\subsubsection{Typischer Ablauf}
\slides{09}{37}
}
\subsection{Zeitlicher Ablauf}
\slides{09}{38}

\subsubsection{synchrones Zeitverhalten}
\slides{09}{39}

\subsubsection{semi-synchrones Zeitverhalten}
\slides{09}{40}

\subsubsection{asynchroner Zeitverlauf}
\slides{09}{41}

\section{Zuteilung}

\subsection{Busarbitrierung, Bus Arbitration}
\slides{09}{42}
\slides{09}{43}

\subsection{Klassen}
\slides{09}{44}

\subsection{Statisch (Static Bus Arbitration)}
\slides{09}{45}

\subsection{Dynamisch (Dynamic Bus Arbitration)}
\slides{09}{46}

\subsection{Anordnung des Buszuteilers}
\slides{09}{47}
\slides{09}{48}

\subsection{Über dedizierte Leitungen}
\slides{09}{49}

\subsection{Zuteilung}
\slides{09}{50}