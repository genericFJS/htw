\unimptnt{
\section{Einführung}
--
}

\section{Statische Vorhersage}
\subsection{Stall/Freeze}
\slides{06}{2}
\subsection{Delayed Branches}
\slides{06}{3}
\subsection{Predict (Not) Taken}
\slides{06}{4}
\unimptnt{
\subsection{Erweiterung}
\slides{06}{5}
}

\subsection{Compiler-adressiert}
\slides{06}{6}

\subsection{Diskussion}
\slides{06}{7}

\section{Dynamische Vorhersage}
\slides{06}{8}

\subsection{1-Bit-Prädikator}
\slides{06}{9}
\subsubsection{Vor-/Nachteile}
\slides{06}{10}

\subsection{2-Bit-Prädikator}
\slides{06}{11}
\subsubsection{Varianten}
\slides{06}{12}
\subsection{Vorteile}
\slides{06}{13}
\subsubsection{Probleme}
\slides{06}{14}

\unimptnt{
\subsection{n-Bit-Prädikator}
\slides{06}{15}
\subsubsection{Weiter Prädikatoren}
\slides{06}{16}
}

\subsection{Branch History Table / Branch Target Buffer}
\subsubsection{Branch History Table - BHT}
\slides{06}{17}
Puffer wichtig, weil die Vorhersage sofort erreichbar sein muss (nicht im Hauptspeicher, der ggf. auch virtueller Speicher und damit langsam sein kann).
\subsubsection{Branch Target Buffer - BTB}
\slides{06}{18}
\subsubsection*{Vorteile/Probleme}
\slides{06}{19}
\unimptnt{
\subsubsection*{Variationsmöglichkeiten}
\slides{06}{20}
}

%\subsection{Leistungsfähigkeit}